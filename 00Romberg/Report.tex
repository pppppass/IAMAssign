% !TeX encoding = UTF-8
% !TeX program = LuaLaTeX
% !TeX spellcheck = en_US

% Author : Zhihan Li
% Description : Report for Romberg integration

\documentclass[english, nochinese]{../textmpls/pkupaper}

\usepackage[paper]{../textmpls/def}

\addbibresource{Bibliography.bib}

\newcommand{\cuniversity}{Peking University}
\newcommand{\cthesisname}{Introduction to Applied Mathematics}
\newcommand{\titlemark}{Report for Romberg integration}

\DeclareRobustCommand{\authoring}%
{%
\begin{tabular}{c}%
Zhihan Li \\%
PKU SMS%
\end{tabular}%
}

\title{\titlemark}
\author{\authoring}

\begin{document}

\maketitle

\section{Another proof of positive coefficient}

Another way to proof that coefficients of Romberg integration (using $ 1 / 2 $ decay rate of step size) without mathematical induction follows.

Let the trapezoid quadrature of step size $h$ be $ T \rbr{h} $ and the Romberg quadrature gives
\begin{gather}
R_0 \rbr{h} = T \rbr{h} \\
R_k \rbr{h} = \frac{ 4^k R_{ k - 1 } \rbr{\frac{h}{2}} - R_{ k - 1 } \rbr{h} }{ 4^k - 1 }.
\end{gather}
Therefore,
\begin{equation}
R_k \rbr{h} = a_{ k 0 } T \rbr{h} + a_{ k 1 } T \rbr{\frac{h}{2}} + \cdots + a_{ k k } T \rbr{\frac{h}{2^k}}
\end{equation}
satisfies $ a_0 + a_1 x + \cdots + a_k x^k \sim \rbr{ 4 x - 1 } \rbr{ 16 x - 1 } \cdots \rbr{ 4^k x - 1 } $, where $\sim$ means equals up to a positive proportional constant here.

Note that $ f \rbr{ a + \frac{1}{2^l} \rbr{ 2 m + 1 } \rbr{ b - a } } $ only occurs in $ T \rbr{\frac{h}{2^l}}, T \rbr{\frac{h}{2^{ l + 1 }}}, \cdots $, and for $ l \ge 1 $ the coefficient of this term in $ T \rbr{\frac{h}{2^k}} $ is $ b_{ l k } = \frac{1}{2^k} $. Therefore, we may compute the coefficient of such term in $ R_k \rbr{h} $
\begin{equation}
c_{ k l } = \frac{1}{2^k} a_{ k k } + \frac{1}{ 2^{ k - 1 } } a_{ k \rbr{ k - 1 } } + \cdots + \frac{1}{2^l} a_{ k l }
\end{equation}
for $ l \ge 1 $.

It is trivial to check $ c_{ k 0 } > 0 $. To prove $ c_{ k l } > 0 $ for $ l \ge 1 $, it is equivalent to show
\begin{equation}
\rbr{ 2 x - 1 } \rbr{ 8 x - 1 } \cdots \rbr{ 2^{ 2 k - 1 } x - 1 } = \widetilde{a}_0 + \widetilde{a}_1 x + \cdots + \widetilde{a}_k x^k
\end{equation}
satisfies $ \sume{i}{l}{k}{ \widetilde{a}_k } > 0 $ for $ l \ge 1 $.

An example of such polynomial of $ k = 3 $ is $ 512 x^3 - 336 x^2 + 42 x - 1 $ and $ 512 > 0 $, $ 512 - 336 > 0 $, $ 512 - 336 + 42 > 0 $. Noticing signs alternating, it is sufficient to prove
\begin{equation}
\rbr{ 2 x + 1 } \rbr{ 8 x + 1 } \cdots \rbr{ 2^{ 2 k - 1 } x + 1 }
\end{equation}
has strictly decreasing coefficients, which is equivalent to
\begin{equation}
p \rbr{x} = \rbr{ x - 1 } \rbr{ x + 2 } \rbr{ x + 8 } \cdots \rbr{ x + 2^{ 2 k - 1 } }
\end{equation}
has positive coefficients except for the constant term.

It is therefore sufficient to prove $ p^{\rbr{l}} \rbr{0} > 0 $ for $ l = 1, 2, \cdots, k $. Because that $ p \rbr{x} $ is a polynomial of degree $ k + 1 $ with roots $ -2^{ 2 k - 1 }, \cdots, -8, -2, 1 $, therefore using Rolle's mean value theorem it remains to prove $ p' \rbr{0} > 0 $. (Therefore the rightmost zero of $p^{\rbr{l}}$ must lies on the left to that of $p'$, which is negative) Moreover $ p' \rbr{0} > 0 $ directly follows from
\begin{equation}
p' \rbr{0} \sim 1 - \frac{1}{2} - \frac{1}{8} - \cdots - \frac{1}{2^{ 2 k - 1 }} > 0.
\end{equation}

\sqed

There are other quadrature using $ R_k \rbr{ h / k } $ and $ R_k \rbr{ h / \rbr{ k + 1 } } $ to perform extrapolation. Further details and error estimation results are shown in \parencite{havie_error_1972} \parencite{havie_romberg_1977}.

\section{Largest algebraic degree of precision}

Finding the largest algebraic degree can be cast as a problem related to linear programming, that is, find the largest $k$, such that
\begin{equation} \label{Eq:Orig}
\begin{array}{ll}
\text{min} & 0, \\
\text{w. r. t.} & a_0, a_1, \cdots, a_n, \\
\text{s. t.} & \sume{i}{0}{n}{ a_i i^0 } = 0, \\
& \sume{i}{0}{n}{ a_i i^1 } = 0, \\
& \vdots, \\
& \sume{i}{0}{n}{ a_i i^l } = 0, \\
& a_0, a_1, \cdots, a_n \ge 0
\end{array}
\end{equation}
is feasible.

Note that this linear programming involvese Vandermonde matrix, and therefore precondition is made and the problem turns out to be
\begin{equation}
\begin{array}{ll}
\text{min} & 0, \\
\text{w. r. t.} & a_0, a_1, \cdots, a_n, \\
\text{s. t.} & \sume{i}{0}{n}{ T_0 \rbr{ -1 + \frac{ 2 i }{n} } } = \intb{-1}{1}{ T_0 \rbr{x} \sd x }, \\
& \sume{i}{0}{n}{ T_1 \rbr{ -1 + \frac{ 2 i }{n} } } = \intb{-1}{1}{ T_1 \rbr{x} \sd x }, \\
& \vdots, \\
& \sume{i}{0}{n}{ T_l \rbr{ -1 + \frac{ 2 i }{n} } } = \intb{-1}{1}{ T_l \rbr{x} \sd x }, \\
& a_0, a_1, \cdots, a_n \ge 0.
\end{array}
\end{equation}

The correspondence between $n$ and $l_{\text{max}}$ is shown in Table \ref{Tbl:MaxL}.

\begin{table}[htbp]
\centering
\caption{Correspondence between $n$ and $l_{\text{max}}$}
\label{Tbl:MaxL}
\begin{tabular}{|c|c|c|c|c|c|c|c|c|c|c|}
\hline
$n$ & 1 & 2 & 3 & 4 & 5 & 6 & 7 & 8 & 9 & 10 \\
\hline
$l_{\text{max}}$ & 1 & 3 & 3 & 5 & 5 & 7 & 7 & 7 & 9 & 9 \\
\hline
$n$ & 11 & 12 & 13 & 14 & 15 & 16 & 17 & 18 & 19 & 20 \\
\hline
$l_{\text{max}}$ & 9 & 9 & 11 & 11 & 11 & 11 & 13 & 13 & 13 & 13 \\
\hline
\end{tabular}
\end{table}

Observation yields that $l_{\text{max}}$ increases with respect to $n$ and therefore the smallest $n$ given a fixed $l_{\text{max}}$ is shown in Table \ref{Tbl:MinN}.

\begin{table}[htbp]
\centering
\caption{Smallest $n$ given a fixed $l_{\text{max}}$}
\label{Tbl:MinN}
\begin{tabular}{|c|c|c|c|c|c|c|c|c|c|c|}
\hline
$l_{\text{max}}$ & 1 & 3 & 5 & 7 & 9 & 11 & 13 & 15 & 17 & 19 \\
\hline
$n_{\text{min}}$ & 1 & 2 & 4 & 6 & 9 & 13 & 17 & 22 & 26 & 32 \\
\hline
$l_{\text{max}}$ & 21 & 23 & 25 & 27 & 29 & 31 & 33 & 35 & 37 & 39 \\
\hline
$n_{\text{min}}$ & 38 & 45 & 52 & 60 & 69 & 78 & 88 & 98 & 109 & 123 \\
\hline    
\end{tabular}
\end{table}

Looking up the table in \href{https://oeis.org/A129337}{OEIS} yields that \parencite{forster_weighted_1986} have encountered the same sequence in a different setting. (Chebyshev-type quadrature)

\section{Discussion}

There are several other problems involved.

\begin{partlist}
\item Original linear programming \eqref{Eq:Orig} has rational coefficients, while directly solving it using simplex method leads to severe numerical instability. Using simplex methods manipulating only \emph{integers} may help here.
\item It remains to prove that the sequence in Table \ref{Tbl:MaxL} increases, and coincides the one in \parencite{forster_weighted_1986}.
\item High dimensional integration is rather complicated and solving such linear programming may indicate better quadrature in these cases.
\item There is some structure in Romberg integration, that is, nodes
\begin{equation}
f \rbr{ a + \frac{1}{2^l} \rbr{ 2 m + 1 } \rbr{ b - a } }
\end{equation}
have identical coefficients for different $m$. Largest algebraic degree of quadrature having this structure can be computed.
\item It is well known that trapezoid integration and Simpson integration can be cast as \emph{reconstruction} schemes. That is, for a function $f$, we have samples $ P f $ as a $ \rbr{ n + 1 } $-tuple, where $P$ is the restriction operator. Therefore, We may reconstruct from $ P f $ to get $ R P f $, which is a function easy to integrate. Moreover, we define the quadrature $ I = \int R P f $. For reconstruction schemes, algebraic degree of precision can be easily derived, and positiveness of coefficients reflects positiveness of integrals to basis functions of $R$. It is a question which quadrature can be casted as a reconstruction scheme, and further which quadrature of positive coefficient can be casted as a reconstruction scheme where basis functions of $R$ are all positive, which is tighter than integrals being positive and serves stability of reconstruction.
\end{partlist}

\printbibliography

\end{document}
