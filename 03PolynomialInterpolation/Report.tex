% !TeX encoding = UTF-8
% !TeX program = LuaLaTeX
% !TeX spellcheck = en_US

% Author : Zhihan Li
% Description : Report for Lecture 3 --- Polynomial Interpolation

\documentclass[english, nochinese]{../TeXTemplate/pkupaper}

\usepackage[paper, pgf, algorithm]{../TeXTemplate/def}

\newcommand{\cuniversity}{Peking University}
\newcommand{\cthesisname}{Introduction to Applied Mathematics}
\newcommand{\titlemark}{Assignment for Lecture 3}

\DeclareRobustCommand{\authoring}%
{%
\begin{tabular}{c}%
Zhihan Li \\%
1600010653%
\end{tabular}%
}

\title{\titlemark}
\author{\authoring}
\begin{document}

\maketitle

\begin{thmquestion}
\ 
\begin{thmanswer}
The graph for interpolations to $ f_1 \rbr{x} = \frac{1}{ 1 + x^2 } $ using equally spaced nodes is shown in Figure \ref{Fig:SpaceTan}.
\begin{figure}[htbp]
\centering \scalebox{0.8}{%% Creator: Matplotlib, PGF backend
%%
%% To include the figure in your LaTeX document, write
%%   \input{<filename>.pgf}
%%
%% Make sure the required packages are loaded in your preamble
%%   \usepackage{pgf}
%%
%% Figures using additional raster images can only be included by \input if
%% they are in the same directory as the main LaTeX file. For loading figures
%% from other directories you can use the `import` package
%%   \usepackage{import}
%% and then include the figures with
%%   \import{<path to file>}{<filename>.pgf}
%%
%% Matplotlib used the following preamble
%%   \usepackage{fontspec}
%%
\begingroup%
\makeatletter%
\begin{pgfpicture}%
\pgfpathrectangle{\pgfpointorigin}{\pgfqpoint{6.000000in}{4.000000in}}%
\pgfusepath{use as bounding box, clip}%
\begin{pgfscope}%
\pgfsetbuttcap%
\pgfsetmiterjoin%
\definecolor{currentfill}{rgb}{1.000000,1.000000,1.000000}%
\pgfsetfillcolor{currentfill}%
\pgfsetlinewidth{0.000000pt}%
\definecolor{currentstroke}{rgb}{1.000000,1.000000,1.000000}%
\pgfsetstrokecolor{currentstroke}%
\pgfsetdash{}{0pt}%
\pgfpathmoveto{\pgfqpoint{0.000000in}{0.000000in}}%
\pgfpathlineto{\pgfqpoint{6.000000in}{0.000000in}}%
\pgfpathlineto{\pgfqpoint{6.000000in}{4.000000in}}%
\pgfpathlineto{\pgfqpoint{0.000000in}{4.000000in}}%
\pgfpathclose%
\pgfusepath{fill}%
\end{pgfscope}%
\begin{pgfscope}%
\pgfsetbuttcap%
\pgfsetmiterjoin%
\definecolor{currentfill}{rgb}{1.000000,1.000000,1.000000}%
\pgfsetfillcolor{currentfill}%
\pgfsetlinewidth{0.000000pt}%
\definecolor{currentstroke}{rgb}{0.000000,0.000000,0.000000}%
\pgfsetstrokecolor{currentstroke}%
\pgfsetstrokeopacity{0.000000}%
\pgfsetdash{}{0pt}%
\pgfpathmoveto{\pgfqpoint{0.750000in}{0.500000in}}%
\pgfpathlineto{\pgfqpoint{5.400000in}{0.500000in}}%
\pgfpathlineto{\pgfqpoint{5.400000in}{3.520000in}}%
\pgfpathlineto{\pgfqpoint{0.750000in}{3.520000in}}%
\pgfpathclose%
\pgfusepath{fill}%
\end{pgfscope}%
\begin{pgfscope}%
\pgfpathrectangle{\pgfqpoint{0.750000in}{0.500000in}}{\pgfqpoint{4.650000in}{3.020000in}}%
\pgfusepath{clip}%
\pgfsetrectcap%
\pgfsetroundjoin%
\pgfsetlinewidth{0.803000pt}%
\definecolor{currentstroke}{rgb}{0.690196,0.690196,0.690196}%
\pgfsetstrokecolor{currentstroke}%
\pgfsetdash{}{0pt}%
\pgfpathmoveto{\pgfqpoint{0.961364in}{0.500000in}}%
\pgfpathlineto{\pgfqpoint{0.961364in}{3.520000in}}%
\pgfusepath{stroke}%
\end{pgfscope}%
\begin{pgfscope}%
\pgfsetbuttcap%
\pgfsetroundjoin%
\definecolor{currentfill}{rgb}{0.000000,0.000000,0.000000}%
\pgfsetfillcolor{currentfill}%
\pgfsetlinewidth{0.803000pt}%
\definecolor{currentstroke}{rgb}{0.000000,0.000000,0.000000}%
\pgfsetstrokecolor{currentstroke}%
\pgfsetdash{}{0pt}%
\pgfsys@defobject{currentmarker}{\pgfqpoint{0.000000in}{-0.048611in}}{\pgfqpoint{0.000000in}{0.000000in}}{%
\pgfpathmoveto{\pgfqpoint{0.000000in}{0.000000in}}%
\pgfpathlineto{\pgfqpoint{0.000000in}{-0.048611in}}%
\pgfusepath{stroke,fill}%
}%
\begin{pgfscope}%
\pgfsys@transformshift{0.961364in}{0.500000in}%
\pgfsys@useobject{currentmarker}{}%
\end{pgfscope}%
\end{pgfscope}%
\begin{pgfscope}%
\pgftext[x=0.961364in,y=0.402778in,,top]{\rmfamily\fontsize{10.000000}{12.000000}\selectfont \(\displaystyle -1.00\)}%
\end{pgfscope}%
\begin{pgfscope}%
\pgfpathrectangle{\pgfqpoint{0.750000in}{0.500000in}}{\pgfqpoint{4.650000in}{3.020000in}}%
\pgfusepath{clip}%
\pgfsetrectcap%
\pgfsetroundjoin%
\pgfsetlinewidth{0.803000pt}%
\definecolor{currentstroke}{rgb}{0.690196,0.690196,0.690196}%
\pgfsetstrokecolor{currentstroke}%
\pgfsetdash{}{0pt}%
\pgfpathmoveto{\pgfqpoint{1.489773in}{0.500000in}}%
\pgfpathlineto{\pgfqpoint{1.489773in}{3.520000in}}%
\pgfusepath{stroke}%
\end{pgfscope}%
\begin{pgfscope}%
\pgfsetbuttcap%
\pgfsetroundjoin%
\definecolor{currentfill}{rgb}{0.000000,0.000000,0.000000}%
\pgfsetfillcolor{currentfill}%
\pgfsetlinewidth{0.803000pt}%
\definecolor{currentstroke}{rgb}{0.000000,0.000000,0.000000}%
\pgfsetstrokecolor{currentstroke}%
\pgfsetdash{}{0pt}%
\pgfsys@defobject{currentmarker}{\pgfqpoint{0.000000in}{-0.048611in}}{\pgfqpoint{0.000000in}{0.000000in}}{%
\pgfpathmoveto{\pgfqpoint{0.000000in}{0.000000in}}%
\pgfpathlineto{\pgfqpoint{0.000000in}{-0.048611in}}%
\pgfusepath{stroke,fill}%
}%
\begin{pgfscope}%
\pgfsys@transformshift{1.489773in}{0.500000in}%
\pgfsys@useobject{currentmarker}{}%
\end{pgfscope}%
\end{pgfscope}%
\begin{pgfscope}%
\pgftext[x=1.489773in,y=0.402778in,,top]{\rmfamily\fontsize{10.000000}{12.000000}\selectfont \(\displaystyle -0.75\)}%
\end{pgfscope}%
\begin{pgfscope}%
\pgfpathrectangle{\pgfqpoint{0.750000in}{0.500000in}}{\pgfqpoint{4.650000in}{3.020000in}}%
\pgfusepath{clip}%
\pgfsetrectcap%
\pgfsetroundjoin%
\pgfsetlinewidth{0.803000pt}%
\definecolor{currentstroke}{rgb}{0.690196,0.690196,0.690196}%
\pgfsetstrokecolor{currentstroke}%
\pgfsetdash{}{0pt}%
\pgfpathmoveto{\pgfqpoint{2.018182in}{0.500000in}}%
\pgfpathlineto{\pgfqpoint{2.018182in}{3.520000in}}%
\pgfusepath{stroke}%
\end{pgfscope}%
\begin{pgfscope}%
\pgfsetbuttcap%
\pgfsetroundjoin%
\definecolor{currentfill}{rgb}{0.000000,0.000000,0.000000}%
\pgfsetfillcolor{currentfill}%
\pgfsetlinewidth{0.803000pt}%
\definecolor{currentstroke}{rgb}{0.000000,0.000000,0.000000}%
\pgfsetstrokecolor{currentstroke}%
\pgfsetdash{}{0pt}%
\pgfsys@defobject{currentmarker}{\pgfqpoint{0.000000in}{-0.048611in}}{\pgfqpoint{0.000000in}{0.000000in}}{%
\pgfpathmoveto{\pgfqpoint{0.000000in}{0.000000in}}%
\pgfpathlineto{\pgfqpoint{0.000000in}{-0.048611in}}%
\pgfusepath{stroke,fill}%
}%
\begin{pgfscope}%
\pgfsys@transformshift{2.018182in}{0.500000in}%
\pgfsys@useobject{currentmarker}{}%
\end{pgfscope}%
\end{pgfscope}%
\begin{pgfscope}%
\pgftext[x=2.018182in,y=0.402778in,,top]{\rmfamily\fontsize{10.000000}{12.000000}\selectfont \(\displaystyle -0.50\)}%
\end{pgfscope}%
\begin{pgfscope}%
\pgfpathrectangle{\pgfqpoint{0.750000in}{0.500000in}}{\pgfqpoint{4.650000in}{3.020000in}}%
\pgfusepath{clip}%
\pgfsetrectcap%
\pgfsetroundjoin%
\pgfsetlinewidth{0.803000pt}%
\definecolor{currentstroke}{rgb}{0.690196,0.690196,0.690196}%
\pgfsetstrokecolor{currentstroke}%
\pgfsetdash{}{0pt}%
\pgfpathmoveto{\pgfqpoint{2.546591in}{0.500000in}}%
\pgfpathlineto{\pgfqpoint{2.546591in}{3.520000in}}%
\pgfusepath{stroke}%
\end{pgfscope}%
\begin{pgfscope}%
\pgfsetbuttcap%
\pgfsetroundjoin%
\definecolor{currentfill}{rgb}{0.000000,0.000000,0.000000}%
\pgfsetfillcolor{currentfill}%
\pgfsetlinewidth{0.803000pt}%
\definecolor{currentstroke}{rgb}{0.000000,0.000000,0.000000}%
\pgfsetstrokecolor{currentstroke}%
\pgfsetdash{}{0pt}%
\pgfsys@defobject{currentmarker}{\pgfqpoint{0.000000in}{-0.048611in}}{\pgfqpoint{0.000000in}{0.000000in}}{%
\pgfpathmoveto{\pgfqpoint{0.000000in}{0.000000in}}%
\pgfpathlineto{\pgfqpoint{0.000000in}{-0.048611in}}%
\pgfusepath{stroke,fill}%
}%
\begin{pgfscope}%
\pgfsys@transformshift{2.546591in}{0.500000in}%
\pgfsys@useobject{currentmarker}{}%
\end{pgfscope}%
\end{pgfscope}%
\begin{pgfscope}%
\pgftext[x=2.546591in,y=0.402778in,,top]{\rmfamily\fontsize{10.000000}{12.000000}\selectfont \(\displaystyle -0.25\)}%
\end{pgfscope}%
\begin{pgfscope}%
\pgfpathrectangle{\pgfqpoint{0.750000in}{0.500000in}}{\pgfqpoint{4.650000in}{3.020000in}}%
\pgfusepath{clip}%
\pgfsetrectcap%
\pgfsetroundjoin%
\pgfsetlinewidth{0.803000pt}%
\definecolor{currentstroke}{rgb}{0.690196,0.690196,0.690196}%
\pgfsetstrokecolor{currentstroke}%
\pgfsetdash{}{0pt}%
\pgfpathmoveto{\pgfqpoint{3.075000in}{0.500000in}}%
\pgfpathlineto{\pgfqpoint{3.075000in}{3.520000in}}%
\pgfusepath{stroke}%
\end{pgfscope}%
\begin{pgfscope}%
\pgfsetbuttcap%
\pgfsetroundjoin%
\definecolor{currentfill}{rgb}{0.000000,0.000000,0.000000}%
\pgfsetfillcolor{currentfill}%
\pgfsetlinewidth{0.803000pt}%
\definecolor{currentstroke}{rgb}{0.000000,0.000000,0.000000}%
\pgfsetstrokecolor{currentstroke}%
\pgfsetdash{}{0pt}%
\pgfsys@defobject{currentmarker}{\pgfqpoint{0.000000in}{-0.048611in}}{\pgfqpoint{0.000000in}{0.000000in}}{%
\pgfpathmoveto{\pgfqpoint{0.000000in}{0.000000in}}%
\pgfpathlineto{\pgfqpoint{0.000000in}{-0.048611in}}%
\pgfusepath{stroke,fill}%
}%
\begin{pgfscope}%
\pgfsys@transformshift{3.075000in}{0.500000in}%
\pgfsys@useobject{currentmarker}{}%
\end{pgfscope}%
\end{pgfscope}%
\begin{pgfscope}%
\pgftext[x=3.075000in,y=0.402778in,,top]{\rmfamily\fontsize{10.000000}{12.000000}\selectfont \(\displaystyle 0.00\)}%
\end{pgfscope}%
\begin{pgfscope}%
\pgfpathrectangle{\pgfqpoint{0.750000in}{0.500000in}}{\pgfqpoint{4.650000in}{3.020000in}}%
\pgfusepath{clip}%
\pgfsetrectcap%
\pgfsetroundjoin%
\pgfsetlinewidth{0.803000pt}%
\definecolor{currentstroke}{rgb}{0.690196,0.690196,0.690196}%
\pgfsetstrokecolor{currentstroke}%
\pgfsetdash{}{0pt}%
\pgfpathmoveto{\pgfqpoint{3.603409in}{0.500000in}}%
\pgfpathlineto{\pgfqpoint{3.603409in}{3.520000in}}%
\pgfusepath{stroke}%
\end{pgfscope}%
\begin{pgfscope}%
\pgfsetbuttcap%
\pgfsetroundjoin%
\definecolor{currentfill}{rgb}{0.000000,0.000000,0.000000}%
\pgfsetfillcolor{currentfill}%
\pgfsetlinewidth{0.803000pt}%
\definecolor{currentstroke}{rgb}{0.000000,0.000000,0.000000}%
\pgfsetstrokecolor{currentstroke}%
\pgfsetdash{}{0pt}%
\pgfsys@defobject{currentmarker}{\pgfqpoint{0.000000in}{-0.048611in}}{\pgfqpoint{0.000000in}{0.000000in}}{%
\pgfpathmoveto{\pgfqpoint{0.000000in}{0.000000in}}%
\pgfpathlineto{\pgfqpoint{0.000000in}{-0.048611in}}%
\pgfusepath{stroke,fill}%
}%
\begin{pgfscope}%
\pgfsys@transformshift{3.603409in}{0.500000in}%
\pgfsys@useobject{currentmarker}{}%
\end{pgfscope}%
\end{pgfscope}%
\begin{pgfscope}%
\pgftext[x=3.603409in,y=0.402778in,,top]{\rmfamily\fontsize{10.000000}{12.000000}\selectfont \(\displaystyle 0.25\)}%
\end{pgfscope}%
\begin{pgfscope}%
\pgfpathrectangle{\pgfqpoint{0.750000in}{0.500000in}}{\pgfqpoint{4.650000in}{3.020000in}}%
\pgfusepath{clip}%
\pgfsetrectcap%
\pgfsetroundjoin%
\pgfsetlinewidth{0.803000pt}%
\definecolor{currentstroke}{rgb}{0.690196,0.690196,0.690196}%
\pgfsetstrokecolor{currentstroke}%
\pgfsetdash{}{0pt}%
\pgfpathmoveto{\pgfqpoint{4.131818in}{0.500000in}}%
\pgfpathlineto{\pgfqpoint{4.131818in}{3.520000in}}%
\pgfusepath{stroke}%
\end{pgfscope}%
\begin{pgfscope}%
\pgfsetbuttcap%
\pgfsetroundjoin%
\definecolor{currentfill}{rgb}{0.000000,0.000000,0.000000}%
\pgfsetfillcolor{currentfill}%
\pgfsetlinewidth{0.803000pt}%
\definecolor{currentstroke}{rgb}{0.000000,0.000000,0.000000}%
\pgfsetstrokecolor{currentstroke}%
\pgfsetdash{}{0pt}%
\pgfsys@defobject{currentmarker}{\pgfqpoint{0.000000in}{-0.048611in}}{\pgfqpoint{0.000000in}{0.000000in}}{%
\pgfpathmoveto{\pgfqpoint{0.000000in}{0.000000in}}%
\pgfpathlineto{\pgfqpoint{0.000000in}{-0.048611in}}%
\pgfusepath{stroke,fill}%
}%
\begin{pgfscope}%
\pgfsys@transformshift{4.131818in}{0.500000in}%
\pgfsys@useobject{currentmarker}{}%
\end{pgfscope}%
\end{pgfscope}%
\begin{pgfscope}%
\pgftext[x=4.131818in,y=0.402778in,,top]{\rmfamily\fontsize{10.000000}{12.000000}\selectfont \(\displaystyle 0.50\)}%
\end{pgfscope}%
\begin{pgfscope}%
\pgfpathrectangle{\pgfqpoint{0.750000in}{0.500000in}}{\pgfqpoint{4.650000in}{3.020000in}}%
\pgfusepath{clip}%
\pgfsetrectcap%
\pgfsetroundjoin%
\pgfsetlinewidth{0.803000pt}%
\definecolor{currentstroke}{rgb}{0.690196,0.690196,0.690196}%
\pgfsetstrokecolor{currentstroke}%
\pgfsetdash{}{0pt}%
\pgfpathmoveto{\pgfqpoint{4.660227in}{0.500000in}}%
\pgfpathlineto{\pgfqpoint{4.660227in}{3.520000in}}%
\pgfusepath{stroke}%
\end{pgfscope}%
\begin{pgfscope}%
\pgfsetbuttcap%
\pgfsetroundjoin%
\definecolor{currentfill}{rgb}{0.000000,0.000000,0.000000}%
\pgfsetfillcolor{currentfill}%
\pgfsetlinewidth{0.803000pt}%
\definecolor{currentstroke}{rgb}{0.000000,0.000000,0.000000}%
\pgfsetstrokecolor{currentstroke}%
\pgfsetdash{}{0pt}%
\pgfsys@defobject{currentmarker}{\pgfqpoint{0.000000in}{-0.048611in}}{\pgfqpoint{0.000000in}{0.000000in}}{%
\pgfpathmoveto{\pgfqpoint{0.000000in}{0.000000in}}%
\pgfpathlineto{\pgfqpoint{0.000000in}{-0.048611in}}%
\pgfusepath{stroke,fill}%
}%
\begin{pgfscope}%
\pgfsys@transformshift{4.660227in}{0.500000in}%
\pgfsys@useobject{currentmarker}{}%
\end{pgfscope}%
\end{pgfscope}%
\begin{pgfscope}%
\pgftext[x=4.660227in,y=0.402778in,,top]{\rmfamily\fontsize{10.000000}{12.000000}\selectfont \(\displaystyle 0.75\)}%
\end{pgfscope}%
\begin{pgfscope}%
\pgfpathrectangle{\pgfqpoint{0.750000in}{0.500000in}}{\pgfqpoint{4.650000in}{3.020000in}}%
\pgfusepath{clip}%
\pgfsetrectcap%
\pgfsetroundjoin%
\pgfsetlinewidth{0.803000pt}%
\definecolor{currentstroke}{rgb}{0.690196,0.690196,0.690196}%
\pgfsetstrokecolor{currentstroke}%
\pgfsetdash{}{0pt}%
\pgfpathmoveto{\pgfqpoint{5.188636in}{0.500000in}}%
\pgfpathlineto{\pgfqpoint{5.188636in}{3.520000in}}%
\pgfusepath{stroke}%
\end{pgfscope}%
\begin{pgfscope}%
\pgfsetbuttcap%
\pgfsetroundjoin%
\definecolor{currentfill}{rgb}{0.000000,0.000000,0.000000}%
\pgfsetfillcolor{currentfill}%
\pgfsetlinewidth{0.803000pt}%
\definecolor{currentstroke}{rgb}{0.000000,0.000000,0.000000}%
\pgfsetstrokecolor{currentstroke}%
\pgfsetdash{}{0pt}%
\pgfsys@defobject{currentmarker}{\pgfqpoint{0.000000in}{-0.048611in}}{\pgfqpoint{0.000000in}{0.000000in}}{%
\pgfpathmoveto{\pgfqpoint{0.000000in}{0.000000in}}%
\pgfpathlineto{\pgfqpoint{0.000000in}{-0.048611in}}%
\pgfusepath{stroke,fill}%
}%
\begin{pgfscope}%
\pgfsys@transformshift{5.188636in}{0.500000in}%
\pgfsys@useobject{currentmarker}{}%
\end{pgfscope}%
\end{pgfscope}%
\begin{pgfscope}%
\pgftext[x=5.188636in,y=0.402778in,,top]{\rmfamily\fontsize{10.000000}{12.000000}\selectfont \(\displaystyle 1.00\)}%
\end{pgfscope}%
\begin{pgfscope}%
\pgfpathrectangle{\pgfqpoint{0.750000in}{0.500000in}}{\pgfqpoint{4.650000in}{3.020000in}}%
\pgfusepath{clip}%
\pgfsetrectcap%
\pgfsetroundjoin%
\pgfsetlinewidth{0.803000pt}%
\definecolor{currentstroke}{rgb}{0.690196,0.690196,0.690196}%
\pgfsetstrokecolor{currentstroke}%
\pgfsetdash{}{0pt}%
\pgfpathmoveto{\pgfqpoint{0.750000in}{0.751667in}}%
\pgfpathlineto{\pgfqpoint{5.400000in}{0.751667in}}%
\pgfusepath{stroke}%
\end{pgfscope}%
\begin{pgfscope}%
\pgfsetbuttcap%
\pgfsetroundjoin%
\definecolor{currentfill}{rgb}{0.000000,0.000000,0.000000}%
\pgfsetfillcolor{currentfill}%
\pgfsetlinewidth{0.803000pt}%
\definecolor{currentstroke}{rgb}{0.000000,0.000000,0.000000}%
\pgfsetstrokecolor{currentstroke}%
\pgfsetdash{}{0pt}%
\pgfsys@defobject{currentmarker}{\pgfqpoint{-0.048611in}{0.000000in}}{\pgfqpoint{0.000000in}{0.000000in}}{%
\pgfpathmoveto{\pgfqpoint{0.000000in}{0.000000in}}%
\pgfpathlineto{\pgfqpoint{-0.048611in}{0.000000in}}%
\pgfusepath{stroke,fill}%
}%
\begin{pgfscope}%
\pgfsys@transformshift{0.750000in}{0.751667in}%
\pgfsys@useobject{currentmarker}{}%
\end{pgfscope}%
\end{pgfscope}%
\begin{pgfscope}%
\pgftext[x=0.475308in,y=0.703472in,left,base]{\rmfamily\fontsize{10.000000}{12.000000}\selectfont \(\displaystyle 0.0\)}%
\end{pgfscope}%
\begin{pgfscope}%
\pgfpathrectangle{\pgfqpoint{0.750000in}{0.500000in}}{\pgfqpoint{4.650000in}{3.020000in}}%
\pgfusepath{clip}%
\pgfsetrectcap%
\pgfsetroundjoin%
\pgfsetlinewidth{0.803000pt}%
\definecolor{currentstroke}{rgb}{0.690196,0.690196,0.690196}%
\pgfsetstrokecolor{currentstroke}%
\pgfsetdash{}{0pt}%
\pgfpathmoveto{\pgfqpoint{0.750000in}{1.255000in}}%
\pgfpathlineto{\pgfqpoint{5.400000in}{1.255000in}}%
\pgfusepath{stroke}%
\end{pgfscope}%
\begin{pgfscope}%
\pgfsetbuttcap%
\pgfsetroundjoin%
\definecolor{currentfill}{rgb}{0.000000,0.000000,0.000000}%
\pgfsetfillcolor{currentfill}%
\pgfsetlinewidth{0.803000pt}%
\definecolor{currentstroke}{rgb}{0.000000,0.000000,0.000000}%
\pgfsetstrokecolor{currentstroke}%
\pgfsetdash{}{0pt}%
\pgfsys@defobject{currentmarker}{\pgfqpoint{-0.048611in}{0.000000in}}{\pgfqpoint{0.000000in}{0.000000in}}{%
\pgfpathmoveto{\pgfqpoint{0.000000in}{0.000000in}}%
\pgfpathlineto{\pgfqpoint{-0.048611in}{0.000000in}}%
\pgfusepath{stroke,fill}%
}%
\begin{pgfscope}%
\pgfsys@transformshift{0.750000in}{1.255000in}%
\pgfsys@useobject{currentmarker}{}%
\end{pgfscope}%
\end{pgfscope}%
\begin{pgfscope}%
\pgftext[x=0.475308in,y=1.206806in,left,base]{\rmfamily\fontsize{10.000000}{12.000000}\selectfont \(\displaystyle 0.2\)}%
\end{pgfscope}%
\begin{pgfscope}%
\pgfpathrectangle{\pgfqpoint{0.750000in}{0.500000in}}{\pgfqpoint{4.650000in}{3.020000in}}%
\pgfusepath{clip}%
\pgfsetrectcap%
\pgfsetroundjoin%
\pgfsetlinewidth{0.803000pt}%
\definecolor{currentstroke}{rgb}{0.690196,0.690196,0.690196}%
\pgfsetstrokecolor{currentstroke}%
\pgfsetdash{}{0pt}%
\pgfpathmoveto{\pgfqpoint{0.750000in}{1.758333in}}%
\pgfpathlineto{\pgfqpoint{5.400000in}{1.758333in}}%
\pgfusepath{stroke}%
\end{pgfscope}%
\begin{pgfscope}%
\pgfsetbuttcap%
\pgfsetroundjoin%
\definecolor{currentfill}{rgb}{0.000000,0.000000,0.000000}%
\pgfsetfillcolor{currentfill}%
\pgfsetlinewidth{0.803000pt}%
\definecolor{currentstroke}{rgb}{0.000000,0.000000,0.000000}%
\pgfsetstrokecolor{currentstroke}%
\pgfsetdash{}{0pt}%
\pgfsys@defobject{currentmarker}{\pgfqpoint{-0.048611in}{0.000000in}}{\pgfqpoint{0.000000in}{0.000000in}}{%
\pgfpathmoveto{\pgfqpoint{0.000000in}{0.000000in}}%
\pgfpathlineto{\pgfqpoint{-0.048611in}{0.000000in}}%
\pgfusepath{stroke,fill}%
}%
\begin{pgfscope}%
\pgfsys@transformshift{0.750000in}{1.758333in}%
\pgfsys@useobject{currentmarker}{}%
\end{pgfscope}%
\end{pgfscope}%
\begin{pgfscope}%
\pgftext[x=0.475308in,y=1.710139in,left,base]{\rmfamily\fontsize{10.000000}{12.000000}\selectfont \(\displaystyle 0.4\)}%
\end{pgfscope}%
\begin{pgfscope}%
\pgfpathrectangle{\pgfqpoint{0.750000in}{0.500000in}}{\pgfqpoint{4.650000in}{3.020000in}}%
\pgfusepath{clip}%
\pgfsetrectcap%
\pgfsetroundjoin%
\pgfsetlinewidth{0.803000pt}%
\definecolor{currentstroke}{rgb}{0.690196,0.690196,0.690196}%
\pgfsetstrokecolor{currentstroke}%
\pgfsetdash{}{0pt}%
\pgfpathmoveto{\pgfqpoint{0.750000in}{2.261667in}}%
\pgfpathlineto{\pgfqpoint{5.400000in}{2.261667in}}%
\pgfusepath{stroke}%
\end{pgfscope}%
\begin{pgfscope}%
\pgfsetbuttcap%
\pgfsetroundjoin%
\definecolor{currentfill}{rgb}{0.000000,0.000000,0.000000}%
\pgfsetfillcolor{currentfill}%
\pgfsetlinewidth{0.803000pt}%
\definecolor{currentstroke}{rgb}{0.000000,0.000000,0.000000}%
\pgfsetstrokecolor{currentstroke}%
\pgfsetdash{}{0pt}%
\pgfsys@defobject{currentmarker}{\pgfqpoint{-0.048611in}{0.000000in}}{\pgfqpoint{0.000000in}{0.000000in}}{%
\pgfpathmoveto{\pgfqpoint{0.000000in}{0.000000in}}%
\pgfpathlineto{\pgfqpoint{-0.048611in}{0.000000in}}%
\pgfusepath{stroke,fill}%
}%
\begin{pgfscope}%
\pgfsys@transformshift{0.750000in}{2.261667in}%
\pgfsys@useobject{currentmarker}{}%
\end{pgfscope}%
\end{pgfscope}%
\begin{pgfscope}%
\pgftext[x=0.475308in,y=2.213472in,left,base]{\rmfamily\fontsize{10.000000}{12.000000}\selectfont \(\displaystyle 0.6\)}%
\end{pgfscope}%
\begin{pgfscope}%
\pgfpathrectangle{\pgfqpoint{0.750000in}{0.500000in}}{\pgfqpoint{4.650000in}{3.020000in}}%
\pgfusepath{clip}%
\pgfsetrectcap%
\pgfsetroundjoin%
\pgfsetlinewidth{0.803000pt}%
\definecolor{currentstroke}{rgb}{0.690196,0.690196,0.690196}%
\pgfsetstrokecolor{currentstroke}%
\pgfsetdash{}{0pt}%
\pgfpathmoveto{\pgfqpoint{0.750000in}{2.765000in}}%
\pgfpathlineto{\pgfqpoint{5.400000in}{2.765000in}}%
\pgfusepath{stroke}%
\end{pgfscope}%
\begin{pgfscope}%
\pgfsetbuttcap%
\pgfsetroundjoin%
\definecolor{currentfill}{rgb}{0.000000,0.000000,0.000000}%
\pgfsetfillcolor{currentfill}%
\pgfsetlinewidth{0.803000pt}%
\definecolor{currentstroke}{rgb}{0.000000,0.000000,0.000000}%
\pgfsetstrokecolor{currentstroke}%
\pgfsetdash{}{0pt}%
\pgfsys@defobject{currentmarker}{\pgfqpoint{-0.048611in}{0.000000in}}{\pgfqpoint{0.000000in}{0.000000in}}{%
\pgfpathmoveto{\pgfqpoint{0.000000in}{0.000000in}}%
\pgfpathlineto{\pgfqpoint{-0.048611in}{0.000000in}}%
\pgfusepath{stroke,fill}%
}%
\begin{pgfscope}%
\pgfsys@transformshift{0.750000in}{2.765000in}%
\pgfsys@useobject{currentmarker}{}%
\end{pgfscope}%
\end{pgfscope}%
\begin{pgfscope}%
\pgftext[x=0.475308in,y=2.716806in,left,base]{\rmfamily\fontsize{10.000000}{12.000000}\selectfont \(\displaystyle 0.8\)}%
\end{pgfscope}%
\begin{pgfscope}%
\pgfpathrectangle{\pgfqpoint{0.750000in}{0.500000in}}{\pgfqpoint{4.650000in}{3.020000in}}%
\pgfusepath{clip}%
\pgfsetrectcap%
\pgfsetroundjoin%
\pgfsetlinewidth{0.803000pt}%
\definecolor{currentstroke}{rgb}{0.690196,0.690196,0.690196}%
\pgfsetstrokecolor{currentstroke}%
\pgfsetdash{}{0pt}%
\pgfpathmoveto{\pgfqpoint{0.750000in}{3.268333in}}%
\pgfpathlineto{\pgfqpoint{5.400000in}{3.268333in}}%
\pgfusepath{stroke}%
\end{pgfscope}%
\begin{pgfscope}%
\pgfsetbuttcap%
\pgfsetroundjoin%
\definecolor{currentfill}{rgb}{0.000000,0.000000,0.000000}%
\pgfsetfillcolor{currentfill}%
\pgfsetlinewidth{0.803000pt}%
\definecolor{currentstroke}{rgb}{0.000000,0.000000,0.000000}%
\pgfsetstrokecolor{currentstroke}%
\pgfsetdash{}{0pt}%
\pgfsys@defobject{currentmarker}{\pgfqpoint{-0.048611in}{0.000000in}}{\pgfqpoint{0.000000in}{0.000000in}}{%
\pgfpathmoveto{\pgfqpoint{0.000000in}{0.000000in}}%
\pgfpathlineto{\pgfqpoint{-0.048611in}{0.000000in}}%
\pgfusepath{stroke,fill}%
}%
\begin{pgfscope}%
\pgfsys@transformshift{0.750000in}{3.268333in}%
\pgfsys@useobject{currentmarker}{}%
\end{pgfscope}%
\end{pgfscope}%
\begin{pgfscope}%
\pgftext[x=0.475308in,y=3.220139in,left,base]{\rmfamily\fontsize{10.000000}{12.000000}\selectfont \(\displaystyle 1.0\)}%
\end{pgfscope}%
\begin{pgfscope}%
\pgfpathrectangle{\pgfqpoint{0.750000in}{0.500000in}}{\pgfqpoint{4.650000in}{3.020000in}}%
\pgfusepath{clip}%
\pgfsetrectcap%
\pgfsetroundjoin%
\pgfsetlinewidth{1.505625pt}%
\definecolor{currentstroke}{rgb}{0.121569,0.466667,0.705882}%
\pgfsetstrokecolor{currentstroke}%
\pgfsetdash{}{0pt}%
\pgfpathmoveto{\pgfqpoint{0.961364in}{3.268333in}}%
\pgfpathlineto{\pgfqpoint{1.020605in}{3.062597in}}%
\pgfpathlineto{\pgfqpoint{1.079846in}{2.868391in}}%
\pgfpathlineto{\pgfqpoint{1.134855in}{2.698090in}}%
\pgfpathlineto{\pgfqpoint{1.189865in}{2.537177in}}%
\pgfpathlineto{\pgfqpoint{1.244874in}{2.385386in}}%
\pgfpathlineto{\pgfqpoint{1.299884in}{2.242452in}}%
\pgfpathlineto{\pgfqpoint{1.354894in}{2.108108in}}%
\pgfpathlineto{\pgfqpoint{1.409903in}{1.982087in}}%
\pgfpathlineto{\pgfqpoint{1.464913in}{1.864124in}}%
\pgfpathlineto{\pgfqpoint{1.515691in}{1.762156in}}%
\pgfpathlineto{\pgfqpoint{1.566469in}{1.666618in}}%
\pgfpathlineto{\pgfqpoint{1.617247in}{1.577299in}}%
\pgfpathlineto{\pgfqpoint{1.668025in}{1.493992in}}%
\pgfpathlineto{\pgfqpoint{1.718803in}{1.416486in}}%
\pgfpathlineto{\pgfqpoint{1.769581in}{1.344571in}}%
\pgfpathlineto{\pgfqpoint{1.820359in}{1.278039in}}%
\pgfpathlineto{\pgfqpoint{1.871137in}{1.216681in}}%
\pgfpathlineto{\pgfqpoint{1.921915in}{1.160286in}}%
\pgfpathlineto{\pgfqpoint{1.972693in}{1.108645in}}%
\pgfpathlineto{\pgfqpoint{2.023471in}{1.061550in}}%
\pgfpathlineto{\pgfqpoint{2.074249in}{1.018791in}}%
\pgfpathlineto{\pgfqpoint{2.125027in}{0.980157in}}%
\pgfpathlineto{\pgfqpoint{2.175805in}{0.945441in}}%
\pgfpathlineto{\pgfqpoint{2.226583in}{0.914432in}}%
\pgfpathlineto{\pgfqpoint{2.277361in}{0.886922in}}%
\pgfpathlineto{\pgfqpoint{2.328140in}{0.862700in}}%
\pgfpathlineto{\pgfqpoint{2.383149in}{0.839928in}}%
\pgfpathlineto{\pgfqpoint{2.438159in}{0.820505in}}%
\pgfpathlineto{\pgfqpoint{2.497400in}{0.803026in}}%
\pgfpathlineto{\pgfqpoint{2.556641in}{0.788788in}}%
\pgfpathlineto{\pgfqpoint{2.620113in}{0.776753in}}%
\pgfpathlineto{\pgfqpoint{2.687817in}{0.767136in}}%
\pgfpathlineto{\pgfqpoint{2.759753in}{0.760017in}}%
\pgfpathlineto{\pgfqpoint{2.840152in}{0.755119in}}%
\pgfpathlineto{\pgfqpoint{2.937476in}{0.752360in}}%
\pgfpathlineto{\pgfqpoint{3.089810in}{0.751668in}}%
\pgfpathlineto{\pgfqpoint{3.254839in}{0.753217in}}%
\pgfpathlineto{\pgfqpoint{3.347932in}{0.757085in}}%
\pgfpathlineto{\pgfqpoint{3.428331in}{0.763423in}}%
\pgfpathlineto{\pgfqpoint{3.500266in}{0.772165in}}%
\pgfpathlineto{\pgfqpoint{3.567970in}{0.783597in}}%
\pgfpathlineto{\pgfqpoint{3.631443in}{0.797586in}}%
\pgfpathlineto{\pgfqpoint{3.690684in}{0.813869in}}%
\pgfpathlineto{\pgfqpoint{3.745693in}{0.832076in}}%
\pgfpathlineto{\pgfqpoint{3.800703in}{0.853529in}}%
\pgfpathlineto{\pgfqpoint{3.855713in}{0.878493in}}%
\pgfpathlineto{\pgfqpoint{3.906491in}{0.904884in}}%
\pgfpathlineto{\pgfqpoint{3.957269in}{0.934703in}}%
\pgfpathlineto{\pgfqpoint{4.008047in}{0.968160in}}%
\pgfpathlineto{\pgfqpoint{4.058825in}{1.005465in}}%
\pgfpathlineto{\pgfqpoint{4.109603in}{1.046826in}}%
\pgfpathlineto{\pgfqpoint{4.160381in}{1.092452in}}%
\pgfpathlineto{\pgfqpoint{4.211159in}{1.142554in}}%
\pgfpathlineto{\pgfqpoint{4.261937in}{1.197341in}}%
\pgfpathlineto{\pgfqpoint{4.312715in}{1.257022in}}%
\pgfpathlineto{\pgfqpoint{4.363493in}{1.321806in}}%
\pgfpathlineto{\pgfqpoint{4.414271in}{1.391903in}}%
\pgfpathlineto{\pgfqpoint{4.465049in}{1.467522in}}%
\pgfpathlineto{\pgfqpoint{4.515827in}{1.548873in}}%
\pgfpathlineto{\pgfqpoint{4.566605in}{1.636164in}}%
\pgfpathlineto{\pgfqpoint{4.617383in}{1.729606in}}%
\pgfpathlineto{\pgfqpoint{4.668161in}{1.829407in}}%
\pgfpathlineto{\pgfqpoint{4.718939in}{1.935777in}}%
\pgfpathlineto{\pgfqpoint{4.769717in}{2.048926in}}%
\pgfpathlineto{\pgfqpoint{4.824727in}{2.179396in}}%
\pgfpathlineto{\pgfqpoint{4.879737in}{2.318332in}}%
\pgfpathlineto{\pgfqpoint{4.934746in}{2.466002in}}%
\pgfpathlineto{\pgfqpoint{4.989756in}{2.622672in}}%
\pgfpathlineto{\pgfqpoint{5.044765in}{2.788607in}}%
\pgfpathlineto{\pgfqpoint{5.099775in}{2.964074in}}%
\pgfpathlineto{\pgfqpoint{5.154784in}{3.149339in}}%
\pgfpathlineto{\pgfqpoint{5.188636in}{3.268333in}}%
\pgfpathlineto{\pgfqpoint{5.188636in}{3.268333in}}%
\pgfusepath{stroke}%
\end{pgfscope}%
\begin{pgfscope}%
\pgfpathrectangle{\pgfqpoint{0.750000in}{0.500000in}}{\pgfqpoint{4.650000in}{3.020000in}}%
\pgfusepath{clip}%
\pgfsetrectcap%
\pgfsetroundjoin%
\pgfsetlinewidth{1.505625pt}%
\definecolor{currentstroke}{rgb}{1.000000,0.498039,0.054902}%
\pgfsetstrokecolor{currentstroke}%
\pgfsetdash{}{0pt}%
\pgfpathmoveto{\pgfqpoint{0.961364in}{3.288902in}}%
\pgfpathlineto{\pgfqpoint{1.012142in}{3.099057in}}%
\pgfpathlineto{\pgfqpoint{1.062920in}{2.920401in}}%
\pgfpathlineto{\pgfqpoint{1.113698in}{2.752449in}}%
\pgfpathlineto{\pgfqpoint{1.164476in}{2.594728in}}%
\pgfpathlineto{\pgfqpoint{1.215254in}{2.446777in}}%
\pgfpathlineto{\pgfqpoint{1.266032in}{2.308147in}}%
\pgfpathlineto{\pgfqpoint{1.316810in}{2.178400in}}%
\pgfpathlineto{\pgfqpoint{1.367588in}{2.057112in}}%
\pgfpathlineto{\pgfqpoint{1.418366in}{1.943869in}}%
\pgfpathlineto{\pgfqpoint{1.469144in}{1.838272in}}%
\pgfpathlineto{\pgfqpoint{1.519922in}{1.739930in}}%
\pgfpathlineto{\pgfqpoint{1.570700in}{1.648469in}}%
\pgfpathlineto{\pgfqpoint{1.621478in}{1.563523in}}%
\pgfpathlineto{\pgfqpoint{1.672256in}{1.484741in}}%
\pgfpathlineto{\pgfqpoint{1.723034in}{1.411781in}}%
\pgfpathlineto{\pgfqpoint{1.773812in}{1.344316in}}%
\pgfpathlineto{\pgfqpoint{1.824590in}{1.282029in}}%
\pgfpathlineto{\pgfqpoint{1.875369in}{1.224617in}}%
\pgfpathlineto{\pgfqpoint{1.926147in}{1.171789in}}%
\pgfpathlineto{\pgfqpoint{1.976925in}{1.123263in}}%
\pgfpathlineto{\pgfqpoint{2.027703in}{1.078772in}}%
\pgfpathlineto{\pgfqpoint{2.082712in}{1.034832in}}%
\pgfpathlineto{\pgfqpoint{2.137722in}{0.995019in}}%
\pgfpathlineto{\pgfqpoint{2.192731in}{0.959042in}}%
\pgfpathlineto{\pgfqpoint{2.247741in}{0.926624in}}%
\pgfpathlineto{\pgfqpoint{2.306982in}{0.895396in}}%
\pgfpathlineto{\pgfqpoint{2.366223in}{0.867695in}}%
\pgfpathlineto{\pgfqpoint{2.425464in}{0.843241in}}%
\pgfpathlineto{\pgfqpoint{2.488937in}{0.820352in}}%
\pgfpathlineto{\pgfqpoint{2.556641in}{0.799398in}}%
\pgfpathlineto{\pgfqpoint{2.624345in}{0.781711in}}%
\pgfpathlineto{\pgfqpoint{2.696280in}{0.766194in}}%
\pgfpathlineto{\pgfqpoint{2.768216in}{0.753776in}}%
\pgfpathlineto{\pgfqpoint{2.844383in}{0.743748in}}%
\pgfpathlineto{\pgfqpoint{2.924782in}{0.736413in}}%
\pgfpathlineto{\pgfqpoint{3.005180in}{0.732240in}}%
\pgfpathlineto{\pgfqpoint{3.085579in}{0.731124in}}%
\pgfpathlineto{\pgfqpoint{3.165977in}{0.733038in}}%
\pgfpathlineto{\pgfqpoint{3.246376in}{0.738031in}}%
\pgfpathlineto{\pgfqpoint{3.326775in}{0.746224in}}%
\pgfpathlineto{\pgfqpoint{3.402942in}{0.757121in}}%
\pgfpathlineto{\pgfqpoint{3.474877in}{0.770425in}}%
\pgfpathlineto{\pgfqpoint{3.546813in}{0.786905in}}%
\pgfpathlineto{\pgfqpoint{3.614517in}{0.805582in}}%
\pgfpathlineto{\pgfqpoint{3.677989in}{0.826137in}}%
\pgfpathlineto{\pgfqpoint{3.741462in}{0.849912in}}%
\pgfpathlineto{\pgfqpoint{3.800703in}{0.875266in}}%
\pgfpathlineto{\pgfqpoint{3.859944in}{0.903946in}}%
\pgfpathlineto{\pgfqpoint{3.919185in}{0.936235in}}%
\pgfpathlineto{\pgfqpoint{3.974195in}{0.969720in}}%
\pgfpathlineto{\pgfqpoint{4.029204in}{1.006848in}}%
\pgfpathlineto{\pgfqpoint{4.084214in}{1.047899in}}%
\pgfpathlineto{\pgfqpoint{4.134992in}{1.089531in}}%
\pgfpathlineto{\pgfqpoint{4.185770in}{1.135006in}}%
\pgfpathlineto{\pgfqpoint{4.236548in}{1.184582in}}%
\pgfpathlineto{\pgfqpoint{4.287326in}{1.238530in}}%
\pgfpathlineto{\pgfqpoint{4.338104in}{1.297132in}}%
\pgfpathlineto{\pgfqpoint{4.388882in}{1.360684in}}%
\pgfpathlineto{\pgfqpoint{4.439660in}{1.429493in}}%
\pgfpathlineto{\pgfqpoint{4.490438in}{1.503877in}}%
\pgfpathlineto{\pgfqpoint{4.541216in}{1.584169in}}%
\pgfpathlineto{\pgfqpoint{4.591994in}{1.670710in}}%
\pgfpathlineto{\pgfqpoint{4.642772in}{1.763856in}}%
\pgfpathlineto{\pgfqpoint{4.693550in}{1.863976in}}%
\pgfpathlineto{\pgfqpoint{4.744328in}{1.971448in}}%
\pgfpathlineto{\pgfqpoint{4.795106in}{2.086663in}}%
\pgfpathlineto{\pgfqpoint{4.845885in}{2.210027in}}%
\pgfpathlineto{\pgfqpoint{4.896663in}{2.341955in}}%
\pgfpathlineto{\pgfqpoint{4.947441in}{2.482874in}}%
\pgfpathlineto{\pgfqpoint{4.998219in}{2.633225in}}%
\pgfpathlineto{\pgfqpoint{5.048997in}{2.793460in}}%
\pgfpathlineto{\pgfqpoint{5.099775in}{2.964042in}}%
\pgfpathlineto{\pgfqpoint{5.150553in}{3.145450in}}%
\pgfpathlineto{\pgfqpoint{5.188636in}{3.288902in}}%
\pgfpathlineto{\pgfqpoint{5.188636in}{3.288902in}}%
\pgfusepath{stroke}%
\end{pgfscope}%
\begin{pgfscope}%
\pgfpathrectangle{\pgfqpoint{0.750000in}{0.500000in}}{\pgfqpoint{4.650000in}{3.020000in}}%
\pgfusepath{clip}%
\pgfsetrectcap%
\pgfsetroundjoin%
\pgfsetlinewidth{1.505625pt}%
\definecolor{currentstroke}{rgb}{0.172549,0.627451,0.172549}%
\pgfsetstrokecolor{currentstroke}%
\pgfsetdash{}{0pt}%
\pgfpathmoveto{\pgfqpoint{0.961364in}{3.290684in}}%
\pgfpathlineto{\pgfqpoint{1.012142in}{3.100408in}}%
\pgfpathlineto{\pgfqpoint{1.062920in}{2.921357in}}%
\pgfpathlineto{\pgfqpoint{1.113698in}{2.753045in}}%
\pgfpathlineto{\pgfqpoint{1.164476in}{2.594994in}}%
\pgfpathlineto{\pgfqpoint{1.215254in}{2.446744in}}%
\pgfpathlineto{\pgfqpoint{1.266032in}{2.307843in}}%
\pgfpathlineto{\pgfqpoint{1.316810in}{2.177852in}}%
\pgfpathlineto{\pgfqpoint{1.367588in}{2.056345in}}%
\pgfpathlineto{\pgfqpoint{1.418366in}{1.942907in}}%
\pgfpathlineto{\pgfqpoint{1.469144in}{1.837136in}}%
\pgfpathlineto{\pgfqpoint{1.519922in}{1.738642in}}%
\pgfpathlineto{\pgfqpoint{1.570700in}{1.647048in}}%
\pgfpathlineto{\pgfqpoint{1.621478in}{1.561986in}}%
\pgfpathlineto{\pgfqpoint{1.672256in}{1.483104in}}%
\pgfpathlineto{\pgfqpoint{1.723034in}{1.410060in}}%
\pgfpathlineto{\pgfqpoint{1.773812in}{1.342524in}}%
\pgfpathlineto{\pgfqpoint{1.824590in}{1.280179in}}%
\pgfpathlineto{\pgfqpoint{1.875369in}{1.222721in}}%
\pgfpathlineto{\pgfqpoint{1.926147in}{1.169856in}}%
\pgfpathlineto{\pgfqpoint{1.976925in}{1.121303in}}%
\pgfpathlineto{\pgfqpoint{2.027703in}{1.076794in}}%
\pgfpathlineto{\pgfqpoint{2.082712in}{1.032842in}}%
\pgfpathlineto{\pgfqpoint{2.137722in}{0.993025in}}%
\pgfpathlineto{\pgfqpoint{2.192731in}{0.957049in}}%
\pgfpathlineto{\pgfqpoint{2.247741in}{0.924637in}}%
\pgfpathlineto{\pgfqpoint{2.306982in}{0.893422in}}%
\pgfpathlineto{\pgfqpoint{2.366223in}{0.865737in}}%
\pgfpathlineto{\pgfqpoint{2.425464in}{0.841301in}}%
\pgfpathlineto{\pgfqpoint{2.488937in}{0.818434in}}%
\pgfpathlineto{\pgfqpoint{2.556641in}{0.797504in}}%
\pgfpathlineto{\pgfqpoint{2.624345in}{0.779840in}}%
\pgfpathlineto{\pgfqpoint{2.696280in}{0.764346in}}%
\pgfpathlineto{\pgfqpoint{2.768216in}{0.751949in}}%
\pgfpathlineto{\pgfqpoint{2.844383in}{0.741940in}}%
\pgfpathlineto{\pgfqpoint{2.924782in}{0.734619in}}%
\pgfpathlineto{\pgfqpoint{3.005180in}{0.730455in}}%
\pgfpathlineto{\pgfqpoint{3.085579in}{0.729342in}}%
\pgfpathlineto{\pgfqpoint{3.165977in}{0.731252in}}%
\pgfpathlineto{\pgfqpoint{3.246376in}{0.736234in}}%
\pgfpathlineto{\pgfqpoint{3.326775in}{0.744412in}}%
\pgfpathlineto{\pgfqpoint{3.402942in}{0.755289in}}%
\pgfpathlineto{\pgfqpoint{3.474877in}{0.768570in}}%
\pgfpathlineto{\pgfqpoint{3.546813in}{0.785026in}}%
\pgfpathlineto{\pgfqpoint{3.614517in}{0.803679in}}%
\pgfpathlineto{\pgfqpoint{3.677989in}{0.824213in}}%
\pgfpathlineto{\pgfqpoint{3.741462in}{0.847966in}}%
\pgfpathlineto{\pgfqpoint{3.800703in}{0.873303in}}%
\pgfpathlineto{\pgfqpoint{3.859944in}{0.901968in}}%
\pgfpathlineto{\pgfqpoint{3.919185in}{0.934246in}}%
\pgfpathlineto{\pgfqpoint{3.974195in}{0.967726in}}%
\pgfpathlineto{\pgfqpoint{4.029204in}{1.004854in}}%
\pgfpathlineto{\pgfqpoint{4.084214in}{1.045911in}}%
\pgfpathlineto{\pgfqpoint{4.134992in}{1.087557in}}%
\pgfpathlineto{\pgfqpoint{4.185770in}{1.133052in}}%
\pgfpathlineto{\pgfqpoint{4.236548in}{1.182657in}}%
\pgfpathlineto{\pgfqpoint{4.287326in}{1.236644in}}%
\pgfpathlineto{\pgfqpoint{4.338104in}{1.295296in}}%
\pgfpathlineto{\pgfqpoint{4.388882in}{1.358909in}}%
\pgfpathlineto{\pgfqpoint{4.439660in}{1.427792in}}%
\pgfpathlineto{\pgfqpoint{4.490438in}{1.502264in}}%
\pgfpathlineto{\pgfqpoint{4.541216in}{1.582658in}}%
\pgfpathlineto{\pgfqpoint{4.591994in}{1.669320in}}%
\pgfpathlineto{\pgfqpoint{4.642772in}{1.762604in}}%
\pgfpathlineto{\pgfqpoint{4.693550in}{1.862882in}}%
\pgfpathlineto{\pgfqpoint{4.744328in}{1.970532in}}%
\pgfpathlineto{\pgfqpoint{4.795106in}{2.085949in}}%
\pgfpathlineto{\pgfqpoint{4.845885in}{2.209538in}}%
\pgfpathlineto{\pgfqpoint{4.896663in}{2.341716in}}%
\pgfpathlineto{\pgfqpoint{4.947441in}{2.482913in}}%
\pgfpathlineto{\pgfqpoint{4.998219in}{2.633570in}}%
\pgfpathlineto{\pgfqpoint{5.048997in}{2.794142in}}%
\pgfpathlineto{\pgfqpoint{5.099775in}{2.965095in}}%
\pgfpathlineto{\pgfqpoint{5.150553in}{3.146906in}}%
\pgfpathlineto{\pgfqpoint{5.188636in}{3.290684in}}%
\pgfpathlineto{\pgfqpoint{5.188636in}{3.290684in}}%
\pgfusepath{stroke}%
\end{pgfscope}%
\begin{pgfscope}%
\pgfsetrectcap%
\pgfsetmiterjoin%
\pgfsetlinewidth{0.803000pt}%
\definecolor{currentstroke}{rgb}{0.000000,0.000000,0.000000}%
\pgfsetstrokecolor{currentstroke}%
\pgfsetdash{}{0pt}%
\pgfpathmoveto{\pgfqpoint{0.750000in}{0.500000in}}%
\pgfpathlineto{\pgfqpoint{0.750000in}{3.520000in}}%
\pgfusepath{stroke}%
\end{pgfscope}%
\begin{pgfscope}%
\pgfsetrectcap%
\pgfsetmiterjoin%
\pgfsetlinewidth{0.803000pt}%
\definecolor{currentstroke}{rgb}{0.000000,0.000000,0.000000}%
\pgfsetstrokecolor{currentstroke}%
\pgfsetdash{}{0pt}%
\pgfpathmoveto{\pgfqpoint{5.400000in}{0.500000in}}%
\pgfpathlineto{\pgfqpoint{5.400000in}{3.520000in}}%
\pgfusepath{stroke}%
\end{pgfscope}%
\begin{pgfscope}%
\pgfsetrectcap%
\pgfsetmiterjoin%
\pgfsetlinewidth{0.803000pt}%
\definecolor{currentstroke}{rgb}{0.000000,0.000000,0.000000}%
\pgfsetstrokecolor{currentstroke}%
\pgfsetdash{}{0pt}%
\pgfpathmoveto{\pgfqpoint{0.750000in}{0.500000in}}%
\pgfpathlineto{\pgfqpoint{5.400000in}{0.500000in}}%
\pgfusepath{stroke}%
\end{pgfscope}%
\begin{pgfscope}%
\pgfsetrectcap%
\pgfsetmiterjoin%
\pgfsetlinewidth{0.803000pt}%
\definecolor{currentstroke}{rgb}{0.000000,0.000000,0.000000}%
\pgfsetstrokecolor{currentstroke}%
\pgfsetdash{}{0pt}%
\pgfpathmoveto{\pgfqpoint{0.750000in}{3.520000in}}%
\pgfpathlineto{\pgfqpoint{5.400000in}{3.520000in}}%
\pgfusepath{stroke}%
\end{pgfscope}%
\begin{pgfscope}%
\pgfsetbuttcap%
\pgfsetmiterjoin%
\definecolor{currentfill}{rgb}{1.000000,1.000000,1.000000}%
\pgfsetfillcolor{currentfill}%
\pgfsetfillopacity{0.800000}%
\pgfsetlinewidth{1.003750pt}%
\definecolor{currentstroke}{rgb}{0.800000,0.800000,0.800000}%
\pgfsetstrokecolor{currentstroke}%
\pgfsetstrokeopacity{0.800000}%
\pgfsetdash{}{0pt}%
\pgfpathmoveto{\pgfqpoint{0.847222in}{0.569444in}}%
\pgfpathlineto{\pgfqpoint{1.423852in}{0.569444in}}%
\pgfpathquadraticcurveto{\pgfqpoint{1.451630in}{0.569444in}}{\pgfqpoint{1.451630in}{0.597222in}}%
\pgfpathlineto{\pgfqpoint{1.451630in}{1.164352in}}%
\pgfpathquadraticcurveto{\pgfqpoint{1.451630in}{1.192129in}}{\pgfqpoint{1.423852in}{1.192129in}}%
\pgfpathlineto{\pgfqpoint{0.847222in}{1.192129in}}%
\pgfpathquadraticcurveto{\pgfqpoint{0.819444in}{1.192129in}}{\pgfqpoint{0.819444in}{1.164352in}}%
\pgfpathlineto{\pgfqpoint{0.819444in}{0.597222in}}%
\pgfpathquadraticcurveto{\pgfqpoint{0.819444in}{0.569444in}}{\pgfqpoint{0.847222in}{0.569444in}}%
\pgfpathclose%
\pgfusepath{stroke,fill}%
\end{pgfscope}%
\begin{pgfscope}%
\pgfsetrectcap%
\pgfsetroundjoin%
\pgfsetlinewidth{1.505625pt}%
\definecolor{currentstroke}{rgb}{0.121569,0.466667,0.705882}%
\pgfsetstrokecolor{currentstroke}%
\pgfsetdash{}{0pt}%
\pgfpathmoveto{\pgfqpoint{0.875000in}{1.087963in}}%
\pgfpathlineto{\pgfqpoint{1.152778in}{1.087963in}}%
\pgfusepath{stroke}%
\end{pgfscope}%
\begin{pgfscope}%
\pgftext[x=1.263889in,y=1.039352in,left,base]{\rmfamily\fontsize{10.000000}{12.000000}\selectfont \(\displaystyle f\)}%
\end{pgfscope}%
\begin{pgfscope}%
\pgfsetrectcap%
\pgfsetroundjoin%
\pgfsetlinewidth{1.505625pt}%
\definecolor{currentstroke}{rgb}{1.000000,0.498039,0.054902}%
\pgfsetstrokecolor{currentstroke}%
\pgfsetdash{}{0pt}%
\pgfpathmoveto{\pgfqpoint{0.875000in}{0.894290in}}%
\pgfpathlineto{\pgfqpoint{1.152778in}{0.894290in}}%
\pgfusepath{stroke}%
\end{pgfscope}%
\begin{pgfscope}%
\pgftext[x=1.263889in,y=0.845679in,left,base]{\rmfamily\fontsize{10.000000}{12.000000}\selectfont \(\displaystyle p_1\)}%
\end{pgfscope}%
\begin{pgfscope}%
\pgfsetrectcap%
\pgfsetroundjoin%
\pgfsetlinewidth{1.505625pt}%
\definecolor{currentstroke}{rgb}{0.172549,0.627451,0.172549}%
\pgfsetstrokecolor{currentstroke}%
\pgfsetdash{}{0pt}%
\pgfpathmoveto{\pgfqpoint{0.875000in}{0.700617in}}%
\pgfpathlineto{\pgfqpoint{1.152778in}{0.700617in}}%
\pgfusepath{stroke}%
\end{pgfscope}%
\begin{pgfscope}%
\pgftext[x=1.263889in,y=0.652006in,left,base]{\rmfamily\fontsize{10.000000}{12.000000}\selectfont \(\displaystyle p_2\)}%
\end{pgfscope}%
\end{pgfpicture}%
\makeatother%
\endgroup%
}
\centering \scalebox{0.8}{%% Creator: Matplotlib, PGF backend
%%
%% To include the figure in your LaTeX document, write
%%   \input{<filename>.pgf}
%%
%% Make sure the required packages are loaded in your preamble
%%   \usepackage{pgf}
%%
%% Figures using additional raster images can only be included by \input if
%% they are in the same directory as the main LaTeX file. For loading figures
%% from other directories you can use the `import` package
%%   \usepackage{import}
%% and then include the figures with
%%   \import{<path to file>}{<filename>.pgf}
%%
%% Matplotlib used the following preamble
%%   \usepackage{fontspec}
%%
\begingroup%
\makeatletter%
\begin{pgfpicture}%
\pgfpathrectangle{\pgfpointorigin}{\pgfqpoint{6.000000in}{4.000000in}}%
\pgfusepath{use as bounding box, clip}%
\begin{pgfscope}%
\pgfsetbuttcap%
\pgfsetmiterjoin%
\definecolor{currentfill}{rgb}{1.000000,1.000000,1.000000}%
\pgfsetfillcolor{currentfill}%
\pgfsetlinewidth{0.000000pt}%
\definecolor{currentstroke}{rgb}{1.000000,1.000000,1.000000}%
\pgfsetstrokecolor{currentstroke}%
\pgfsetdash{}{0pt}%
\pgfpathmoveto{\pgfqpoint{0.000000in}{0.000000in}}%
\pgfpathlineto{\pgfqpoint{6.000000in}{0.000000in}}%
\pgfpathlineto{\pgfqpoint{6.000000in}{4.000000in}}%
\pgfpathlineto{\pgfqpoint{0.000000in}{4.000000in}}%
\pgfpathclose%
\pgfusepath{fill}%
\end{pgfscope}%
\begin{pgfscope}%
\pgfsetbuttcap%
\pgfsetmiterjoin%
\definecolor{currentfill}{rgb}{1.000000,1.000000,1.000000}%
\pgfsetfillcolor{currentfill}%
\pgfsetlinewidth{0.000000pt}%
\definecolor{currentstroke}{rgb}{0.000000,0.000000,0.000000}%
\pgfsetstrokecolor{currentstroke}%
\pgfsetstrokeopacity{0.000000}%
\pgfsetdash{}{0pt}%
\pgfpathmoveto{\pgfqpoint{0.750000in}{0.500000in}}%
\pgfpathlineto{\pgfqpoint{5.400000in}{0.500000in}}%
\pgfpathlineto{\pgfqpoint{5.400000in}{3.520000in}}%
\pgfpathlineto{\pgfqpoint{0.750000in}{3.520000in}}%
\pgfpathclose%
\pgfusepath{fill}%
\end{pgfscope}%
\begin{pgfscope}%
\pgfpathrectangle{\pgfqpoint{0.750000in}{0.500000in}}{\pgfqpoint{4.650000in}{3.020000in}}%
\pgfusepath{clip}%
\pgfsetrectcap%
\pgfsetroundjoin%
\pgfsetlinewidth{0.803000pt}%
\definecolor{currentstroke}{rgb}{0.690196,0.690196,0.690196}%
\pgfsetstrokecolor{currentstroke}%
\pgfsetdash{}{0pt}%
\pgfpathmoveto{\pgfqpoint{0.961364in}{0.500000in}}%
\pgfpathlineto{\pgfqpoint{0.961364in}{3.520000in}}%
\pgfusepath{stroke}%
\end{pgfscope}%
\begin{pgfscope}%
\pgfsetbuttcap%
\pgfsetroundjoin%
\definecolor{currentfill}{rgb}{0.000000,0.000000,0.000000}%
\pgfsetfillcolor{currentfill}%
\pgfsetlinewidth{0.803000pt}%
\definecolor{currentstroke}{rgb}{0.000000,0.000000,0.000000}%
\pgfsetstrokecolor{currentstroke}%
\pgfsetdash{}{0pt}%
\pgfsys@defobject{currentmarker}{\pgfqpoint{0.000000in}{-0.048611in}}{\pgfqpoint{0.000000in}{0.000000in}}{%
\pgfpathmoveto{\pgfqpoint{0.000000in}{0.000000in}}%
\pgfpathlineto{\pgfqpoint{0.000000in}{-0.048611in}}%
\pgfusepath{stroke,fill}%
}%
\begin{pgfscope}%
\pgfsys@transformshift{0.961364in}{0.500000in}%
\pgfsys@useobject{currentmarker}{}%
\end{pgfscope}%
\end{pgfscope}%
\begin{pgfscope}%
\pgftext[x=0.961364in,y=0.402778in,,top]{\rmfamily\fontsize{10.000000}{12.000000}\selectfont \(\displaystyle -1.00\)}%
\end{pgfscope}%
\begin{pgfscope}%
\pgfpathrectangle{\pgfqpoint{0.750000in}{0.500000in}}{\pgfqpoint{4.650000in}{3.020000in}}%
\pgfusepath{clip}%
\pgfsetrectcap%
\pgfsetroundjoin%
\pgfsetlinewidth{0.803000pt}%
\definecolor{currentstroke}{rgb}{0.690196,0.690196,0.690196}%
\pgfsetstrokecolor{currentstroke}%
\pgfsetdash{}{0pt}%
\pgfpathmoveto{\pgfqpoint{1.489773in}{0.500000in}}%
\pgfpathlineto{\pgfqpoint{1.489773in}{3.520000in}}%
\pgfusepath{stroke}%
\end{pgfscope}%
\begin{pgfscope}%
\pgfsetbuttcap%
\pgfsetroundjoin%
\definecolor{currentfill}{rgb}{0.000000,0.000000,0.000000}%
\pgfsetfillcolor{currentfill}%
\pgfsetlinewidth{0.803000pt}%
\definecolor{currentstroke}{rgb}{0.000000,0.000000,0.000000}%
\pgfsetstrokecolor{currentstroke}%
\pgfsetdash{}{0pt}%
\pgfsys@defobject{currentmarker}{\pgfqpoint{0.000000in}{-0.048611in}}{\pgfqpoint{0.000000in}{0.000000in}}{%
\pgfpathmoveto{\pgfqpoint{0.000000in}{0.000000in}}%
\pgfpathlineto{\pgfqpoint{0.000000in}{-0.048611in}}%
\pgfusepath{stroke,fill}%
}%
\begin{pgfscope}%
\pgfsys@transformshift{1.489773in}{0.500000in}%
\pgfsys@useobject{currentmarker}{}%
\end{pgfscope}%
\end{pgfscope}%
\begin{pgfscope}%
\pgftext[x=1.489773in,y=0.402778in,,top]{\rmfamily\fontsize{10.000000}{12.000000}\selectfont \(\displaystyle -0.75\)}%
\end{pgfscope}%
\begin{pgfscope}%
\pgfpathrectangle{\pgfqpoint{0.750000in}{0.500000in}}{\pgfqpoint{4.650000in}{3.020000in}}%
\pgfusepath{clip}%
\pgfsetrectcap%
\pgfsetroundjoin%
\pgfsetlinewidth{0.803000pt}%
\definecolor{currentstroke}{rgb}{0.690196,0.690196,0.690196}%
\pgfsetstrokecolor{currentstroke}%
\pgfsetdash{}{0pt}%
\pgfpathmoveto{\pgfqpoint{2.018182in}{0.500000in}}%
\pgfpathlineto{\pgfqpoint{2.018182in}{3.520000in}}%
\pgfusepath{stroke}%
\end{pgfscope}%
\begin{pgfscope}%
\pgfsetbuttcap%
\pgfsetroundjoin%
\definecolor{currentfill}{rgb}{0.000000,0.000000,0.000000}%
\pgfsetfillcolor{currentfill}%
\pgfsetlinewidth{0.803000pt}%
\definecolor{currentstroke}{rgb}{0.000000,0.000000,0.000000}%
\pgfsetstrokecolor{currentstroke}%
\pgfsetdash{}{0pt}%
\pgfsys@defobject{currentmarker}{\pgfqpoint{0.000000in}{-0.048611in}}{\pgfqpoint{0.000000in}{0.000000in}}{%
\pgfpathmoveto{\pgfqpoint{0.000000in}{0.000000in}}%
\pgfpathlineto{\pgfqpoint{0.000000in}{-0.048611in}}%
\pgfusepath{stroke,fill}%
}%
\begin{pgfscope}%
\pgfsys@transformshift{2.018182in}{0.500000in}%
\pgfsys@useobject{currentmarker}{}%
\end{pgfscope}%
\end{pgfscope}%
\begin{pgfscope}%
\pgftext[x=2.018182in,y=0.402778in,,top]{\rmfamily\fontsize{10.000000}{12.000000}\selectfont \(\displaystyle -0.50\)}%
\end{pgfscope}%
\begin{pgfscope}%
\pgfpathrectangle{\pgfqpoint{0.750000in}{0.500000in}}{\pgfqpoint{4.650000in}{3.020000in}}%
\pgfusepath{clip}%
\pgfsetrectcap%
\pgfsetroundjoin%
\pgfsetlinewidth{0.803000pt}%
\definecolor{currentstroke}{rgb}{0.690196,0.690196,0.690196}%
\pgfsetstrokecolor{currentstroke}%
\pgfsetdash{}{0pt}%
\pgfpathmoveto{\pgfqpoint{2.546591in}{0.500000in}}%
\pgfpathlineto{\pgfqpoint{2.546591in}{3.520000in}}%
\pgfusepath{stroke}%
\end{pgfscope}%
\begin{pgfscope}%
\pgfsetbuttcap%
\pgfsetroundjoin%
\definecolor{currentfill}{rgb}{0.000000,0.000000,0.000000}%
\pgfsetfillcolor{currentfill}%
\pgfsetlinewidth{0.803000pt}%
\definecolor{currentstroke}{rgb}{0.000000,0.000000,0.000000}%
\pgfsetstrokecolor{currentstroke}%
\pgfsetdash{}{0pt}%
\pgfsys@defobject{currentmarker}{\pgfqpoint{0.000000in}{-0.048611in}}{\pgfqpoint{0.000000in}{0.000000in}}{%
\pgfpathmoveto{\pgfqpoint{0.000000in}{0.000000in}}%
\pgfpathlineto{\pgfqpoint{0.000000in}{-0.048611in}}%
\pgfusepath{stroke,fill}%
}%
\begin{pgfscope}%
\pgfsys@transformshift{2.546591in}{0.500000in}%
\pgfsys@useobject{currentmarker}{}%
\end{pgfscope}%
\end{pgfscope}%
\begin{pgfscope}%
\pgftext[x=2.546591in,y=0.402778in,,top]{\rmfamily\fontsize{10.000000}{12.000000}\selectfont \(\displaystyle -0.25\)}%
\end{pgfscope}%
\begin{pgfscope}%
\pgfpathrectangle{\pgfqpoint{0.750000in}{0.500000in}}{\pgfqpoint{4.650000in}{3.020000in}}%
\pgfusepath{clip}%
\pgfsetrectcap%
\pgfsetroundjoin%
\pgfsetlinewidth{0.803000pt}%
\definecolor{currentstroke}{rgb}{0.690196,0.690196,0.690196}%
\pgfsetstrokecolor{currentstroke}%
\pgfsetdash{}{0pt}%
\pgfpathmoveto{\pgfqpoint{3.075000in}{0.500000in}}%
\pgfpathlineto{\pgfqpoint{3.075000in}{3.520000in}}%
\pgfusepath{stroke}%
\end{pgfscope}%
\begin{pgfscope}%
\pgfsetbuttcap%
\pgfsetroundjoin%
\definecolor{currentfill}{rgb}{0.000000,0.000000,0.000000}%
\pgfsetfillcolor{currentfill}%
\pgfsetlinewidth{0.803000pt}%
\definecolor{currentstroke}{rgb}{0.000000,0.000000,0.000000}%
\pgfsetstrokecolor{currentstroke}%
\pgfsetdash{}{0pt}%
\pgfsys@defobject{currentmarker}{\pgfqpoint{0.000000in}{-0.048611in}}{\pgfqpoint{0.000000in}{0.000000in}}{%
\pgfpathmoveto{\pgfqpoint{0.000000in}{0.000000in}}%
\pgfpathlineto{\pgfqpoint{0.000000in}{-0.048611in}}%
\pgfusepath{stroke,fill}%
}%
\begin{pgfscope}%
\pgfsys@transformshift{3.075000in}{0.500000in}%
\pgfsys@useobject{currentmarker}{}%
\end{pgfscope}%
\end{pgfscope}%
\begin{pgfscope}%
\pgftext[x=3.075000in,y=0.402778in,,top]{\rmfamily\fontsize{10.000000}{12.000000}\selectfont \(\displaystyle 0.00\)}%
\end{pgfscope}%
\begin{pgfscope}%
\pgfpathrectangle{\pgfqpoint{0.750000in}{0.500000in}}{\pgfqpoint{4.650000in}{3.020000in}}%
\pgfusepath{clip}%
\pgfsetrectcap%
\pgfsetroundjoin%
\pgfsetlinewidth{0.803000pt}%
\definecolor{currentstroke}{rgb}{0.690196,0.690196,0.690196}%
\pgfsetstrokecolor{currentstroke}%
\pgfsetdash{}{0pt}%
\pgfpathmoveto{\pgfqpoint{3.603409in}{0.500000in}}%
\pgfpathlineto{\pgfqpoint{3.603409in}{3.520000in}}%
\pgfusepath{stroke}%
\end{pgfscope}%
\begin{pgfscope}%
\pgfsetbuttcap%
\pgfsetroundjoin%
\definecolor{currentfill}{rgb}{0.000000,0.000000,0.000000}%
\pgfsetfillcolor{currentfill}%
\pgfsetlinewidth{0.803000pt}%
\definecolor{currentstroke}{rgb}{0.000000,0.000000,0.000000}%
\pgfsetstrokecolor{currentstroke}%
\pgfsetdash{}{0pt}%
\pgfsys@defobject{currentmarker}{\pgfqpoint{0.000000in}{-0.048611in}}{\pgfqpoint{0.000000in}{0.000000in}}{%
\pgfpathmoveto{\pgfqpoint{0.000000in}{0.000000in}}%
\pgfpathlineto{\pgfqpoint{0.000000in}{-0.048611in}}%
\pgfusepath{stroke,fill}%
}%
\begin{pgfscope}%
\pgfsys@transformshift{3.603409in}{0.500000in}%
\pgfsys@useobject{currentmarker}{}%
\end{pgfscope}%
\end{pgfscope}%
\begin{pgfscope}%
\pgftext[x=3.603409in,y=0.402778in,,top]{\rmfamily\fontsize{10.000000}{12.000000}\selectfont \(\displaystyle 0.25\)}%
\end{pgfscope}%
\begin{pgfscope}%
\pgfpathrectangle{\pgfqpoint{0.750000in}{0.500000in}}{\pgfqpoint{4.650000in}{3.020000in}}%
\pgfusepath{clip}%
\pgfsetrectcap%
\pgfsetroundjoin%
\pgfsetlinewidth{0.803000pt}%
\definecolor{currentstroke}{rgb}{0.690196,0.690196,0.690196}%
\pgfsetstrokecolor{currentstroke}%
\pgfsetdash{}{0pt}%
\pgfpathmoveto{\pgfqpoint{4.131818in}{0.500000in}}%
\pgfpathlineto{\pgfqpoint{4.131818in}{3.520000in}}%
\pgfusepath{stroke}%
\end{pgfscope}%
\begin{pgfscope}%
\pgfsetbuttcap%
\pgfsetroundjoin%
\definecolor{currentfill}{rgb}{0.000000,0.000000,0.000000}%
\pgfsetfillcolor{currentfill}%
\pgfsetlinewidth{0.803000pt}%
\definecolor{currentstroke}{rgb}{0.000000,0.000000,0.000000}%
\pgfsetstrokecolor{currentstroke}%
\pgfsetdash{}{0pt}%
\pgfsys@defobject{currentmarker}{\pgfqpoint{0.000000in}{-0.048611in}}{\pgfqpoint{0.000000in}{0.000000in}}{%
\pgfpathmoveto{\pgfqpoint{0.000000in}{0.000000in}}%
\pgfpathlineto{\pgfqpoint{0.000000in}{-0.048611in}}%
\pgfusepath{stroke,fill}%
}%
\begin{pgfscope}%
\pgfsys@transformshift{4.131818in}{0.500000in}%
\pgfsys@useobject{currentmarker}{}%
\end{pgfscope}%
\end{pgfscope}%
\begin{pgfscope}%
\pgftext[x=4.131818in,y=0.402778in,,top]{\rmfamily\fontsize{10.000000}{12.000000}\selectfont \(\displaystyle 0.50\)}%
\end{pgfscope}%
\begin{pgfscope}%
\pgfpathrectangle{\pgfqpoint{0.750000in}{0.500000in}}{\pgfqpoint{4.650000in}{3.020000in}}%
\pgfusepath{clip}%
\pgfsetrectcap%
\pgfsetroundjoin%
\pgfsetlinewidth{0.803000pt}%
\definecolor{currentstroke}{rgb}{0.690196,0.690196,0.690196}%
\pgfsetstrokecolor{currentstroke}%
\pgfsetdash{}{0pt}%
\pgfpathmoveto{\pgfqpoint{4.660227in}{0.500000in}}%
\pgfpathlineto{\pgfqpoint{4.660227in}{3.520000in}}%
\pgfusepath{stroke}%
\end{pgfscope}%
\begin{pgfscope}%
\pgfsetbuttcap%
\pgfsetroundjoin%
\definecolor{currentfill}{rgb}{0.000000,0.000000,0.000000}%
\pgfsetfillcolor{currentfill}%
\pgfsetlinewidth{0.803000pt}%
\definecolor{currentstroke}{rgb}{0.000000,0.000000,0.000000}%
\pgfsetstrokecolor{currentstroke}%
\pgfsetdash{}{0pt}%
\pgfsys@defobject{currentmarker}{\pgfqpoint{0.000000in}{-0.048611in}}{\pgfqpoint{0.000000in}{0.000000in}}{%
\pgfpathmoveto{\pgfqpoint{0.000000in}{0.000000in}}%
\pgfpathlineto{\pgfqpoint{0.000000in}{-0.048611in}}%
\pgfusepath{stroke,fill}%
}%
\begin{pgfscope}%
\pgfsys@transformshift{4.660227in}{0.500000in}%
\pgfsys@useobject{currentmarker}{}%
\end{pgfscope}%
\end{pgfscope}%
\begin{pgfscope}%
\pgftext[x=4.660227in,y=0.402778in,,top]{\rmfamily\fontsize{10.000000}{12.000000}\selectfont \(\displaystyle 0.75\)}%
\end{pgfscope}%
\begin{pgfscope}%
\pgfpathrectangle{\pgfqpoint{0.750000in}{0.500000in}}{\pgfqpoint{4.650000in}{3.020000in}}%
\pgfusepath{clip}%
\pgfsetrectcap%
\pgfsetroundjoin%
\pgfsetlinewidth{0.803000pt}%
\definecolor{currentstroke}{rgb}{0.690196,0.690196,0.690196}%
\pgfsetstrokecolor{currentstroke}%
\pgfsetdash{}{0pt}%
\pgfpathmoveto{\pgfqpoint{5.188636in}{0.500000in}}%
\pgfpathlineto{\pgfqpoint{5.188636in}{3.520000in}}%
\pgfusepath{stroke}%
\end{pgfscope}%
\begin{pgfscope}%
\pgfsetbuttcap%
\pgfsetroundjoin%
\definecolor{currentfill}{rgb}{0.000000,0.000000,0.000000}%
\pgfsetfillcolor{currentfill}%
\pgfsetlinewidth{0.803000pt}%
\definecolor{currentstroke}{rgb}{0.000000,0.000000,0.000000}%
\pgfsetstrokecolor{currentstroke}%
\pgfsetdash{}{0pt}%
\pgfsys@defobject{currentmarker}{\pgfqpoint{0.000000in}{-0.048611in}}{\pgfqpoint{0.000000in}{0.000000in}}{%
\pgfpathmoveto{\pgfqpoint{0.000000in}{0.000000in}}%
\pgfpathlineto{\pgfqpoint{0.000000in}{-0.048611in}}%
\pgfusepath{stroke,fill}%
}%
\begin{pgfscope}%
\pgfsys@transformshift{5.188636in}{0.500000in}%
\pgfsys@useobject{currentmarker}{}%
\end{pgfscope}%
\end{pgfscope}%
\begin{pgfscope}%
\pgftext[x=5.188636in,y=0.402778in,,top]{\rmfamily\fontsize{10.000000}{12.000000}\selectfont \(\displaystyle 1.00\)}%
\end{pgfscope}%
\begin{pgfscope}%
\pgfpathrectangle{\pgfqpoint{0.750000in}{0.500000in}}{\pgfqpoint{4.650000in}{3.020000in}}%
\pgfusepath{clip}%
\pgfsetrectcap%
\pgfsetroundjoin%
\pgfsetlinewidth{0.803000pt}%
\definecolor{currentstroke}{rgb}{0.690196,0.690196,0.690196}%
\pgfsetstrokecolor{currentstroke}%
\pgfsetdash{}{0pt}%
\pgfpathmoveto{\pgfqpoint{0.750000in}{0.500000in}}%
\pgfpathlineto{\pgfqpoint{5.400000in}{0.500000in}}%
\pgfusepath{stroke}%
\end{pgfscope}%
\begin{pgfscope}%
\pgfsetbuttcap%
\pgfsetroundjoin%
\definecolor{currentfill}{rgb}{0.000000,0.000000,0.000000}%
\pgfsetfillcolor{currentfill}%
\pgfsetlinewidth{0.803000pt}%
\definecolor{currentstroke}{rgb}{0.000000,0.000000,0.000000}%
\pgfsetstrokecolor{currentstroke}%
\pgfsetdash{}{0pt}%
\pgfsys@defobject{currentmarker}{\pgfqpoint{-0.048611in}{0.000000in}}{\pgfqpoint{0.000000in}{0.000000in}}{%
\pgfpathmoveto{\pgfqpoint{0.000000in}{0.000000in}}%
\pgfpathlineto{\pgfqpoint{-0.048611in}{0.000000in}}%
\pgfusepath{stroke,fill}%
}%
\begin{pgfscope}%
\pgfsys@transformshift{0.750000in}{0.500000in}%
\pgfsys@useobject{currentmarker}{}%
\end{pgfscope}%
\end{pgfscope}%
\begin{pgfscope}%
\pgftext[x=0.228394in,y=0.451806in,left,base]{\rmfamily\fontsize{10.000000}{12.000000}\selectfont \(\displaystyle -0.015\)}%
\end{pgfscope}%
\begin{pgfscope}%
\pgfpathrectangle{\pgfqpoint{0.750000in}{0.500000in}}{\pgfqpoint{4.650000in}{3.020000in}}%
\pgfusepath{clip}%
\pgfsetrectcap%
\pgfsetroundjoin%
\pgfsetlinewidth{0.803000pt}%
\definecolor{currentstroke}{rgb}{0.690196,0.690196,0.690196}%
\pgfsetstrokecolor{currentstroke}%
\pgfsetdash{}{0pt}%
\pgfpathmoveto{\pgfqpoint{0.750000in}{1.003333in}}%
\pgfpathlineto{\pgfqpoint{5.400000in}{1.003333in}}%
\pgfusepath{stroke}%
\end{pgfscope}%
\begin{pgfscope}%
\pgfsetbuttcap%
\pgfsetroundjoin%
\definecolor{currentfill}{rgb}{0.000000,0.000000,0.000000}%
\pgfsetfillcolor{currentfill}%
\pgfsetlinewidth{0.803000pt}%
\definecolor{currentstroke}{rgb}{0.000000,0.000000,0.000000}%
\pgfsetstrokecolor{currentstroke}%
\pgfsetdash{}{0pt}%
\pgfsys@defobject{currentmarker}{\pgfqpoint{-0.048611in}{0.000000in}}{\pgfqpoint{0.000000in}{0.000000in}}{%
\pgfpathmoveto{\pgfqpoint{0.000000in}{0.000000in}}%
\pgfpathlineto{\pgfqpoint{-0.048611in}{0.000000in}}%
\pgfusepath{stroke,fill}%
}%
\begin{pgfscope}%
\pgfsys@transformshift{0.750000in}{1.003333in}%
\pgfsys@useobject{currentmarker}{}%
\end{pgfscope}%
\end{pgfscope}%
\begin{pgfscope}%
\pgftext[x=0.228394in,y=0.955139in,left,base]{\rmfamily\fontsize{10.000000}{12.000000}\selectfont \(\displaystyle -0.010\)}%
\end{pgfscope}%
\begin{pgfscope}%
\pgfpathrectangle{\pgfqpoint{0.750000in}{0.500000in}}{\pgfqpoint{4.650000in}{3.020000in}}%
\pgfusepath{clip}%
\pgfsetrectcap%
\pgfsetroundjoin%
\pgfsetlinewidth{0.803000pt}%
\definecolor{currentstroke}{rgb}{0.690196,0.690196,0.690196}%
\pgfsetstrokecolor{currentstroke}%
\pgfsetdash{}{0pt}%
\pgfpathmoveto{\pgfqpoint{0.750000in}{1.506667in}}%
\pgfpathlineto{\pgfqpoint{5.400000in}{1.506667in}}%
\pgfusepath{stroke}%
\end{pgfscope}%
\begin{pgfscope}%
\pgfsetbuttcap%
\pgfsetroundjoin%
\definecolor{currentfill}{rgb}{0.000000,0.000000,0.000000}%
\pgfsetfillcolor{currentfill}%
\pgfsetlinewidth{0.803000pt}%
\definecolor{currentstroke}{rgb}{0.000000,0.000000,0.000000}%
\pgfsetstrokecolor{currentstroke}%
\pgfsetdash{}{0pt}%
\pgfsys@defobject{currentmarker}{\pgfqpoint{-0.048611in}{0.000000in}}{\pgfqpoint{0.000000in}{0.000000in}}{%
\pgfpathmoveto{\pgfqpoint{0.000000in}{0.000000in}}%
\pgfpathlineto{\pgfqpoint{-0.048611in}{0.000000in}}%
\pgfusepath{stroke,fill}%
}%
\begin{pgfscope}%
\pgfsys@transformshift{0.750000in}{1.506667in}%
\pgfsys@useobject{currentmarker}{}%
\end{pgfscope}%
\end{pgfscope}%
\begin{pgfscope}%
\pgftext[x=0.228394in,y=1.458472in,left,base]{\rmfamily\fontsize{10.000000}{12.000000}\selectfont \(\displaystyle -0.005\)}%
\end{pgfscope}%
\begin{pgfscope}%
\pgfpathrectangle{\pgfqpoint{0.750000in}{0.500000in}}{\pgfqpoint{4.650000in}{3.020000in}}%
\pgfusepath{clip}%
\pgfsetrectcap%
\pgfsetroundjoin%
\pgfsetlinewidth{0.803000pt}%
\definecolor{currentstroke}{rgb}{0.690196,0.690196,0.690196}%
\pgfsetstrokecolor{currentstroke}%
\pgfsetdash{}{0pt}%
\pgfpathmoveto{\pgfqpoint{0.750000in}{2.010000in}}%
\pgfpathlineto{\pgfqpoint{5.400000in}{2.010000in}}%
\pgfusepath{stroke}%
\end{pgfscope}%
\begin{pgfscope}%
\pgfsetbuttcap%
\pgfsetroundjoin%
\definecolor{currentfill}{rgb}{0.000000,0.000000,0.000000}%
\pgfsetfillcolor{currentfill}%
\pgfsetlinewidth{0.803000pt}%
\definecolor{currentstroke}{rgb}{0.000000,0.000000,0.000000}%
\pgfsetstrokecolor{currentstroke}%
\pgfsetdash{}{0pt}%
\pgfsys@defobject{currentmarker}{\pgfqpoint{-0.048611in}{0.000000in}}{\pgfqpoint{0.000000in}{0.000000in}}{%
\pgfpathmoveto{\pgfqpoint{0.000000in}{0.000000in}}%
\pgfpathlineto{\pgfqpoint{-0.048611in}{0.000000in}}%
\pgfusepath{stroke,fill}%
}%
\begin{pgfscope}%
\pgfsys@transformshift{0.750000in}{2.010000in}%
\pgfsys@useobject{currentmarker}{}%
\end{pgfscope}%
\end{pgfscope}%
\begin{pgfscope}%
\pgftext[x=0.336419in,y=1.961806in,left,base]{\rmfamily\fontsize{10.000000}{12.000000}\selectfont \(\displaystyle 0.000\)}%
\end{pgfscope}%
\begin{pgfscope}%
\pgfpathrectangle{\pgfqpoint{0.750000in}{0.500000in}}{\pgfqpoint{4.650000in}{3.020000in}}%
\pgfusepath{clip}%
\pgfsetrectcap%
\pgfsetroundjoin%
\pgfsetlinewidth{0.803000pt}%
\definecolor{currentstroke}{rgb}{0.690196,0.690196,0.690196}%
\pgfsetstrokecolor{currentstroke}%
\pgfsetdash{}{0pt}%
\pgfpathmoveto{\pgfqpoint{0.750000in}{2.513333in}}%
\pgfpathlineto{\pgfqpoint{5.400000in}{2.513333in}}%
\pgfusepath{stroke}%
\end{pgfscope}%
\begin{pgfscope}%
\pgfsetbuttcap%
\pgfsetroundjoin%
\definecolor{currentfill}{rgb}{0.000000,0.000000,0.000000}%
\pgfsetfillcolor{currentfill}%
\pgfsetlinewidth{0.803000pt}%
\definecolor{currentstroke}{rgb}{0.000000,0.000000,0.000000}%
\pgfsetstrokecolor{currentstroke}%
\pgfsetdash{}{0pt}%
\pgfsys@defobject{currentmarker}{\pgfqpoint{-0.048611in}{0.000000in}}{\pgfqpoint{0.000000in}{0.000000in}}{%
\pgfpathmoveto{\pgfqpoint{0.000000in}{0.000000in}}%
\pgfpathlineto{\pgfqpoint{-0.048611in}{0.000000in}}%
\pgfusepath{stroke,fill}%
}%
\begin{pgfscope}%
\pgfsys@transformshift{0.750000in}{2.513333in}%
\pgfsys@useobject{currentmarker}{}%
\end{pgfscope}%
\end{pgfscope}%
\begin{pgfscope}%
\pgftext[x=0.336419in,y=2.465139in,left,base]{\rmfamily\fontsize{10.000000}{12.000000}\selectfont \(\displaystyle 0.005\)}%
\end{pgfscope}%
\begin{pgfscope}%
\pgfpathrectangle{\pgfqpoint{0.750000in}{0.500000in}}{\pgfqpoint{4.650000in}{3.020000in}}%
\pgfusepath{clip}%
\pgfsetrectcap%
\pgfsetroundjoin%
\pgfsetlinewidth{0.803000pt}%
\definecolor{currentstroke}{rgb}{0.690196,0.690196,0.690196}%
\pgfsetstrokecolor{currentstroke}%
\pgfsetdash{}{0pt}%
\pgfpathmoveto{\pgfqpoint{0.750000in}{3.016667in}}%
\pgfpathlineto{\pgfqpoint{5.400000in}{3.016667in}}%
\pgfusepath{stroke}%
\end{pgfscope}%
\begin{pgfscope}%
\pgfsetbuttcap%
\pgfsetroundjoin%
\definecolor{currentfill}{rgb}{0.000000,0.000000,0.000000}%
\pgfsetfillcolor{currentfill}%
\pgfsetlinewidth{0.803000pt}%
\definecolor{currentstroke}{rgb}{0.000000,0.000000,0.000000}%
\pgfsetstrokecolor{currentstroke}%
\pgfsetdash{}{0pt}%
\pgfsys@defobject{currentmarker}{\pgfqpoint{-0.048611in}{0.000000in}}{\pgfqpoint{0.000000in}{0.000000in}}{%
\pgfpathmoveto{\pgfqpoint{0.000000in}{0.000000in}}%
\pgfpathlineto{\pgfqpoint{-0.048611in}{0.000000in}}%
\pgfusepath{stroke,fill}%
}%
\begin{pgfscope}%
\pgfsys@transformshift{0.750000in}{3.016667in}%
\pgfsys@useobject{currentmarker}{}%
\end{pgfscope}%
\end{pgfscope}%
\begin{pgfscope}%
\pgftext[x=0.336419in,y=2.968472in,left,base]{\rmfamily\fontsize{10.000000}{12.000000}\selectfont \(\displaystyle 0.010\)}%
\end{pgfscope}%
\begin{pgfscope}%
\pgfpathrectangle{\pgfqpoint{0.750000in}{0.500000in}}{\pgfqpoint{4.650000in}{3.020000in}}%
\pgfusepath{clip}%
\pgfsetrectcap%
\pgfsetroundjoin%
\pgfsetlinewidth{0.803000pt}%
\definecolor{currentstroke}{rgb}{0.690196,0.690196,0.690196}%
\pgfsetstrokecolor{currentstroke}%
\pgfsetdash{}{0pt}%
\pgfpathmoveto{\pgfqpoint{0.750000in}{3.520000in}}%
\pgfpathlineto{\pgfqpoint{5.400000in}{3.520000in}}%
\pgfusepath{stroke}%
\end{pgfscope}%
\begin{pgfscope}%
\pgfsetbuttcap%
\pgfsetroundjoin%
\definecolor{currentfill}{rgb}{0.000000,0.000000,0.000000}%
\pgfsetfillcolor{currentfill}%
\pgfsetlinewidth{0.803000pt}%
\definecolor{currentstroke}{rgb}{0.000000,0.000000,0.000000}%
\pgfsetstrokecolor{currentstroke}%
\pgfsetdash{}{0pt}%
\pgfsys@defobject{currentmarker}{\pgfqpoint{-0.048611in}{0.000000in}}{\pgfqpoint{0.000000in}{0.000000in}}{%
\pgfpathmoveto{\pgfqpoint{0.000000in}{0.000000in}}%
\pgfpathlineto{\pgfqpoint{-0.048611in}{0.000000in}}%
\pgfusepath{stroke,fill}%
}%
\begin{pgfscope}%
\pgfsys@transformshift{0.750000in}{3.520000in}%
\pgfsys@useobject{currentmarker}{}%
\end{pgfscope}%
\end{pgfscope}%
\begin{pgfscope}%
\pgftext[x=0.336419in,y=3.471806in,left,base]{\rmfamily\fontsize{10.000000}{12.000000}\selectfont \(\displaystyle 0.015\)}%
\end{pgfscope}%
\begin{pgfscope}%
\pgfpathrectangle{\pgfqpoint{0.750000in}{0.500000in}}{\pgfqpoint{4.650000in}{3.020000in}}%
\pgfusepath{clip}%
\pgfsetrectcap%
\pgfsetroundjoin%
\pgfsetlinewidth{1.505625pt}%
\definecolor{currentstroke}{rgb}{0.121569,0.466667,0.705882}%
\pgfsetstrokecolor{currentstroke}%
\pgfsetdash{}{0pt}%
\pgfpathmoveto{\pgfqpoint{0.961364in}{2.904044in}}%
\pgfpathlineto{\pgfqpoint{0.990984in}{2.581764in}}%
\pgfpathlineto{\pgfqpoint{1.016373in}{2.336359in}}%
\pgfpathlineto{\pgfqpoint{1.041762in}{2.117906in}}%
\pgfpathlineto{\pgfqpoint{1.067151in}{1.925005in}}%
\pgfpathlineto{\pgfqpoint{1.092540in}{1.756288in}}%
\pgfpathlineto{\pgfqpoint{1.117929in}{1.610414in}}%
\pgfpathlineto{\pgfqpoint{1.139087in}{1.505358in}}%
\pgfpathlineto{\pgfqpoint{1.160244in}{1.414515in}}%
\pgfpathlineto{\pgfqpoint{1.181402in}{1.337160in}}%
\pgfpathlineto{\pgfqpoint{1.202559in}{1.272580in}}%
\pgfpathlineto{\pgfqpoint{1.219485in}{1.229645in}}%
\pgfpathlineto{\pgfqpoint{1.236411in}{1.194089in}}%
\pgfpathlineto{\pgfqpoint{1.253337in}{1.165568in}}%
\pgfpathlineto{\pgfqpoint{1.270263in}{1.143744in}}%
\pgfpathlineto{\pgfqpoint{1.282958in}{1.131571in}}%
\pgfpathlineto{\pgfqpoint{1.295652in}{1.122840in}}%
\pgfpathlineto{\pgfqpoint{1.308347in}{1.117415in}}%
\pgfpathlineto{\pgfqpoint{1.321041in}{1.115162in}}%
\pgfpathlineto{\pgfqpoint{1.333736in}{1.115948in}}%
\pgfpathlineto{\pgfqpoint{1.346431in}{1.119644in}}%
\pgfpathlineto{\pgfqpoint{1.359125in}{1.126120in}}%
\pgfpathlineto{\pgfqpoint{1.376051in}{1.138863in}}%
\pgfpathlineto{\pgfqpoint{1.392977in}{1.156029in}}%
\pgfpathlineto{\pgfqpoint{1.409903in}{1.177330in}}%
\pgfpathlineto{\pgfqpoint{1.431061in}{1.209342in}}%
\pgfpathlineto{\pgfqpoint{1.452218in}{1.246834in}}%
\pgfpathlineto{\pgfqpoint{1.477607in}{1.298322in}}%
\pgfpathlineto{\pgfqpoint{1.507228in}{1.366243in}}%
\pgfpathlineto{\pgfqpoint{1.541080in}{1.452581in}}%
\pgfpathlineto{\pgfqpoint{1.579163in}{1.558540in}}%
\pgfpathlineto{\pgfqpoint{1.629941in}{1.710079in}}%
\pgfpathlineto{\pgfqpoint{1.718803in}{1.987847in}}%
\pgfpathlineto{\pgfqpoint{1.794970in}{2.221536in}}%
\pgfpathlineto{\pgfqpoint{1.845748in}{2.367412in}}%
\pgfpathlineto{\pgfqpoint{1.888063in}{2.479769in}}%
\pgfpathlineto{\pgfqpoint{1.926147in}{2.572091in}}%
\pgfpathlineto{\pgfqpoint{1.959999in}{2.646159in}}%
\pgfpathlineto{\pgfqpoint{1.989619in}{2.704194in}}%
\pgfpathlineto{\pgfqpoint{2.019240in}{2.755457in}}%
\pgfpathlineto{\pgfqpoint{2.044629in}{2.793725in}}%
\pgfpathlineto{\pgfqpoint{2.070018in}{2.826543in}}%
\pgfpathlineto{\pgfqpoint{2.095407in}{2.853743in}}%
\pgfpathlineto{\pgfqpoint{2.116564in}{2.872015in}}%
\pgfpathlineto{\pgfqpoint{2.137722in}{2.886223in}}%
\pgfpathlineto{\pgfqpoint{2.158879in}{2.896316in}}%
\pgfpathlineto{\pgfqpoint{2.175805in}{2.901404in}}%
\pgfpathlineto{\pgfqpoint{2.192731in}{2.903824in}}%
\pgfpathlineto{\pgfqpoint{2.209657in}{2.903572in}}%
\pgfpathlineto{\pgfqpoint{2.226583in}{2.900647in}}%
\pgfpathlineto{\pgfqpoint{2.243509in}{2.895056in}}%
\pgfpathlineto{\pgfqpoint{2.264667in}{2.884338in}}%
\pgfpathlineto{\pgfqpoint{2.285824in}{2.869512in}}%
\pgfpathlineto{\pgfqpoint{2.306982in}{2.850631in}}%
\pgfpathlineto{\pgfqpoint{2.328140in}{2.827762in}}%
\pgfpathlineto{\pgfqpoint{2.353529in}{2.795171in}}%
\pgfpathlineto{\pgfqpoint{2.378918in}{2.757125in}}%
\pgfpathlineto{\pgfqpoint{2.404307in}{2.713822in}}%
\pgfpathlineto{\pgfqpoint{2.433927in}{2.656966in}}%
\pgfpathlineto{\pgfqpoint{2.463548in}{2.593687in}}%
\pgfpathlineto{\pgfqpoint{2.497400in}{2.514112in}}%
\pgfpathlineto{\pgfqpoint{2.535483in}{2.416297in}}%
\pgfpathlineto{\pgfqpoint{2.577798in}{2.298826in}}%
\pgfpathlineto{\pgfqpoint{2.628576in}{2.148370in}}%
\pgfpathlineto{\pgfqpoint{2.704743in}{1.911488in}}%
\pgfpathlineto{\pgfqpoint{2.793605in}{1.636510in}}%
\pgfpathlineto{\pgfqpoint{2.840152in}{1.501874in}}%
\pgfpathlineto{\pgfqpoint{2.878235in}{1.400642in}}%
\pgfpathlineto{\pgfqpoint{2.912087in}{1.319706in}}%
\pgfpathlineto{\pgfqpoint{2.941708in}{1.257487in}}%
\pgfpathlineto{\pgfqpoint{2.967097in}{1.211640in}}%
\pgfpathlineto{\pgfqpoint{2.988254in}{1.179399in}}%
\pgfpathlineto{\pgfqpoint{3.009412in}{1.153153in}}%
\pgfpathlineto{\pgfqpoint{3.026338in}{1.136846in}}%
\pgfpathlineto{\pgfqpoint{3.043264in}{1.125019in}}%
\pgfpathlineto{\pgfqpoint{3.055958in}{1.119267in}}%
\pgfpathlineto{\pgfqpoint{3.068653in}{1.116330in}}%
\pgfpathlineto{\pgfqpoint{3.081347in}{1.116330in}}%
\pgfpathlineto{\pgfqpoint{3.094042in}{1.119267in}}%
\pgfpathlineto{\pgfqpoint{3.106736in}{1.125019in}}%
\pgfpathlineto{\pgfqpoint{3.123662in}{1.136846in}}%
\pgfpathlineto{\pgfqpoint{3.140588in}{1.153153in}}%
\pgfpathlineto{\pgfqpoint{3.157514in}{1.173652in}}%
\pgfpathlineto{\pgfqpoint{3.178672in}{1.204738in}}%
\pgfpathlineto{\pgfqpoint{3.199829in}{1.241388in}}%
\pgfpathlineto{\pgfqpoint{3.225218in}{1.291971in}}%
\pgfpathlineto{\pgfqpoint{3.254839in}{1.358979in}}%
\pgfpathlineto{\pgfqpoint{3.288691in}{1.444452in}}%
\pgfpathlineto{\pgfqpoint{3.326775in}{1.549658in}}%
\pgfpathlineto{\pgfqpoint{3.373321in}{1.687616in}}%
\pgfpathlineto{\pgfqpoint{3.449488in}{1.924795in}}%
\pgfpathlineto{\pgfqpoint{3.538350in}{2.199466in}}%
\pgfpathlineto{\pgfqpoint{3.589128in}{2.346810in}}%
\pgfpathlineto{\pgfqpoint{3.631443in}{2.460781in}}%
\pgfpathlineto{\pgfqpoint{3.669526in}{2.554823in}}%
\pgfpathlineto{\pgfqpoint{3.703378in}{2.630607in}}%
\pgfpathlineto{\pgfqpoint{3.732999in}{2.690270in}}%
\pgfpathlineto{\pgfqpoint{3.762619in}{2.743264in}}%
\pgfpathlineto{\pgfqpoint{3.788008in}{2.783086in}}%
\pgfpathlineto{\pgfqpoint{3.813397in}{2.817516in}}%
\pgfpathlineto{\pgfqpoint{3.838787in}{2.846374in}}%
\pgfpathlineto{\pgfqpoint{3.859944in}{2.866058in}}%
\pgfpathlineto{\pgfqpoint{3.881102in}{2.881700in}}%
\pgfpathlineto{\pgfqpoint{3.902259in}{2.893243in}}%
\pgfpathlineto{\pgfqpoint{3.923417in}{2.900647in}}%
\pgfpathlineto{\pgfqpoint{3.940343in}{2.903572in}}%
\pgfpathlineto{\pgfqpoint{3.957269in}{2.903824in}}%
\pgfpathlineto{\pgfqpoint{3.974195in}{2.901404in}}%
\pgfpathlineto{\pgfqpoint{3.991121in}{2.896316in}}%
\pgfpathlineto{\pgfqpoint{4.012278in}{2.886223in}}%
\pgfpathlineto{\pgfqpoint{4.033436in}{2.872015in}}%
\pgfpathlineto{\pgfqpoint{4.054593in}{2.853743in}}%
\pgfpathlineto{\pgfqpoint{4.075751in}{2.831471in}}%
\pgfpathlineto{\pgfqpoint{4.101140in}{2.799578in}}%
\pgfpathlineto{\pgfqpoint{4.126529in}{2.762205in}}%
\pgfpathlineto{\pgfqpoint{4.151918in}{2.719549in}}%
\pgfpathlineto{\pgfqpoint{4.181538in}{2.663408in}}%
\pgfpathlineto{\pgfqpoint{4.211159in}{2.600793in}}%
\pgfpathlineto{\pgfqpoint{4.245011in}{2.521910in}}%
\pgfpathlineto{\pgfqpoint{4.283094in}{2.424774in}}%
\pgfpathlineto{\pgfqpoint{4.325410in}{2.307920in}}%
\pgfpathlineto{\pgfqpoint{4.376188in}{2.157988in}}%
\pgfpathlineto{\pgfqpoint{4.448123in}{1.934692in}}%
\pgfpathlineto{\pgfqpoint{4.545448in}{1.633148in}}%
\pgfpathlineto{\pgfqpoint{4.591994in}{1.498664in}}%
\pgfpathlineto{\pgfqpoint{4.630078in}{1.397626in}}%
\pgfpathlineto{\pgfqpoint{4.663930in}{1.316919in}}%
\pgfpathlineto{\pgfqpoint{4.693550in}{1.254944in}}%
\pgfpathlineto{\pgfqpoint{4.718939in}{1.209342in}}%
\pgfpathlineto{\pgfqpoint{4.740097in}{1.177330in}}%
\pgfpathlineto{\pgfqpoint{4.761254in}{1.151338in}}%
\pgfpathlineto{\pgfqpoint{4.778180in}{1.135251in}}%
\pgfpathlineto{\pgfqpoint{4.795106in}{1.123660in}}%
\pgfpathlineto{\pgfqpoint{4.807801in}{1.118097in}}%
\pgfpathlineto{\pgfqpoint{4.820495in}{1.115357in}}%
\pgfpathlineto{\pgfqpoint{4.833190in}{1.115569in}}%
\pgfpathlineto{\pgfqpoint{4.845885in}{1.118865in}}%
\pgfpathlineto{\pgfqpoint{4.858579in}{1.125377in}}%
\pgfpathlineto{\pgfqpoint{4.871274in}{1.135240in}}%
\pgfpathlineto{\pgfqpoint{4.883968in}{1.148591in}}%
\pgfpathlineto{\pgfqpoint{4.896663in}{1.165568in}}%
\pgfpathlineto{\pgfqpoint{4.913589in}{1.194089in}}%
\pgfpathlineto{\pgfqpoint{4.930515in}{1.229645in}}%
\pgfpathlineto{\pgfqpoint{4.947441in}{1.272580in}}%
\pgfpathlineto{\pgfqpoint{4.964367in}{1.323244in}}%
\pgfpathlineto{\pgfqpoint{4.985524in}{1.397988in}}%
\pgfpathlineto{\pgfqpoint{5.006682in}{1.486076in}}%
\pgfpathlineto{\pgfqpoint{5.027839in}{1.588230in}}%
\pgfpathlineto{\pgfqpoint{5.048997in}{1.705190in}}%
\pgfpathlineto{\pgfqpoint{5.070154in}{1.837708in}}%
\pgfpathlineto{\pgfqpoint{5.095543in}{2.018348in}}%
\pgfpathlineto{\pgfqpoint{5.120932in}{2.223852in}}%
\pgfpathlineto{\pgfqpoint{5.146321in}{2.455604in}}%
\pgfpathlineto{\pgfqpoint{5.171710in}{2.715019in}}%
\pgfpathlineto{\pgfqpoint{5.188636in}{2.904044in}}%
\pgfpathlineto{\pgfqpoint{5.188636in}{2.904044in}}%
\pgfusepath{stroke}%
\end{pgfscope}%
\begin{pgfscope}%
\pgfsetrectcap%
\pgfsetmiterjoin%
\pgfsetlinewidth{0.803000pt}%
\definecolor{currentstroke}{rgb}{0.000000,0.000000,0.000000}%
\pgfsetstrokecolor{currentstroke}%
\pgfsetdash{}{0pt}%
\pgfpathmoveto{\pgfqpoint{0.750000in}{0.500000in}}%
\pgfpathlineto{\pgfqpoint{0.750000in}{3.520000in}}%
\pgfusepath{stroke}%
\end{pgfscope}%
\begin{pgfscope}%
\pgfsetrectcap%
\pgfsetmiterjoin%
\pgfsetlinewidth{0.803000pt}%
\definecolor{currentstroke}{rgb}{0.000000,0.000000,0.000000}%
\pgfsetstrokecolor{currentstroke}%
\pgfsetdash{}{0pt}%
\pgfpathmoveto{\pgfqpoint{5.400000in}{0.500000in}}%
\pgfpathlineto{\pgfqpoint{5.400000in}{3.520000in}}%
\pgfusepath{stroke}%
\end{pgfscope}%
\begin{pgfscope}%
\pgfsetrectcap%
\pgfsetmiterjoin%
\pgfsetlinewidth{0.803000pt}%
\definecolor{currentstroke}{rgb}{0.000000,0.000000,0.000000}%
\pgfsetstrokecolor{currentstroke}%
\pgfsetdash{}{0pt}%
\pgfpathmoveto{\pgfqpoint{0.750000in}{0.500000in}}%
\pgfpathlineto{\pgfqpoint{5.400000in}{0.500000in}}%
\pgfusepath{stroke}%
\end{pgfscope}%
\begin{pgfscope}%
\pgfsetrectcap%
\pgfsetmiterjoin%
\pgfsetlinewidth{0.803000pt}%
\definecolor{currentstroke}{rgb}{0.000000,0.000000,0.000000}%
\pgfsetstrokecolor{currentstroke}%
\pgfsetdash{}{0pt}%
\pgfpathmoveto{\pgfqpoint{0.750000in}{3.520000in}}%
\pgfpathlineto{\pgfqpoint{5.400000in}{3.520000in}}%
\pgfusepath{stroke}%
\end{pgfscope}%
\begin{pgfscope}%
\pgfsetbuttcap%
\pgfsetmiterjoin%
\definecolor{currentfill}{rgb}{1.000000,1.000000,1.000000}%
\pgfsetfillcolor{currentfill}%
\pgfsetfillopacity{0.800000}%
\pgfsetlinewidth{1.003750pt}%
\definecolor{currentstroke}{rgb}{0.800000,0.800000,0.800000}%
\pgfsetstrokecolor{currentstroke}%
\pgfsetstrokeopacity{0.800000}%
\pgfsetdash{}{0pt}%
\pgfpathmoveto{\pgfqpoint{4.473447in}{3.215216in}}%
\pgfpathlineto{\pgfqpoint{5.302778in}{3.215216in}}%
\pgfpathquadraticcurveto{\pgfqpoint{5.330556in}{3.215216in}}{\pgfqpoint{5.330556in}{3.242994in}}%
\pgfpathlineto{\pgfqpoint{5.330556in}{3.422778in}}%
\pgfpathquadraticcurveto{\pgfqpoint{5.330556in}{3.450556in}}{\pgfqpoint{5.302778in}{3.450556in}}%
\pgfpathlineto{\pgfqpoint{4.473447in}{3.450556in}}%
\pgfpathquadraticcurveto{\pgfqpoint{4.445670in}{3.450556in}}{\pgfqpoint{4.445670in}{3.422778in}}%
\pgfpathlineto{\pgfqpoint{4.445670in}{3.242994in}}%
\pgfpathquadraticcurveto{\pgfqpoint{4.445670in}{3.215216in}}{\pgfqpoint{4.473447in}{3.215216in}}%
\pgfpathclose%
\pgfusepath{stroke,fill}%
\end{pgfscope}%
\begin{pgfscope}%
\pgfsetrectcap%
\pgfsetroundjoin%
\pgfsetlinewidth{1.505625pt}%
\definecolor{currentstroke}{rgb}{0.121569,0.466667,0.705882}%
\pgfsetstrokecolor{currentstroke}%
\pgfsetdash{}{0pt}%
\pgfpathmoveto{\pgfqpoint{4.501225in}{3.346389in}}%
\pgfpathlineto{\pgfqpoint{4.779003in}{3.346389in}}%
\pgfusepath{stroke}%
\end{pgfscope}%
\begin{pgfscope}%
\pgftext[x=4.890114in,y=3.297778in,left,base]{\rmfamily\fontsize{10.000000}{12.000000}\selectfont \(\displaystyle  p_2 - f \)}%
\end{pgfscope}%
\end{pgfpicture}%
\makeatother%
\endgroup%
}
\caption{Interpolating polynomials of degree $n$ to $f_1$ using equally spaced nodes} \label{Fig:SpaceTan}
\end{figure}
The graph for interpolations of $ f_2 \rbr{x} = \se^{-x^2} $ using equally spaced nodes is shown in Figure \ref{Fig:SpaceExp}.
\begin{figure}[htbp]
\centering \scalebox{0.8}{%% Creator: Matplotlib, PGF backend
%%
%% To include the figure in your LaTeX document, write
%%   \input{<filename>.pgf}
%%
%% Make sure the required packages are loaded in your preamble
%%   \usepackage{pgf}
%%
%% Figures using additional raster images can only be included by \input if
%% they are in the same directory as the main LaTeX file. For loading figures
%% from other directories you can use the `import` package
%%   \usepackage{import}
%% and then include the figures with
%%   \import{<path to file>}{<filename>.pgf}
%%
%% Matplotlib used the following preamble
%%   \usepackage{fontspec}
%%
\begingroup%
\makeatletter%
\begin{pgfpicture}%
\pgfpathrectangle{\pgfpointorigin}{\pgfqpoint{6.000000in}{4.000000in}}%
\pgfusepath{use as bounding box, clip}%
\begin{pgfscope}%
\pgfsetbuttcap%
\pgfsetmiterjoin%
\definecolor{currentfill}{rgb}{1.000000,1.000000,1.000000}%
\pgfsetfillcolor{currentfill}%
\pgfsetlinewidth{0.000000pt}%
\definecolor{currentstroke}{rgb}{1.000000,1.000000,1.000000}%
\pgfsetstrokecolor{currentstroke}%
\pgfsetdash{}{0pt}%
\pgfpathmoveto{\pgfqpoint{0.000000in}{0.000000in}}%
\pgfpathlineto{\pgfqpoint{6.000000in}{0.000000in}}%
\pgfpathlineto{\pgfqpoint{6.000000in}{4.000000in}}%
\pgfpathlineto{\pgfqpoint{0.000000in}{4.000000in}}%
\pgfpathclose%
\pgfusepath{fill}%
\end{pgfscope}%
\begin{pgfscope}%
\pgfsetbuttcap%
\pgfsetmiterjoin%
\definecolor{currentfill}{rgb}{1.000000,1.000000,1.000000}%
\pgfsetfillcolor{currentfill}%
\pgfsetlinewidth{0.000000pt}%
\definecolor{currentstroke}{rgb}{0.000000,0.000000,0.000000}%
\pgfsetstrokecolor{currentstroke}%
\pgfsetstrokeopacity{0.000000}%
\pgfsetdash{}{0pt}%
\pgfpathmoveto{\pgfqpoint{0.750000in}{0.500000in}}%
\pgfpathlineto{\pgfqpoint{5.400000in}{0.500000in}}%
\pgfpathlineto{\pgfqpoint{5.400000in}{3.520000in}}%
\pgfpathlineto{\pgfqpoint{0.750000in}{3.520000in}}%
\pgfpathclose%
\pgfusepath{fill}%
\end{pgfscope}%
\begin{pgfscope}%
\pgfpathrectangle{\pgfqpoint{0.750000in}{0.500000in}}{\pgfqpoint{4.650000in}{3.020000in}}%
\pgfusepath{clip}%
\pgfsetrectcap%
\pgfsetroundjoin%
\pgfsetlinewidth{0.803000pt}%
\definecolor{currentstroke}{rgb}{0.690196,0.690196,0.690196}%
\pgfsetstrokecolor{currentstroke}%
\pgfsetdash{}{0pt}%
\pgfpathmoveto{\pgfqpoint{0.750000in}{0.500000in}}%
\pgfpathlineto{\pgfqpoint{0.750000in}{3.520000in}}%
\pgfusepath{stroke}%
\end{pgfscope}%
\begin{pgfscope}%
\pgfsetbuttcap%
\pgfsetroundjoin%
\definecolor{currentfill}{rgb}{0.000000,0.000000,0.000000}%
\pgfsetfillcolor{currentfill}%
\pgfsetlinewidth{0.803000pt}%
\definecolor{currentstroke}{rgb}{0.000000,0.000000,0.000000}%
\pgfsetstrokecolor{currentstroke}%
\pgfsetdash{}{0pt}%
\pgfsys@defobject{currentmarker}{\pgfqpoint{0.000000in}{-0.048611in}}{\pgfqpoint{0.000000in}{0.000000in}}{%
\pgfpathmoveto{\pgfqpoint{0.000000in}{0.000000in}}%
\pgfpathlineto{\pgfqpoint{0.000000in}{-0.048611in}}%
\pgfusepath{stroke,fill}%
}%
\begin{pgfscope}%
\pgfsys@transformshift{0.750000in}{0.500000in}%
\pgfsys@useobject{currentmarker}{}%
\end{pgfscope}%
\end{pgfscope}%
\begin{pgfscope}%
\pgftext[x=0.750000in,y=0.402778in,,top]{\rmfamily\fontsize{10.000000}{12.000000}\selectfont \(\displaystyle -6\)}%
\end{pgfscope}%
\begin{pgfscope}%
\pgfpathrectangle{\pgfqpoint{0.750000in}{0.500000in}}{\pgfqpoint{4.650000in}{3.020000in}}%
\pgfusepath{clip}%
\pgfsetrectcap%
\pgfsetroundjoin%
\pgfsetlinewidth{0.803000pt}%
\definecolor{currentstroke}{rgb}{0.690196,0.690196,0.690196}%
\pgfsetstrokecolor{currentstroke}%
\pgfsetdash{}{0pt}%
\pgfpathmoveto{\pgfqpoint{1.525000in}{0.500000in}}%
\pgfpathlineto{\pgfqpoint{1.525000in}{3.520000in}}%
\pgfusepath{stroke}%
\end{pgfscope}%
\begin{pgfscope}%
\pgfsetbuttcap%
\pgfsetroundjoin%
\definecolor{currentfill}{rgb}{0.000000,0.000000,0.000000}%
\pgfsetfillcolor{currentfill}%
\pgfsetlinewidth{0.803000pt}%
\definecolor{currentstroke}{rgb}{0.000000,0.000000,0.000000}%
\pgfsetstrokecolor{currentstroke}%
\pgfsetdash{}{0pt}%
\pgfsys@defobject{currentmarker}{\pgfqpoint{0.000000in}{-0.048611in}}{\pgfqpoint{0.000000in}{0.000000in}}{%
\pgfpathmoveto{\pgfqpoint{0.000000in}{0.000000in}}%
\pgfpathlineto{\pgfqpoint{0.000000in}{-0.048611in}}%
\pgfusepath{stroke,fill}%
}%
\begin{pgfscope}%
\pgfsys@transformshift{1.525000in}{0.500000in}%
\pgfsys@useobject{currentmarker}{}%
\end{pgfscope}%
\end{pgfscope}%
\begin{pgfscope}%
\pgftext[x=1.525000in,y=0.402778in,,top]{\rmfamily\fontsize{10.000000}{12.000000}\selectfont \(\displaystyle -4\)}%
\end{pgfscope}%
\begin{pgfscope}%
\pgfpathrectangle{\pgfqpoint{0.750000in}{0.500000in}}{\pgfqpoint{4.650000in}{3.020000in}}%
\pgfusepath{clip}%
\pgfsetrectcap%
\pgfsetroundjoin%
\pgfsetlinewidth{0.803000pt}%
\definecolor{currentstroke}{rgb}{0.690196,0.690196,0.690196}%
\pgfsetstrokecolor{currentstroke}%
\pgfsetdash{}{0pt}%
\pgfpathmoveto{\pgfqpoint{2.300000in}{0.500000in}}%
\pgfpathlineto{\pgfqpoint{2.300000in}{3.520000in}}%
\pgfusepath{stroke}%
\end{pgfscope}%
\begin{pgfscope}%
\pgfsetbuttcap%
\pgfsetroundjoin%
\definecolor{currentfill}{rgb}{0.000000,0.000000,0.000000}%
\pgfsetfillcolor{currentfill}%
\pgfsetlinewidth{0.803000pt}%
\definecolor{currentstroke}{rgb}{0.000000,0.000000,0.000000}%
\pgfsetstrokecolor{currentstroke}%
\pgfsetdash{}{0pt}%
\pgfsys@defobject{currentmarker}{\pgfqpoint{0.000000in}{-0.048611in}}{\pgfqpoint{0.000000in}{0.000000in}}{%
\pgfpathmoveto{\pgfqpoint{0.000000in}{0.000000in}}%
\pgfpathlineto{\pgfqpoint{0.000000in}{-0.048611in}}%
\pgfusepath{stroke,fill}%
}%
\begin{pgfscope}%
\pgfsys@transformshift{2.300000in}{0.500000in}%
\pgfsys@useobject{currentmarker}{}%
\end{pgfscope}%
\end{pgfscope}%
\begin{pgfscope}%
\pgftext[x=2.300000in,y=0.402778in,,top]{\rmfamily\fontsize{10.000000}{12.000000}\selectfont \(\displaystyle -2\)}%
\end{pgfscope}%
\begin{pgfscope}%
\pgfpathrectangle{\pgfqpoint{0.750000in}{0.500000in}}{\pgfqpoint{4.650000in}{3.020000in}}%
\pgfusepath{clip}%
\pgfsetrectcap%
\pgfsetroundjoin%
\pgfsetlinewidth{0.803000pt}%
\definecolor{currentstroke}{rgb}{0.690196,0.690196,0.690196}%
\pgfsetstrokecolor{currentstroke}%
\pgfsetdash{}{0pt}%
\pgfpathmoveto{\pgfqpoint{3.075000in}{0.500000in}}%
\pgfpathlineto{\pgfqpoint{3.075000in}{3.520000in}}%
\pgfusepath{stroke}%
\end{pgfscope}%
\begin{pgfscope}%
\pgfsetbuttcap%
\pgfsetroundjoin%
\definecolor{currentfill}{rgb}{0.000000,0.000000,0.000000}%
\pgfsetfillcolor{currentfill}%
\pgfsetlinewidth{0.803000pt}%
\definecolor{currentstroke}{rgb}{0.000000,0.000000,0.000000}%
\pgfsetstrokecolor{currentstroke}%
\pgfsetdash{}{0pt}%
\pgfsys@defobject{currentmarker}{\pgfqpoint{0.000000in}{-0.048611in}}{\pgfqpoint{0.000000in}{0.000000in}}{%
\pgfpathmoveto{\pgfqpoint{0.000000in}{0.000000in}}%
\pgfpathlineto{\pgfqpoint{0.000000in}{-0.048611in}}%
\pgfusepath{stroke,fill}%
}%
\begin{pgfscope}%
\pgfsys@transformshift{3.075000in}{0.500000in}%
\pgfsys@useobject{currentmarker}{}%
\end{pgfscope}%
\end{pgfscope}%
\begin{pgfscope}%
\pgftext[x=3.075000in,y=0.402778in,,top]{\rmfamily\fontsize{10.000000}{12.000000}\selectfont \(\displaystyle 0\)}%
\end{pgfscope}%
\begin{pgfscope}%
\pgfpathrectangle{\pgfqpoint{0.750000in}{0.500000in}}{\pgfqpoint{4.650000in}{3.020000in}}%
\pgfusepath{clip}%
\pgfsetrectcap%
\pgfsetroundjoin%
\pgfsetlinewidth{0.803000pt}%
\definecolor{currentstroke}{rgb}{0.690196,0.690196,0.690196}%
\pgfsetstrokecolor{currentstroke}%
\pgfsetdash{}{0pt}%
\pgfpathmoveto{\pgfqpoint{3.850000in}{0.500000in}}%
\pgfpathlineto{\pgfqpoint{3.850000in}{3.520000in}}%
\pgfusepath{stroke}%
\end{pgfscope}%
\begin{pgfscope}%
\pgfsetbuttcap%
\pgfsetroundjoin%
\definecolor{currentfill}{rgb}{0.000000,0.000000,0.000000}%
\pgfsetfillcolor{currentfill}%
\pgfsetlinewidth{0.803000pt}%
\definecolor{currentstroke}{rgb}{0.000000,0.000000,0.000000}%
\pgfsetstrokecolor{currentstroke}%
\pgfsetdash{}{0pt}%
\pgfsys@defobject{currentmarker}{\pgfqpoint{0.000000in}{-0.048611in}}{\pgfqpoint{0.000000in}{0.000000in}}{%
\pgfpathmoveto{\pgfqpoint{0.000000in}{0.000000in}}%
\pgfpathlineto{\pgfqpoint{0.000000in}{-0.048611in}}%
\pgfusepath{stroke,fill}%
}%
\begin{pgfscope}%
\pgfsys@transformshift{3.850000in}{0.500000in}%
\pgfsys@useobject{currentmarker}{}%
\end{pgfscope}%
\end{pgfscope}%
\begin{pgfscope}%
\pgftext[x=3.850000in,y=0.402778in,,top]{\rmfamily\fontsize{10.000000}{12.000000}\selectfont \(\displaystyle 2\)}%
\end{pgfscope}%
\begin{pgfscope}%
\pgfpathrectangle{\pgfqpoint{0.750000in}{0.500000in}}{\pgfqpoint{4.650000in}{3.020000in}}%
\pgfusepath{clip}%
\pgfsetrectcap%
\pgfsetroundjoin%
\pgfsetlinewidth{0.803000pt}%
\definecolor{currentstroke}{rgb}{0.690196,0.690196,0.690196}%
\pgfsetstrokecolor{currentstroke}%
\pgfsetdash{}{0pt}%
\pgfpathmoveto{\pgfqpoint{4.625000in}{0.500000in}}%
\pgfpathlineto{\pgfqpoint{4.625000in}{3.520000in}}%
\pgfusepath{stroke}%
\end{pgfscope}%
\begin{pgfscope}%
\pgfsetbuttcap%
\pgfsetroundjoin%
\definecolor{currentfill}{rgb}{0.000000,0.000000,0.000000}%
\pgfsetfillcolor{currentfill}%
\pgfsetlinewidth{0.803000pt}%
\definecolor{currentstroke}{rgb}{0.000000,0.000000,0.000000}%
\pgfsetstrokecolor{currentstroke}%
\pgfsetdash{}{0pt}%
\pgfsys@defobject{currentmarker}{\pgfqpoint{0.000000in}{-0.048611in}}{\pgfqpoint{0.000000in}{0.000000in}}{%
\pgfpathmoveto{\pgfqpoint{0.000000in}{0.000000in}}%
\pgfpathlineto{\pgfqpoint{0.000000in}{-0.048611in}}%
\pgfusepath{stroke,fill}%
}%
\begin{pgfscope}%
\pgfsys@transformshift{4.625000in}{0.500000in}%
\pgfsys@useobject{currentmarker}{}%
\end{pgfscope}%
\end{pgfscope}%
\begin{pgfscope}%
\pgftext[x=4.625000in,y=0.402778in,,top]{\rmfamily\fontsize{10.000000}{12.000000}\selectfont \(\displaystyle 4\)}%
\end{pgfscope}%
\begin{pgfscope}%
\pgfpathrectangle{\pgfqpoint{0.750000in}{0.500000in}}{\pgfqpoint{4.650000in}{3.020000in}}%
\pgfusepath{clip}%
\pgfsetrectcap%
\pgfsetroundjoin%
\pgfsetlinewidth{0.803000pt}%
\definecolor{currentstroke}{rgb}{0.690196,0.690196,0.690196}%
\pgfsetstrokecolor{currentstroke}%
\pgfsetdash{}{0pt}%
\pgfpathmoveto{\pgfqpoint{5.400000in}{0.500000in}}%
\pgfpathlineto{\pgfqpoint{5.400000in}{3.520000in}}%
\pgfusepath{stroke}%
\end{pgfscope}%
\begin{pgfscope}%
\pgfsetbuttcap%
\pgfsetroundjoin%
\definecolor{currentfill}{rgb}{0.000000,0.000000,0.000000}%
\pgfsetfillcolor{currentfill}%
\pgfsetlinewidth{0.803000pt}%
\definecolor{currentstroke}{rgb}{0.000000,0.000000,0.000000}%
\pgfsetstrokecolor{currentstroke}%
\pgfsetdash{}{0pt}%
\pgfsys@defobject{currentmarker}{\pgfqpoint{0.000000in}{-0.048611in}}{\pgfqpoint{0.000000in}{0.000000in}}{%
\pgfpathmoveto{\pgfqpoint{0.000000in}{0.000000in}}%
\pgfpathlineto{\pgfqpoint{0.000000in}{-0.048611in}}%
\pgfusepath{stroke,fill}%
}%
\begin{pgfscope}%
\pgfsys@transformshift{5.400000in}{0.500000in}%
\pgfsys@useobject{currentmarker}{}%
\end{pgfscope}%
\end{pgfscope}%
\begin{pgfscope}%
\pgftext[x=5.400000in,y=0.402778in,,top]{\rmfamily\fontsize{10.000000}{12.000000}\selectfont \(\displaystyle 6\)}%
\end{pgfscope}%
\begin{pgfscope}%
\pgfpathrectangle{\pgfqpoint{0.750000in}{0.500000in}}{\pgfqpoint{4.650000in}{3.020000in}}%
\pgfusepath{clip}%
\pgfsetrectcap%
\pgfsetroundjoin%
\pgfsetlinewidth{0.803000pt}%
\definecolor{currentstroke}{rgb}{0.690196,0.690196,0.690196}%
\pgfsetstrokecolor{currentstroke}%
\pgfsetdash{}{0pt}%
\pgfpathmoveto{\pgfqpoint{0.750000in}{0.500000in}}%
\pgfpathlineto{\pgfqpoint{5.400000in}{0.500000in}}%
\pgfusepath{stroke}%
\end{pgfscope}%
\begin{pgfscope}%
\pgfsetbuttcap%
\pgfsetroundjoin%
\definecolor{currentfill}{rgb}{0.000000,0.000000,0.000000}%
\pgfsetfillcolor{currentfill}%
\pgfsetlinewidth{0.803000pt}%
\definecolor{currentstroke}{rgb}{0.000000,0.000000,0.000000}%
\pgfsetstrokecolor{currentstroke}%
\pgfsetdash{}{0pt}%
\pgfsys@defobject{currentmarker}{\pgfqpoint{-0.048611in}{0.000000in}}{\pgfqpoint{0.000000in}{0.000000in}}{%
\pgfpathmoveto{\pgfqpoint{0.000000in}{0.000000in}}%
\pgfpathlineto{\pgfqpoint{-0.048611in}{0.000000in}}%
\pgfusepath{stroke,fill}%
}%
\begin{pgfscope}%
\pgfsys@transformshift{0.750000in}{0.500000in}%
\pgfsys@useobject{currentmarker}{}%
\end{pgfscope}%
\end{pgfscope}%
\begin{pgfscope}%
\pgftext[x=0.297838in,y=0.451806in,left,base]{\rmfamily\fontsize{10.000000}{12.000000}\selectfont \(\displaystyle -0.50\)}%
\end{pgfscope}%
\begin{pgfscope}%
\pgfpathrectangle{\pgfqpoint{0.750000in}{0.500000in}}{\pgfqpoint{4.650000in}{3.020000in}}%
\pgfusepath{clip}%
\pgfsetrectcap%
\pgfsetroundjoin%
\pgfsetlinewidth{0.803000pt}%
\definecolor{currentstroke}{rgb}{0.690196,0.690196,0.690196}%
\pgfsetstrokecolor{currentstroke}%
\pgfsetdash{}{0pt}%
\pgfpathmoveto{\pgfqpoint{0.750000in}{0.877500in}}%
\pgfpathlineto{\pgfqpoint{5.400000in}{0.877500in}}%
\pgfusepath{stroke}%
\end{pgfscope}%
\begin{pgfscope}%
\pgfsetbuttcap%
\pgfsetroundjoin%
\definecolor{currentfill}{rgb}{0.000000,0.000000,0.000000}%
\pgfsetfillcolor{currentfill}%
\pgfsetlinewidth{0.803000pt}%
\definecolor{currentstroke}{rgb}{0.000000,0.000000,0.000000}%
\pgfsetstrokecolor{currentstroke}%
\pgfsetdash{}{0pt}%
\pgfsys@defobject{currentmarker}{\pgfqpoint{-0.048611in}{0.000000in}}{\pgfqpoint{0.000000in}{0.000000in}}{%
\pgfpathmoveto{\pgfqpoint{0.000000in}{0.000000in}}%
\pgfpathlineto{\pgfqpoint{-0.048611in}{0.000000in}}%
\pgfusepath{stroke,fill}%
}%
\begin{pgfscope}%
\pgfsys@transformshift{0.750000in}{0.877500in}%
\pgfsys@useobject{currentmarker}{}%
\end{pgfscope}%
\end{pgfscope}%
\begin{pgfscope}%
\pgftext[x=0.297838in,y=0.829306in,left,base]{\rmfamily\fontsize{10.000000}{12.000000}\selectfont \(\displaystyle -0.25\)}%
\end{pgfscope}%
\begin{pgfscope}%
\pgfpathrectangle{\pgfqpoint{0.750000in}{0.500000in}}{\pgfqpoint{4.650000in}{3.020000in}}%
\pgfusepath{clip}%
\pgfsetrectcap%
\pgfsetroundjoin%
\pgfsetlinewidth{0.803000pt}%
\definecolor{currentstroke}{rgb}{0.690196,0.690196,0.690196}%
\pgfsetstrokecolor{currentstroke}%
\pgfsetdash{}{0pt}%
\pgfpathmoveto{\pgfqpoint{0.750000in}{1.255000in}}%
\pgfpathlineto{\pgfqpoint{5.400000in}{1.255000in}}%
\pgfusepath{stroke}%
\end{pgfscope}%
\begin{pgfscope}%
\pgfsetbuttcap%
\pgfsetroundjoin%
\definecolor{currentfill}{rgb}{0.000000,0.000000,0.000000}%
\pgfsetfillcolor{currentfill}%
\pgfsetlinewidth{0.803000pt}%
\definecolor{currentstroke}{rgb}{0.000000,0.000000,0.000000}%
\pgfsetstrokecolor{currentstroke}%
\pgfsetdash{}{0pt}%
\pgfsys@defobject{currentmarker}{\pgfqpoint{-0.048611in}{0.000000in}}{\pgfqpoint{0.000000in}{0.000000in}}{%
\pgfpathmoveto{\pgfqpoint{0.000000in}{0.000000in}}%
\pgfpathlineto{\pgfqpoint{-0.048611in}{0.000000in}}%
\pgfusepath{stroke,fill}%
}%
\begin{pgfscope}%
\pgfsys@transformshift{0.750000in}{1.255000in}%
\pgfsys@useobject{currentmarker}{}%
\end{pgfscope}%
\end{pgfscope}%
\begin{pgfscope}%
\pgftext[x=0.405863in,y=1.206806in,left,base]{\rmfamily\fontsize{10.000000}{12.000000}\selectfont \(\displaystyle 0.00\)}%
\end{pgfscope}%
\begin{pgfscope}%
\pgfpathrectangle{\pgfqpoint{0.750000in}{0.500000in}}{\pgfqpoint{4.650000in}{3.020000in}}%
\pgfusepath{clip}%
\pgfsetrectcap%
\pgfsetroundjoin%
\pgfsetlinewidth{0.803000pt}%
\definecolor{currentstroke}{rgb}{0.690196,0.690196,0.690196}%
\pgfsetstrokecolor{currentstroke}%
\pgfsetdash{}{0pt}%
\pgfpathmoveto{\pgfqpoint{0.750000in}{1.632500in}}%
\pgfpathlineto{\pgfqpoint{5.400000in}{1.632500in}}%
\pgfusepath{stroke}%
\end{pgfscope}%
\begin{pgfscope}%
\pgfsetbuttcap%
\pgfsetroundjoin%
\definecolor{currentfill}{rgb}{0.000000,0.000000,0.000000}%
\pgfsetfillcolor{currentfill}%
\pgfsetlinewidth{0.803000pt}%
\definecolor{currentstroke}{rgb}{0.000000,0.000000,0.000000}%
\pgfsetstrokecolor{currentstroke}%
\pgfsetdash{}{0pt}%
\pgfsys@defobject{currentmarker}{\pgfqpoint{-0.048611in}{0.000000in}}{\pgfqpoint{0.000000in}{0.000000in}}{%
\pgfpathmoveto{\pgfqpoint{0.000000in}{0.000000in}}%
\pgfpathlineto{\pgfqpoint{-0.048611in}{0.000000in}}%
\pgfusepath{stroke,fill}%
}%
\begin{pgfscope}%
\pgfsys@transformshift{0.750000in}{1.632500in}%
\pgfsys@useobject{currentmarker}{}%
\end{pgfscope}%
\end{pgfscope}%
\begin{pgfscope}%
\pgftext[x=0.405863in,y=1.584306in,left,base]{\rmfamily\fontsize{10.000000}{12.000000}\selectfont \(\displaystyle 0.25\)}%
\end{pgfscope}%
\begin{pgfscope}%
\pgfpathrectangle{\pgfqpoint{0.750000in}{0.500000in}}{\pgfqpoint{4.650000in}{3.020000in}}%
\pgfusepath{clip}%
\pgfsetrectcap%
\pgfsetroundjoin%
\pgfsetlinewidth{0.803000pt}%
\definecolor{currentstroke}{rgb}{0.690196,0.690196,0.690196}%
\pgfsetstrokecolor{currentstroke}%
\pgfsetdash{}{0pt}%
\pgfpathmoveto{\pgfqpoint{0.750000in}{2.010000in}}%
\pgfpathlineto{\pgfqpoint{5.400000in}{2.010000in}}%
\pgfusepath{stroke}%
\end{pgfscope}%
\begin{pgfscope}%
\pgfsetbuttcap%
\pgfsetroundjoin%
\definecolor{currentfill}{rgb}{0.000000,0.000000,0.000000}%
\pgfsetfillcolor{currentfill}%
\pgfsetlinewidth{0.803000pt}%
\definecolor{currentstroke}{rgb}{0.000000,0.000000,0.000000}%
\pgfsetstrokecolor{currentstroke}%
\pgfsetdash{}{0pt}%
\pgfsys@defobject{currentmarker}{\pgfqpoint{-0.048611in}{0.000000in}}{\pgfqpoint{0.000000in}{0.000000in}}{%
\pgfpathmoveto{\pgfqpoint{0.000000in}{0.000000in}}%
\pgfpathlineto{\pgfqpoint{-0.048611in}{0.000000in}}%
\pgfusepath{stroke,fill}%
}%
\begin{pgfscope}%
\pgfsys@transformshift{0.750000in}{2.010000in}%
\pgfsys@useobject{currentmarker}{}%
\end{pgfscope}%
\end{pgfscope}%
\begin{pgfscope}%
\pgftext[x=0.405863in,y=1.961806in,left,base]{\rmfamily\fontsize{10.000000}{12.000000}\selectfont \(\displaystyle 0.50\)}%
\end{pgfscope}%
\begin{pgfscope}%
\pgfpathrectangle{\pgfqpoint{0.750000in}{0.500000in}}{\pgfqpoint{4.650000in}{3.020000in}}%
\pgfusepath{clip}%
\pgfsetrectcap%
\pgfsetroundjoin%
\pgfsetlinewidth{0.803000pt}%
\definecolor{currentstroke}{rgb}{0.690196,0.690196,0.690196}%
\pgfsetstrokecolor{currentstroke}%
\pgfsetdash{}{0pt}%
\pgfpathmoveto{\pgfqpoint{0.750000in}{2.387500in}}%
\pgfpathlineto{\pgfqpoint{5.400000in}{2.387500in}}%
\pgfusepath{stroke}%
\end{pgfscope}%
\begin{pgfscope}%
\pgfsetbuttcap%
\pgfsetroundjoin%
\definecolor{currentfill}{rgb}{0.000000,0.000000,0.000000}%
\pgfsetfillcolor{currentfill}%
\pgfsetlinewidth{0.803000pt}%
\definecolor{currentstroke}{rgb}{0.000000,0.000000,0.000000}%
\pgfsetstrokecolor{currentstroke}%
\pgfsetdash{}{0pt}%
\pgfsys@defobject{currentmarker}{\pgfqpoint{-0.048611in}{0.000000in}}{\pgfqpoint{0.000000in}{0.000000in}}{%
\pgfpathmoveto{\pgfqpoint{0.000000in}{0.000000in}}%
\pgfpathlineto{\pgfqpoint{-0.048611in}{0.000000in}}%
\pgfusepath{stroke,fill}%
}%
\begin{pgfscope}%
\pgfsys@transformshift{0.750000in}{2.387500in}%
\pgfsys@useobject{currentmarker}{}%
\end{pgfscope}%
\end{pgfscope}%
\begin{pgfscope}%
\pgftext[x=0.405863in,y=2.339306in,left,base]{\rmfamily\fontsize{10.000000}{12.000000}\selectfont \(\displaystyle 0.75\)}%
\end{pgfscope}%
\begin{pgfscope}%
\pgfpathrectangle{\pgfqpoint{0.750000in}{0.500000in}}{\pgfqpoint{4.650000in}{3.020000in}}%
\pgfusepath{clip}%
\pgfsetrectcap%
\pgfsetroundjoin%
\pgfsetlinewidth{0.803000pt}%
\definecolor{currentstroke}{rgb}{0.690196,0.690196,0.690196}%
\pgfsetstrokecolor{currentstroke}%
\pgfsetdash{}{0pt}%
\pgfpathmoveto{\pgfqpoint{0.750000in}{2.765000in}}%
\pgfpathlineto{\pgfqpoint{5.400000in}{2.765000in}}%
\pgfusepath{stroke}%
\end{pgfscope}%
\begin{pgfscope}%
\pgfsetbuttcap%
\pgfsetroundjoin%
\definecolor{currentfill}{rgb}{0.000000,0.000000,0.000000}%
\pgfsetfillcolor{currentfill}%
\pgfsetlinewidth{0.803000pt}%
\definecolor{currentstroke}{rgb}{0.000000,0.000000,0.000000}%
\pgfsetstrokecolor{currentstroke}%
\pgfsetdash{}{0pt}%
\pgfsys@defobject{currentmarker}{\pgfqpoint{-0.048611in}{0.000000in}}{\pgfqpoint{0.000000in}{0.000000in}}{%
\pgfpathmoveto{\pgfqpoint{0.000000in}{0.000000in}}%
\pgfpathlineto{\pgfqpoint{-0.048611in}{0.000000in}}%
\pgfusepath{stroke,fill}%
}%
\begin{pgfscope}%
\pgfsys@transformshift{0.750000in}{2.765000in}%
\pgfsys@useobject{currentmarker}{}%
\end{pgfscope}%
\end{pgfscope}%
\begin{pgfscope}%
\pgftext[x=0.405863in,y=2.716806in,left,base]{\rmfamily\fontsize{10.000000}{12.000000}\selectfont \(\displaystyle 1.00\)}%
\end{pgfscope}%
\begin{pgfscope}%
\pgfpathrectangle{\pgfqpoint{0.750000in}{0.500000in}}{\pgfqpoint{4.650000in}{3.020000in}}%
\pgfusepath{clip}%
\pgfsetrectcap%
\pgfsetroundjoin%
\pgfsetlinewidth{0.803000pt}%
\definecolor{currentstroke}{rgb}{0.690196,0.690196,0.690196}%
\pgfsetstrokecolor{currentstroke}%
\pgfsetdash{}{0pt}%
\pgfpathmoveto{\pgfqpoint{0.750000in}{3.142500in}}%
\pgfpathlineto{\pgfqpoint{5.400000in}{3.142500in}}%
\pgfusepath{stroke}%
\end{pgfscope}%
\begin{pgfscope}%
\pgfsetbuttcap%
\pgfsetroundjoin%
\definecolor{currentfill}{rgb}{0.000000,0.000000,0.000000}%
\pgfsetfillcolor{currentfill}%
\pgfsetlinewidth{0.803000pt}%
\definecolor{currentstroke}{rgb}{0.000000,0.000000,0.000000}%
\pgfsetstrokecolor{currentstroke}%
\pgfsetdash{}{0pt}%
\pgfsys@defobject{currentmarker}{\pgfqpoint{-0.048611in}{0.000000in}}{\pgfqpoint{0.000000in}{0.000000in}}{%
\pgfpathmoveto{\pgfqpoint{0.000000in}{0.000000in}}%
\pgfpathlineto{\pgfqpoint{-0.048611in}{0.000000in}}%
\pgfusepath{stroke,fill}%
}%
\begin{pgfscope}%
\pgfsys@transformshift{0.750000in}{3.142500in}%
\pgfsys@useobject{currentmarker}{}%
\end{pgfscope}%
\end{pgfscope}%
\begin{pgfscope}%
\pgftext[x=0.405863in,y=3.094306in,left,base]{\rmfamily\fontsize{10.000000}{12.000000}\selectfont \(\displaystyle 1.25\)}%
\end{pgfscope}%
\begin{pgfscope}%
\pgfpathrectangle{\pgfqpoint{0.750000in}{0.500000in}}{\pgfqpoint{4.650000in}{3.020000in}}%
\pgfusepath{clip}%
\pgfsetrectcap%
\pgfsetroundjoin%
\pgfsetlinewidth{0.803000pt}%
\definecolor{currentstroke}{rgb}{0.690196,0.690196,0.690196}%
\pgfsetstrokecolor{currentstroke}%
\pgfsetdash{}{0pt}%
\pgfpathmoveto{\pgfqpoint{0.750000in}{3.520000in}}%
\pgfpathlineto{\pgfqpoint{5.400000in}{3.520000in}}%
\pgfusepath{stroke}%
\end{pgfscope}%
\begin{pgfscope}%
\pgfsetbuttcap%
\pgfsetroundjoin%
\definecolor{currentfill}{rgb}{0.000000,0.000000,0.000000}%
\pgfsetfillcolor{currentfill}%
\pgfsetlinewidth{0.803000pt}%
\definecolor{currentstroke}{rgb}{0.000000,0.000000,0.000000}%
\pgfsetstrokecolor{currentstroke}%
\pgfsetdash{}{0pt}%
\pgfsys@defobject{currentmarker}{\pgfqpoint{-0.048611in}{0.000000in}}{\pgfqpoint{0.000000in}{0.000000in}}{%
\pgfpathmoveto{\pgfqpoint{0.000000in}{0.000000in}}%
\pgfpathlineto{\pgfqpoint{-0.048611in}{0.000000in}}%
\pgfusepath{stroke,fill}%
}%
\begin{pgfscope}%
\pgfsys@transformshift{0.750000in}{3.520000in}%
\pgfsys@useobject{currentmarker}{}%
\end{pgfscope}%
\end{pgfscope}%
\begin{pgfscope}%
\pgftext[x=0.405863in,y=3.471806in,left,base]{\rmfamily\fontsize{10.000000}{12.000000}\selectfont \(\displaystyle 1.50\)}%
\end{pgfscope}%
\begin{pgfscope}%
\pgfpathrectangle{\pgfqpoint{0.750000in}{0.500000in}}{\pgfqpoint{4.650000in}{3.020000in}}%
\pgfusepath{clip}%
\pgfsetrectcap%
\pgfsetroundjoin%
\pgfsetlinewidth{1.505625pt}%
\definecolor{currentstroke}{rgb}{0.121569,0.466667,0.705882}%
\pgfsetstrokecolor{currentstroke}%
\pgfsetdash{}{0pt}%
\pgfpathmoveto{\pgfqpoint{0.750000in}{1.255000in}}%
\pgfpathlineto{\pgfqpoint{5.400000in}{1.255000in}}%
\pgfpathlineto{\pgfqpoint{5.400000in}{1.255000in}}%
\pgfusepath{stroke}%
\end{pgfscope}%
\begin{pgfscope}%
\pgfpathrectangle{\pgfqpoint{0.750000in}{0.500000in}}{\pgfqpoint{4.650000in}{3.020000in}}%
\pgfusepath{clip}%
\pgfsetrectcap%
\pgfsetroundjoin%
\pgfsetlinewidth{1.505625pt}%
\definecolor{currentstroke}{rgb}{1.000000,0.498039,0.054902}%
\pgfsetstrokecolor{currentstroke}%
\pgfsetdash{}{0pt}%
\pgfpathmoveto{\pgfqpoint{0.750000in}{1.208526in}}%
\pgfpathlineto{\pgfqpoint{0.987387in}{1.237999in}}%
\pgfpathlineto{\pgfqpoint{1.220120in}{1.263816in}}%
\pgfpathlineto{\pgfqpoint{1.452853in}{1.286585in}}%
\pgfpathlineto{\pgfqpoint{1.685586in}{1.306305in}}%
\pgfpathlineto{\pgfqpoint{1.918318in}{1.322978in}}%
\pgfpathlineto{\pgfqpoint{2.151051in}{1.336603in}}%
\pgfpathlineto{\pgfqpoint{2.383784in}{1.347179in}}%
\pgfpathlineto{\pgfqpoint{2.616517in}{1.354708in}}%
\pgfpathlineto{\pgfqpoint{2.849249in}{1.359188in}}%
\pgfpathlineto{\pgfqpoint{3.081982in}{1.360621in}}%
\pgfpathlineto{\pgfqpoint{3.314715in}{1.359006in}}%
\pgfpathlineto{\pgfqpoint{3.547447in}{1.354342in}}%
\pgfpathlineto{\pgfqpoint{3.780180in}{1.346631in}}%
\pgfpathlineto{\pgfqpoint{4.012913in}{1.335871in}}%
\pgfpathlineto{\pgfqpoint{4.245646in}{1.322064in}}%
\pgfpathlineto{\pgfqpoint{4.478378in}{1.305208in}}%
\pgfpathlineto{\pgfqpoint{4.711111in}{1.285304in}}%
\pgfpathlineto{\pgfqpoint{4.943844in}{1.262353in}}%
\pgfpathlineto{\pgfqpoint{5.176577in}{1.236353in}}%
\pgfpathlineto{\pgfqpoint{5.400000in}{1.208526in}}%
\pgfpathlineto{\pgfqpoint{5.400000in}{1.208526in}}%
\pgfusepath{stroke}%
\end{pgfscope}%
\begin{pgfscope}%
\pgfpathrectangle{\pgfqpoint{0.750000in}{0.500000in}}{\pgfqpoint{4.650000in}{3.020000in}}%
\pgfusepath{clip}%
\pgfsetrectcap%
\pgfsetroundjoin%
\pgfsetlinewidth{1.505625pt}%
\definecolor{currentstroke}{rgb}{0.172549,0.627451,0.172549}%
\pgfsetstrokecolor{currentstroke}%
\pgfsetdash{}{0pt}%
\pgfpathmoveto{\pgfqpoint{0.750000in}{2.113725in}}%
\pgfpathlineto{\pgfqpoint{0.791892in}{1.977248in}}%
\pgfpathlineto{\pgfqpoint{0.829129in}{1.865841in}}%
\pgfpathlineto{\pgfqpoint{0.866366in}{1.763383in}}%
\pgfpathlineto{\pgfqpoint{0.903604in}{1.669520in}}%
\pgfpathlineto{\pgfqpoint{0.940841in}{1.583903in}}%
\pgfpathlineto{\pgfqpoint{0.978078in}{1.506190in}}%
\pgfpathlineto{\pgfqpoint{1.015315in}{1.436046in}}%
\pgfpathlineto{\pgfqpoint{1.052553in}{1.373139in}}%
\pgfpathlineto{\pgfqpoint{1.085135in}{1.323778in}}%
\pgfpathlineto{\pgfqpoint{1.117718in}{1.279496in}}%
\pgfpathlineto{\pgfqpoint{1.150300in}{1.240084in}}%
\pgfpathlineto{\pgfqpoint{1.182883in}{1.205334in}}%
\pgfpathlineto{\pgfqpoint{1.215465in}{1.175044in}}%
\pgfpathlineto{\pgfqpoint{1.248048in}{1.149013in}}%
\pgfpathlineto{\pgfqpoint{1.280631in}{1.127046in}}%
\pgfpathlineto{\pgfqpoint{1.313213in}{1.108950in}}%
\pgfpathlineto{\pgfqpoint{1.345796in}{1.094534in}}%
\pgfpathlineto{\pgfqpoint{1.378378in}{1.083615in}}%
\pgfpathlineto{\pgfqpoint{1.410961in}{1.076008in}}%
\pgfpathlineto{\pgfqpoint{1.443544in}{1.071536in}}%
\pgfpathlineto{\pgfqpoint{1.476126in}{1.070022in}}%
\pgfpathlineto{\pgfqpoint{1.508709in}{1.071295in}}%
\pgfpathlineto{\pgfqpoint{1.545946in}{1.075946in}}%
\pgfpathlineto{\pgfqpoint{1.583183in}{1.083772in}}%
\pgfpathlineto{\pgfqpoint{1.620420in}{1.094531in}}%
\pgfpathlineto{\pgfqpoint{1.662312in}{1.109850in}}%
\pgfpathlineto{\pgfqpoint{1.708859in}{1.130485in}}%
\pgfpathlineto{\pgfqpoint{1.755405in}{1.154508in}}%
\pgfpathlineto{\pgfqpoint{1.806607in}{1.184347in}}%
\pgfpathlineto{\pgfqpoint{1.862462in}{1.220370in}}%
\pgfpathlineto{\pgfqpoint{1.927628in}{1.266113in}}%
\pgfpathlineto{\pgfqpoint{2.006757in}{1.325662in}}%
\pgfpathlineto{\pgfqpoint{2.118468in}{1.414138in}}%
\pgfpathlineto{\pgfqpoint{2.323273in}{1.576836in}}%
\pgfpathlineto{\pgfqpoint{2.411712in}{1.642655in}}%
\pgfpathlineto{\pgfqpoint{2.486186in}{1.694313in}}%
\pgfpathlineto{\pgfqpoint{2.551351in}{1.736013in}}%
\pgfpathlineto{\pgfqpoint{2.611862in}{1.771373in}}%
\pgfpathlineto{\pgfqpoint{2.672372in}{1.803149in}}%
\pgfpathlineto{\pgfqpoint{2.728228in}{1.829045in}}%
\pgfpathlineto{\pgfqpoint{2.784084in}{1.851434in}}%
\pgfpathlineto{\pgfqpoint{2.835285in}{1.868731in}}%
\pgfpathlineto{\pgfqpoint{2.886486in}{1.882828in}}%
\pgfpathlineto{\pgfqpoint{2.937688in}{1.893636in}}%
\pgfpathlineto{\pgfqpoint{2.988889in}{1.901089in}}%
\pgfpathlineto{\pgfqpoint{3.040090in}{1.905140in}}%
\pgfpathlineto{\pgfqpoint{3.086637in}{1.905849in}}%
\pgfpathlineto{\pgfqpoint{3.133183in}{1.903722in}}%
\pgfpathlineto{\pgfqpoint{3.179730in}{1.898770in}}%
\pgfpathlineto{\pgfqpoint{3.230931in}{1.890091in}}%
\pgfpathlineto{\pgfqpoint{3.282132in}{1.878078in}}%
\pgfpathlineto{\pgfqpoint{3.333333in}{1.862806in}}%
\pgfpathlineto{\pgfqpoint{3.384535in}{1.844372in}}%
\pgfpathlineto{\pgfqpoint{3.440390in}{1.820793in}}%
\pgfpathlineto{\pgfqpoint{3.496246in}{1.793772in}}%
\pgfpathlineto{\pgfqpoint{3.556757in}{1.760860in}}%
\pgfpathlineto{\pgfqpoint{3.617267in}{1.724463in}}%
\pgfpathlineto{\pgfqpoint{3.682432in}{1.681772in}}%
\pgfpathlineto{\pgfqpoint{3.756907in}{1.629161in}}%
\pgfpathlineto{\pgfqpoint{3.840691in}{1.566089in}}%
\pgfpathlineto{\pgfqpoint{3.952402in}{1.477812in}}%
\pgfpathlineto{\pgfqpoint{4.166517in}{1.307783in}}%
\pgfpathlineto{\pgfqpoint{4.245646in}{1.249375in}}%
\pgfpathlineto{\pgfqpoint{4.310811in}{1.204961in}}%
\pgfpathlineto{\pgfqpoint{4.366667in}{1.170372in}}%
\pgfpathlineto{\pgfqpoint{4.417868in}{1.142100in}}%
\pgfpathlineto{\pgfqpoint{4.464414in}{1.119718in}}%
\pgfpathlineto{\pgfqpoint{4.506306in}{1.102641in}}%
\pgfpathlineto{\pgfqpoint{4.548198in}{1.088799in}}%
\pgfpathlineto{\pgfqpoint{4.585435in}{1.079477in}}%
\pgfpathlineto{\pgfqpoint{4.622673in}{1.073209in}}%
\pgfpathlineto{\pgfqpoint{4.659910in}{1.070236in}}%
\pgfpathlineto{\pgfqpoint{4.692492in}{1.070535in}}%
\pgfpathlineto{\pgfqpoint{4.725075in}{1.073718in}}%
\pgfpathlineto{\pgfqpoint{4.757658in}{1.079959in}}%
\pgfpathlineto{\pgfqpoint{4.790240in}{1.089437in}}%
\pgfpathlineto{\pgfqpoint{4.822823in}{1.102332in}}%
\pgfpathlineto{\pgfqpoint{4.855405in}{1.118828in}}%
\pgfpathlineto{\pgfqpoint{4.887988in}{1.139112in}}%
\pgfpathlineto{\pgfqpoint{4.920571in}{1.163378in}}%
\pgfpathlineto{\pgfqpoint{4.953153in}{1.191818in}}%
\pgfpathlineto{\pgfqpoint{4.985736in}{1.224632in}}%
\pgfpathlineto{\pgfqpoint{5.018318in}{1.262021in}}%
\pgfpathlineto{\pgfqpoint{5.050901in}{1.304191in}}%
\pgfpathlineto{\pgfqpoint{5.083483in}{1.351349in}}%
\pgfpathlineto{\pgfqpoint{5.116066in}{1.403708in}}%
\pgfpathlineto{\pgfqpoint{5.148649in}{1.461484in}}%
\pgfpathlineto{\pgfqpoint{5.181231in}{1.524896in}}%
\pgfpathlineto{\pgfqpoint{5.218468in}{1.604553in}}%
\pgfpathlineto{\pgfqpoint{5.255706in}{1.692199in}}%
\pgfpathlineto{\pgfqpoint{5.292943in}{1.788178in}}%
\pgfpathlineto{\pgfqpoint{5.330180in}{1.892840in}}%
\pgfpathlineto{\pgfqpoint{5.367417in}{2.006540in}}%
\pgfpathlineto{\pgfqpoint{5.400000in}{2.113725in}}%
\pgfpathlineto{\pgfqpoint{5.400000in}{2.113725in}}%
\pgfusepath{stroke}%
\end{pgfscope}%
\begin{pgfscope}%
\pgfpathrectangle{\pgfqpoint{0.750000in}{0.500000in}}{\pgfqpoint{4.650000in}{3.020000in}}%
\pgfusepath{clip}%
\pgfsetrectcap%
\pgfsetroundjoin%
\pgfsetlinewidth{1.505625pt}%
\definecolor{currentstroke}{rgb}{0.839216,0.152941,0.156863}%
\pgfsetstrokecolor{currentstroke}%
\pgfsetdash{}{0pt}%
\pgfpathmoveto{\pgfqpoint{1.024090in}{0.486111in}}%
\pgfpathlineto{\pgfqpoint{1.043243in}{0.654520in}}%
\pgfpathlineto{\pgfqpoint{1.066517in}{0.836379in}}%
\pgfpathlineto{\pgfqpoint{1.089790in}{0.995037in}}%
\pgfpathlineto{\pgfqpoint{1.113063in}{1.132231in}}%
\pgfpathlineto{\pgfqpoint{1.136336in}{1.249622in}}%
\pgfpathlineto{\pgfqpoint{1.159610in}{1.348790in}}%
\pgfpathlineto{\pgfqpoint{1.178228in}{1.416021in}}%
\pgfpathlineto{\pgfqpoint{1.196847in}{1.473294in}}%
\pgfpathlineto{\pgfqpoint{1.215465in}{1.521320in}}%
\pgfpathlineto{\pgfqpoint{1.234084in}{1.560778in}}%
\pgfpathlineto{\pgfqpoint{1.252703in}{1.592321in}}%
\pgfpathlineto{\pgfqpoint{1.271321in}{1.616577in}}%
\pgfpathlineto{\pgfqpoint{1.285285in}{1.630347in}}%
\pgfpathlineto{\pgfqpoint{1.299249in}{1.640597in}}%
\pgfpathlineto{\pgfqpoint{1.313213in}{1.647564in}}%
\pgfpathlineto{\pgfqpoint{1.327177in}{1.651474in}}%
\pgfpathlineto{\pgfqpoint{1.341141in}{1.652547in}}%
\pgfpathlineto{\pgfqpoint{1.355105in}{1.650994in}}%
\pgfpathlineto{\pgfqpoint{1.373724in}{1.645191in}}%
\pgfpathlineto{\pgfqpoint{1.392342in}{1.635547in}}%
\pgfpathlineto{\pgfqpoint{1.410961in}{1.622502in}}%
\pgfpathlineto{\pgfqpoint{1.434234in}{1.602051in}}%
\pgfpathlineto{\pgfqpoint{1.462162in}{1.572443in}}%
\pgfpathlineto{\pgfqpoint{1.494745in}{1.532466in}}%
\pgfpathlineto{\pgfqpoint{1.536637in}{1.475194in}}%
\pgfpathlineto{\pgfqpoint{1.620420in}{1.353060in}}%
\pgfpathlineto{\pgfqpoint{1.676276in}{1.274591in}}%
\pgfpathlineto{\pgfqpoint{1.718168in}{1.220882in}}%
\pgfpathlineto{\pgfqpoint{1.755405in}{1.178327in}}%
\pgfpathlineto{\pgfqpoint{1.787988in}{1.145852in}}%
\pgfpathlineto{\pgfqpoint{1.815916in}{1.121944in}}%
\pgfpathlineto{\pgfqpoint{1.843844in}{1.101924in}}%
\pgfpathlineto{\pgfqpoint{1.871772in}{1.085981in}}%
\pgfpathlineto{\pgfqpoint{1.895045in}{1.075911in}}%
\pgfpathlineto{\pgfqpoint{1.918318in}{1.068826in}}%
\pgfpathlineto{\pgfqpoint{1.941592in}{1.064759in}}%
\pgfpathlineto{\pgfqpoint{1.964865in}{1.063722in}}%
\pgfpathlineto{\pgfqpoint{1.988138in}{1.065711in}}%
\pgfpathlineto{\pgfqpoint{2.011411in}{1.070701in}}%
\pgfpathlineto{\pgfqpoint{2.034685in}{1.078650in}}%
\pgfpathlineto{\pgfqpoint{2.057958in}{1.089503in}}%
\pgfpathlineto{\pgfqpoint{2.085886in}{1.106256in}}%
\pgfpathlineto{\pgfqpoint{2.113814in}{1.126937in}}%
\pgfpathlineto{\pgfqpoint{2.141742in}{1.151368in}}%
\pgfpathlineto{\pgfqpoint{2.174324in}{1.184348in}}%
\pgfpathlineto{\pgfqpoint{2.206907in}{1.221813in}}%
\pgfpathlineto{\pgfqpoint{2.244144in}{1.269629in}}%
\pgfpathlineto{\pgfqpoint{2.286036in}{1.329053in}}%
\pgfpathlineto{\pgfqpoint{2.332583in}{1.400973in}}%
\pgfpathlineto{\pgfqpoint{2.388438in}{1.493612in}}%
\pgfpathlineto{\pgfqpoint{2.462913in}{1.623905in}}%
\pgfpathlineto{\pgfqpoint{2.611862in}{1.886061in}}%
\pgfpathlineto{\pgfqpoint{2.667718in}{1.977751in}}%
\pgfpathlineto{\pgfqpoint{2.714264in}{2.048923in}}%
\pgfpathlineto{\pgfqpoint{2.756156in}{2.107922in}}%
\pgfpathlineto{\pgfqpoint{2.793393in}{2.155721in}}%
\pgfpathlineto{\pgfqpoint{2.830631in}{2.198661in}}%
\pgfpathlineto{\pgfqpoint{2.863213in}{2.231919in}}%
\pgfpathlineto{\pgfqpoint{2.895796in}{2.260891in}}%
\pgfpathlineto{\pgfqpoint{2.928378in}{2.285365in}}%
\pgfpathlineto{\pgfqpoint{2.956306in}{2.302626in}}%
\pgfpathlineto{\pgfqpoint{2.984234in}{2.316356in}}%
\pgfpathlineto{\pgfqpoint{3.012162in}{2.326484in}}%
\pgfpathlineto{\pgfqpoint{3.040090in}{2.332955in}}%
\pgfpathlineto{\pgfqpoint{3.068018in}{2.335735in}}%
\pgfpathlineto{\pgfqpoint{3.091291in}{2.335220in}}%
\pgfpathlineto{\pgfqpoint{3.119219in}{2.331207in}}%
\pgfpathlineto{\pgfqpoint{3.147147in}{2.323512in}}%
\pgfpathlineto{\pgfqpoint{3.175075in}{2.312177in}}%
\pgfpathlineto{\pgfqpoint{3.203003in}{2.297260in}}%
\pgfpathlineto{\pgfqpoint{3.230931in}{2.278843in}}%
\pgfpathlineto{\pgfqpoint{3.258859in}{2.257023in}}%
\pgfpathlineto{\pgfqpoint{3.291441in}{2.227425in}}%
\pgfpathlineto{\pgfqpoint{3.324024in}{2.193574in}}%
\pgfpathlineto{\pgfqpoint{3.361261in}{2.150003in}}%
\pgfpathlineto{\pgfqpoint{3.398498in}{2.101628in}}%
\pgfpathlineto{\pgfqpoint{3.440390in}{2.042056in}}%
\pgfpathlineto{\pgfqpoint{3.486937in}{1.970348in}}%
\pgfpathlineto{\pgfqpoint{3.542793in}{1.878170in}}%
\pgfpathlineto{\pgfqpoint{3.617267in}{1.748361in}}%
\pgfpathlineto{\pgfqpoint{3.780180in}{1.462098in}}%
\pgfpathlineto{\pgfqpoint{3.836036in}{1.371541in}}%
\pgfpathlineto{\pgfqpoint{3.882583in}{1.301961in}}%
\pgfpathlineto{\pgfqpoint{3.924474in}{1.245089in}}%
\pgfpathlineto{\pgfqpoint{3.961712in}{1.199877in}}%
\pgfpathlineto{\pgfqpoint{3.994294in}{1.164930in}}%
\pgfpathlineto{\pgfqpoint{4.026877in}{1.134673in}}%
\pgfpathlineto{\pgfqpoint{4.054805in}{1.112722in}}%
\pgfpathlineto{\pgfqpoint{4.082733in}{1.094641in}}%
\pgfpathlineto{\pgfqpoint{4.110661in}{1.080591in}}%
\pgfpathlineto{\pgfqpoint{4.133934in}{1.072056in}}%
\pgfpathlineto{\pgfqpoint{4.157207in}{1.066470in}}%
\pgfpathlineto{\pgfqpoint{4.180480in}{1.063879in}}%
\pgfpathlineto{\pgfqpoint{4.203754in}{1.064309in}}%
\pgfpathlineto{\pgfqpoint{4.227027in}{1.067770in}}%
\pgfpathlineto{\pgfqpoint{4.250300in}{1.074254in}}%
\pgfpathlineto{\pgfqpoint{4.273574in}{1.083731in}}%
\pgfpathlineto{\pgfqpoint{4.296847in}{1.096149in}}%
\pgfpathlineto{\pgfqpoint{4.324775in}{1.114829in}}%
\pgfpathlineto{\pgfqpoint{4.352703in}{1.137465in}}%
\pgfpathlineto{\pgfqpoint{4.380631in}{1.163831in}}%
\pgfpathlineto{\pgfqpoint{4.413213in}{1.198927in}}%
\pgfpathlineto{\pgfqpoint{4.450450in}{1.244071in}}%
\pgfpathlineto{\pgfqpoint{4.492342in}{1.300053in}}%
\pgfpathlineto{\pgfqpoint{4.552853in}{1.387063in}}%
\pgfpathlineto{\pgfqpoint{4.636637in}{1.507650in}}%
\pgfpathlineto{\pgfqpoint{4.673874in}{1.555929in}}%
\pgfpathlineto{\pgfqpoint{4.701802in}{1.587868in}}%
\pgfpathlineto{\pgfqpoint{4.725075in}{1.610740in}}%
\pgfpathlineto{\pgfqpoint{4.748348in}{1.629423in}}%
\pgfpathlineto{\pgfqpoint{4.766967in}{1.640821in}}%
\pgfpathlineto{\pgfqpoint{4.785586in}{1.648601in}}%
\pgfpathlineto{\pgfqpoint{4.804204in}{1.652311in}}%
\pgfpathlineto{\pgfqpoint{4.818168in}{1.652136in}}%
\pgfpathlineto{\pgfqpoint{4.832132in}{1.649196in}}%
\pgfpathlineto{\pgfqpoint{4.846096in}{1.643273in}}%
\pgfpathlineto{\pgfqpoint{4.860060in}{1.634143in}}%
\pgfpathlineto{\pgfqpoint{4.874024in}{1.621573in}}%
\pgfpathlineto{\pgfqpoint{4.887988in}{1.605322in}}%
\pgfpathlineto{\pgfqpoint{4.901952in}{1.585143in}}%
\pgfpathlineto{\pgfqpoint{4.920571in}{1.551680in}}%
\pgfpathlineto{\pgfqpoint{4.939189in}{1.510142in}}%
\pgfpathlineto{\pgfqpoint{4.957808in}{1.459870in}}%
\pgfpathlineto{\pgfqpoint{4.976426in}{1.400175in}}%
\pgfpathlineto{\pgfqpoint{4.995045in}{1.330341in}}%
\pgfpathlineto{\pgfqpoint{5.013664in}{1.249622in}}%
\pgfpathlineto{\pgfqpoint{5.032282in}{1.157242in}}%
\pgfpathlineto{\pgfqpoint{5.055556in}{1.024139in}}%
\pgfpathlineto{\pgfqpoint{5.078829in}{0.869910in}}%
\pgfpathlineto{\pgfqpoint{5.102102in}{0.692834in}}%
\pgfpathlineto{\pgfqpoint{5.125910in}{0.486111in}}%
\pgfpathlineto{\pgfqpoint{5.125910in}{0.486111in}}%
\pgfusepath{stroke}%
\end{pgfscope}%
\begin{pgfscope}%
\pgfpathrectangle{\pgfqpoint{0.750000in}{0.500000in}}{\pgfqpoint{4.650000in}{3.020000in}}%
\pgfusepath{clip}%
\pgfsetrectcap%
\pgfsetroundjoin%
\pgfsetlinewidth{1.505625pt}%
\definecolor{currentstroke}{rgb}{0.580392,0.403922,0.741176}%
\pgfsetstrokecolor{currentstroke}%
\pgfsetdash{}{0pt}%
\pgfpathmoveto{\pgfqpoint{1.035759in}{3.533889in}}%
\pgfpathlineto{\pgfqpoint{1.052553in}{3.005999in}}%
\pgfpathlineto{\pgfqpoint{1.071171in}{2.499883in}}%
\pgfpathlineto{\pgfqpoint{1.089790in}{2.068572in}}%
\pgfpathlineto{\pgfqpoint{1.108408in}{1.704697in}}%
\pgfpathlineto{\pgfqpoint{1.127027in}{1.401397in}}%
\pgfpathlineto{\pgfqpoint{1.145646in}{1.152301in}}%
\pgfpathlineto{\pgfqpoint{1.164264in}{0.951495in}}%
\pgfpathlineto{\pgfqpoint{1.178228in}{0.829273in}}%
\pgfpathlineto{\pgfqpoint{1.192192in}{0.728973in}}%
\pgfpathlineto{\pgfqpoint{1.206156in}{0.648567in}}%
\pgfpathlineto{\pgfqpoint{1.220120in}{0.586148in}}%
\pgfpathlineto{\pgfqpoint{1.234084in}{0.539925in}}%
\pgfpathlineto{\pgfqpoint{1.243393in}{0.517275in}}%
\pgfpathlineto{\pgfqpoint{1.252703in}{0.500604in}}%
\pgfpathlineto{\pgfqpoint{1.262012in}{0.489464in}}%
\pgfpathlineto{\pgfqpoint{1.266308in}{0.486111in}}%
\pgfpathmoveto{\pgfqpoint{1.291662in}{0.486111in}}%
\pgfpathlineto{\pgfqpoint{1.299249in}{0.491881in}}%
\pgfpathlineto{\pgfqpoint{1.308559in}{0.502316in}}%
\pgfpathlineto{\pgfqpoint{1.322523in}{0.523920in}}%
\pgfpathlineto{\pgfqpoint{1.336486in}{0.551719in}}%
\pgfpathlineto{\pgfqpoint{1.355105in}{0.596718in}}%
\pgfpathlineto{\pgfqpoint{1.378378in}{0.662743in}}%
\pgfpathlineto{\pgfqpoint{1.410961in}{0.766964in}}%
\pgfpathlineto{\pgfqpoint{1.518018in}{1.119746in}}%
\pgfpathlineto{\pgfqpoint{1.545946in}{1.198856in}}%
\pgfpathlineto{\pgfqpoint{1.573874in}{1.268742in}}%
\pgfpathlineto{\pgfqpoint{1.597147in}{1.319184in}}%
\pgfpathlineto{\pgfqpoint{1.620420in}{1.362157in}}%
\pgfpathlineto{\pgfqpoint{1.639039in}{1.391049in}}%
\pgfpathlineto{\pgfqpoint{1.657658in}{1.415061in}}%
\pgfpathlineto{\pgfqpoint{1.676276in}{1.434257in}}%
\pgfpathlineto{\pgfqpoint{1.694895in}{1.448755in}}%
\pgfpathlineto{\pgfqpoint{1.713514in}{1.458722in}}%
\pgfpathlineto{\pgfqpoint{1.732132in}{1.464366in}}%
\pgfpathlineto{\pgfqpoint{1.750751in}{1.465933in}}%
\pgfpathlineto{\pgfqpoint{1.769369in}{1.463697in}}%
\pgfpathlineto{\pgfqpoint{1.787988in}{1.457960in}}%
\pgfpathlineto{\pgfqpoint{1.806607in}{1.449044in}}%
\pgfpathlineto{\pgfqpoint{1.829880in}{1.433942in}}%
\pgfpathlineto{\pgfqpoint{1.853153in}{1.415084in}}%
\pgfpathlineto{\pgfqpoint{1.881081in}{1.388484in}}%
\pgfpathlineto{\pgfqpoint{1.918318in}{1.348285in}}%
\pgfpathlineto{\pgfqpoint{2.043994in}{1.207583in}}%
\pgfpathlineto{\pgfqpoint{2.076577in}{1.177952in}}%
\pgfpathlineto{\pgfqpoint{2.104505in}{1.156938in}}%
\pgfpathlineto{\pgfqpoint{2.127778in}{1.143059in}}%
\pgfpathlineto{\pgfqpoint{2.151051in}{1.132855in}}%
\pgfpathlineto{\pgfqpoint{2.169670in}{1.127536in}}%
\pgfpathlineto{\pgfqpoint{2.188288in}{1.124883in}}%
\pgfpathlineto{\pgfqpoint{2.206907in}{1.125000in}}%
\pgfpathlineto{\pgfqpoint{2.225526in}{1.127967in}}%
\pgfpathlineto{\pgfqpoint{2.244144in}{1.133846in}}%
\pgfpathlineto{\pgfqpoint{2.262763in}{1.142673in}}%
\pgfpathlineto{\pgfqpoint{2.281381in}{1.154465in}}%
\pgfpathlineto{\pgfqpoint{2.304655in}{1.173368in}}%
\pgfpathlineto{\pgfqpoint{2.327928in}{1.196846in}}%
\pgfpathlineto{\pgfqpoint{2.351201in}{1.224800in}}%
\pgfpathlineto{\pgfqpoint{2.374474in}{1.257087in}}%
\pgfpathlineto{\pgfqpoint{2.402402in}{1.301288in}}%
\pgfpathlineto{\pgfqpoint{2.430330in}{1.351066in}}%
\pgfpathlineto{\pgfqpoint{2.462913in}{1.415561in}}%
\pgfpathlineto{\pgfqpoint{2.500150in}{1.496679in}}%
\pgfpathlineto{\pgfqpoint{2.542042in}{1.595647in}}%
\pgfpathlineto{\pgfqpoint{2.597898in}{1.736486in}}%
\pgfpathlineto{\pgfqpoint{2.756156in}{2.141813in}}%
\pgfpathlineto{\pgfqpoint{2.798048in}{2.238505in}}%
\pgfpathlineto{\pgfqpoint{2.835285in}{2.317018in}}%
\pgfpathlineto{\pgfqpoint{2.867868in}{2.378796in}}%
\pgfpathlineto{\pgfqpoint{2.895796in}{2.425914in}}%
\pgfpathlineto{\pgfqpoint{2.923724in}{2.467137in}}%
\pgfpathlineto{\pgfqpoint{2.946997in}{2.496679in}}%
\pgfpathlineto{\pgfqpoint{2.970270in}{2.521624in}}%
\pgfpathlineto{\pgfqpoint{2.993544in}{2.541796in}}%
\pgfpathlineto{\pgfqpoint{3.012162in}{2.554398in}}%
\pgfpathlineto{\pgfqpoint{3.030781in}{2.563797in}}%
\pgfpathlineto{\pgfqpoint{3.049399in}{2.569948in}}%
\pgfpathlineto{\pgfqpoint{3.068018in}{2.572824in}}%
\pgfpathlineto{\pgfqpoint{3.086637in}{2.572413in}}%
\pgfpathlineto{\pgfqpoint{3.105255in}{2.568716in}}%
\pgfpathlineto{\pgfqpoint{3.123874in}{2.561750in}}%
\pgfpathlineto{\pgfqpoint{3.142492in}{2.551546in}}%
\pgfpathlineto{\pgfqpoint{3.161111in}{2.538151in}}%
\pgfpathlineto{\pgfqpoint{3.184384in}{2.517012in}}%
\pgfpathlineto{\pgfqpoint{3.207658in}{2.491132in}}%
\pgfpathlineto{\pgfqpoint{3.230931in}{2.460694in}}%
\pgfpathlineto{\pgfqpoint{3.258859in}{2.418458in}}%
\pgfpathlineto{\pgfqpoint{3.286787in}{2.370402in}}%
\pgfpathlineto{\pgfqpoint{3.319369in}{2.307641in}}%
\pgfpathlineto{\pgfqpoint{3.356607in}{2.228163in}}%
\pgfpathlineto{\pgfqpoint{3.398498in}{2.130613in}}%
\pgfpathlineto{\pgfqpoint{3.449700in}{2.002885in}}%
\pgfpathlineto{\pgfqpoint{3.635886in}{1.528861in}}%
\pgfpathlineto{\pgfqpoint{3.677778in}{1.435148in}}%
\pgfpathlineto{\pgfqpoint{3.715015in}{1.359873in}}%
\pgfpathlineto{\pgfqpoint{3.747598in}{1.301288in}}%
\pgfpathlineto{\pgfqpoint{3.775526in}{1.257087in}}%
\pgfpathlineto{\pgfqpoint{3.803453in}{1.218857in}}%
\pgfpathlineto{\pgfqpoint{3.826727in}{1.191788in}}%
\pgfpathlineto{\pgfqpoint{3.850000in}{1.169219in}}%
\pgfpathlineto{\pgfqpoint{3.873273in}{1.151239in}}%
\pgfpathlineto{\pgfqpoint{3.891892in}{1.140188in}}%
\pgfpathlineto{\pgfqpoint{3.910511in}{1.132101in}}%
\pgfpathlineto{\pgfqpoint{3.929129in}{1.126955in}}%
\pgfpathlineto{\pgfqpoint{3.947748in}{1.124706in}}%
\pgfpathlineto{\pgfqpoint{3.966366in}{1.125290in}}%
\pgfpathlineto{\pgfqpoint{3.984985in}{1.128620in}}%
\pgfpathlineto{\pgfqpoint{4.003604in}{1.134587in}}%
\pgfpathlineto{\pgfqpoint{4.026877in}{1.145551in}}%
\pgfpathlineto{\pgfqpoint{4.050150in}{1.160126in}}%
\pgfpathlineto{\pgfqpoint{4.078078in}{1.181872in}}%
\pgfpathlineto{\pgfqpoint{4.106006in}{1.207583in}}%
\pgfpathlineto{\pgfqpoint{4.143243in}{1.246604in}}%
\pgfpathlineto{\pgfqpoint{4.199099in}{1.310722in}}%
\pgfpathlineto{\pgfqpoint{4.254955in}{1.373917in}}%
\pgfpathlineto{\pgfqpoint{4.287538in}{1.406646in}}%
\pgfpathlineto{\pgfqpoint{4.315465in}{1.430449in}}%
\pgfpathlineto{\pgfqpoint{4.338739in}{1.446357in}}%
\pgfpathlineto{\pgfqpoint{4.357357in}{1.456016in}}%
\pgfpathlineto{\pgfqpoint{4.375976in}{1.462579in}}%
\pgfpathlineto{\pgfqpoint{4.394595in}{1.465719in}}%
\pgfpathlineto{\pgfqpoint{4.413213in}{1.465130in}}%
\pgfpathlineto{\pgfqpoint{4.431832in}{1.460529in}}%
\pgfpathlineto{\pgfqpoint{4.450450in}{1.451664in}}%
\pgfpathlineto{\pgfqpoint{4.469069in}{1.438316in}}%
\pgfpathlineto{\pgfqpoint{4.487688in}{1.420308in}}%
\pgfpathlineto{\pgfqpoint{4.506306in}{1.397508in}}%
\pgfpathlineto{\pgfqpoint{4.524925in}{1.369838in}}%
\pgfpathlineto{\pgfqpoint{4.543544in}{1.337282in}}%
\pgfpathlineto{\pgfqpoint{4.566817in}{1.289797in}}%
\pgfpathlineto{\pgfqpoint{4.590090in}{1.235027in}}%
\pgfpathlineto{\pgfqpoint{4.618018in}{1.160370in}}%
\pgfpathlineto{\pgfqpoint{4.650601in}{1.062623in}}%
\pgfpathlineto{\pgfqpoint{4.692492in}{0.924914in}}%
\pgfpathlineto{\pgfqpoint{4.766967in}{0.676951in}}%
\pgfpathlineto{\pgfqpoint{4.790240in}{0.609158in}}%
\pgfpathlineto{\pgfqpoint{4.808859in}{0.562189in}}%
\pgfpathlineto{\pgfqpoint{4.827477in}{0.523920in}}%
\pgfpathlineto{\pgfqpoint{4.841441in}{0.502316in}}%
\pgfpathlineto{\pgfqpoint{4.850751in}{0.491881in}}%
\pgfpathlineto{\pgfqpoint{4.858338in}{0.486111in}}%
\pgfpathmoveto{\pgfqpoint{4.883692in}{0.486111in}}%
\pgfpathlineto{\pgfqpoint{4.892643in}{0.494370in}}%
\pgfpathlineto{\pgfqpoint{4.901952in}{0.508221in}}%
\pgfpathlineto{\pgfqpoint{4.911261in}{0.527824in}}%
\pgfpathlineto{\pgfqpoint{4.920571in}{0.553639in}}%
\pgfpathlineto{\pgfqpoint{4.934535in}{0.605068in}}%
\pgfpathlineto{\pgfqpoint{4.948498in}{0.673279in}}%
\pgfpathlineto{\pgfqpoint{4.962462in}{0.760099in}}%
\pgfpathlineto{\pgfqpoint{4.976426in}{0.867474in}}%
\pgfpathlineto{\pgfqpoint{4.990390in}{0.997476in}}%
\pgfpathlineto{\pgfqpoint{5.004354in}{1.152301in}}%
\pgfpathlineto{\pgfqpoint{5.022973in}{1.401397in}}%
\pgfpathlineto{\pgfqpoint{5.041592in}{1.704697in}}%
\pgfpathlineto{\pgfqpoint{5.060210in}{2.068572in}}%
\pgfpathlineto{\pgfqpoint{5.078829in}{2.499883in}}%
\pgfpathlineto{\pgfqpoint{5.097447in}{3.005999in}}%
\pgfpathlineto{\pgfqpoint{5.114241in}{3.533889in}}%
\pgfpathlineto{\pgfqpoint{5.114241in}{3.533889in}}%
\pgfusepath{stroke}%
\end{pgfscope}%
\begin{pgfscope}%
\pgfpathrectangle{\pgfqpoint{0.750000in}{0.500000in}}{\pgfqpoint{4.650000in}{3.020000in}}%
\pgfusepath{clip}%
\pgfsetrectcap%
\pgfsetroundjoin%
\pgfsetlinewidth{1.505625pt}%
\definecolor{currentstroke}{rgb}{0.549020,0.337255,0.294118}%
\pgfsetstrokecolor{currentstroke}%
\pgfsetdash{}{0pt}%
\pgfpathmoveto{\pgfqpoint{0.750000in}{1.255000in}}%
\pgfpathlineto{\pgfqpoint{2.057958in}{1.256539in}}%
\pgfpathlineto{\pgfqpoint{2.146396in}{1.259841in}}%
\pgfpathlineto{\pgfqpoint{2.206907in}{1.264986in}}%
\pgfpathlineto{\pgfqpoint{2.253453in}{1.271860in}}%
\pgfpathlineto{\pgfqpoint{2.290691in}{1.280108in}}%
\pgfpathlineto{\pgfqpoint{2.323273in}{1.290040in}}%
\pgfpathlineto{\pgfqpoint{2.355856in}{1.303215in}}%
\pgfpathlineto{\pgfqpoint{2.383784in}{1.317676in}}%
\pgfpathlineto{\pgfqpoint{2.411712in}{1.335633in}}%
\pgfpathlineto{\pgfqpoint{2.439640in}{1.357662in}}%
\pgfpathlineto{\pgfqpoint{2.467568in}{1.384360in}}%
\pgfpathlineto{\pgfqpoint{2.490841in}{1.410600in}}%
\pgfpathlineto{\pgfqpoint{2.514114in}{1.440817in}}%
\pgfpathlineto{\pgfqpoint{2.542042in}{1.482741in}}%
\pgfpathlineto{\pgfqpoint{2.569970in}{1.531239in}}%
\pgfpathlineto{\pgfqpoint{2.597898in}{1.586602in}}%
\pgfpathlineto{\pgfqpoint{2.625826in}{1.648946in}}%
\pgfpathlineto{\pgfqpoint{2.658408in}{1.730363in}}%
\pgfpathlineto{\pgfqpoint{2.690991in}{1.820552in}}%
\pgfpathlineto{\pgfqpoint{2.728228in}{1.932919in}}%
\pgfpathlineto{\pgfqpoint{2.779429in}{2.098920in}}%
\pgfpathlineto{\pgfqpoint{2.872523in}{2.404216in}}%
\pgfpathlineto{\pgfqpoint{2.905105in}{2.500929in}}%
\pgfpathlineto{\pgfqpoint{2.933033in}{2.575335in}}%
\pgfpathlineto{\pgfqpoint{2.956306in}{2.629770in}}%
\pgfpathlineto{\pgfqpoint{2.979580in}{2.676159in}}%
\pgfpathlineto{\pgfqpoint{2.998198in}{2.706833in}}%
\pgfpathlineto{\pgfqpoint{3.016817in}{2.731338in}}%
\pgfpathlineto{\pgfqpoint{3.030781in}{2.745464in}}%
\pgfpathlineto{\pgfqpoint{3.044745in}{2.755823in}}%
\pgfpathlineto{\pgfqpoint{3.058709in}{2.762333in}}%
\pgfpathlineto{\pgfqpoint{3.072673in}{2.764946in}}%
\pgfpathlineto{\pgfqpoint{3.086637in}{2.763639in}}%
\pgfpathlineto{\pgfqpoint{3.100601in}{2.758424in}}%
\pgfpathlineto{\pgfqpoint{3.114565in}{2.749340in}}%
\pgfpathlineto{\pgfqpoint{3.128529in}{2.736459in}}%
\pgfpathlineto{\pgfqpoint{3.142492in}{2.719880in}}%
\pgfpathlineto{\pgfqpoint{3.161111in}{2.692243in}}%
\pgfpathlineto{\pgfqpoint{3.179730in}{2.658633in}}%
\pgfpathlineto{\pgfqpoint{3.203003in}{2.608903in}}%
\pgfpathlineto{\pgfqpoint{3.226276in}{2.551548in}}%
\pgfpathlineto{\pgfqpoint{3.254204in}{2.474253in}}%
\pgfpathlineto{\pgfqpoint{3.286787in}{2.375076in}}%
\pgfpathlineto{\pgfqpoint{3.337988in}{2.207662in}}%
\pgfpathlineto{\pgfqpoint{3.421772in}{1.932919in}}%
\pgfpathlineto{\pgfqpoint{3.463664in}{1.807167in}}%
\pgfpathlineto{\pgfqpoint{3.496246in}{1.718176in}}%
\pgfpathlineto{\pgfqpoint{3.528829in}{1.638072in}}%
\pgfpathlineto{\pgfqpoint{3.556757in}{1.576890in}}%
\pgfpathlineto{\pgfqpoint{3.584685in}{1.522685in}}%
\pgfpathlineto{\pgfqpoint{3.612613in}{1.475307in}}%
\pgfpathlineto{\pgfqpoint{3.640541in}{1.434440in}}%
\pgfpathlineto{\pgfqpoint{3.668468in}{1.399644in}}%
\pgfpathlineto{\pgfqpoint{3.696396in}{1.370390in}}%
\pgfpathlineto{\pgfqpoint{3.724324in}{1.346102in}}%
\pgfpathlineto{\pgfqpoint{3.752252in}{1.326182in}}%
\pgfpathlineto{\pgfqpoint{3.780180in}{1.310043in}}%
\pgfpathlineto{\pgfqpoint{3.808108in}{1.297124in}}%
\pgfpathlineto{\pgfqpoint{3.840691in}{1.285429in}}%
\pgfpathlineto{\pgfqpoint{3.877928in}{1.275622in}}%
\pgfpathlineto{\pgfqpoint{3.919820in}{1.268023in}}%
\pgfpathlineto{\pgfqpoint{3.971021in}{1.262193in}}%
\pgfpathlineto{\pgfqpoint{4.036186in}{1.258213in}}%
\pgfpathlineto{\pgfqpoint{4.129279in}{1.255921in}}%
\pgfpathlineto{\pgfqpoint{4.320120in}{1.255050in}}%
\pgfpathlineto{\pgfqpoint{5.400000in}{1.255000in}}%
\pgfpathlineto{\pgfqpoint{5.400000in}{1.255000in}}%
\pgfusepath{stroke}%
\end{pgfscope}%
\begin{pgfscope}%
\pgfsetrectcap%
\pgfsetmiterjoin%
\pgfsetlinewidth{0.803000pt}%
\definecolor{currentstroke}{rgb}{0.000000,0.000000,0.000000}%
\pgfsetstrokecolor{currentstroke}%
\pgfsetdash{}{0pt}%
\pgfpathmoveto{\pgfqpoint{0.750000in}{0.500000in}}%
\pgfpathlineto{\pgfqpoint{0.750000in}{3.520000in}}%
\pgfusepath{stroke}%
\end{pgfscope}%
\begin{pgfscope}%
\pgfsetrectcap%
\pgfsetmiterjoin%
\pgfsetlinewidth{0.803000pt}%
\definecolor{currentstroke}{rgb}{0.000000,0.000000,0.000000}%
\pgfsetstrokecolor{currentstroke}%
\pgfsetdash{}{0pt}%
\pgfpathmoveto{\pgfqpoint{5.400000in}{0.500000in}}%
\pgfpathlineto{\pgfqpoint{5.400000in}{3.520000in}}%
\pgfusepath{stroke}%
\end{pgfscope}%
\begin{pgfscope}%
\pgfsetrectcap%
\pgfsetmiterjoin%
\pgfsetlinewidth{0.803000pt}%
\definecolor{currentstroke}{rgb}{0.000000,0.000000,0.000000}%
\pgfsetstrokecolor{currentstroke}%
\pgfsetdash{}{0pt}%
\pgfpathmoveto{\pgfqpoint{0.750000in}{0.500000in}}%
\pgfpathlineto{\pgfqpoint{5.400000in}{0.500000in}}%
\pgfusepath{stroke}%
\end{pgfscope}%
\begin{pgfscope}%
\pgfsetrectcap%
\pgfsetmiterjoin%
\pgfsetlinewidth{0.803000pt}%
\definecolor{currentstroke}{rgb}{0.000000,0.000000,0.000000}%
\pgfsetstrokecolor{currentstroke}%
\pgfsetdash{}{0pt}%
\pgfpathmoveto{\pgfqpoint{0.750000in}{3.520000in}}%
\pgfpathlineto{\pgfqpoint{5.400000in}{3.520000in}}%
\pgfusepath{stroke}%
\end{pgfscope}%
\begin{pgfscope}%
\pgfsetbuttcap%
\pgfsetmiterjoin%
\definecolor{currentfill}{rgb}{1.000000,1.000000,1.000000}%
\pgfsetfillcolor{currentfill}%
\pgfsetfillopacity{0.800000}%
\pgfsetlinewidth{1.003750pt}%
\definecolor{currentstroke}{rgb}{0.800000,0.800000,0.800000}%
\pgfsetstrokecolor{currentstroke}%
\pgfsetstrokeopacity{0.800000}%
\pgfsetdash{}{0pt}%
\pgfpathmoveto{\pgfqpoint{4.520339in}{2.247161in}}%
\pgfpathlineto{\pgfqpoint{5.302778in}{2.247161in}}%
\pgfpathquadraticcurveto{\pgfqpoint{5.330556in}{2.247161in}}{\pgfqpoint{5.330556in}{2.274939in}}%
\pgfpathlineto{\pgfqpoint{5.330556in}{3.422778in}}%
\pgfpathquadraticcurveto{\pgfqpoint{5.330556in}{3.450556in}}{\pgfqpoint{5.302778in}{3.450556in}}%
\pgfpathlineto{\pgfqpoint{4.520339in}{3.450556in}}%
\pgfpathquadraticcurveto{\pgfqpoint{4.492561in}{3.450556in}}{\pgfqpoint{4.492561in}{3.422778in}}%
\pgfpathlineto{\pgfqpoint{4.492561in}{2.274939in}}%
\pgfpathquadraticcurveto{\pgfqpoint{4.492561in}{2.247161in}}{\pgfqpoint{4.520339in}{2.247161in}}%
\pgfpathclose%
\pgfusepath{stroke,fill}%
\end{pgfscope}%
\begin{pgfscope}%
\pgfsetrectcap%
\pgfsetroundjoin%
\pgfsetlinewidth{1.505625pt}%
\definecolor{currentstroke}{rgb}{0.121569,0.466667,0.705882}%
\pgfsetstrokecolor{currentstroke}%
\pgfsetdash{}{0pt}%
\pgfpathmoveto{\pgfqpoint{4.548117in}{3.346389in}}%
\pgfpathlineto{\pgfqpoint{4.825895in}{3.346389in}}%
\pgfusepath{stroke}%
\end{pgfscope}%
\begin{pgfscope}%
\pgftext[x=4.937006in,y=3.297778in,left,base]{\rmfamily\fontsize{10.000000}{12.000000}\selectfont \(\displaystyle  n = 1 \)}%
\end{pgfscope}%
\begin{pgfscope}%
\pgfsetrectcap%
\pgfsetroundjoin%
\pgfsetlinewidth{1.505625pt}%
\definecolor{currentstroke}{rgb}{1.000000,0.498039,0.054902}%
\pgfsetstrokecolor{currentstroke}%
\pgfsetdash{}{0pt}%
\pgfpathmoveto{\pgfqpoint{4.548117in}{3.152778in}}%
\pgfpathlineto{\pgfqpoint{4.825895in}{3.152778in}}%
\pgfusepath{stroke}%
\end{pgfscope}%
\begin{pgfscope}%
\pgftext[x=4.937006in,y=3.104167in,left,base]{\rmfamily\fontsize{10.000000}{12.000000}\selectfont \(\displaystyle  n = 3 \)}%
\end{pgfscope}%
\begin{pgfscope}%
\pgfsetrectcap%
\pgfsetroundjoin%
\pgfsetlinewidth{1.505625pt}%
\definecolor{currentstroke}{rgb}{0.172549,0.627451,0.172549}%
\pgfsetstrokecolor{currentstroke}%
\pgfsetdash{}{0pt}%
\pgfpathmoveto{\pgfqpoint{4.548117in}{2.959167in}}%
\pgfpathlineto{\pgfqpoint{4.825895in}{2.959167in}}%
\pgfusepath{stroke}%
\end{pgfscope}%
\begin{pgfscope}%
\pgftext[x=4.937006in,y=2.910556in,left,base]{\rmfamily\fontsize{10.000000}{12.000000}\selectfont \(\displaystyle  n = 5 \)}%
\end{pgfscope}%
\begin{pgfscope}%
\pgfsetrectcap%
\pgfsetroundjoin%
\pgfsetlinewidth{1.505625pt}%
\definecolor{currentstroke}{rgb}{0.839216,0.152941,0.156863}%
\pgfsetstrokecolor{currentstroke}%
\pgfsetdash{}{0pt}%
\pgfpathmoveto{\pgfqpoint{4.548117in}{2.765556in}}%
\pgfpathlineto{\pgfqpoint{4.825895in}{2.765556in}}%
\pgfusepath{stroke}%
\end{pgfscope}%
\begin{pgfscope}%
\pgftext[x=4.937006in,y=2.716945in,left,base]{\rmfamily\fontsize{10.000000}{12.000000}\selectfont \(\displaystyle  n = 7 \)}%
\end{pgfscope}%
\begin{pgfscope}%
\pgfsetrectcap%
\pgfsetroundjoin%
\pgfsetlinewidth{1.505625pt}%
\definecolor{currentstroke}{rgb}{0.580392,0.403922,0.741176}%
\pgfsetstrokecolor{currentstroke}%
\pgfsetdash{}{0pt}%
\pgfpathmoveto{\pgfqpoint{4.548117in}{2.571945in}}%
\pgfpathlineto{\pgfqpoint{4.825895in}{2.571945in}}%
\pgfusepath{stroke}%
\end{pgfscope}%
\begin{pgfscope}%
\pgftext[x=4.937006in,y=2.523334in,left,base]{\rmfamily\fontsize{10.000000}{12.000000}\selectfont \(\displaystyle  n = 9 \)}%
\end{pgfscope}%
\begin{pgfscope}%
\pgfsetrectcap%
\pgfsetroundjoin%
\pgfsetlinewidth{1.505625pt}%
\definecolor{currentstroke}{rgb}{0.549020,0.337255,0.294118}%
\pgfsetstrokecolor{currentstroke}%
\pgfsetdash{}{0pt}%
\pgfpathmoveto{\pgfqpoint{4.548117in}{2.378334in}}%
\pgfpathlineto{\pgfqpoint{4.825895in}{2.378334in}}%
\pgfusepath{stroke}%
\end{pgfscope}%
\begin{pgfscope}%
\pgftext[x=4.937006in,y=2.329723in,left,base]{\rmfamily\fontsize{10.000000}{12.000000}\selectfont \(\displaystyle f_2\)}%
\end{pgfscope}%
\end{pgfpicture}%
\makeatother%
\endgroup%
}
\centering \scalebox{0.8}{%% Creator: Matplotlib, PGF backend
%%
%% To include the figure in your LaTeX document, write
%%   \input{<filename>.pgf}
%%
%% Make sure the required packages are loaded in your preamble
%%   \usepackage{pgf}
%%
%% Figures using additional raster images can only be included by \input if
%% they are in the same directory as the main LaTeX file. For loading figures
%% from other directories you can use the `import` package
%%   \usepackage{import}
%% and then include the figures with
%%   \import{<path to file>}{<filename>.pgf}
%%
%% Matplotlib used the following preamble
%%   \usepackage{fontspec}
%%
\begingroup%
\makeatletter%
\begin{pgfpicture}%
\pgfpathrectangle{\pgfpointorigin}{\pgfqpoint{6.000000in}{4.000000in}}%
\pgfusepath{use as bounding box, clip}%
\begin{pgfscope}%
\pgfsetbuttcap%
\pgfsetmiterjoin%
\definecolor{currentfill}{rgb}{1.000000,1.000000,1.000000}%
\pgfsetfillcolor{currentfill}%
\pgfsetlinewidth{0.000000pt}%
\definecolor{currentstroke}{rgb}{1.000000,1.000000,1.000000}%
\pgfsetstrokecolor{currentstroke}%
\pgfsetdash{}{0pt}%
\pgfpathmoveto{\pgfqpoint{0.000000in}{0.000000in}}%
\pgfpathlineto{\pgfqpoint{6.000000in}{0.000000in}}%
\pgfpathlineto{\pgfqpoint{6.000000in}{4.000000in}}%
\pgfpathlineto{\pgfqpoint{0.000000in}{4.000000in}}%
\pgfpathclose%
\pgfusepath{fill}%
\end{pgfscope}%
\begin{pgfscope}%
\pgfsetbuttcap%
\pgfsetmiterjoin%
\definecolor{currentfill}{rgb}{1.000000,1.000000,1.000000}%
\pgfsetfillcolor{currentfill}%
\pgfsetlinewidth{0.000000pt}%
\definecolor{currentstroke}{rgb}{0.000000,0.000000,0.000000}%
\pgfsetstrokecolor{currentstroke}%
\pgfsetstrokeopacity{0.000000}%
\pgfsetdash{}{0pt}%
\pgfpathmoveto{\pgfqpoint{0.750000in}{0.500000in}}%
\pgfpathlineto{\pgfqpoint{5.400000in}{0.500000in}}%
\pgfpathlineto{\pgfqpoint{5.400000in}{3.520000in}}%
\pgfpathlineto{\pgfqpoint{0.750000in}{3.520000in}}%
\pgfpathclose%
\pgfusepath{fill}%
\end{pgfscope}%
\begin{pgfscope}%
\pgfpathrectangle{\pgfqpoint{0.750000in}{0.500000in}}{\pgfqpoint{4.650000in}{3.020000in}}%
\pgfusepath{clip}%
\pgfsetrectcap%
\pgfsetroundjoin%
\pgfsetlinewidth{0.803000pt}%
\definecolor{currentstroke}{rgb}{0.690196,0.690196,0.690196}%
\pgfsetstrokecolor{currentstroke}%
\pgfsetdash{}{0pt}%
\pgfpathmoveto{\pgfqpoint{0.750000in}{0.500000in}}%
\pgfpathlineto{\pgfqpoint{0.750000in}{3.520000in}}%
\pgfusepath{stroke}%
\end{pgfscope}%
\begin{pgfscope}%
\pgfsetbuttcap%
\pgfsetroundjoin%
\definecolor{currentfill}{rgb}{0.000000,0.000000,0.000000}%
\pgfsetfillcolor{currentfill}%
\pgfsetlinewidth{0.803000pt}%
\definecolor{currentstroke}{rgb}{0.000000,0.000000,0.000000}%
\pgfsetstrokecolor{currentstroke}%
\pgfsetdash{}{0pt}%
\pgfsys@defobject{currentmarker}{\pgfqpoint{0.000000in}{-0.048611in}}{\pgfqpoint{0.000000in}{0.000000in}}{%
\pgfpathmoveto{\pgfqpoint{0.000000in}{0.000000in}}%
\pgfpathlineto{\pgfqpoint{0.000000in}{-0.048611in}}%
\pgfusepath{stroke,fill}%
}%
\begin{pgfscope}%
\pgfsys@transformshift{0.750000in}{0.500000in}%
\pgfsys@useobject{currentmarker}{}%
\end{pgfscope}%
\end{pgfscope}%
\begin{pgfscope}%
\pgftext[x=0.750000in,y=0.402778in,,top]{\rmfamily\fontsize{10.000000}{12.000000}\selectfont \(\displaystyle -6\)}%
\end{pgfscope}%
\begin{pgfscope}%
\pgfpathrectangle{\pgfqpoint{0.750000in}{0.500000in}}{\pgfqpoint{4.650000in}{3.020000in}}%
\pgfusepath{clip}%
\pgfsetrectcap%
\pgfsetroundjoin%
\pgfsetlinewidth{0.803000pt}%
\definecolor{currentstroke}{rgb}{0.690196,0.690196,0.690196}%
\pgfsetstrokecolor{currentstroke}%
\pgfsetdash{}{0pt}%
\pgfpathmoveto{\pgfqpoint{1.525000in}{0.500000in}}%
\pgfpathlineto{\pgfqpoint{1.525000in}{3.520000in}}%
\pgfusepath{stroke}%
\end{pgfscope}%
\begin{pgfscope}%
\pgfsetbuttcap%
\pgfsetroundjoin%
\definecolor{currentfill}{rgb}{0.000000,0.000000,0.000000}%
\pgfsetfillcolor{currentfill}%
\pgfsetlinewidth{0.803000pt}%
\definecolor{currentstroke}{rgb}{0.000000,0.000000,0.000000}%
\pgfsetstrokecolor{currentstroke}%
\pgfsetdash{}{0pt}%
\pgfsys@defobject{currentmarker}{\pgfqpoint{0.000000in}{-0.048611in}}{\pgfqpoint{0.000000in}{0.000000in}}{%
\pgfpathmoveto{\pgfqpoint{0.000000in}{0.000000in}}%
\pgfpathlineto{\pgfqpoint{0.000000in}{-0.048611in}}%
\pgfusepath{stroke,fill}%
}%
\begin{pgfscope}%
\pgfsys@transformshift{1.525000in}{0.500000in}%
\pgfsys@useobject{currentmarker}{}%
\end{pgfscope}%
\end{pgfscope}%
\begin{pgfscope}%
\pgftext[x=1.525000in,y=0.402778in,,top]{\rmfamily\fontsize{10.000000}{12.000000}\selectfont \(\displaystyle -4\)}%
\end{pgfscope}%
\begin{pgfscope}%
\pgfpathrectangle{\pgfqpoint{0.750000in}{0.500000in}}{\pgfqpoint{4.650000in}{3.020000in}}%
\pgfusepath{clip}%
\pgfsetrectcap%
\pgfsetroundjoin%
\pgfsetlinewidth{0.803000pt}%
\definecolor{currentstroke}{rgb}{0.690196,0.690196,0.690196}%
\pgfsetstrokecolor{currentstroke}%
\pgfsetdash{}{0pt}%
\pgfpathmoveto{\pgfqpoint{2.300000in}{0.500000in}}%
\pgfpathlineto{\pgfqpoint{2.300000in}{3.520000in}}%
\pgfusepath{stroke}%
\end{pgfscope}%
\begin{pgfscope}%
\pgfsetbuttcap%
\pgfsetroundjoin%
\definecolor{currentfill}{rgb}{0.000000,0.000000,0.000000}%
\pgfsetfillcolor{currentfill}%
\pgfsetlinewidth{0.803000pt}%
\definecolor{currentstroke}{rgb}{0.000000,0.000000,0.000000}%
\pgfsetstrokecolor{currentstroke}%
\pgfsetdash{}{0pt}%
\pgfsys@defobject{currentmarker}{\pgfqpoint{0.000000in}{-0.048611in}}{\pgfqpoint{0.000000in}{0.000000in}}{%
\pgfpathmoveto{\pgfqpoint{0.000000in}{0.000000in}}%
\pgfpathlineto{\pgfqpoint{0.000000in}{-0.048611in}}%
\pgfusepath{stroke,fill}%
}%
\begin{pgfscope}%
\pgfsys@transformshift{2.300000in}{0.500000in}%
\pgfsys@useobject{currentmarker}{}%
\end{pgfscope}%
\end{pgfscope}%
\begin{pgfscope}%
\pgftext[x=2.300000in,y=0.402778in,,top]{\rmfamily\fontsize{10.000000}{12.000000}\selectfont \(\displaystyle -2\)}%
\end{pgfscope}%
\begin{pgfscope}%
\pgfpathrectangle{\pgfqpoint{0.750000in}{0.500000in}}{\pgfqpoint{4.650000in}{3.020000in}}%
\pgfusepath{clip}%
\pgfsetrectcap%
\pgfsetroundjoin%
\pgfsetlinewidth{0.803000pt}%
\definecolor{currentstroke}{rgb}{0.690196,0.690196,0.690196}%
\pgfsetstrokecolor{currentstroke}%
\pgfsetdash{}{0pt}%
\pgfpathmoveto{\pgfqpoint{3.075000in}{0.500000in}}%
\pgfpathlineto{\pgfqpoint{3.075000in}{3.520000in}}%
\pgfusepath{stroke}%
\end{pgfscope}%
\begin{pgfscope}%
\pgfsetbuttcap%
\pgfsetroundjoin%
\definecolor{currentfill}{rgb}{0.000000,0.000000,0.000000}%
\pgfsetfillcolor{currentfill}%
\pgfsetlinewidth{0.803000pt}%
\definecolor{currentstroke}{rgb}{0.000000,0.000000,0.000000}%
\pgfsetstrokecolor{currentstroke}%
\pgfsetdash{}{0pt}%
\pgfsys@defobject{currentmarker}{\pgfqpoint{0.000000in}{-0.048611in}}{\pgfqpoint{0.000000in}{0.000000in}}{%
\pgfpathmoveto{\pgfqpoint{0.000000in}{0.000000in}}%
\pgfpathlineto{\pgfqpoint{0.000000in}{-0.048611in}}%
\pgfusepath{stroke,fill}%
}%
\begin{pgfscope}%
\pgfsys@transformshift{3.075000in}{0.500000in}%
\pgfsys@useobject{currentmarker}{}%
\end{pgfscope}%
\end{pgfscope}%
\begin{pgfscope}%
\pgftext[x=3.075000in,y=0.402778in,,top]{\rmfamily\fontsize{10.000000}{12.000000}\selectfont \(\displaystyle 0\)}%
\end{pgfscope}%
\begin{pgfscope}%
\pgfpathrectangle{\pgfqpoint{0.750000in}{0.500000in}}{\pgfqpoint{4.650000in}{3.020000in}}%
\pgfusepath{clip}%
\pgfsetrectcap%
\pgfsetroundjoin%
\pgfsetlinewidth{0.803000pt}%
\definecolor{currentstroke}{rgb}{0.690196,0.690196,0.690196}%
\pgfsetstrokecolor{currentstroke}%
\pgfsetdash{}{0pt}%
\pgfpathmoveto{\pgfqpoint{3.850000in}{0.500000in}}%
\pgfpathlineto{\pgfqpoint{3.850000in}{3.520000in}}%
\pgfusepath{stroke}%
\end{pgfscope}%
\begin{pgfscope}%
\pgfsetbuttcap%
\pgfsetroundjoin%
\definecolor{currentfill}{rgb}{0.000000,0.000000,0.000000}%
\pgfsetfillcolor{currentfill}%
\pgfsetlinewidth{0.803000pt}%
\definecolor{currentstroke}{rgb}{0.000000,0.000000,0.000000}%
\pgfsetstrokecolor{currentstroke}%
\pgfsetdash{}{0pt}%
\pgfsys@defobject{currentmarker}{\pgfqpoint{0.000000in}{-0.048611in}}{\pgfqpoint{0.000000in}{0.000000in}}{%
\pgfpathmoveto{\pgfqpoint{0.000000in}{0.000000in}}%
\pgfpathlineto{\pgfqpoint{0.000000in}{-0.048611in}}%
\pgfusepath{stroke,fill}%
}%
\begin{pgfscope}%
\pgfsys@transformshift{3.850000in}{0.500000in}%
\pgfsys@useobject{currentmarker}{}%
\end{pgfscope}%
\end{pgfscope}%
\begin{pgfscope}%
\pgftext[x=3.850000in,y=0.402778in,,top]{\rmfamily\fontsize{10.000000}{12.000000}\selectfont \(\displaystyle 2\)}%
\end{pgfscope}%
\begin{pgfscope}%
\pgfpathrectangle{\pgfqpoint{0.750000in}{0.500000in}}{\pgfqpoint{4.650000in}{3.020000in}}%
\pgfusepath{clip}%
\pgfsetrectcap%
\pgfsetroundjoin%
\pgfsetlinewidth{0.803000pt}%
\definecolor{currentstroke}{rgb}{0.690196,0.690196,0.690196}%
\pgfsetstrokecolor{currentstroke}%
\pgfsetdash{}{0pt}%
\pgfpathmoveto{\pgfqpoint{4.625000in}{0.500000in}}%
\pgfpathlineto{\pgfqpoint{4.625000in}{3.520000in}}%
\pgfusepath{stroke}%
\end{pgfscope}%
\begin{pgfscope}%
\pgfsetbuttcap%
\pgfsetroundjoin%
\definecolor{currentfill}{rgb}{0.000000,0.000000,0.000000}%
\pgfsetfillcolor{currentfill}%
\pgfsetlinewidth{0.803000pt}%
\definecolor{currentstroke}{rgb}{0.000000,0.000000,0.000000}%
\pgfsetstrokecolor{currentstroke}%
\pgfsetdash{}{0pt}%
\pgfsys@defobject{currentmarker}{\pgfqpoint{0.000000in}{-0.048611in}}{\pgfqpoint{0.000000in}{0.000000in}}{%
\pgfpathmoveto{\pgfqpoint{0.000000in}{0.000000in}}%
\pgfpathlineto{\pgfqpoint{0.000000in}{-0.048611in}}%
\pgfusepath{stroke,fill}%
}%
\begin{pgfscope}%
\pgfsys@transformshift{4.625000in}{0.500000in}%
\pgfsys@useobject{currentmarker}{}%
\end{pgfscope}%
\end{pgfscope}%
\begin{pgfscope}%
\pgftext[x=4.625000in,y=0.402778in,,top]{\rmfamily\fontsize{10.000000}{12.000000}\selectfont \(\displaystyle 4\)}%
\end{pgfscope}%
\begin{pgfscope}%
\pgfpathrectangle{\pgfqpoint{0.750000in}{0.500000in}}{\pgfqpoint{4.650000in}{3.020000in}}%
\pgfusepath{clip}%
\pgfsetrectcap%
\pgfsetroundjoin%
\pgfsetlinewidth{0.803000pt}%
\definecolor{currentstroke}{rgb}{0.690196,0.690196,0.690196}%
\pgfsetstrokecolor{currentstroke}%
\pgfsetdash{}{0pt}%
\pgfpathmoveto{\pgfqpoint{5.400000in}{0.500000in}}%
\pgfpathlineto{\pgfqpoint{5.400000in}{3.520000in}}%
\pgfusepath{stroke}%
\end{pgfscope}%
\begin{pgfscope}%
\pgfsetbuttcap%
\pgfsetroundjoin%
\definecolor{currentfill}{rgb}{0.000000,0.000000,0.000000}%
\pgfsetfillcolor{currentfill}%
\pgfsetlinewidth{0.803000pt}%
\definecolor{currentstroke}{rgb}{0.000000,0.000000,0.000000}%
\pgfsetstrokecolor{currentstroke}%
\pgfsetdash{}{0pt}%
\pgfsys@defobject{currentmarker}{\pgfqpoint{0.000000in}{-0.048611in}}{\pgfqpoint{0.000000in}{0.000000in}}{%
\pgfpathmoveto{\pgfqpoint{0.000000in}{0.000000in}}%
\pgfpathlineto{\pgfqpoint{0.000000in}{-0.048611in}}%
\pgfusepath{stroke,fill}%
}%
\begin{pgfscope}%
\pgfsys@transformshift{5.400000in}{0.500000in}%
\pgfsys@useobject{currentmarker}{}%
\end{pgfscope}%
\end{pgfscope}%
\begin{pgfscope}%
\pgftext[x=5.400000in,y=0.402778in,,top]{\rmfamily\fontsize{10.000000}{12.000000}\selectfont \(\displaystyle 6\)}%
\end{pgfscope}%
\begin{pgfscope}%
\pgfpathrectangle{\pgfqpoint{0.750000in}{0.500000in}}{\pgfqpoint{4.650000in}{3.020000in}}%
\pgfusepath{clip}%
\pgfsetrectcap%
\pgfsetroundjoin%
\pgfsetlinewidth{0.803000pt}%
\definecolor{currentstroke}{rgb}{0.690196,0.690196,0.690196}%
\pgfsetstrokecolor{currentstroke}%
\pgfsetdash{}{0pt}%
\pgfpathmoveto{\pgfqpoint{0.750000in}{0.500000in}}%
\pgfpathlineto{\pgfqpoint{5.400000in}{0.500000in}}%
\pgfusepath{stroke}%
\end{pgfscope}%
\begin{pgfscope}%
\pgfsetbuttcap%
\pgfsetroundjoin%
\definecolor{currentfill}{rgb}{0.000000,0.000000,0.000000}%
\pgfsetfillcolor{currentfill}%
\pgfsetlinewidth{0.803000pt}%
\definecolor{currentstroke}{rgb}{0.000000,0.000000,0.000000}%
\pgfsetstrokecolor{currentstroke}%
\pgfsetdash{}{0pt}%
\pgfsys@defobject{currentmarker}{\pgfqpoint{-0.048611in}{0.000000in}}{\pgfqpoint{0.000000in}{0.000000in}}{%
\pgfpathmoveto{\pgfqpoint{0.000000in}{0.000000in}}%
\pgfpathlineto{\pgfqpoint{-0.048611in}{0.000000in}}%
\pgfusepath{stroke,fill}%
}%
\begin{pgfscope}%
\pgfsys@transformshift{0.750000in}{0.500000in}%
\pgfsys@useobject{currentmarker}{}%
\end{pgfscope}%
\end{pgfscope}%
\begin{pgfscope}%
\pgftext[x=0.297838in,y=0.451806in,left,base]{\rmfamily\fontsize{10.000000}{12.000000}\selectfont \(\displaystyle -0.50\)}%
\end{pgfscope}%
\begin{pgfscope}%
\pgfpathrectangle{\pgfqpoint{0.750000in}{0.500000in}}{\pgfqpoint{4.650000in}{3.020000in}}%
\pgfusepath{clip}%
\pgfsetrectcap%
\pgfsetroundjoin%
\pgfsetlinewidth{0.803000pt}%
\definecolor{currentstroke}{rgb}{0.690196,0.690196,0.690196}%
\pgfsetstrokecolor{currentstroke}%
\pgfsetdash{}{0pt}%
\pgfpathmoveto{\pgfqpoint{0.750000in}{0.877500in}}%
\pgfpathlineto{\pgfqpoint{5.400000in}{0.877500in}}%
\pgfusepath{stroke}%
\end{pgfscope}%
\begin{pgfscope}%
\pgfsetbuttcap%
\pgfsetroundjoin%
\definecolor{currentfill}{rgb}{0.000000,0.000000,0.000000}%
\pgfsetfillcolor{currentfill}%
\pgfsetlinewidth{0.803000pt}%
\definecolor{currentstroke}{rgb}{0.000000,0.000000,0.000000}%
\pgfsetstrokecolor{currentstroke}%
\pgfsetdash{}{0pt}%
\pgfsys@defobject{currentmarker}{\pgfqpoint{-0.048611in}{0.000000in}}{\pgfqpoint{0.000000in}{0.000000in}}{%
\pgfpathmoveto{\pgfqpoint{0.000000in}{0.000000in}}%
\pgfpathlineto{\pgfqpoint{-0.048611in}{0.000000in}}%
\pgfusepath{stroke,fill}%
}%
\begin{pgfscope}%
\pgfsys@transformshift{0.750000in}{0.877500in}%
\pgfsys@useobject{currentmarker}{}%
\end{pgfscope}%
\end{pgfscope}%
\begin{pgfscope}%
\pgftext[x=0.297838in,y=0.829306in,left,base]{\rmfamily\fontsize{10.000000}{12.000000}\selectfont \(\displaystyle -0.25\)}%
\end{pgfscope}%
\begin{pgfscope}%
\pgfpathrectangle{\pgfqpoint{0.750000in}{0.500000in}}{\pgfqpoint{4.650000in}{3.020000in}}%
\pgfusepath{clip}%
\pgfsetrectcap%
\pgfsetroundjoin%
\pgfsetlinewidth{0.803000pt}%
\definecolor{currentstroke}{rgb}{0.690196,0.690196,0.690196}%
\pgfsetstrokecolor{currentstroke}%
\pgfsetdash{}{0pt}%
\pgfpathmoveto{\pgfqpoint{0.750000in}{1.255000in}}%
\pgfpathlineto{\pgfqpoint{5.400000in}{1.255000in}}%
\pgfusepath{stroke}%
\end{pgfscope}%
\begin{pgfscope}%
\pgfsetbuttcap%
\pgfsetroundjoin%
\definecolor{currentfill}{rgb}{0.000000,0.000000,0.000000}%
\pgfsetfillcolor{currentfill}%
\pgfsetlinewidth{0.803000pt}%
\definecolor{currentstroke}{rgb}{0.000000,0.000000,0.000000}%
\pgfsetstrokecolor{currentstroke}%
\pgfsetdash{}{0pt}%
\pgfsys@defobject{currentmarker}{\pgfqpoint{-0.048611in}{0.000000in}}{\pgfqpoint{0.000000in}{0.000000in}}{%
\pgfpathmoveto{\pgfqpoint{0.000000in}{0.000000in}}%
\pgfpathlineto{\pgfqpoint{-0.048611in}{0.000000in}}%
\pgfusepath{stroke,fill}%
}%
\begin{pgfscope}%
\pgfsys@transformshift{0.750000in}{1.255000in}%
\pgfsys@useobject{currentmarker}{}%
\end{pgfscope}%
\end{pgfscope}%
\begin{pgfscope}%
\pgftext[x=0.405863in,y=1.206806in,left,base]{\rmfamily\fontsize{10.000000}{12.000000}\selectfont \(\displaystyle 0.00\)}%
\end{pgfscope}%
\begin{pgfscope}%
\pgfpathrectangle{\pgfqpoint{0.750000in}{0.500000in}}{\pgfqpoint{4.650000in}{3.020000in}}%
\pgfusepath{clip}%
\pgfsetrectcap%
\pgfsetroundjoin%
\pgfsetlinewidth{0.803000pt}%
\definecolor{currentstroke}{rgb}{0.690196,0.690196,0.690196}%
\pgfsetstrokecolor{currentstroke}%
\pgfsetdash{}{0pt}%
\pgfpathmoveto{\pgfqpoint{0.750000in}{1.632500in}}%
\pgfpathlineto{\pgfqpoint{5.400000in}{1.632500in}}%
\pgfusepath{stroke}%
\end{pgfscope}%
\begin{pgfscope}%
\pgfsetbuttcap%
\pgfsetroundjoin%
\definecolor{currentfill}{rgb}{0.000000,0.000000,0.000000}%
\pgfsetfillcolor{currentfill}%
\pgfsetlinewidth{0.803000pt}%
\definecolor{currentstroke}{rgb}{0.000000,0.000000,0.000000}%
\pgfsetstrokecolor{currentstroke}%
\pgfsetdash{}{0pt}%
\pgfsys@defobject{currentmarker}{\pgfqpoint{-0.048611in}{0.000000in}}{\pgfqpoint{0.000000in}{0.000000in}}{%
\pgfpathmoveto{\pgfqpoint{0.000000in}{0.000000in}}%
\pgfpathlineto{\pgfqpoint{-0.048611in}{0.000000in}}%
\pgfusepath{stroke,fill}%
}%
\begin{pgfscope}%
\pgfsys@transformshift{0.750000in}{1.632500in}%
\pgfsys@useobject{currentmarker}{}%
\end{pgfscope}%
\end{pgfscope}%
\begin{pgfscope}%
\pgftext[x=0.405863in,y=1.584306in,left,base]{\rmfamily\fontsize{10.000000}{12.000000}\selectfont \(\displaystyle 0.25\)}%
\end{pgfscope}%
\begin{pgfscope}%
\pgfpathrectangle{\pgfqpoint{0.750000in}{0.500000in}}{\pgfqpoint{4.650000in}{3.020000in}}%
\pgfusepath{clip}%
\pgfsetrectcap%
\pgfsetroundjoin%
\pgfsetlinewidth{0.803000pt}%
\definecolor{currentstroke}{rgb}{0.690196,0.690196,0.690196}%
\pgfsetstrokecolor{currentstroke}%
\pgfsetdash{}{0pt}%
\pgfpathmoveto{\pgfqpoint{0.750000in}{2.010000in}}%
\pgfpathlineto{\pgfqpoint{5.400000in}{2.010000in}}%
\pgfusepath{stroke}%
\end{pgfscope}%
\begin{pgfscope}%
\pgfsetbuttcap%
\pgfsetroundjoin%
\definecolor{currentfill}{rgb}{0.000000,0.000000,0.000000}%
\pgfsetfillcolor{currentfill}%
\pgfsetlinewidth{0.803000pt}%
\definecolor{currentstroke}{rgb}{0.000000,0.000000,0.000000}%
\pgfsetstrokecolor{currentstroke}%
\pgfsetdash{}{0pt}%
\pgfsys@defobject{currentmarker}{\pgfqpoint{-0.048611in}{0.000000in}}{\pgfqpoint{0.000000in}{0.000000in}}{%
\pgfpathmoveto{\pgfqpoint{0.000000in}{0.000000in}}%
\pgfpathlineto{\pgfqpoint{-0.048611in}{0.000000in}}%
\pgfusepath{stroke,fill}%
}%
\begin{pgfscope}%
\pgfsys@transformshift{0.750000in}{2.010000in}%
\pgfsys@useobject{currentmarker}{}%
\end{pgfscope}%
\end{pgfscope}%
\begin{pgfscope}%
\pgftext[x=0.405863in,y=1.961806in,left,base]{\rmfamily\fontsize{10.000000}{12.000000}\selectfont \(\displaystyle 0.50\)}%
\end{pgfscope}%
\begin{pgfscope}%
\pgfpathrectangle{\pgfqpoint{0.750000in}{0.500000in}}{\pgfqpoint{4.650000in}{3.020000in}}%
\pgfusepath{clip}%
\pgfsetrectcap%
\pgfsetroundjoin%
\pgfsetlinewidth{0.803000pt}%
\definecolor{currentstroke}{rgb}{0.690196,0.690196,0.690196}%
\pgfsetstrokecolor{currentstroke}%
\pgfsetdash{}{0pt}%
\pgfpathmoveto{\pgfqpoint{0.750000in}{2.387500in}}%
\pgfpathlineto{\pgfqpoint{5.400000in}{2.387500in}}%
\pgfusepath{stroke}%
\end{pgfscope}%
\begin{pgfscope}%
\pgfsetbuttcap%
\pgfsetroundjoin%
\definecolor{currentfill}{rgb}{0.000000,0.000000,0.000000}%
\pgfsetfillcolor{currentfill}%
\pgfsetlinewidth{0.803000pt}%
\definecolor{currentstroke}{rgb}{0.000000,0.000000,0.000000}%
\pgfsetstrokecolor{currentstroke}%
\pgfsetdash{}{0pt}%
\pgfsys@defobject{currentmarker}{\pgfqpoint{-0.048611in}{0.000000in}}{\pgfqpoint{0.000000in}{0.000000in}}{%
\pgfpathmoveto{\pgfqpoint{0.000000in}{0.000000in}}%
\pgfpathlineto{\pgfqpoint{-0.048611in}{0.000000in}}%
\pgfusepath{stroke,fill}%
}%
\begin{pgfscope}%
\pgfsys@transformshift{0.750000in}{2.387500in}%
\pgfsys@useobject{currentmarker}{}%
\end{pgfscope}%
\end{pgfscope}%
\begin{pgfscope}%
\pgftext[x=0.405863in,y=2.339306in,left,base]{\rmfamily\fontsize{10.000000}{12.000000}\selectfont \(\displaystyle 0.75\)}%
\end{pgfscope}%
\begin{pgfscope}%
\pgfpathrectangle{\pgfqpoint{0.750000in}{0.500000in}}{\pgfqpoint{4.650000in}{3.020000in}}%
\pgfusepath{clip}%
\pgfsetrectcap%
\pgfsetroundjoin%
\pgfsetlinewidth{0.803000pt}%
\definecolor{currentstroke}{rgb}{0.690196,0.690196,0.690196}%
\pgfsetstrokecolor{currentstroke}%
\pgfsetdash{}{0pt}%
\pgfpathmoveto{\pgfqpoint{0.750000in}{2.765000in}}%
\pgfpathlineto{\pgfqpoint{5.400000in}{2.765000in}}%
\pgfusepath{stroke}%
\end{pgfscope}%
\begin{pgfscope}%
\pgfsetbuttcap%
\pgfsetroundjoin%
\definecolor{currentfill}{rgb}{0.000000,0.000000,0.000000}%
\pgfsetfillcolor{currentfill}%
\pgfsetlinewidth{0.803000pt}%
\definecolor{currentstroke}{rgb}{0.000000,0.000000,0.000000}%
\pgfsetstrokecolor{currentstroke}%
\pgfsetdash{}{0pt}%
\pgfsys@defobject{currentmarker}{\pgfqpoint{-0.048611in}{0.000000in}}{\pgfqpoint{0.000000in}{0.000000in}}{%
\pgfpathmoveto{\pgfqpoint{0.000000in}{0.000000in}}%
\pgfpathlineto{\pgfqpoint{-0.048611in}{0.000000in}}%
\pgfusepath{stroke,fill}%
}%
\begin{pgfscope}%
\pgfsys@transformshift{0.750000in}{2.765000in}%
\pgfsys@useobject{currentmarker}{}%
\end{pgfscope}%
\end{pgfscope}%
\begin{pgfscope}%
\pgftext[x=0.405863in,y=2.716806in,left,base]{\rmfamily\fontsize{10.000000}{12.000000}\selectfont \(\displaystyle 1.00\)}%
\end{pgfscope}%
\begin{pgfscope}%
\pgfpathrectangle{\pgfqpoint{0.750000in}{0.500000in}}{\pgfqpoint{4.650000in}{3.020000in}}%
\pgfusepath{clip}%
\pgfsetrectcap%
\pgfsetroundjoin%
\pgfsetlinewidth{0.803000pt}%
\definecolor{currentstroke}{rgb}{0.690196,0.690196,0.690196}%
\pgfsetstrokecolor{currentstroke}%
\pgfsetdash{}{0pt}%
\pgfpathmoveto{\pgfqpoint{0.750000in}{3.142500in}}%
\pgfpathlineto{\pgfqpoint{5.400000in}{3.142500in}}%
\pgfusepath{stroke}%
\end{pgfscope}%
\begin{pgfscope}%
\pgfsetbuttcap%
\pgfsetroundjoin%
\definecolor{currentfill}{rgb}{0.000000,0.000000,0.000000}%
\pgfsetfillcolor{currentfill}%
\pgfsetlinewidth{0.803000pt}%
\definecolor{currentstroke}{rgb}{0.000000,0.000000,0.000000}%
\pgfsetstrokecolor{currentstroke}%
\pgfsetdash{}{0pt}%
\pgfsys@defobject{currentmarker}{\pgfqpoint{-0.048611in}{0.000000in}}{\pgfqpoint{0.000000in}{0.000000in}}{%
\pgfpathmoveto{\pgfqpoint{0.000000in}{0.000000in}}%
\pgfpathlineto{\pgfqpoint{-0.048611in}{0.000000in}}%
\pgfusepath{stroke,fill}%
}%
\begin{pgfscope}%
\pgfsys@transformshift{0.750000in}{3.142500in}%
\pgfsys@useobject{currentmarker}{}%
\end{pgfscope}%
\end{pgfscope}%
\begin{pgfscope}%
\pgftext[x=0.405863in,y=3.094306in,left,base]{\rmfamily\fontsize{10.000000}{12.000000}\selectfont \(\displaystyle 1.25\)}%
\end{pgfscope}%
\begin{pgfscope}%
\pgfpathrectangle{\pgfqpoint{0.750000in}{0.500000in}}{\pgfqpoint{4.650000in}{3.020000in}}%
\pgfusepath{clip}%
\pgfsetrectcap%
\pgfsetroundjoin%
\pgfsetlinewidth{0.803000pt}%
\definecolor{currentstroke}{rgb}{0.690196,0.690196,0.690196}%
\pgfsetstrokecolor{currentstroke}%
\pgfsetdash{}{0pt}%
\pgfpathmoveto{\pgfqpoint{0.750000in}{3.520000in}}%
\pgfpathlineto{\pgfqpoint{5.400000in}{3.520000in}}%
\pgfusepath{stroke}%
\end{pgfscope}%
\begin{pgfscope}%
\pgfsetbuttcap%
\pgfsetroundjoin%
\definecolor{currentfill}{rgb}{0.000000,0.000000,0.000000}%
\pgfsetfillcolor{currentfill}%
\pgfsetlinewidth{0.803000pt}%
\definecolor{currentstroke}{rgb}{0.000000,0.000000,0.000000}%
\pgfsetstrokecolor{currentstroke}%
\pgfsetdash{}{0pt}%
\pgfsys@defobject{currentmarker}{\pgfqpoint{-0.048611in}{0.000000in}}{\pgfqpoint{0.000000in}{0.000000in}}{%
\pgfpathmoveto{\pgfqpoint{0.000000in}{0.000000in}}%
\pgfpathlineto{\pgfqpoint{-0.048611in}{0.000000in}}%
\pgfusepath{stroke,fill}%
}%
\begin{pgfscope}%
\pgfsys@transformshift{0.750000in}{3.520000in}%
\pgfsys@useobject{currentmarker}{}%
\end{pgfscope}%
\end{pgfscope}%
\begin{pgfscope}%
\pgftext[x=0.405863in,y=3.471806in,left,base]{\rmfamily\fontsize{10.000000}{12.000000}\selectfont \(\displaystyle 1.50\)}%
\end{pgfscope}%
\begin{pgfscope}%
\pgfpathrectangle{\pgfqpoint{0.750000in}{0.500000in}}{\pgfqpoint{4.650000in}{3.020000in}}%
\pgfusepath{clip}%
\pgfsetrectcap%
\pgfsetroundjoin%
\pgfsetlinewidth{1.505625pt}%
\definecolor{currentstroke}{rgb}{0.121569,0.466667,0.705882}%
\pgfsetstrokecolor{currentstroke}%
\pgfsetdash{}{0pt}%
\pgfpathmoveto{\pgfqpoint{0.750000in}{0.590600in}}%
\pgfpathlineto{\pgfqpoint{0.838438in}{0.752874in}}%
\pgfpathlineto{\pgfqpoint{0.926877in}{0.908855in}}%
\pgfpathlineto{\pgfqpoint{1.015315in}{1.058544in}}%
\pgfpathlineto{\pgfqpoint{1.103754in}{1.201941in}}%
\pgfpathlineto{\pgfqpoint{1.187538in}{1.331987in}}%
\pgfpathlineto{\pgfqpoint{1.271321in}{1.456385in}}%
\pgfpathlineto{\pgfqpoint{1.355105in}{1.575136in}}%
\pgfpathlineto{\pgfqpoint{1.434234in}{1.682104in}}%
\pgfpathlineto{\pgfqpoint{1.513363in}{1.784035in}}%
\pgfpathlineto{\pgfqpoint{1.592492in}{1.880929in}}%
\pgfpathlineto{\pgfqpoint{1.671622in}{1.972785in}}%
\pgfpathlineto{\pgfqpoint{1.746096in}{2.054636in}}%
\pgfpathlineto{\pgfqpoint{1.820571in}{2.132026in}}%
\pgfpathlineto{\pgfqpoint{1.895045in}{2.204953in}}%
\pgfpathlineto{\pgfqpoint{1.969520in}{2.273418in}}%
\pgfpathlineto{\pgfqpoint{2.039339in}{2.333552in}}%
\pgfpathlineto{\pgfqpoint{2.109159in}{2.389764in}}%
\pgfpathlineto{\pgfqpoint{2.178979in}{2.442054in}}%
\pgfpathlineto{\pgfqpoint{2.248799in}{2.490422in}}%
\pgfpathlineto{\pgfqpoint{2.318619in}{2.534869in}}%
\pgfpathlineto{\pgfqpoint{2.383784in}{2.572814in}}%
\pgfpathlineto{\pgfqpoint{2.448949in}{2.607343in}}%
\pgfpathlineto{\pgfqpoint{2.514114in}{2.638456in}}%
\pgfpathlineto{\pgfqpoint{2.579279in}{2.666152in}}%
\pgfpathlineto{\pgfqpoint{2.644444in}{2.690432in}}%
\pgfpathlineto{\pgfqpoint{2.709610in}{2.711296in}}%
\pgfpathlineto{\pgfqpoint{2.774775in}{2.728743in}}%
\pgfpathlineto{\pgfqpoint{2.839940in}{2.742775in}}%
\pgfpathlineto{\pgfqpoint{2.900450in}{2.752745in}}%
\pgfpathlineto{\pgfqpoint{2.960961in}{2.759769in}}%
\pgfpathlineto{\pgfqpoint{3.021471in}{2.763847in}}%
\pgfpathlineto{\pgfqpoint{3.081982in}{2.764980in}}%
\pgfpathlineto{\pgfqpoint{3.142492in}{2.763168in}}%
\pgfpathlineto{\pgfqpoint{3.203003in}{2.758409in}}%
\pgfpathlineto{\pgfqpoint{3.263514in}{2.750705in}}%
\pgfpathlineto{\pgfqpoint{3.324024in}{2.740055in}}%
\pgfpathlineto{\pgfqpoint{3.384535in}{2.726460in}}%
\pgfpathlineto{\pgfqpoint{3.449700in}{2.708524in}}%
\pgfpathlineto{\pgfqpoint{3.514865in}{2.687173in}}%
\pgfpathlineto{\pgfqpoint{3.580030in}{2.662405in}}%
\pgfpathlineto{\pgfqpoint{3.645195in}{2.634220in}}%
\pgfpathlineto{\pgfqpoint{3.710360in}{2.602620in}}%
\pgfpathlineto{\pgfqpoint{3.775526in}{2.567603in}}%
\pgfpathlineto{\pgfqpoint{3.840691in}{2.529169in}}%
\pgfpathlineto{\pgfqpoint{3.905856in}{2.487320in}}%
\pgfpathlineto{\pgfqpoint{3.975676in}{2.438690in}}%
\pgfpathlineto{\pgfqpoint{4.045495in}{2.386138in}}%
\pgfpathlineto{\pgfqpoint{4.115315in}{2.329665in}}%
\pgfpathlineto{\pgfqpoint{4.185135in}{2.269270in}}%
\pgfpathlineto{\pgfqpoint{4.254955in}{2.204953in}}%
\pgfpathlineto{\pgfqpoint{4.329429in}{2.132026in}}%
\pgfpathlineto{\pgfqpoint{4.403904in}{2.054636in}}%
\pgfpathlineto{\pgfqpoint{4.478378in}{1.972785in}}%
\pgfpathlineto{\pgfqpoint{4.552853in}{1.886471in}}%
\pgfpathlineto{\pgfqpoint{4.631982in}{1.789874in}}%
\pgfpathlineto{\pgfqpoint{4.711111in}{1.688240in}}%
\pgfpathlineto{\pgfqpoint{4.790240in}{1.581568in}}%
\pgfpathlineto{\pgfqpoint{4.869369in}{1.469859in}}%
\pgfpathlineto{\pgfqpoint{4.953153in}{1.346088in}}%
\pgfpathlineto{\pgfqpoint{5.036937in}{1.216670in}}%
\pgfpathlineto{\pgfqpoint{5.120721in}{1.081604in}}%
\pgfpathlineto{\pgfqpoint{5.204505in}{0.940892in}}%
\pgfpathlineto{\pgfqpoint{5.292943in}{0.786235in}}%
\pgfpathlineto{\pgfqpoint{5.381381in}{0.625286in}}%
\pgfpathlineto{\pgfqpoint{5.400000in}{0.590600in}}%
\pgfpathlineto{\pgfqpoint{5.400000in}{0.590600in}}%
\pgfusepath{stroke}%
\end{pgfscope}%
\begin{pgfscope}%
\pgfpathrectangle{\pgfqpoint{0.750000in}{0.500000in}}{\pgfqpoint{4.650000in}{3.020000in}}%
\pgfusepath{clip}%
\pgfsetrectcap%
\pgfsetroundjoin%
\pgfsetlinewidth{1.505625pt}%
\definecolor{currentstroke}{rgb}{1.000000,0.498039,0.054902}%
\pgfsetstrokecolor{currentstroke}%
\pgfsetdash{}{0pt}%
\pgfpathmoveto{\pgfqpoint{0.827588in}{3.533889in}}%
\pgfpathlineto{\pgfqpoint{0.861712in}{3.191891in}}%
\pgfpathlineto{\pgfqpoint{0.898949in}{2.846771in}}%
\pgfpathlineto{\pgfqpoint{0.936186in}{2.529771in}}%
\pgfpathlineto{\pgfqpoint{0.973423in}{2.239748in}}%
\pgfpathlineto{\pgfqpoint{1.010661in}{1.975578in}}%
\pgfpathlineto{\pgfqpoint{1.043243in}{1.764769in}}%
\pgfpathlineto{\pgfqpoint{1.075826in}{1.572182in}}%
\pgfpathlineto{\pgfqpoint{1.108408in}{1.397102in}}%
\pgfpathlineto{\pgfqpoint{1.140991in}{1.238824in}}%
\pgfpathlineto{\pgfqpoint{1.173574in}{1.096658in}}%
\pgfpathlineto{\pgfqpoint{1.201502in}{0.987108in}}%
\pgfpathlineto{\pgfqpoint{1.229429in}{0.888472in}}%
\pgfpathlineto{\pgfqpoint{1.257357in}{0.800334in}}%
\pgfpathlineto{\pgfqpoint{1.285285in}{0.722287in}}%
\pgfpathlineto{\pgfqpoint{1.313213in}{0.653925in}}%
\pgfpathlineto{\pgfqpoint{1.336486in}{0.604070in}}%
\pgfpathlineto{\pgfqpoint{1.359760in}{0.560439in}}%
\pgfpathlineto{\pgfqpoint{1.383033in}{0.522810in}}%
\pgfpathlineto{\pgfqpoint{1.406306in}{0.490961in}}%
\pgfpathlineto{\pgfqpoint{1.410270in}{0.486111in}}%
\pgfpathmoveto{\pgfqpoint{1.688103in}{0.486111in}}%
\pgfpathlineto{\pgfqpoint{1.713514in}{0.513494in}}%
\pgfpathlineto{\pgfqpoint{1.741441in}{0.547692in}}%
\pgfpathlineto{\pgfqpoint{1.774024in}{0.592606in}}%
\pgfpathlineto{\pgfqpoint{1.811261in}{0.649980in}}%
\pgfpathlineto{\pgfqpoint{1.848498in}{0.713150in}}%
\pgfpathlineto{\pgfqpoint{1.890390in}{0.790324in}}%
\pgfpathlineto{\pgfqpoint{1.936937in}{0.882593in}}%
\pgfpathlineto{\pgfqpoint{1.992793in}{1.000742in}}%
\pgfpathlineto{\pgfqpoint{2.057958in}{1.146321in}}%
\pgfpathlineto{\pgfqpoint{2.146396in}{1.352158in}}%
\pgfpathlineto{\pgfqpoint{2.341892in}{1.809778in}}%
\pgfpathlineto{\pgfqpoint{2.416366in}{1.974775in}}%
\pgfpathlineto{\pgfqpoint{2.476877in}{2.101674in}}%
\pgfpathlineto{\pgfqpoint{2.532733in}{2.211772in}}%
\pgfpathlineto{\pgfqpoint{2.583934in}{2.305857in}}%
\pgfpathlineto{\pgfqpoint{2.630480in}{2.385095in}}%
\pgfpathlineto{\pgfqpoint{2.672372in}{2.450866in}}%
\pgfpathlineto{\pgfqpoint{2.714264in}{2.511055in}}%
\pgfpathlineto{\pgfqpoint{2.751502in}{2.559637in}}%
\pgfpathlineto{\pgfqpoint{2.788739in}{2.603398in}}%
\pgfpathlineto{\pgfqpoint{2.825976in}{2.642178in}}%
\pgfpathlineto{\pgfqpoint{2.858559in}{2.671912in}}%
\pgfpathlineto{\pgfqpoint{2.891141in}{2.697640in}}%
\pgfpathlineto{\pgfqpoint{2.923724in}{2.719292in}}%
\pgfpathlineto{\pgfqpoint{2.956306in}{2.736808in}}%
\pgfpathlineto{\pgfqpoint{2.984234in}{2.748494in}}%
\pgfpathlineto{\pgfqpoint{3.012162in}{2.757081in}}%
\pgfpathlineto{\pgfqpoint{3.040090in}{2.762555in}}%
\pgfpathlineto{\pgfqpoint{3.068018in}{2.764902in}}%
\pgfpathlineto{\pgfqpoint{3.095946in}{2.764120in}}%
\pgfpathlineto{\pgfqpoint{3.123874in}{2.760208in}}%
\pgfpathlineto{\pgfqpoint{3.151802in}{2.753176in}}%
\pgfpathlineto{\pgfqpoint{3.179730in}{2.743037in}}%
\pgfpathlineto{\pgfqpoint{3.207658in}{2.729811in}}%
\pgfpathlineto{\pgfqpoint{3.235586in}{2.713526in}}%
\pgfpathlineto{\pgfqpoint{3.268168in}{2.690703in}}%
\pgfpathlineto{\pgfqpoint{3.300751in}{2.663822in}}%
\pgfpathlineto{\pgfqpoint{3.333333in}{2.632958in}}%
\pgfpathlineto{\pgfqpoint{3.365916in}{2.598197in}}%
\pgfpathlineto{\pgfqpoint{3.403153in}{2.553824in}}%
\pgfpathlineto{\pgfqpoint{3.440390in}{2.504653in}}%
\pgfpathlineto{\pgfqpoint{3.482282in}{2.443828in}}%
\pgfpathlineto{\pgfqpoint{3.524174in}{2.377456in}}%
\pgfpathlineto{\pgfqpoint{3.570721in}{2.297595in}}%
\pgfpathlineto{\pgfqpoint{3.617267in}{2.211772in}}%
\pgfpathlineto{\pgfqpoint{3.668468in}{2.111123in}}%
\pgfpathlineto{\pgfqpoint{3.724324in}{1.994761in}}%
\pgfpathlineto{\pgfqpoint{3.789489in}{1.851799in}}%
\pgfpathlineto{\pgfqpoint{3.868619in}{1.670458in}}%
\pgfpathlineto{\pgfqpoint{4.026877in}{1.297404in}}%
\pgfpathlineto{\pgfqpoint{4.115315in}{1.093546in}}%
\pgfpathlineto{\pgfqpoint{4.180480in}{0.950644in}}%
\pgfpathlineto{\pgfqpoint{4.236336in}{0.835672in}}%
\pgfpathlineto{\pgfqpoint{4.282883in}{0.746702in}}%
\pgfpathlineto{\pgfqpoint{4.324775in}{0.673026in}}%
\pgfpathlineto{\pgfqpoint{4.362012in}{0.613400in}}%
\pgfpathlineto{\pgfqpoint{4.394595in}{0.566304in}}%
\pgfpathlineto{\pgfqpoint{4.427177in}{0.524432in}}%
\pgfpathlineto{\pgfqpoint{4.455105in}{0.493058in}}%
\pgfpathlineto{\pgfqpoint{4.461897in}{0.486111in}}%
\pgfpathmoveto{\pgfqpoint{4.739730in}{0.486111in}}%
\pgfpathlineto{\pgfqpoint{4.762312in}{0.515985in}}%
\pgfpathlineto{\pgfqpoint{4.785586in}{0.552440in}}%
\pgfpathlineto{\pgfqpoint{4.808859in}{0.594853in}}%
\pgfpathlineto{\pgfqpoint{4.832132in}{0.643445in}}%
\pgfpathlineto{\pgfqpoint{4.855405in}{0.698443in}}%
\pgfpathlineto{\pgfqpoint{4.883333in}{0.773217in}}%
\pgfpathlineto{\pgfqpoint{4.911261in}{0.857946in}}%
\pgfpathlineto{\pgfqpoint{4.939189in}{0.953037in}}%
\pgfpathlineto{\pgfqpoint{4.967117in}{1.058903in}}%
\pgfpathlineto{\pgfqpoint{4.995045in}{1.175963in}}%
\pgfpathlineto{\pgfqpoint{5.027628in}{1.327252in}}%
\pgfpathlineto{\pgfqpoint{5.060210in}{1.495045in}}%
\pgfpathlineto{\pgfqpoint{5.092793in}{1.680042in}}%
\pgfpathlineto{\pgfqpoint{5.125375in}{1.882953in}}%
\pgfpathlineto{\pgfqpoint{5.157958in}{2.104500in}}%
\pgfpathlineto{\pgfqpoint{5.190541in}{2.345417in}}%
\pgfpathlineto{\pgfqpoint{5.227778in}{2.645424in}}%
\pgfpathlineto{\pgfqpoint{5.265015in}{2.972834in}}%
\pgfpathlineto{\pgfqpoint{5.302252in}{3.328797in}}%
\pgfpathlineto{\pgfqpoint{5.322412in}{3.533889in}}%
\pgfpathlineto{\pgfqpoint{5.322412in}{3.533889in}}%
\pgfusepath{stroke}%
\end{pgfscope}%
\begin{pgfscope}%
\pgfpathrectangle{\pgfqpoint{0.750000in}{0.500000in}}{\pgfqpoint{4.650000in}{3.020000in}}%
\pgfusepath{clip}%
\pgfsetrectcap%
\pgfsetroundjoin%
\pgfsetlinewidth{1.505625pt}%
\definecolor{currentstroke}{rgb}{0.172549,0.627451,0.172549}%
\pgfsetstrokecolor{currentstroke}%
\pgfsetdash{}{0pt}%
\pgfpathmoveto{\pgfqpoint{1.090177in}{0.486111in}}%
\pgfpathlineto{\pgfqpoint{1.113063in}{0.885853in}}%
\pgfpathlineto{\pgfqpoint{1.136336in}{1.238702in}}%
\pgfpathlineto{\pgfqpoint{1.159610in}{1.541575in}}%
\pgfpathlineto{\pgfqpoint{1.182883in}{1.798347in}}%
\pgfpathlineto{\pgfqpoint{1.201502in}{1.973063in}}%
\pgfpathlineto{\pgfqpoint{1.220120in}{2.122452in}}%
\pgfpathlineto{\pgfqpoint{1.238739in}{2.248257in}}%
\pgfpathlineto{\pgfqpoint{1.257357in}{2.352146in}}%
\pgfpathlineto{\pgfqpoint{1.275976in}{2.435723in}}%
\pgfpathlineto{\pgfqpoint{1.289940in}{2.486004in}}%
\pgfpathlineto{\pgfqpoint{1.303904in}{2.526350in}}%
\pgfpathlineto{\pgfqpoint{1.317868in}{2.557365in}}%
\pgfpathlineto{\pgfqpoint{1.331832in}{2.579635in}}%
\pgfpathlineto{\pgfqpoint{1.341141in}{2.589903in}}%
\pgfpathlineto{\pgfqpoint{1.350450in}{2.596698in}}%
\pgfpathlineto{\pgfqpoint{1.359760in}{2.600181in}}%
\pgfpathlineto{\pgfqpoint{1.369069in}{2.600505in}}%
\pgfpathlineto{\pgfqpoint{1.378378in}{2.597824in}}%
\pgfpathlineto{\pgfqpoint{1.387688in}{2.592285in}}%
\pgfpathlineto{\pgfqpoint{1.401652in}{2.578935in}}%
\pgfpathlineto{\pgfqpoint{1.415616in}{2.559954in}}%
\pgfpathlineto{\pgfqpoint{1.429580in}{2.535800in}}%
\pgfpathlineto{\pgfqpoint{1.448198in}{2.496304in}}%
\pgfpathlineto{\pgfqpoint{1.466817in}{2.449386in}}%
\pgfpathlineto{\pgfqpoint{1.490090in}{2.381739in}}%
\pgfpathlineto{\pgfqpoint{1.518018in}{2.289639in}}%
\pgfpathlineto{\pgfqpoint{1.550601in}{2.170620in}}%
\pgfpathlineto{\pgfqpoint{1.592492in}{2.005400in}}%
\pgfpathlineto{\pgfqpoint{1.746096in}{1.387807in}}%
\pgfpathlineto{\pgfqpoint{1.783333in}{1.255023in}}%
\pgfpathlineto{\pgfqpoint{1.815916in}{1.148721in}}%
\pgfpathlineto{\pgfqpoint{1.848498in}{1.052966in}}%
\pgfpathlineto{\pgfqpoint{1.876426in}{0.980048in}}%
\pgfpathlineto{\pgfqpoint{1.904354in}{0.916104in}}%
\pgfpathlineto{\pgfqpoint{1.927628in}{0.869943in}}%
\pgfpathlineto{\pgfqpoint{1.950901in}{0.830426in}}%
\pgfpathlineto{\pgfqpoint{1.974174in}{0.797655in}}%
\pgfpathlineto{\pgfqpoint{1.992793in}{0.776335in}}%
\pgfpathlineto{\pgfqpoint{2.011411in}{0.759374in}}%
\pgfpathlineto{\pgfqpoint{2.030030in}{0.746763in}}%
\pgfpathlineto{\pgfqpoint{2.048649in}{0.738474in}}%
\pgfpathlineto{\pgfqpoint{2.067267in}{0.734462in}}%
\pgfpathlineto{\pgfqpoint{2.081231in}{0.734226in}}%
\pgfpathlineto{\pgfqpoint{2.095195in}{0.736335in}}%
\pgfpathlineto{\pgfqpoint{2.113814in}{0.742731in}}%
\pgfpathlineto{\pgfqpoint{2.132432in}{0.753142in}}%
\pgfpathlineto{\pgfqpoint{2.151051in}{0.767460in}}%
\pgfpathlineto{\pgfqpoint{2.169670in}{0.785572in}}%
\pgfpathlineto{\pgfqpoint{2.192943in}{0.813353in}}%
\pgfpathlineto{\pgfqpoint{2.216216in}{0.846596in}}%
\pgfpathlineto{\pgfqpoint{2.239489in}{0.885011in}}%
\pgfpathlineto{\pgfqpoint{2.267417in}{0.937503in}}%
\pgfpathlineto{\pgfqpoint{2.295345in}{0.996436in}}%
\pgfpathlineto{\pgfqpoint{2.327928in}{1.072538in}}%
\pgfpathlineto{\pgfqpoint{2.365165in}{1.167991in}}%
\pgfpathlineto{\pgfqpoint{2.407057in}{1.284359in}}%
\pgfpathlineto{\pgfqpoint{2.458258in}{1.436357in}}%
\pgfpathlineto{\pgfqpoint{2.532733in}{1.668581in}}%
\pgfpathlineto{\pgfqpoint{2.644444in}{2.016744in}}%
\pgfpathlineto{\pgfqpoint{2.700300in}{2.179902in}}%
\pgfpathlineto{\pgfqpoint{2.746847in}{2.305835in}}%
\pgfpathlineto{\pgfqpoint{2.784084in}{2.398312in}}%
\pgfpathlineto{\pgfqpoint{2.821321in}{2.482255in}}%
\pgfpathlineto{\pgfqpoint{2.853904in}{2.547936in}}%
\pgfpathlineto{\pgfqpoint{2.881832in}{2.597998in}}%
\pgfpathlineto{\pgfqpoint{2.909760in}{2.641964in}}%
\pgfpathlineto{\pgfqpoint{2.937688in}{2.679551in}}%
\pgfpathlineto{\pgfqpoint{2.960961in}{2.705829in}}%
\pgfpathlineto{\pgfqpoint{2.984234in}{2.727395in}}%
\pgfpathlineto{\pgfqpoint{3.007508in}{2.744156in}}%
\pgfpathlineto{\pgfqpoint{3.026126in}{2.754054in}}%
\pgfpathlineto{\pgfqpoint{3.044745in}{2.760801in}}%
\pgfpathlineto{\pgfqpoint{3.063363in}{2.764379in}}%
\pgfpathlineto{\pgfqpoint{3.081982in}{2.764776in}}%
\pgfpathlineto{\pgfqpoint{3.100601in}{2.761993in}}%
\pgfpathlineto{\pgfqpoint{3.119219in}{2.756037in}}%
\pgfpathlineto{\pgfqpoint{3.137838in}{2.746924in}}%
\pgfpathlineto{\pgfqpoint{3.156456in}{2.734681in}}%
\pgfpathlineto{\pgfqpoint{3.179730in}{2.715027in}}%
\pgfpathlineto{\pgfqpoint{3.203003in}{2.690621in}}%
\pgfpathlineto{\pgfqpoint{3.226276in}{2.661571in}}%
\pgfpathlineto{\pgfqpoint{3.249550in}{2.628004in}}%
\pgfpathlineto{\pgfqpoint{3.277477in}{2.581973in}}%
\pgfpathlineto{\pgfqpoint{3.305405in}{2.529948in}}%
\pgfpathlineto{\pgfqpoint{3.337988in}{2.462124in}}%
\pgfpathlineto{\pgfqpoint{3.370571in}{2.387198in}}%
\pgfpathlineto{\pgfqpoint{3.407808in}{2.293729in}}%
\pgfpathlineto{\pgfqpoint{3.449700in}{2.179902in}}%
\pgfpathlineto{\pgfqpoint{3.500901in}{2.030744in}}%
\pgfpathlineto{\pgfqpoint{3.566066in}{1.830099in}}%
\pgfpathlineto{\pgfqpoint{3.719670in}{1.352333in}}%
\pgfpathlineto{\pgfqpoint{3.766216in}{1.218657in}}%
\pgfpathlineto{\pgfqpoint{3.808108in}{1.107346in}}%
\pgfpathlineto{\pgfqpoint{3.845345in}{1.017410in}}%
\pgfpathlineto{\pgfqpoint{3.877928in}{0.946894in}}%
\pgfpathlineto{\pgfqpoint{3.905856in}{0.893288in}}%
\pgfpathlineto{\pgfqpoint{3.933784in}{0.846596in}}%
\pgfpathlineto{\pgfqpoint{3.957057in}{0.813353in}}%
\pgfpathlineto{\pgfqpoint{3.980330in}{0.785572in}}%
\pgfpathlineto{\pgfqpoint{3.998949in}{0.767460in}}%
\pgfpathlineto{\pgfqpoint{4.017568in}{0.753142in}}%
\pgfpathlineto{\pgfqpoint{4.036186in}{0.742731in}}%
\pgfpathlineto{\pgfqpoint{4.054805in}{0.736335in}}%
\pgfpathlineto{\pgfqpoint{4.073423in}{0.734043in}}%
\pgfpathlineto{\pgfqpoint{4.087387in}{0.735067in}}%
\pgfpathlineto{\pgfqpoint{4.101351in}{0.738474in}}%
\pgfpathlineto{\pgfqpoint{4.119970in}{0.746763in}}%
\pgfpathlineto{\pgfqpoint{4.138589in}{0.759374in}}%
\pgfpathlineto{\pgfqpoint{4.157207in}{0.776335in}}%
\pgfpathlineto{\pgfqpoint{4.175826in}{0.797655in}}%
\pgfpathlineto{\pgfqpoint{4.194444in}{0.823330in}}%
\pgfpathlineto{\pgfqpoint{4.217718in}{0.861504in}}%
\pgfpathlineto{\pgfqpoint{4.240991in}{0.906347in}}%
\pgfpathlineto{\pgfqpoint{4.264264in}{0.957716in}}%
\pgfpathlineto{\pgfqpoint{4.292192in}{1.027691in}}%
\pgfpathlineto{\pgfqpoint{4.320120in}{1.106328in}}%
\pgfpathlineto{\pgfqpoint{4.352703in}{1.208248in}}%
\pgfpathlineto{\pgfqpoint{4.389940in}{1.336727in}}%
\pgfpathlineto{\pgfqpoint{4.431832in}{1.493990in}}%
\pgfpathlineto{\pgfqpoint{4.483033in}{1.699099in}}%
\pgfpathlineto{\pgfqpoint{4.599399in}{2.170620in}}%
\pgfpathlineto{\pgfqpoint{4.636637in}{2.305707in}}%
\pgfpathlineto{\pgfqpoint{4.664565in}{2.395993in}}%
\pgfpathlineto{\pgfqpoint{4.687838in}{2.461758in}}%
\pgfpathlineto{\pgfqpoint{4.706456in}{2.506911in}}%
\pgfpathlineto{\pgfqpoint{4.725075in}{2.544399in}}%
\pgfpathlineto{\pgfqpoint{4.739039in}{2.566878in}}%
\pgfpathlineto{\pgfqpoint{4.753003in}{2.584034in}}%
\pgfpathlineto{\pgfqpoint{4.766967in}{2.595403in}}%
\pgfpathlineto{\pgfqpoint{4.776276in}{2.599531in}}%
\pgfpathlineto{\pgfqpoint{4.785586in}{2.600728in}}%
\pgfpathlineto{\pgfqpoint{4.794895in}{2.598844in}}%
\pgfpathlineto{\pgfqpoint{4.804204in}{2.593725in}}%
\pgfpathlineto{\pgfqpoint{4.813514in}{2.585213in}}%
\pgfpathlineto{\pgfqpoint{4.822823in}{2.573148in}}%
\pgfpathlineto{\pgfqpoint{4.836787in}{2.548027in}}%
\pgfpathlineto{\pgfqpoint{4.850751in}{2.513967in}}%
\pgfpathlineto{\pgfqpoint{4.864715in}{2.470378in}}%
\pgfpathlineto{\pgfqpoint{4.878679in}{2.416648in}}%
\pgfpathlineto{\pgfqpoint{4.892643in}{2.352146in}}%
\pgfpathlineto{\pgfqpoint{4.911261in}{2.248257in}}%
\pgfpathlineto{\pgfqpoint{4.929880in}{2.122452in}}%
\pgfpathlineto{\pgfqpoint{4.948498in}{1.973063in}}%
\pgfpathlineto{\pgfqpoint{4.967117in}{1.798347in}}%
\pgfpathlineto{\pgfqpoint{4.985736in}{1.596497in}}%
\pgfpathlineto{\pgfqpoint{5.004354in}{1.365627in}}%
\pgfpathlineto{\pgfqpoint{5.027628in}{1.033246in}}%
\pgfpathlineto{\pgfqpoint{5.050901in}{0.648472in}}%
\pgfpathlineto{\pgfqpoint{5.059823in}{0.486111in}}%
\pgfpathlineto{\pgfqpoint{5.059823in}{0.486111in}}%
\pgfusepath{stroke}%
\end{pgfscope}%
\begin{pgfscope}%
\pgfpathrectangle{\pgfqpoint{0.750000in}{0.500000in}}{\pgfqpoint{4.650000in}{3.020000in}}%
\pgfusepath{clip}%
\pgfsetrectcap%
\pgfsetroundjoin%
\pgfsetlinewidth{1.505625pt}%
\definecolor{currentstroke}{rgb}{0.839216,0.152941,0.156863}%
\pgfsetstrokecolor{currentstroke}%
\pgfsetdash{}{0pt}%
\pgfpathmoveto{\pgfqpoint{1.087092in}{3.533889in}}%
\pgfpathlineto{\pgfqpoint{1.103754in}{2.660632in}}%
\pgfpathlineto{\pgfqpoint{1.122372in}{1.829269in}}%
\pgfpathlineto{\pgfqpoint{1.140991in}{1.134472in}}%
\pgfpathlineto{\pgfqpoint{1.159610in}{0.562288in}}%
\pgfpathlineto{\pgfqpoint{1.162437in}{0.486111in}}%
\pgfpathmoveto{\pgfqpoint{1.514363in}{0.486111in}}%
\pgfpathlineto{\pgfqpoint{1.550601in}{0.776577in}}%
\pgfpathlineto{\pgfqpoint{1.583183in}{1.012635in}}%
\pgfpathlineto{\pgfqpoint{1.611111in}{1.192019in}}%
\pgfpathlineto{\pgfqpoint{1.634384in}{1.323725in}}%
\pgfpathlineto{\pgfqpoint{1.657658in}{1.438525in}}%
\pgfpathlineto{\pgfqpoint{1.680931in}{1.536124in}}%
\pgfpathlineto{\pgfqpoint{1.699550in}{1.601841in}}%
\pgfpathlineto{\pgfqpoint{1.718168in}{1.656731in}}%
\pgfpathlineto{\pgfqpoint{1.736787in}{1.701060in}}%
\pgfpathlineto{\pgfqpoint{1.750751in}{1.727588in}}%
\pgfpathlineto{\pgfqpoint{1.764715in}{1.748554in}}%
\pgfpathlineto{\pgfqpoint{1.778679in}{1.764163in}}%
\pgfpathlineto{\pgfqpoint{1.792643in}{1.774636in}}%
\pgfpathlineto{\pgfqpoint{1.806607in}{1.780215in}}%
\pgfpathlineto{\pgfqpoint{1.820571in}{1.781156in}}%
\pgfpathlineto{\pgfqpoint{1.834535in}{1.777726in}}%
\pgfpathlineto{\pgfqpoint{1.848498in}{1.770206in}}%
\pgfpathlineto{\pgfqpoint{1.862462in}{1.758882in}}%
\pgfpathlineto{\pgfqpoint{1.876426in}{1.744050in}}%
\pgfpathlineto{\pgfqpoint{1.895045in}{1.719334in}}%
\pgfpathlineto{\pgfqpoint{1.913664in}{1.689632in}}%
\pgfpathlineto{\pgfqpoint{1.936937in}{1.646591in}}%
\pgfpathlineto{\pgfqpoint{1.964865in}{1.588119in}}%
\pgfpathlineto{\pgfqpoint{2.002102in}{1.502464in}}%
\pgfpathlineto{\pgfqpoint{2.104505in}{1.261697in}}%
\pgfpathlineto{\pgfqpoint{2.137087in}{1.194639in}}%
\pgfpathlineto{\pgfqpoint{2.165015in}{1.144211in}}%
\pgfpathlineto{\pgfqpoint{2.188288in}{1.108134in}}%
\pgfpathlineto{\pgfqpoint{2.211562in}{1.078123in}}%
\pgfpathlineto{\pgfqpoint{2.230180in}{1.058832in}}%
\pgfpathlineto{\pgfqpoint{2.248799in}{1.043971in}}%
\pgfpathlineto{\pgfqpoint{2.267417in}{1.033712in}}%
\pgfpathlineto{\pgfqpoint{2.281381in}{1.029121in}}%
\pgfpathlineto{\pgfqpoint{2.295345in}{1.027236in}}%
\pgfpathlineto{\pgfqpoint{2.309309in}{1.028085in}}%
\pgfpathlineto{\pgfqpoint{2.323273in}{1.031684in}}%
\pgfpathlineto{\pgfqpoint{2.337237in}{1.038037in}}%
\pgfpathlineto{\pgfqpoint{2.351201in}{1.047137in}}%
\pgfpathlineto{\pgfqpoint{2.369820in}{1.063508in}}%
\pgfpathlineto{\pgfqpoint{2.388438in}{1.084648in}}%
\pgfpathlineto{\pgfqpoint{2.407057in}{1.110447in}}%
\pgfpathlineto{\pgfqpoint{2.430330in}{1.149030in}}%
\pgfpathlineto{\pgfqpoint{2.453604in}{1.194326in}}%
\pgfpathlineto{\pgfqpoint{2.476877in}{1.245924in}}%
\pgfpathlineto{\pgfqpoint{2.504805in}{1.315490in}}%
\pgfpathlineto{\pgfqpoint{2.537387in}{1.406022in}}%
\pgfpathlineto{\pgfqpoint{2.574625in}{1.519839in}}%
\pgfpathlineto{\pgfqpoint{2.621171in}{1.673713in}}%
\pgfpathlineto{\pgfqpoint{2.695646in}{1.933487in}}%
\pgfpathlineto{\pgfqpoint{2.765465in}{2.173825in}}%
\pgfpathlineto{\pgfqpoint{2.807357in}{2.308399in}}%
\pgfpathlineto{\pgfqpoint{2.844595in}{2.418139in}}%
\pgfpathlineto{\pgfqpoint{2.877177in}{2.504532in}}%
\pgfpathlineto{\pgfqpoint{2.905105in}{2.570265in}}%
\pgfpathlineto{\pgfqpoint{2.933033in}{2.627460in}}%
\pgfpathlineto{\pgfqpoint{2.956306in}{2.668092in}}%
\pgfpathlineto{\pgfqpoint{2.979580in}{2.701960in}}%
\pgfpathlineto{\pgfqpoint{2.998198in}{2.723990in}}%
\pgfpathlineto{\pgfqpoint{3.016817in}{2.741386in}}%
\pgfpathlineto{\pgfqpoint{3.035435in}{2.754055in}}%
\pgfpathlineto{\pgfqpoint{3.049399in}{2.760412in}}%
\pgfpathlineto{\pgfqpoint{3.063363in}{2.764051in}}%
\pgfpathlineto{\pgfqpoint{3.077327in}{2.764962in}}%
\pgfpathlineto{\pgfqpoint{3.091291in}{2.763141in}}%
\pgfpathlineto{\pgfqpoint{3.105255in}{2.758594in}}%
\pgfpathlineto{\pgfqpoint{3.119219in}{2.751335in}}%
\pgfpathlineto{\pgfqpoint{3.137838in}{2.737477in}}%
\pgfpathlineto{\pgfqpoint{3.156456in}{2.718912in}}%
\pgfpathlineto{\pgfqpoint{3.175075in}{2.695742in}}%
\pgfpathlineto{\pgfqpoint{3.193694in}{2.668092in}}%
\pgfpathlineto{\pgfqpoint{3.216967in}{2.627460in}}%
\pgfpathlineto{\pgfqpoint{3.240240in}{2.580410in}}%
\pgfpathlineto{\pgfqpoint{3.268168in}{2.516047in}}%
\pgfpathlineto{\pgfqpoint{3.296096in}{2.443814in}}%
\pgfpathlineto{\pgfqpoint{3.328679in}{2.350789in}}%
\pgfpathlineto{\pgfqpoint{3.365916in}{2.234870in}}%
\pgfpathlineto{\pgfqpoint{3.412462in}{2.079354in}}%
\pgfpathlineto{\pgfqpoint{3.589339in}{1.476016in}}%
\pgfpathlineto{\pgfqpoint{3.626577in}{1.366079in}}%
\pgfpathlineto{\pgfqpoint{3.659159in}{1.279713in}}%
\pgfpathlineto{\pgfqpoint{3.687087in}{1.214234in}}%
\pgfpathlineto{\pgfqpoint{3.710360in}{1.166364in}}%
\pgfpathlineto{\pgfqpoint{3.733634in}{1.125051in}}%
\pgfpathlineto{\pgfqpoint{3.756907in}{1.090666in}}%
\pgfpathlineto{\pgfqpoint{3.775526in}{1.068349in}}%
\pgfpathlineto{\pgfqpoint{3.794144in}{1.050778in}}%
\pgfpathlineto{\pgfqpoint{3.812763in}{1.038037in}}%
\pgfpathlineto{\pgfqpoint{3.826727in}{1.031684in}}%
\pgfpathlineto{\pgfqpoint{3.840691in}{1.028085in}}%
\pgfpathlineto{\pgfqpoint{3.854655in}{1.027236in}}%
\pgfpathlineto{\pgfqpoint{3.868619in}{1.029121in}}%
\pgfpathlineto{\pgfqpoint{3.882583in}{1.033712in}}%
\pgfpathlineto{\pgfqpoint{3.896547in}{1.040969in}}%
\pgfpathlineto{\pgfqpoint{3.915165in}{1.054694in}}%
\pgfpathlineto{\pgfqpoint{3.933784in}{1.072897in}}%
\pgfpathlineto{\pgfqpoint{3.952402in}{1.095371in}}%
\pgfpathlineto{\pgfqpoint{3.975676in}{1.129089in}}%
\pgfpathlineto{\pgfqpoint{3.998949in}{1.168518in}}%
\pgfpathlineto{\pgfqpoint{4.026877in}{1.222410in}}%
\pgfpathlineto{\pgfqpoint{4.059459in}{1.292610in}}%
\pgfpathlineto{\pgfqpoint{4.106006in}{1.401730in}}%
\pgfpathlineto{\pgfqpoint{4.180480in}{1.577805in}}%
\pgfpathlineto{\pgfqpoint{4.213063in}{1.646591in}}%
\pgfpathlineto{\pgfqpoint{4.236336in}{1.689632in}}%
\pgfpathlineto{\pgfqpoint{4.259610in}{1.726009in}}%
\pgfpathlineto{\pgfqpoint{4.278228in}{1.749366in}}%
\pgfpathlineto{\pgfqpoint{4.292192in}{1.763061in}}%
\pgfpathlineto{\pgfqpoint{4.306156in}{1.773150in}}%
\pgfpathlineto{\pgfqpoint{4.320120in}{1.779338in}}%
\pgfpathlineto{\pgfqpoint{4.334084in}{1.781341in}}%
\pgfpathlineto{\pgfqpoint{4.348048in}{1.778884in}}%
\pgfpathlineto{\pgfqpoint{4.362012in}{1.771701in}}%
\pgfpathlineto{\pgfqpoint{4.375976in}{1.759542in}}%
\pgfpathlineto{\pgfqpoint{4.389940in}{1.742171in}}%
\pgfpathlineto{\pgfqpoint{4.403904in}{1.719373in}}%
\pgfpathlineto{\pgfqpoint{4.417868in}{1.690950in}}%
\pgfpathlineto{\pgfqpoint{4.431832in}{1.656731in}}%
\pgfpathlineto{\pgfqpoint{4.450450in}{1.601841in}}%
\pgfpathlineto{\pgfqpoint{4.469069in}{1.536124in}}%
\pgfpathlineto{\pgfqpoint{4.487688in}{1.459424in}}%
\pgfpathlineto{\pgfqpoint{4.510961in}{1.348052in}}%
\pgfpathlineto{\pgfqpoint{4.534234in}{1.219686in}}%
\pgfpathlineto{\pgfqpoint{4.557508in}{1.074920in}}%
\pgfpathlineto{\pgfqpoint{4.585435in}{0.881024in}}%
\pgfpathlineto{\pgfqpoint{4.618018in}{0.630500in}}%
\pgfpathlineto{\pgfqpoint{4.635637in}{0.486111in}}%
\pgfpathmoveto{\pgfqpoint{4.987563in}{0.486111in}}%
\pgfpathlineto{\pgfqpoint{4.999700in}{0.833886in}}%
\pgfpathlineto{\pgfqpoint{5.018318in}{1.465696in}}%
\pgfpathlineto{\pgfqpoint{5.036937in}{2.226968in}}%
\pgfpathlineto{\pgfqpoint{5.055556in}{3.132169in}}%
\pgfpathlineto{\pgfqpoint{5.062908in}{3.533889in}}%
\pgfpathlineto{\pgfqpoint{5.062908in}{3.533889in}}%
\pgfusepath{stroke}%
\end{pgfscope}%
\begin{pgfscope}%
\pgfpathrectangle{\pgfqpoint{0.750000in}{0.500000in}}{\pgfqpoint{4.650000in}{3.020000in}}%
\pgfusepath{clip}%
\pgfsetrectcap%
\pgfsetroundjoin%
\pgfsetlinewidth{1.505625pt}%
\definecolor{currentstroke}{rgb}{0.580392,0.403922,0.741176}%
\pgfsetstrokecolor{currentstroke}%
\pgfsetdash{}{0pt}%
\pgfpathmoveto{\pgfqpoint{1.127945in}{0.486111in}}%
\pgfpathlineto{\pgfqpoint{1.140991in}{1.510488in}}%
\pgfpathlineto{\pgfqpoint{1.154955in}{2.410044in}}%
\pgfpathlineto{\pgfqpoint{1.168919in}{3.130874in}}%
\pgfpathlineto{\pgfqpoint{1.178532in}{3.533889in}}%
\pgfpathmoveto{\pgfqpoint{1.364073in}{3.533889in}}%
\pgfpathlineto{\pgfqpoint{1.438889in}{2.307348in}}%
\pgfpathlineto{\pgfqpoint{1.466817in}{1.909407in}}%
\pgfpathlineto{\pgfqpoint{1.490090in}{1.617620in}}%
\pgfpathlineto{\pgfqpoint{1.513363in}{1.365614in}}%
\pgfpathlineto{\pgfqpoint{1.531982in}{1.193580in}}%
\pgfpathlineto{\pgfqpoint{1.550601in}{1.047796in}}%
\pgfpathlineto{\pgfqpoint{1.569219in}{0.927665in}}%
\pgfpathlineto{\pgfqpoint{1.583183in}{0.853795in}}%
\pgfpathlineto{\pgfqpoint{1.597147in}{0.793217in}}%
\pgfpathlineto{\pgfqpoint{1.611111in}{0.745276in}}%
\pgfpathlineto{\pgfqpoint{1.625075in}{0.709246in}}%
\pgfpathlineto{\pgfqpoint{1.634384in}{0.691459in}}%
\pgfpathlineto{\pgfqpoint{1.643694in}{0.678374in}}%
\pgfpathlineto{\pgfqpoint{1.653003in}{0.669743in}}%
\pgfpathlineto{\pgfqpoint{1.662312in}{0.665312in}}%
\pgfpathlineto{\pgfqpoint{1.671622in}{0.664822in}}%
\pgfpathlineto{\pgfqpoint{1.680931in}{0.668013in}}%
\pgfpathlineto{\pgfqpoint{1.690240in}{0.674625in}}%
\pgfpathlineto{\pgfqpoint{1.699550in}{0.684398in}}%
\pgfpathlineto{\pgfqpoint{1.713514in}{0.704418in}}%
\pgfpathlineto{\pgfqpoint{1.727477in}{0.730109in}}%
\pgfpathlineto{\pgfqpoint{1.746096in}{0.771736in}}%
\pgfpathlineto{\pgfqpoint{1.769369in}{0.832944in}}%
\pgfpathlineto{\pgfqpoint{1.801952in}{0.929562in}}%
\pgfpathlineto{\pgfqpoint{1.885736in}{1.184392in}}%
\pgfpathlineto{\pgfqpoint{1.913664in}{1.258069in}}%
\pgfpathlineto{\pgfqpoint{1.936937in}{1.311856in}}%
\pgfpathlineto{\pgfqpoint{1.960210in}{1.357732in}}%
\pgfpathlineto{\pgfqpoint{1.978829in}{1.388344in}}%
\pgfpathlineto{\pgfqpoint{1.997447in}{1.413376in}}%
\pgfpathlineto{\pgfqpoint{2.016066in}{1.432807in}}%
\pgfpathlineto{\pgfqpoint{2.030030in}{1.443752in}}%
\pgfpathlineto{\pgfqpoint{2.043994in}{1.451668in}}%
\pgfpathlineto{\pgfqpoint{2.057958in}{1.456659in}}%
\pgfpathlineto{\pgfqpoint{2.071922in}{1.458857in}}%
\pgfpathlineto{\pgfqpoint{2.085886in}{1.458418in}}%
\pgfpathlineto{\pgfqpoint{2.099850in}{1.455519in}}%
\pgfpathlineto{\pgfqpoint{2.118468in}{1.448170in}}%
\pgfpathlineto{\pgfqpoint{2.137087in}{1.437311in}}%
\pgfpathlineto{\pgfqpoint{2.160360in}{1.419649in}}%
\pgfpathlineto{\pgfqpoint{2.188288in}{1.393980in}}%
\pgfpathlineto{\pgfqpoint{2.234835in}{1.345724in}}%
\pgfpathlineto{\pgfqpoint{2.281381in}{1.298923in}}%
\pgfpathlineto{\pgfqpoint{2.309309in}{1.275320in}}%
\pgfpathlineto{\pgfqpoint{2.332583in}{1.259800in}}%
\pgfpathlineto{\pgfqpoint{2.351201in}{1.250753in}}%
\pgfpathlineto{\pgfqpoint{2.369820in}{1.245145in}}%
\pgfpathlineto{\pgfqpoint{2.388438in}{1.243336in}}%
\pgfpathlineto{\pgfqpoint{2.402402in}{1.244656in}}%
\pgfpathlineto{\pgfqpoint{2.416366in}{1.248393in}}%
\pgfpathlineto{\pgfqpoint{2.430330in}{1.254636in}}%
\pgfpathlineto{\pgfqpoint{2.448949in}{1.266981in}}%
\pgfpathlineto{\pgfqpoint{2.467568in}{1.284026in}}%
\pgfpathlineto{\pgfqpoint{2.486186in}{1.305831in}}%
\pgfpathlineto{\pgfqpoint{2.504805in}{1.332399in}}%
\pgfpathlineto{\pgfqpoint{2.523423in}{1.363674in}}%
\pgfpathlineto{\pgfqpoint{2.546697in}{1.409215in}}%
\pgfpathlineto{\pgfqpoint{2.569970in}{1.461603in}}%
\pgfpathlineto{\pgfqpoint{2.597898in}{1.532855in}}%
\pgfpathlineto{\pgfqpoint{2.625826in}{1.612302in}}%
\pgfpathlineto{\pgfqpoint{2.658408in}{1.713723in}}%
\pgfpathlineto{\pgfqpoint{2.700300in}{1.854469in}}%
\pgfpathlineto{\pgfqpoint{2.853904in}{2.383520in}}%
\pgfpathlineto{\pgfqpoint{2.886486in}{2.480207in}}%
\pgfpathlineto{\pgfqpoint{2.914414in}{2.554306in}}%
\pgfpathlineto{\pgfqpoint{2.937688in}{2.608825in}}%
\pgfpathlineto{\pgfqpoint{2.960961in}{2.656028in}}%
\pgfpathlineto{\pgfqpoint{2.979580in}{2.688118in}}%
\pgfpathlineto{\pgfqpoint{2.998198in}{2.714878in}}%
\pgfpathlineto{\pgfqpoint{3.016817in}{2.736092in}}%
\pgfpathlineto{\pgfqpoint{3.030781in}{2.748256in}}%
\pgfpathlineto{\pgfqpoint{3.044745in}{2.757145in}}%
\pgfpathlineto{\pgfqpoint{3.058709in}{2.762720in}}%
\pgfpathlineto{\pgfqpoint{3.072673in}{2.764953in}}%
\pgfpathlineto{\pgfqpoint{3.086637in}{2.763836in}}%
\pgfpathlineto{\pgfqpoint{3.100601in}{2.759373in}}%
\pgfpathlineto{\pgfqpoint{3.114565in}{2.751585in}}%
\pgfpathlineto{\pgfqpoint{3.128529in}{2.740507in}}%
\pgfpathlineto{\pgfqpoint{3.142492in}{2.726190in}}%
\pgfpathlineto{\pgfqpoint{3.161111in}{2.702179in}}%
\pgfpathlineto{\pgfqpoint{3.179730in}{2.672724in}}%
\pgfpathlineto{\pgfqpoint{3.198348in}{2.638065in}}%
\pgfpathlineto{\pgfqpoint{3.221622in}{2.587858in}}%
\pgfpathlineto{\pgfqpoint{3.244895in}{2.530602in}}%
\pgfpathlineto{\pgfqpoint{3.272823in}{2.453618in}}%
\pgfpathlineto{\pgfqpoint{3.305405in}{2.354180in}}%
\pgfpathlineto{\pgfqpoint{3.347297in}{2.214918in}}%
\pgfpathlineto{\pgfqpoint{3.417117in}{1.968846in}}%
\pgfpathlineto{\pgfqpoint{3.477628in}{1.759555in}}%
\pgfpathlineto{\pgfqpoint{3.514865in}{1.640402in}}%
\pgfpathlineto{\pgfqpoint{3.547447in}{1.545557in}}%
\pgfpathlineto{\pgfqpoint{3.575375in}{1.472864in}}%
\pgfpathlineto{\pgfqpoint{3.603303in}{1.409215in}}%
\pgfpathlineto{\pgfqpoint{3.626577in}{1.363674in}}%
\pgfpathlineto{\pgfqpoint{3.649850in}{1.325312in}}%
\pgfpathlineto{\pgfqpoint{3.668468in}{1.299932in}}%
\pgfpathlineto{\pgfqpoint{3.687087in}{1.279320in}}%
\pgfpathlineto{\pgfqpoint{3.705706in}{1.263458in}}%
\pgfpathlineto{\pgfqpoint{3.724324in}{1.252272in}}%
\pgfpathlineto{\pgfqpoint{3.742943in}{1.245629in}}%
\pgfpathlineto{\pgfqpoint{3.756907in}{1.243514in}}%
\pgfpathlineto{\pgfqpoint{3.775526in}{1.244324in}}%
\pgfpathlineto{\pgfqpoint{3.794144in}{1.249013in}}%
\pgfpathlineto{\pgfqpoint{3.812763in}{1.257237in}}%
\pgfpathlineto{\pgfqpoint{3.831381in}{1.268599in}}%
\pgfpathlineto{\pgfqpoint{3.854655in}{1.286536in}}%
\pgfpathlineto{\pgfqpoint{3.882583in}{1.312257in}}%
\pgfpathlineto{\pgfqpoint{3.929129in}{1.360479in}}%
\pgfpathlineto{\pgfqpoint{3.975676in}{1.407311in}}%
\pgfpathlineto{\pgfqpoint{4.003604in}{1.430738in}}%
\pgfpathlineto{\pgfqpoint{4.026877in}{1.445763in}}%
\pgfpathlineto{\pgfqpoint{4.045495in}{1.454039in}}%
\pgfpathlineto{\pgfqpoint{4.064114in}{1.458418in}}%
\pgfpathlineto{\pgfqpoint{4.078078in}{1.458857in}}%
\pgfpathlineto{\pgfqpoint{4.092042in}{1.456659in}}%
\pgfpathlineto{\pgfqpoint{4.106006in}{1.451668in}}%
\pgfpathlineto{\pgfqpoint{4.119970in}{1.443752in}}%
\pgfpathlineto{\pgfqpoint{4.133934in}{1.432807in}}%
\pgfpathlineto{\pgfqpoint{4.152553in}{1.413376in}}%
\pgfpathlineto{\pgfqpoint{4.171171in}{1.388344in}}%
\pgfpathlineto{\pgfqpoint{4.189790in}{1.357732in}}%
\pgfpathlineto{\pgfqpoint{4.208408in}{1.321687in}}%
\pgfpathlineto{\pgfqpoint{4.231682in}{1.269420in}}%
\pgfpathlineto{\pgfqpoint{4.259610in}{1.197283in}}%
\pgfpathlineto{\pgfqpoint{4.292192in}{1.102920in}}%
\pgfpathlineto{\pgfqpoint{4.394595in}{0.795169in}}%
\pgfpathlineto{\pgfqpoint{4.417868in}{0.739787in}}%
\pgfpathlineto{\pgfqpoint{4.436486in}{0.704418in}}%
\pgfpathlineto{\pgfqpoint{4.450450in}{0.684398in}}%
\pgfpathlineto{\pgfqpoint{4.464414in}{0.670908in}}%
\pgfpathlineto{\pgfqpoint{4.473724in}{0.665973in}}%
\pgfpathlineto{\pgfqpoint{4.483033in}{0.664590in}}%
\pgfpathlineto{\pgfqpoint{4.492342in}{0.667019in}}%
\pgfpathlineto{\pgfqpoint{4.501652in}{0.673518in}}%
\pgfpathlineto{\pgfqpoint{4.510961in}{0.684344in}}%
\pgfpathlineto{\pgfqpoint{4.520270in}{0.699749in}}%
\pgfpathlineto{\pgfqpoint{4.529580in}{0.719980in}}%
\pgfpathlineto{\pgfqpoint{4.543544in}{0.759896in}}%
\pgfpathlineto{\pgfqpoint{4.557508in}{0.811972in}}%
\pgfpathlineto{\pgfqpoint{4.571471in}{0.876912in}}%
\pgfpathlineto{\pgfqpoint{4.585435in}{0.955343in}}%
\pgfpathlineto{\pgfqpoint{4.604054in}{1.081806in}}%
\pgfpathlineto{\pgfqpoint{4.622673in}{1.234115in}}%
\pgfpathlineto{\pgfqpoint{4.641291in}{1.412746in}}%
\pgfpathlineto{\pgfqpoint{4.659910in}{1.617620in}}%
\pgfpathlineto{\pgfqpoint{4.683183in}{1.909407in}}%
\pgfpathlineto{\pgfqpoint{4.711111in}{2.307348in}}%
\pgfpathlineto{\pgfqpoint{4.743694in}{2.824107in}}%
\pgfpathlineto{\pgfqpoint{4.785927in}{3.533889in}}%
\pgfpathmoveto{\pgfqpoint{4.971468in}{3.533889in}}%
\pgfpathlineto{\pgfqpoint{4.981081in}{3.130874in}}%
\pgfpathlineto{\pgfqpoint{4.995045in}{2.410044in}}%
\pgfpathlineto{\pgfqpoint{5.009009in}{1.510488in}}%
\pgfpathlineto{\pgfqpoint{5.022055in}{0.486111in}}%
\pgfpathlineto{\pgfqpoint{5.022055in}{0.486111in}}%
\pgfusepath{stroke}%
\end{pgfscope}%
\begin{pgfscope}%
\pgfpathrectangle{\pgfqpoint{0.750000in}{0.500000in}}{\pgfqpoint{4.650000in}{3.020000in}}%
\pgfusepath{clip}%
\pgfsetrectcap%
\pgfsetroundjoin%
\pgfsetlinewidth{1.505625pt}%
\definecolor{currentstroke}{rgb}{0.549020,0.337255,0.294118}%
\pgfsetstrokecolor{currentstroke}%
\pgfsetdash{}{0pt}%
\pgfpathmoveto{\pgfqpoint{0.750000in}{1.255000in}}%
\pgfpathlineto{\pgfqpoint{2.057958in}{1.256539in}}%
\pgfpathlineto{\pgfqpoint{2.146396in}{1.259841in}}%
\pgfpathlineto{\pgfqpoint{2.206907in}{1.264986in}}%
\pgfpathlineto{\pgfqpoint{2.253453in}{1.271860in}}%
\pgfpathlineto{\pgfqpoint{2.290691in}{1.280108in}}%
\pgfpathlineto{\pgfqpoint{2.323273in}{1.290040in}}%
\pgfpathlineto{\pgfqpoint{2.355856in}{1.303215in}}%
\pgfpathlineto{\pgfqpoint{2.383784in}{1.317676in}}%
\pgfpathlineto{\pgfqpoint{2.411712in}{1.335633in}}%
\pgfpathlineto{\pgfqpoint{2.439640in}{1.357662in}}%
\pgfpathlineto{\pgfqpoint{2.467568in}{1.384360in}}%
\pgfpathlineto{\pgfqpoint{2.490841in}{1.410600in}}%
\pgfpathlineto{\pgfqpoint{2.514114in}{1.440817in}}%
\pgfpathlineto{\pgfqpoint{2.542042in}{1.482741in}}%
\pgfpathlineto{\pgfqpoint{2.569970in}{1.531239in}}%
\pgfpathlineto{\pgfqpoint{2.597898in}{1.586602in}}%
\pgfpathlineto{\pgfqpoint{2.625826in}{1.648946in}}%
\pgfpathlineto{\pgfqpoint{2.658408in}{1.730363in}}%
\pgfpathlineto{\pgfqpoint{2.690991in}{1.820552in}}%
\pgfpathlineto{\pgfqpoint{2.728228in}{1.932919in}}%
\pgfpathlineto{\pgfqpoint{2.779429in}{2.098920in}}%
\pgfpathlineto{\pgfqpoint{2.872523in}{2.404216in}}%
\pgfpathlineto{\pgfqpoint{2.905105in}{2.500929in}}%
\pgfpathlineto{\pgfqpoint{2.933033in}{2.575335in}}%
\pgfpathlineto{\pgfqpoint{2.956306in}{2.629770in}}%
\pgfpathlineto{\pgfqpoint{2.979580in}{2.676159in}}%
\pgfpathlineto{\pgfqpoint{2.998198in}{2.706833in}}%
\pgfpathlineto{\pgfqpoint{3.016817in}{2.731338in}}%
\pgfpathlineto{\pgfqpoint{3.030781in}{2.745464in}}%
\pgfpathlineto{\pgfqpoint{3.044745in}{2.755823in}}%
\pgfpathlineto{\pgfqpoint{3.058709in}{2.762333in}}%
\pgfpathlineto{\pgfqpoint{3.072673in}{2.764946in}}%
\pgfpathlineto{\pgfqpoint{3.086637in}{2.763639in}}%
\pgfpathlineto{\pgfqpoint{3.100601in}{2.758424in}}%
\pgfpathlineto{\pgfqpoint{3.114565in}{2.749340in}}%
\pgfpathlineto{\pgfqpoint{3.128529in}{2.736459in}}%
\pgfpathlineto{\pgfqpoint{3.142492in}{2.719880in}}%
\pgfpathlineto{\pgfqpoint{3.161111in}{2.692243in}}%
\pgfpathlineto{\pgfqpoint{3.179730in}{2.658633in}}%
\pgfpathlineto{\pgfqpoint{3.203003in}{2.608903in}}%
\pgfpathlineto{\pgfqpoint{3.226276in}{2.551548in}}%
\pgfpathlineto{\pgfqpoint{3.254204in}{2.474253in}}%
\pgfpathlineto{\pgfqpoint{3.286787in}{2.375076in}}%
\pgfpathlineto{\pgfqpoint{3.337988in}{2.207662in}}%
\pgfpathlineto{\pgfqpoint{3.421772in}{1.932919in}}%
\pgfpathlineto{\pgfqpoint{3.463664in}{1.807167in}}%
\pgfpathlineto{\pgfqpoint{3.496246in}{1.718176in}}%
\pgfpathlineto{\pgfqpoint{3.528829in}{1.638072in}}%
\pgfpathlineto{\pgfqpoint{3.556757in}{1.576890in}}%
\pgfpathlineto{\pgfqpoint{3.584685in}{1.522685in}}%
\pgfpathlineto{\pgfqpoint{3.612613in}{1.475307in}}%
\pgfpathlineto{\pgfqpoint{3.640541in}{1.434440in}}%
\pgfpathlineto{\pgfqpoint{3.668468in}{1.399644in}}%
\pgfpathlineto{\pgfqpoint{3.696396in}{1.370390in}}%
\pgfpathlineto{\pgfqpoint{3.724324in}{1.346102in}}%
\pgfpathlineto{\pgfqpoint{3.752252in}{1.326182in}}%
\pgfpathlineto{\pgfqpoint{3.780180in}{1.310043in}}%
\pgfpathlineto{\pgfqpoint{3.808108in}{1.297124in}}%
\pgfpathlineto{\pgfqpoint{3.840691in}{1.285429in}}%
\pgfpathlineto{\pgfqpoint{3.877928in}{1.275622in}}%
\pgfpathlineto{\pgfqpoint{3.919820in}{1.268023in}}%
\pgfpathlineto{\pgfqpoint{3.971021in}{1.262193in}}%
\pgfpathlineto{\pgfqpoint{4.036186in}{1.258213in}}%
\pgfpathlineto{\pgfqpoint{4.129279in}{1.255921in}}%
\pgfpathlineto{\pgfqpoint{4.320120in}{1.255050in}}%
\pgfpathlineto{\pgfqpoint{5.400000in}{1.255000in}}%
\pgfpathlineto{\pgfqpoint{5.400000in}{1.255000in}}%
\pgfusepath{stroke}%
\end{pgfscope}%
\begin{pgfscope}%
\pgfsetrectcap%
\pgfsetmiterjoin%
\pgfsetlinewidth{0.803000pt}%
\definecolor{currentstroke}{rgb}{0.000000,0.000000,0.000000}%
\pgfsetstrokecolor{currentstroke}%
\pgfsetdash{}{0pt}%
\pgfpathmoveto{\pgfqpoint{0.750000in}{0.500000in}}%
\pgfpathlineto{\pgfqpoint{0.750000in}{3.520000in}}%
\pgfusepath{stroke}%
\end{pgfscope}%
\begin{pgfscope}%
\pgfsetrectcap%
\pgfsetmiterjoin%
\pgfsetlinewidth{0.803000pt}%
\definecolor{currentstroke}{rgb}{0.000000,0.000000,0.000000}%
\pgfsetstrokecolor{currentstroke}%
\pgfsetdash{}{0pt}%
\pgfpathmoveto{\pgfqpoint{5.400000in}{0.500000in}}%
\pgfpathlineto{\pgfqpoint{5.400000in}{3.520000in}}%
\pgfusepath{stroke}%
\end{pgfscope}%
\begin{pgfscope}%
\pgfsetrectcap%
\pgfsetmiterjoin%
\pgfsetlinewidth{0.803000pt}%
\definecolor{currentstroke}{rgb}{0.000000,0.000000,0.000000}%
\pgfsetstrokecolor{currentstroke}%
\pgfsetdash{}{0pt}%
\pgfpathmoveto{\pgfqpoint{0.750000in}{0.500000in}}%
\pgfpathlineto{\pgfqpoint{5.400000in}{0.500000in}}%
\pgfusepath{stroke}%
\end{pgfscope}%
\begin{pgfscope}%
\pgfsetrectcap%
\pgfsetmiterjoin%
\pgfsetlinewidth{0.803000pt}%
\definecolor{currentstroke}{rgb}{0.000000,0.000000,0.000000}%
\pgfsetstrokecolor{currentstroke}%
\pgfsetdash{}{0pt}%
\pgfpathmoveto{\pgfqpoint{0.750000in}{3.520000in}}%
\pgfpathlineto{\pgfqpoint{5.400000in}{3.520000in}}%
\pgfusepath{stroke}%
\end{pgfscope}%
\begin{pgfscope}%
\pgfsetbuttcap%
\pgfsetmiterjoin%
\definecolor{currentfill}{rgb}{1.000000,1.000000,1.000000}%
\pgfsetfillcolor{currentfill}%
\pgfsetfillopacity{0.800000}%
\pgfsetlinewidth{1.003750pt}%
\definecolor{currentstroke}{rgb}{0.800000,0.800000,0.800000}%
\pgfsetstrokecolor{currentstroke}%
\pgfsetstrokeopacity{0.800000}%
\pgfsetdash{}{0pt}%
\pgfpathmoveto{\pgfqpoint{2.649058in}{0.569444in}}%
\pgfpathlineto{\pgfqpoint{3.500942in}{0.569444in}}%
\pgfpathquadraticcurveto{\pgfqpoint{3.528719in}{0.569444in}}{\pgfqpoint{3.528719in}{0.597222in}}%
\pgfpathlineto{\pgfqpoint{3.528719in}{1.745061in}}%
\pgfpathquadraticcurveto{\pgfqpoint{3.528719in}{1.772839in}}{\pgfqpoint{3.500942in}{1.772839in}}%
\pgfpathlineto{\pgfqpoint{2.649058in}{1.772839in}}%
\pgfpathquadraticcurveto{\pgfqpoint{2.621281in}{1.772839in}}{\pgfqpoint{2.621281in}{1.745061in}}%
\pgfpathlineto{\pgfqpoint{2.621281in}{0.597222in}}%
\pgfpathquadraticcurveto{\pgfqpoint{2.621281in}{0.569444in}}{\pgfqpoint{2.649058in}{0.569444in}}%
\pgfpathclose%
\pgfusepath{stroke,fill}%
\end{pgfscope}%
\begin{pgfscope}%
\pgfsetrectcap%
\pgfsetroundjoin%
\pgfsetlinewidth{1.505625pt}%
\definecolor{currentstroke}{rgb}{0.121569,0.466667,0.705882}%
\pgfsetstrokecolor{currentstroke}%
\pgfsetdash{}{0pt}%
\pgfpathmoveto{\pgfqpoint{2.676836in}{1.668672in}}%
\pgfpathlineto{\pgfqpoint{2.954614in}{1.668672in}}%
\pgfusepath{stroke}%
\end{pgfscope}%
\begin{pgfscope}%
\pgftext[x=3.065725in,y=1.620061in,left,base]{\rmfamily\fontsize{10.000000}{12.000000}\selectfont \(\displaystyle  n = 2 \)}%
\end{pgfscope}%
\begin{pgfscope}%
\pgfsetrectcap%
\pgfsetroundjoin%
\pgfsetlinewidth{1.505625pt}%
\definecolor{currentstroke}{rgb}{1.000000,0.498039,0.054902}%
\pgfsetstrokecolor{currentstroke}%
\pgfsetdash{}{0pt}%
\pgfpathmoveto{\pgfqpoint{2.676836in}{1.475061in}}%
\pgfpathlineto{\pgfqpoint{2.954614in}{1.475061in}}%
\pgfusepath{stroke}%
\end{pgfscope}%
\begin{pgfscope}%
\pgftext[x=3.065725in,y=1.426450in,left,base]{\rmfamily\fontsize{10.000000}{12.000000}\selectfont \(\displaystyle  n = 4 \)}%
\end{pgfscope}%
\begin{pgfscope}%
\pgfsetrectcap%
\pgfsetroundjoin%
\pgfsetlinewidth{1.505625pt}%
\definecolor{currentstroke}{rgb}{0.172549,0.627451,0.172549}%
\pgfsetstrokecolor{currentstroke}%
\pgfsetdash{}{0pt}%
\pgfpathmoveto{\pgfqpoint{2.676836in}{1.281450in}}%
\pgfpathlineto{\pgfqpoint{2.954614in}{1.281450in}}%
\pgfusepath{stroke}%
\end{pgfscope}%
\begin{pgfscope}%
\pgftext[x=3.065725in,y=1.232839in,left,base]{\rmfamily\fontsize{10.000000}{12.000000}\selectfont \(\displaystyle  n = 6 \)}%
\end{pgfscope}%
\begin{pgfscope}%
\pgfsetrectcap%
\pgfsetroundjoin%
\pgfsetlinewidth{1.505625pt}%
\definecolor{currentstroke}{rgb}{0.839216,0.152941,0.156863}%
\pgfsetstrokecolor{currentstroke}%
\pgfsetdash{}{0pt}%
\pgfpathmoveto{\pgfqpoint{2.676836in}{1.087839in}}%
\pgfpathlineto{\pgfqpoint{2.954614in}{1.087839in}}%
\pgfusepath{stroke}%
\end{pgfscope}%
\begin{pgfscope}%
\pgftext[x=3.065725in,y=1.039228in,left,base]{\rmfamily\fontsize{10.000000}{12.000000}\selectfont \(\displaystyle  n = 8 \)}%
\end{pgfscope}%
\begin{pgfscope}%
\pgfsetrectcap%
\pgfsetroundjoin%
\pgfsetlinewidth{1.505625pt}%
\definecolor{currentstroke}{rgb}{0.580392,0.403922,0.741176}%
\pgfsetstrokecolor{currentstroke}%
\pgfsetdash{}{0pt}%
\pgfpathmoveto{\pgfqpoint{2.676836in}{0.894228in}}%
\pgfpathlineto{\pgfqpoint{2.954614in}{0.894228in}}%
\pgfusepath{stroke}%
\end{pgfscope}%
\begin{pgfscope}%
\pgftext[x=3.065725in,y=0.845617in,left,base]{\rmfamily\fontsize{10.000000}{12.000000}\selectfont \(\displaystyle  n = 10 \)}%
\end{pgfscope}%
\begin{pgfscope}%
\pgfsetrectcap%
\pgfsetroundjoin%
\pgfsetlinewidth{1.505625pt}%
\definecolor{currentstroke}{rgb}{0.549020,0.337255,0.294118}%
\pgfsetstrokecolor{currentstroke}%
\pgfsetdash{}{0pt}%
\pgfpathmoveto{\pgfqpoint{2.676836in}{0.700617in}}%
\pgfpathlineto{\pgfqpoint{2.954614in}{0.700617in}}%
\pgfusepath{stroke}%
\end{pgfscope}%
\begin{pgfscope}%
\pgftext[x=3.065725in,y=0.652006in,left,base]{\rmfamily\fontsize{10.000000}{12.000000}\selectfont \(\displaystyle f_2\)}%
\end{pgfscope}%
\end{pgfpicture}%
\makeatother%
\endgroup%
}
\caption{Interpolation polynomials of degree $n$ to $f_2$ using equally spaced nodes} \label{Fig:SpaceExp}
\end{figure}
Note that polynomials of odd and even orders behave differently, and therefore plots are given separately.
\end{thmanswer}
\end{thmquestion}

\begin{thmquestion}
\
\begin{thmanswer}
The coefficient is given by
\begin{equation}
c_n = \sume{i}{0}{n}{ y_i \prodb{\sarr{c}{ j = 0 \\ j \neq i }}{n}{\frac{1}{ x_i - x_j }} }.
\end{equation}
\end{thmanswer}
\end{thmquestion}

\begin{thmquestion}
\
\begin{thmanswer}
The algorithm is described in Algorithm \ref{Alg:Calc}.

\begin{algorithm}
\SetAlgoLined

\KwData{$d_i$ where $i$ ranges from $1$ to $n$}
\KwResult{$u$}

$ u \slar 0 $\;
\For{$i$ from $n$ to $1$}
{
	$ u \slar u + 1$ \;
	$ u \slar u d_i $\;
}

\caption{Calculation of $u$} \label{Alg:Calc}
\end{algorithm}
\end{thmanswer}
\end{thmquestion}

\begin{thmquestion}
\ 
\begin{thmanswer}
Claim that $T_n$ is even if $n$ is even, or odd if $n$ is odd. Perform mathematical induction on $n$.

When $ n = 0, 1 $, $ T_n \rbr{x} = 1, x $ and is even and odd respective, satisfying the claim.

Suppose the case $ n \le k $ is done where $ k \ge 1 $, and then consider the case $ n = k + 1 $. If $k$ is itself even, then $T_k$ is even and $ T_{ k - 1 } $ is odd. Therefore, from
\begin{equation}
T_{ k + 1 } \rbr{x} = 2 x T_k \rbr{x} - T_{ k - 1 } \rbr{x}
\end{equation}
is odd. If $k$ is odd, then $T_k$ is odd and $ T_{ k - 1 } $ is even, and similarly $ T_{ k + 1 } $ being even follows. Combing these two situation, the claim holds for $ n = k + 1 $.

By mathematical induction, the claim that $T_n$ and $n$ have the same oddness and evenness for $ n \in \Nset $ is proven.

\sqed
\end{thmanswer}
\end{thmquestion}

\begin{thmquestion}
\ 
\begin{thmanswer}
Zeros of Chebyshev polynomial $T_n$ (scaled to the interval $ \sbr{ -5, 5 } $) are given by
\begin{equation}
t_i = 5 \cos \frac{ \rbr{ 2 i + 1 } \pi }{ 2 n }. \rbr{ i = 0, 1, \cdots, n - 1 }
\end{equation}

The graph for interpolations to $ f_1 \rbr{x} = \frac{1}{ 1 + x^2 } $ using zeros of Chebyshev polynomials is shown in Figure \ref{Fig:CheTan}.
\begin{figure}[htbp]
\centering \scalebox{0.8}{%% Creator: Matplotlib, PGF backend
%%
%% To include the figure in your LaTeX document, write
%%   \input{<filename>.pgf}
%%
%% Make sure the required packages are loaded in your preamble
%%   \usepackage{pgf}
%%
%% Figures using additional raster images can only be included by \input if
%% they are in the same directory as the main LaTeX file. For loading figures
%% from other directories you can use the `import` package
%%   \usepackage{import}
%% and then include the figures with
%%   \import{<path to file>}{<filename>.pgf}
%%
%% Matplotlib used the following preamble
%%   \usepackage{fontspec}
%%
\begingroup%
\makeatletter%
\begin{pgfpicture}%
\pgfpathrectangle{\pgfpointorigin}{\pgfqpoint{6.000000in}{4.000000in}}%
\pgfusepath{use as bounding box, clip}%
\begin{pgfscope}%
\pgfsetbuttcap%
\pgfsetmiterjoin%
\definecolor{currentfill}{rgb}{1.000000,1.000000,1.000000}%
\pgfsetfillcolor{currentfill}%
\pgfsetlinewidth{0.000000pt}%
\definecolor{currentstroke}{rgb}{1.000000,1.000000,1.000000}%
\pgfsetstrokecolor{currentstroke}%
\pgfsetdash{}{0pt}%
\pgfpathmoveto{\pgfqpoint{0.000000in}{0.000000in}}%
\pgfpathlineto{\pgfqpoint{6.000000in}{0.000000in}}%
\pgfpathlineto{\pgfqpoint{6.000000in}{4.000000in}}%
\pgfpathlineto{\pgfqpoint{0.000000in}{4.000000in}}%
\pgfpathclose%
\pgfusepath{fill}%
\end{pgfscope}%
\begin{pgfscope}%
\pgfsetbuttcap%
\pgfsetmiterjoin%
\definecolor{currentfill}{rgb}{1.000000,1.000000,1.000000}%
\pgfsetfillcolor{currentfill}%
\pgfsetlinewidth{0.000000pt}%
\definecolor{currentstroke}{rgb}{0.000000,0.000000,0.000000}%
\pgfsetstrokecolor{currentstroke}%
\pgfsetstrokeopacity{0.000000}%
\pgfsetdash{}{0pt}%
\pgfpathmoveto{\pgfqpoint{0.750000in}{0.500000in}}%
\pgfpathlineto{\pgfqpoint{5.400000in}{0.500000in}}%
\pgfpathlineto{\pgfqpoint{5.400000in}{3.520000in}}%
\pgfpathlineto{\pgfqpoint{0.750000in}{3.520000in}}%
\pgfpathclose%
\pgfusepath{fill}%
\end{pgfscope}%
\begin{pgfscope}%
\pgfpathrectangle{\pgfqpoint{0.750000in}{0.500000in}}{\pgfqpoint{4.650000in}{3.020000in}}%
\pgfusepath{clip}%
\pgfsetrectcap%
\pgfsetroundjoin%
\pgfsetlinewidth{0.803000pt}%
\definecolor{currentstroke}{rgb}{0.690196,0.690196,0.690196}%
\pgfsetstrokecolor{currentstroke}%
\pgfsetdash{}{0pt}%
\pgfpathmoveto{\pgfqpoint{0.750000in}{0.500000in}}%
\pgfpathlineto{\pgfqpoint{0.750000in}{3.520000in}}%
\pgfusepath{stroke}%
\end{pgfscope}%
\begin{pgfscope}%
\pgfsetbuttcap%
\pgfsetroundjoin%
\definecolor{currentfill}{rgb}{0.000000,0.000000,0.000000}%
\pgfsetfillcolor{currentfill}%
\pgfsetlinewidth{0.803000pt}%
\definecolor{currentstroke}{rgb}{0.000000,0.000000,0.000000}%
\pgfsetstrokecolor{currentstroke}%
\pgfsetdash{}{0pt}%
\pgfsys@defobject{currentmarker}{\pgfqpoint{0.000000in}{-0.048611in}}{\pgfqpoint{0.000000in}{0.000000in}}{%
\pgfpathmoveto{\pgfqpoint{0.000000in}{0.000000in}}%
\pgfpathlineto{\pgfqpoint{0.000000in}{-0.048611in}}%
\pgfusepath{stroke,fill}%
}%
\begin{pgfscope}%
\pgfsys@transformshift{0.750000in}{0.500000in}%
\pgfsys@useobject{currentmarker}{}%
\end{pgfscope}%
\end{pgfscope}%
\begin{pgfscope}%
\pgftext[x=0.750000in,y=0.402778in,,top]{\rmfamily\fontsize{10.000000}{12.000000}\selectfont \(\displaystyle -6\)}%
\end{pgfscope}%
\begin{pgfscope}%
\pgfpathrectangle{\pgfqpoint{0.750000in}{0.500000in}}{\pgfqpoint{4.650000in}{3.020000in}}%
\pgfusepath{clip}%
\pgfsetrectcap%
\pgfsetroundjoin%
\pgfsetlinewidth{0.803000pt}%
\definecolor{currentstroke}{rgb}{0.690196,0.690196,0.690196}%
\pgfsetstrokecolor{currentstroke}%
\pgfsetdash{}{0pt}%
\pgfpathmoveto{\pgfqpoint{1.525000in}{0.500000in}}%
\pgfpathlineto{\pgfqpoint{1.525000in}{3.520000in}}%
\pgfusepath{stroke}%
\end{pgfscope}%
\begin{pgfscope}%
\pgfsetbuttcap%
\pgfsetroundjoin%
\definecolor{currentfill}{rgb}{0.000000,0.000000,0.000000}%
\pgfsetfillcolor{currentfill}%
\pgfsetlinewidth{0.803000pt}%
\definecolor{currentstroke}{rgb}{0.000000,0.000000,0.000000}%
\pgfsetstrokecolor{currentstroke}%
\pgfsetdash{}{0pt}%
\pgfsys@defobject{currentmarker}{\pgfqpoint{0.000000in}{-0.048611in}}{\pgfqpoint{0.000000in}{0.000000in}}{%
\pgfpathmoveto{\pgfqpoint{0.000000in}{0.000000in}}%
\pgfpathlineto{\pgfqpoint{0.000000in}{-0.048611in}}%
\pgfusepath{stroke,fill}%
}%
\begin{pgfscope}%
\pgfsys@transformshift{1.525000in}{0.500000in}%
\pgfsys@useobject{currentmarker}{}%
\end{pgfscope}%
\end{pgfscope}%
\begin{pgfscope}%
\pgftext[x=1.525000in,y=0.402778in,,top]{\rmfamily\fontsize{10.000000}{12.000000}\selectfont \(\displaystyle -4\)}%
\end{pgfscope}%
\begin{pgfscope}%
\pgfpathrectangle{\pgfqpoint{0.750000in}{0.500000in}}{\pgfqpoint{4.650000in}{3.020000in}}%
\pgfusepath{clip}%
\pgfsetrectcap%
\pgfsetroundjoin%
\pgfsetlinewidth{0.803000pt}%
\definecolor{currentstroke}{rgb}{0.690196,0.690196,0.690196}%
\pgfsetstrokecolor{currentstroke}%
\pgfsetdash{}{0pt}%
\pgfpathmoveto{\pgfqpoint{2.300000in}{0.500000in}}%
\pgfpathlineto{\pgfqpoint{2.300000in}{3.520000in}}%
\pgfusepath{stroke}%
\end{pgfscope}%
\begin{pgfscope}%
\pgfsetbuttcap%
\pgfsetroundjoin%
\definecolor{currentfill}{rgb}{0.000000,0.000000,0.000000}%
\pgfsetfillcolor{currentfill}%
\pgfsetlinewidth{0.803000pt}%
\definecolor{currentstroke}{rgb}{0.000000,0.000000,0.000000}%
\pgfsetstrokecolor{currentstroke}%
\pgfsetdash{}{0pt}%
\pgfsys@defobject{currentmarker}{\pgfqpoint{0.000000in}{-0.048611in}}{\pgfqpoint{0.000000in}{0.000000in}}{%
\pgfpathmoveto{\pgfqpoint{0.000000in}{0.000000in}}%
\pgfpathlineto{\pgfqpoint{0.000000in}{-0.048611in}}%
\pgfusepath{stroke,fill}%
}%
\begin{pgfscope}%
\pgfsys@transformshift{2.300000in}{0.500000in}%
\pgfsys@useobject{currentmarker}{}%
\end{pgfscope}%
\end{pgfscope}%
\begin{pgfscope}%
\pgftext[x=2.300000in,y=0.402778in,,top]{\rmfamily\fontsize{10.000000}{12.000000}\selectfont \(\displaystyle -2\)}%
\end{pgfscope}%
\begin{pgfscope}%
\pgfpathrectangle{\pgfqpoint{0.750000in}{0.500000in}}{\pgfqpoint{4.650000in}{3.020000in}}%
\pgfusepath{clip}%
\pgfsetrectcap%
\pgfsetroundjoin%
\pgfsetlinewidth{0.803000pt}%
\definecolor{currentstroke}{rgb}{0.690196,0.690196,0.690196}%
\pgfsetstrokecolor{currentstroke}%
\pgfsetdash{}{0pt}%
\pgfpathmoveto{\pgfqpoint{3.075000in}{0.500000in}}%
\pgfpathlineto{\pgfqpoint{3.075000in}{3.520000in}}%
\pgfusepath{stroke}%
\end{pgfscope}%
\begin{pgfscope}%
\pgfsetbuttcap%
\pgfsetroundjoin%
\definecolor{currentfill}{rgb}{0.000000,0.000000,0.000000}%
\pgfsetfillcolor{currentfill}%
\pgfsetlinewidth{0.803000pt}%
\definecolor{currentstroke}{rgb}{0.000000,0.000000,0.000000}%
\pgfsetstrokecolor{currentstroke}%
\pgfsetdash{}{0pt}%
\pgfsys@defobject{currentmarker}{\pgfqpoint{0.000000in}{-0.048611in}}{\pgfqpoint{0.000000in}{0.000000in}}{%
\pgfpathmoveto{\pgfqpoint{0.000000in}{0.000000in}}%
\pgfpathlineto{\pgfqpoint{0.000000in}{-0.048611in}}%
\pgfusepath{stroke,fill}%
}%
\begin{pgfscope}%
\pgfsys@transformshift{3.075000in}{0.500000in}%
\pgfsys@useobject{currentmarker}{}%
\end{pgfscope}%
\end{pgfscope}%
\begin{pgfscope}%
\pgftext[x=3.075000in,y=0.402778in,,top]{\rmfamily\fontsize{10.000000}{12.000000}\selectfont \(\displaystyle 0\)}%
\end{pgfscope}%
\begin{pgfscope}%
\pgfpathrectangle{\pgfqpoint{0.750000in}{0.500000in}}{\pgfqpoint{4.650000in}{3.020000in}}%
\pgfusepath{clip}%
\pgfsetrectcap%
\pgfsetroundjoin%
\pgfsetlinewidth{0.803000pt}%
\definecolor{currentstroke}{rgb}{0.690196,0.690196,0.690196}%
\pgfsetstrokecolor{currentstroke}%
\pgfsetdash{}{0pt}%
\pgfpathmoveto{\pgfqpoint{3.850000in}{0.500000in}}%
\pgfpathlineto{\pgfqpoint{3.850000in}{3.520000in}}%
\pgfusepath{stroke}%
\end{pgfscope}%
\begin{pgfscope}%
\pgfsetbuttcap%
\pgfsetroundjoin%
\definecolor{currentfill}{rgb}{0.000000,0.000000,0.000000}%
\pgfsetfillcolor{currentfill}%
\pgfsetlinewidth{0.803000pt}%
\definecolor{currentstroke}{rgb}{0.000000,0.000000,0.000000}%
\pgfsetstrokecolor{currentstroke}%
\pgfsetdash{}{0pt}%
\pgfsys@defobject{currentmarker}{\pgfqpoint{0.000000in}{-0.048611in}}{\pgfqpoint{0.000000in}{0.000000in}}{%
\pgfpathmoveto{\pgfqpoint{0.000000in}{0.000000in}}%
\pgfpathlineto{\pgfqpoint{0.000000in}{-0.048611in}}%
\pgfusepath{stroke,fill}%
}%
\begin{pgfscope}%
\pgfsys@transformshift{3.850000in}{0.500000in}%
\pgfsys@useobject{currentmarker}{}%
\end{pgfscope}%
\end{pgfscope}%
\begin{pgfscope}%
\pgftext[x=3.850000in,y=0.402778in,,top]{\rmfamily\fontsize{10.000000}{12.000000}\selectfont \(\displaystyle 2\)}%
\end{pgfscope}%
\begin{pgfscope}%
\pgfpathrectangle{\pgfqpoint{0.750000in}{0.500000in}}{\pgfqpoint{4.650000in}{3.020000in}}%
\pgfusepath{clip}%
\pgfsetrectcap%
\pgfsetroundjoin%
\pgfsetlinewidth{0.803000pt}%
\definecolor{currentstroke}{rgb}{0.690196,0.690196,0.690196}%
\pgfsetstrokecolor{currentstroke}%
\pgfsetdash{}{0pt}%
\pgfpathmoveto{\pgfqpoint{4.625000in}{0.500000in}}%
\pgfpathlineto{\pgfqpoint{4.625000in}{3.520000in}}%
\pgfusepath{stroke}%
\end{pgfscope}%
\begin{pgfscope}%
\pgfsetbuttcap%
\pgfsetroundjoin%
\definecolor{currentfill}{rgb}{0.000000,0.000000,0.000000}%
\pgfsetfillcolor{currentfill}%
\pgfsetlinewidth{0.803000pt}%
\definecolor{currentstroke}{rgb}{0.000000,0.000000,0.000000}%
\pgfsetstrokecolor{currentstroke}%
\pgfsetdash{}{0pt}%
\pgfsys@defobject{currentmarker}{\pgfqpoint{0.000000in}{-0.048611in}}{\pgfqpoint{0.000000in}{0.000000in}}{%
\pgfpathmoveto{\pgfqpoint{0.000000in}{0.000000in}}%
\pgfpathlineto{\pgfqpoint{0.000000in}{-0.048611in}}%
\pgfusepath{stroke,fill}%
}%
\begin{pgfscope}%
\pgfsys@transformshift{4.625000in}{0.500000in}%
\pgfsys@useobject{currentmarker}{}%
\end{pgfscope}%
\end{pgfscope}%
\begin{pgfscope}%
\pgftext[x=4.625000in,y=0.402778in,,top]{\rmfamily\fontsize{10.000000}{12.000000}\selectfont \(\displaystyle 4\)}%
\end{pgfscope}%
\begin{pgfscope}%
\pgfpathrectangle{\pgfqpoint{0.750000in}{0.500000in}}{\pgfqpoint{4.650000in}{3.020000in}}%
\pgfusepath{clip}%
\pgfsetrectcap%
\pgfsetroundjoin%
\pgfsetlinewidth{0.803000pt}%
\definecolor{currentstroke}{rgb}{0.690196,0.690196,0.690196}%
\pgfsetstrokecolor{currentstroke}%
\pgfsetdash{}{0pt}%
\pgfpathmoveto{\pgfqpoint{5.400000in}{0.500000in}}%
\pgfpathlineto{\pgfqpoint{5.400000in}{3.520000in}}%
\pgfusepath{stroke}%
\end{pgfscope}%
\begin{pgfscope}%
\pgfsetbuttcap%
\pgfsetroundjoin%
\definecolor{currentfill}{rgb}{0.000000,0.000000,0.000000}%
\pgfsetfillcolor{currentfill}%
\pgfsetlinewidth{0.803000pt}%
\definecolor{currentstroke}{rgb}{0.000000,0.000000,0.000000}%
\pgfsetstrokecolor{currentstroke}%
\pgfsetdash{}{0pt}%
\pgfsys@defobject{currentmarker}{\pgfqpoint{0.000000in}{-0.048611in}}{\pgfqpoint{0.000000in}{0.000000in}}{%
\pgfpathmoveto{\pgfqpoint{0.000000in}{0.000000in}}%
\pgfpathlineto{\pgfqpoint{0.000000in}{-0.048611in}}%
\pgfusepath{stroke,fill}%
}%
\begin{pgfscope}%
\pgfsys@transformshift{5.400000in}{0.500000in}%
\pgfsys@useobject{currentmarker}{}%
\end{pgfscope}%
\end{pgfscope}%
\begin{pgfscope}%
\pgftext[x=5.400000in,y=0.402778in,,top]{\rmfamily\fontsize{10.000000}{12.000000}\selectfont \(\displaystyle 6\)}%
\end{pgfscope}%
\begin{pgfscope}%
\pgfpathrectangle{\pgfqpoint{0.750000in}{0.500000in}}{\pgfqpoint{4.650000in}{3.020000in}}%
\pgfusepath{clip}%
\pgfsetrectcap%
\pgfsetroundjoin%
\pgfsetlinewidth{0.803000pt}%
\definecolor{currentstroke}{rgb}{0.690196,0.690196,0.690196}%
\pgfsetstrokecolor{currentstroke}%
\pgfsetdash{}{0pt}%
\pgfpathmoveto{\pgfqpoint{0.750000in}{0.500000in}}%
\pgfpathlineto{\pgfqpoint{5.400000in}{0.500000in}}%
\pgfusepath{stroke}%
\end{pgfscope}%
\begin{pgfscope}%
\pgfsetbuttcap%
\pgfsetroundjoin%
\definecolor{currentfill}{rgb}{0.000000,0.000000,0.000000}%
\pgfsetfillcolor{currentfill}%
\pgfsetlinewidth{0.803000pt}%
\definecolor{currentstroke}{rgb}{0.000000,0.000000,0.000000}%
\pgfsetstrokecolor{currentstroke}%
\pgfsetdash{}{0pt}%
\pgfsys@defobject{currentmarker}{\pgfqpoint{-0.048611in}{0.000000in}}{\pgfqpoint{0.000000in}{0.000000in}}{%
\pgfpathmoveto{\pgfqpoint{0.000000in}{0.000000in}}%
\pgfpathlineto{\pgfqpoint{-0.048611in}{0.000000in}}%
\pgfusepath{stroke,fill}%
}%
\begin{pgfscope}%
\pgfsys@transformshift{0.750000in}{0.500000in}%
\pgfsys@useobject{currentmarker}{}%
\end{pgfscope}%
\end{pgfscope}%
\begin{pgfscope}%
\pgftext[x=0.297838in,y=0.451806in,left,base]{\rmfamily\fontsize{10.000000}{12.000000}\selectfont \(\displaystyle -0.50\)}%
\end{pgfscope}%
\begin{pgfscope}%
\pgfpathrectangle{\pgfqpoint{0.750000in}{0.500000in}}{\pgfqpoint{4.650000in}{3.020000in}}%
\pgfusepath{clip}%
\pgfsetrectcap%
\pgfsetroundjoin%
\pgfsetlinewidth{0.803000pt}%
\definecolor{currentstroke}{rgb}{0.690196,0.690196,0.690196}%
\pgfsetstrokecolor{currentstroke}%
\pgfsetdash{}{0pt}%
\pgfpathmoveto{\pgfqpoint{0.750000in}{0.877500in}}%
\pgfpathlineto{\pgfqpoint{5.400000in}{0.877500in}}%
\pgfusepath{stroke}%
\end{pgfscope}%
\begin{pgfscope}%
\pgfsetbuttcap%
\pgfsetroundjoin%
\definecolor{currentfill}{rgb}{0.000000,0.000000,0.000000}%
\pgfsetfillcolor{currentfill}%
\pgfsetlinewidth{0.803000pt}%
\definecolor{currentstroke}{rgb}{0.000000,0.000000,0.000000}%
\pgfsetstrokecolor{currentstroke}%
\pgfsetdash{}{0pt}%
\pgfsys@defobject{currentmarker}{\pgfqpoint{-0.048611in}{0.000000in}}{\pgfqpoint{0.000000in}{0.000000in}}{%
\pgfpathmoveto{\pgfqpoint{0.000000in}{0.000000in}}%
\pgfpathlineto{\pgfqpoint{-0.048611in}{0.000000in}}%
\pgfusepath{stroke,fill}%
}%
\begin{pgfscope}%
\pgfsys@transformshift{0.750000in}{0.877500in}%
\pgfsys@useobject{currentmarker}{}%
\end{pgfscope}%
\end{pgfscope}%
\begin{pgfscope}%
\pgftext[x=0.297838in,y=0.829306in,left,base]{\rmfamily\fontsize{10.000000}{12.000000}\selectfont \(\displaystyle -0.25\)}%
\end{pgfscope}%
\begin{pgfscope}%
\pgfpathrectangle{\pgfqpoint{0.750000in}{0.500000in}}{\pgfqpoint{4.650000in}{3.020000in}}%
\pgfusepath{clip}%
\pgfsetrectcap%
\pgfsetroundjoin%
\pgfsetlinewidth{0.803000pt}%
\definecolor{currentstroke}{rgb}{0.690196,0.690196,0.690196}%
\pgfsetstrokecolor{currentstroke}%
\pgfsetdash{}{0pt}%
\pgfpathmoveto{\pgfqpoint{0.750000in}{1.255000in}}%
\pgfpathlineto{\pgfqpoint{5.400000in}{1.255000in}}%
\pgfusepath{stroke}%
\end{pgfscope}%
\begin{pgfscope}%
\pgfsetbuttcap%
\pgfsetroundjoin%
\definecolor{currentfill}{rgb}{0.000000,0.000000,0.000000}%
\pgfsetfillcolor{currentfill}%
\pgfsetlinewidth{0.803000pt}%
\definecolor{currentstroke}{rgb}{0.000000,0.000000,0.000000}%
\pgfsetstrokecolor{currentstroke}%
\pgfsetdash{}{0pt}%
\pgfsys@defobject{currentmarker}{\pgfqpoint{-0.048611in}{0.000000in}}{\pgfqpoint{0.000000in}{0.000000in}}{%
\pgfpathmoveto{\pgfqpoint{0.000000in}{0.000000in}}%
\pgfpathlineto{\pgfqpoint{-0.048611in}{0.000000in}}%
\pgfusepath{stroke,fill}%
}%
\begin{pgfscope}%
\pgfsys@transformshift{0.750000in}{1.255000in}%
\pgfsys@useobject{currentmarker}{}%
\end{pgfscope}%
\end{pgfscope}%
\begin{pgfscope}%
\pgftext[x=0.405863in,y=1.206806in,left,base]{\rmfamily\fontsize{10.000000}{12.000000}\selectfont \(\displaystyle 0.00\)}%
\end{pgfscope}%
\begin{pgfscope}%
\pgfpathrectangle{\pgfqpoint{0.750000in}{0.500000in}}{\pgfqpoint{4.650000in}{3.020000in}}%
\pgfusepath{clip}%
\pgfsetrectcap%
\pgfsetroundjoin%
\pgfsetlinewidth{0.803000pt}%
\definecolor{currentstroke}{rgb}{0.690196,0.690196,0.690196}%
\pgfsetstrokecolor{currentstroke}%
\pgfsetdash{}{0pt}%
\pgfpathmoveto{\pgfqpoint{0.750000in}{1.632500in}}%
\pgfpathlineto{\pgfqpoint{5.400000in}{1.632500in}}%
\pgfusepath{stroke}%
\end{pgfscope}%
\begin{pgfscope}%
\pgfsetbuttcap%
\pgfsetroundjoin%
\definecolor{currentfill}{rgb}{0.000000,0.000000,0.000000}%
\pgfsetfillcolor{currentfill}%
\pgfsetlinewidth{0.803000pt}%
\definecolor{currentstroke}{rgb}{0.000000,0.000000,0.000000}%
\pgfsetstrokecolor{currentstroke}%
\pgfsetdash{}{0pt}%
\pgfsys@defobject{currentmarker}{\pgfqpoint{-0.048611in}{0.000000in}}{\pgfqpoint{0.000000in}{0.000000in}}{%
\pgfpathmoveto{\pgfqpoint{0.000000in}{0.000000in}}%
\pgfpathlineto{\pgfqpoint{-0.048611in}{0.000000in}}%
\pgfusepath{stroke,fill}%
}%
\begin{pgfscope}%
\pgfsys@transformshift{0.750000in}{1.632500in}%
\pgfsys@useobject{currentmarker}{}%
\end{pgfscope}%
\end{pgfscope}%
\begin{pgfscope}%
\pgftext[x=0.405863in,y=1.584306in,left,base]{\rmfamily\fontsize{10.000000}{12.000000}\selectfont \(\displaystyle 0.25\)}%
\end{pgfscope}%
\begin{pgfscope}%
\pgfpathrectangle{\pgfqpoint{0.750000in}{0.500000in}}{\pgfqpoint{4.650000in}{3.020000in}}%
\pgfusepath{clip}%
\pgfsetrectcap%
\pgfsetroundjoin%
\pgfsetlinewidth{0.803000pt}%
\definecolor{currentstroke}{rgb}{0.690196,0.690196,0.690196}%
\pgfsetstrokecolor{currentstroke}%
\pgfsetdash{}{0pt}%
\pgfpathmoveto{\pgfqpoint{0.750000in}{2.010000in}}%
\pgfpathlineto{\pgfqpoint{5.400000in}{2.010000in}}%
\pgfusepath{stroke}%
\end{pgfscope}%
\begin{pgfscope}%
\pgfsetbuttcap%
\pgfsetroundjoin%
\definecolor{currentfill}{rgb}{0.000000,0.000000,0.000000}%
\pgfsetfillcolor{currentfill}%
\pgfsetlinewidth{0.803000pt}%
\definecolor{currentstroke}{rgb}{0.000000,0.000000,0.000000}%
\pgfsetstrokecolor{currentstroke}%
\pgfsetdash{}{0pt}%
\pgfsys@defobject{currentmarker}{\pgfqpoint{-0.048611in}{0.000000in}}{\pgfqpoint{0.000000in}{0.000000in}}{%
\pgfpathmoveto{\pgfqpoint{0.000000in}{0.000000in}}%
\pgfpathlineto{\pgfqpoint{-0.048611in}{0.000000in}}%
\pgfusepath{stroke,fill}%
}%
\begin{pgfscope}%
\pgfsys@transformshift{0.750000in}{2.010000in}%
\pgfsys@useobject{currentmarker}{}%
\end{pgfscope}%
\end{pgfscope}%
\begin{pgfscope}%
\pgftext[x=0.405863in,y=1.961806in,left,base]{\rmfamily\fontsize{10.000000}{12.000000}\selectfont \(\displaystyle 0.50\)}%
\end{pgfscope}%
\begin{pgfscope}%
\pgfpathrectangle{\pgfqpoint{0.750000in}{0.500000in}}{\pgfqpoint{4.650000in}{3.020000in}}%
\pgfusepath{clip}%
\pgfsetrectcap%
\pgfsetroundjoin%
\pgfsetlinewidth{0.803000pt}%
\definecolor{currentstroke}{rgb}{0.690196,0.690196,0.690196}%
\pgfsetstrokecolor{currentstroke}%
\pgfsetdash{}{0pt}%
\pgfpathmoveto{\pgfqpoint{0.750000in}{2.387500in}}%
\pgfpathlineto{\pgfqpoint{5.400000in}{2.387500in}}%
\pgfusepath{stroke}%
\end{pgfscope}%
\begin{pgfscope}%
\pgfsetbuttcap%
\pgfsetroundjoin%
\definecolor{currentfill}{rgb}{0.000000,0.000000,0.000000}%
\pgfsetfillcolor{currentfill}%
\pgfsetlinewidth{0.803000pt}%
\definecolor{currentstroke}{rgb}{0.000000,0.000000,0.000000}%
\pgfsetstrokecolor{currentstroke}%
\pgfsetdash{}{0pt}%
\pgfsys@defobject{currentmarker}{\pgfqpoint{-0.048611in}{0.000000in}}{\pgfqpoint{0.000000in}{0.000000in}}{%
\pgfpathmoveto{\pgfqpoint{0.000000in}{0.000000in}}%
\pgfpathlineto{\pgfqpoint{-0.048611in}{0.000000in}}%
\pgfusepath{stroke,fill}%
}%
\begin{pgfscope}%
\pgfsys@transformshift{0.750000in}{2.387500in}%
\pgfsys@useobject{currentmarker}{}%
\end{pgfscope}%
\end{pgfscope}%
\begin{pgfscope}%
\pgftext[x=0.405863in,y=2.339306in,left,base]{\rmfamily\fontsize{10.000000}{12.000000}\selectfont \(\displaystyle 0.75\)}%
\end{pgfscope}%
\begin{pgfscope}%
\pgfpathrectangle{\pgfqpoint{0.750000in}{0.500000in}}{\pgfqpoint{4.650000in}{3.020000in}}%
\pgfusepath{clip}%
\pgfsetrectcap%
\pgfsetroundjoin%
\pgfsetlinewidth{0.803000pt}%
\definecolor{currentstroke}{rgb}{0.690196,0.690196,0.690196}%
\pgfsetstrokecolor{currentstroke}%
\pgfsetdash{}{0pt}%
\pgfpathmoveto{\pgfqpoint{0.750000in}{2.765000in}}%
\pgfpathlineto{\pgfqpoint{5.400000in}{2.765000in}}%
\pgfusepath{stroke}%
\end{pgfscope}%
\begin{pgfscope}%
\pgfsetbuttcap%
\pgfsetroundjoin%
\definecolor{currentfill}{rgb}{0.000000,0.000000,0.000000}%
\pgfsetfillcolor{currentfill}%
\pgfsetlinewidth{0.803000pt}%
\definecolor{currentstroke}{rgb}{0.000000,0.000000,0.000000}%
\pgfsetstrokecolor{currentstroke}%
\pgfsetdash{}{0pt}%
\pgfsys@defobject{currentmarker}{\pgfqpoint{-0.048611in}{0.000000in}}{\pgfqpoint{0.000000in}{0.000000in}}{%
\pgfpathmoveto{\pgfqpoint{0.000000in}{0.000000in}}%
\pgfpathlineto{\pgfqpoint{-0.048611in}{0.000000in}}%
\pgfusepath{stroke,fill}%
}%
\begin{pgfscope}%
\pgfsys@transformshift{0.750000in}{2.765000in}%
\pgfsys@useobject{currentmarker}{}%
\end{pgfscope}%
\end{pgfscope}%
\begin{pgfscope}%
\pgftext[x=0.405863in,y=2.716806in,left,base]{\rmfamily\fontsize{10.000000}{12.000000}\selectfont \(\displaystyle 1.00\)}%
\end{pgfscope}%
\begin{pgfscope}%
\pgfpathrectangle{\pgfqpoint{0.750000in}{0.500000in}}{\pgfqpoint{4.650000in}{3.020000in}}%
\pgfusepath{clip}%
\pgfsetrectcap%
\pgfsetroundjoin%
\pgfsetlinewidth{0.803000pt}%
\definecolor{currentstroke}{rgb}{0.690196,0.690196,0.690196}%
\pgfsetstrokecolor{currentstroke}%
\pgfsetdash{}{0pt}%
\pgfpathmoveto{\pgfqpoint{0.750000in}{3.142500in}}%
\pgfpathlineto{\pgfqpoint{5.400000in}{3.142500in}}%
\pgfusepath{stroke}%
\end{pgfscope}%
\begin{pgfscope}%
\pgfsetbuttcap%
\pgfsetroundjoin%
\definecolor{currentfill}{rgb}{0.000000,0.000000,0.000000}%
\pgfsetfillcolor{currentfill}%
\pgfsetlinewidth{0.803000pt}%
\definecolor{currentstroke}{rgb}{0.000000,0.000000,0.000000}%
\pgfsetstrokecolor{currentstroke}%
\pgfsetdash{}{0pt}%
\pgfsys@defobject{currentmarker}{\pgfqpoint{-0.048611in}{0.000000in}}{\pgfqpoint{0.000000in}{0.000000in}}{%
\pgfpathmoveto{\pgfqpoint{0.000000in}{0.000000in}}%
\pgfpathlineto{\pgfqpoint{-0.048611in}{0.000000in}}%
\pgfusepath{stroke,fill}%
}%
\begin{pgfscope}%
\pgfsys@transformshift{0.750000in}{3.142500in}%
\pgfsys@useobject{currentmarker}{}%
\end{pgfscope}%
\end{pgfscope}%
\begin{pgfscope}%
\pgftext[x=0.405863in,y=3.094306in,left,base]{\rmfamily\fontsize{10.000000}{12.000000}\selectfont \(\displaystyle 1.25\)}%
\end{pgfscope}%
\begin{pgfscope}%
\pgfpathrectangle{\pgfqpoint{0.750000in}{0.500000in}}{\pgfqpoint{4.650000in}{3.020000in}}%
\pgfusepath{clip}%
\pgfsetrectcap%
\pgfsetroundjoin%
\pgfsetlinewidth{0.803000pt}%
\definecolor{currentstroke}{rgb}{0.690196,0.690196,0.690196}%
\pgfsetstrokecolor{currentstroke}%
\pgfsetdash{}{0pt}%
\pgfpathmoveto{\pgfqpoint{0.750000in}{3.520000in}}%
\pgfpathlineto{\pgfqpoint{5.400000in}{3.520000in}}%
\pgfusepath{stroke}%
\end{pgfscope}%
\begin{pgfscope}%
\pgfsetbuttcap%
\pgfsetroundjoin%
\definecolor{currentfill}{rgb}{0.000000,0.000000,0.000000}%
\pgfsetfillcolor{currentfill}%
\pgfsetlinewidth{0.803000pt}%
\definecolor{currentstroke}{rgb}{0.000000,0.000000,0.000000}%
\pgfsetstrokecolor{currentstroke}%
\pgfsetdash{}{0pt}%
\pgfsys@defobject{currentmarker}{\pgfqpoint{-0.048611in}{0.000000in}}{\pgfqpoint{0.000000in}{0.000000in}}{%
\pgfpathmoveto{\pgfqpoint{0.000000in}{0.000000in}}%
\pgfpathlineto{\pgfqpoint{-0.048611in}{0.000000in}}%
\pgfusepath{stroke,fill}%
}%
\begin{pgfscope}%
\pgfsys@transformshift{0.750000in}{3.520000in}%
\pgfsys@useobject{currentmarker}{}%
\end{pgfscope}%
\end{pgfscope}%
\begin{pgfscope}%
\pgftext[x=0.405863in,y=3.471806in,left,base]{\rmfamily\fontsize{10.000000}{12.000000}\selectfont \(\displaystyle 1.50\)}%
\end{pgfscope}%
\begin{pgfscope}%
\pgfpathrectangle{\pgfqpoint{0.750000in}{0.500000in}}{\pgfqpoint{4.650000in}{3.020000in}}%
\pgfusepath{clip}%
\pgfsetrectcap%
\pgfsetroundjoin%
\pgfsetlinewidth{1.505625pt}%
\definecolor{currentstroke}{rgb}{0.121569,0.466667,0.705882}%
\pgfsetstrokecolor{currentstroke}%
\pgfsetdash{}{0pt}%
\pgfpathmoveto{\pgfqpoint{0.750000in}{1.366852in}}%
\pgfpathlineto{\pgfqpoint{5.400000in}{1.366852in}}%
\pgfpathlineto{\pgfqpoint{5.400000in}{1.366852in}}%
\pgfusepath{stroke}%
\end{pgfscope}%
\begin{pgfscope}%
\pgfpathrectangle{\pgfqpoint{0.750000in}{0.500000in}}{\pgfqpoint{4.650000in}{3.020000in}}%
\pgfusepath{clip}%
\pgfsetrectcap%
\pgfsetroundjoin%
\pgfsetlinewidth{1.505625pt}%
\definecolor{currentstroke}{rgb}{1.000000,0.498039,0.054902}%
\pgfsetstrokecolor{currentstroke}%
\pgfsetdash{}{0pt}%
\pgfpathmoveto{\pgfqpoint{0.750000in}{1.109982in}}%
\pgfpathlineto{\pgfqpoint{0.880330in}{1.166871in}}%
\pgfpathlineto{\pgfqpoint{1.010661in}{1.220480in}}%
\pgfpathlineto{\pgfqpoint{1.140991in}{1.270807in}}%
\pgfpathlineto{\pgfqpoint{1.271321in}{1.317854in}}%
\pgfpathlineto{\pgfqpoint{1.401652in}{1.361619in}}%
\pgfpathlineto{\pgfqpoint{1.531982in}{1.402104in}}%
\pgfpathlineto{\pgfqpoint{1.662312in}{1.439307in}}%
\pgfpathlineto{\pgfqpoint{1.792643in}{1.473230in}}%
\pgfpathlineto{\pgfqpoint{1.922973in}{1.503872in}}%
\pgfpathlineto{\pgfqpoint{2.048649in}{1.530312in}}%
\pgfpathlineto{\pgfqpoint{2.174324in}{1.553701in}}%
\pgfpathlineto{\pgfqpoint{2.300000in}{1.574040in}}%
\pgfpathlineto{\pgfqpoint{2.425676in}{1.591327in}}%
\pgfpathlineto{\pgfqpoint{2.551351in}{1.605564in}}%
\pgfpathlineto{\pgfqpoint{2.677027in}{1.616751in}}%
\pgfpathlineto{\pgfqpoint{2.802703in}{1.624886in}}%
\pgfpathlineto{\pgfqpoint{2.928378in}{1.629971in}}%
\pgfpathlineto{\pgfqpoint{3.054054in}{1.632004in}}%
\pgfpathlineto{\pgfqpoint{3.179730in}{1.630988in}}%
\pgfpathlineto{\pgfqpoint{3.305405in}{1.626920in}}%
\pgfpathlineto{\pgfqpoint{3.431081in}{1.619801in}}%
\pgfpathlineto{\pgfqpoint{3.556757in}{1.609632in}}%
\pgfpathlineto{\pgfqpoint{3.682432in}{1.596412in}}%
\pgfpathlineto{\pgfqpoint{3.808108in}{1.580141in}}%
\pgfpathlineto{\pgfqpoint{3.933784in}{1.560820in}}%
\pgfpathlineto{\pgfqpoint{4.059459in}{1.538447in}}%
\pgfpathlineto{\pgfqpoint{4.185135in}{1.513024in}}%
\pgfpathlineto{\pgfqpoint{4.310811in}{1.484550in}}%
\pgfpathlineto{\pgfqpoint{4.441141in}{1.451799in}}%
\pgfpathlineto{\pgfqpoint{4.571471in}{1.415767in}}%
\pgfpathlineto{\pgfqpoint{4.701802in}{1.376454in}}%
\pgfpathlineto{\pgfqpoint{4.832132in}{1.333861in}}%
\pgfpathlineto{\pgfqpoint{4.962462in}{1.287986in}}%
\pgfpathlineto{\pgfqpoint{5.092793in}{1.238830in}}%
\pgfpathlineto{\pgfqpoint{5.223123in}{1.186394in}}%
\pgfpathlineto{\pgfqpoint{5.353453in}{1.130676in}}%
\pgfpathlineto{\pgfqpoint{5.400000in}{1.109982in}}%
\pgfpathlineto{\pgfqpoint{5.400000in}{1.109982in}}%
\pgfusepath{stroke}%
\end{pgfscope}%
\begin{pgfscope}%
\pgfpathrectangle{\pgfqpoint{0.750000in}{0.500000in}}{\pgfqpoint{4.650000in}{3.020000in}}%
\pgfusepath{clip}%
\pgfsetrectcap%
\pgfsetroundjoin%
\pgfsetlinewidth{1.505625pt}%
\definecolor{currentstroke}{rgb}{0.172549,0.627451,0.172549}%
\pgfsetstrokecolor{currentstroke}%
\pgfsetdash{}{0pt}%
\pgfpathmoveto{\pgfqpoint{0.750000in}{1.770851in}}%
\pgfpathlineto{\pgfqpoint{0.791892in}{1.699652in}}%
\pgfpathlineto{\pgfqpoint{0.829129in}{1.642102in}}%
\pgfpathlineto{\pgfqpoint{0.866366in}{1.589729in}}%
\pgfpathlineto{\pgfqpoint{0.903604in}{1.542323in}}%
\pgfpathlineto{\pgfqpoint{0.940841in}{1.499677in}}%
\pgfpathlineto{\pgfqpoint{0.978078in}{1.461587in}}%
\pgfpathlineto{\pgfqpoint{1.015315in}{1.427854in}}%
\pgfpathlineto{\pgfqpoint{1.052553in}{1.398281in}}%
\pgfpathlineto{\pgfqpoint{1.089790in}{1.372675in}}%
\pgfpathlineto{\pgfqpoint{1.127027in}{1.350847in}}%
\pgfpathlineto{\pgfqpoint{1.164264in}{1.332611in}}%
\pgfpathlineto{\pgfqpoint{1.201502in}{1.317785in}}%
\pgfpathlineto{\pgfqpoint{1.238739in}{1.306190in}}%
\pgfpathlineto{\pgfqpoint{1.275976in}{1.297650in}}%
\pgfpathlineto{\pgfqpoint{1.313213in}{1.291995in}}%
\pgfpathlineto{\pgfqpoint{1.355105in}{1.288871in}}%
\pgfpathlineto{\pgfqpoint{1.396997in}{1.288950in}}%
\pgfpathlineto{\pgfqpoint{1.438889in}{1.292004in}}%
\pgfpathlineto{\pgfqpoint{1.485435in}{1.298614in}}%
\pgfpathlineto{\pgfqpoint{1.531982in}{1.308324in}}%
\pgfpathlineto{\pgfqpoint{1.583183in}{1.322241in}}%
\pgfpathlineto{\pgfqpoint{1.639039in}{1.340864in}}%
\pgfpathlineto{\pgfqpoint{1.699550in}{1.364564in}}%
\pgfpathlineto{\pgfqpoint{1.764715in}{1.393553in}}%
\pgfpathlineto{\pgfqpoint{1.839189in}{1.430232in}}%
\pgfpathlineto{\pgfqpoint{1.927628in}{1.477435in}}%
\pgfpathlineto{\pgfqpoint{2.048649in}{1.545857in}}%
\pgfpathlineto{\pgfqpoint{2.286036in}{1.680742in}}%
\pgfpathlineto{\pgfqpoint{2.383784in}{1.732380in}}%
\pgfpathlineto{\pgfqpoint{2.467568in}{1.773315in}}%
\pgfpathlineto{\pgfqpoint{2.546697in}{1.808488in}}%
\pgfpathlineto{\pgfqpoint{2.616517in}{1.836286in}}%
\pgfpathlineto{\pgfqpoint{2.686336in}{1.860729in}}%
\pgfpathlineto{\pgfqpoint{2.751502in}{1.880280in}}%
\pgfpathlineto{\pgfqpoint{2.816667in}{1.896497in}}%
\pgfpathlineto{\pgfqpoint{2.877177in}{1.908441in}}%
\pgfpathlineto{\pgfqpoint{2.937688in}{1.917290in}}%
\pgfpathlineto{\pgfqpoint{2.998198in}{1.922977in}}%
\pgfpathlineto{\pgfqpoint{3.058709in}{1.925457in}}%
\pgfpathlineto{\pgfqpoint{3.119219in}{1.924712in}}%
\pgfpathlineto{\pgfqpoint{3.179730in}{1.920748in}}%
\pgfpathlineto{\pgfqpoint{3.240240in}{1.913595in}}%
\pgfpathlineto{\pgfqpoint{3.300751in}{1.903308in}}%
\pgfpathlineto{\pgfqpoint{3.361261in}{1.889965in}}%
\pgfpathlineto{\pgfqpoint{3.426426in}{1.872299in}}%
\pgfpathlineto{\pgfqpoint{3.491592in}{1.851372in}}%
\pgfpathlineto{\pgfqpoint{3.561411in}{1.825553in}}%
\pgfpathlineto{\pgfqpoint{3.631231in}{1.796496in}}%
\pgfpathlineto{\pgfqpoint{3.710360in}{1.760061in}}%
\pgfpathlineto{\pgfqpoint{3.794144in}{1.718006in}}%
\pgfpathlineto{\pgfqpoint{3.891892in}{1.665384in}}%
\pgfpathlineto{\pgfqpoint{4.026877in}{1.588786in}}%
\pgfpathlineto{\pgfqpoint{4.236336in}{1.469776in}}%
\pgfpathlineto{\pgfqpoint{4.329429in}{1.420757in}}%
\pgfpathlineto{\pgfqpoint{4.403904in}{1.384942in}}%
\pgfpathlineto{\pgfqpoint{4.469069in}{1.356914in}}%
\pgfpathlineto{\pgfqpoint{4.529580in}{1.334285in}}%
\pgfpathlineto{\pgfqpoint{4.585435in}{1.316811in}}%
\pgfpathlineto{\pgfqpoint{4.636637in}{1.304087in}}%
\pgfpathlineto{\pgfqpoint{4.683183in}{1.295582in}}%
\pgfpathlineto{\pgfqpoint{4.729730in}{1.290294in}}%
\pgfpathlineto{\pgfqpoint{4.771622in}{1.288534in}}%
\pgfpathlineto{\pgfqpoint{4.813514in}{1.289850in}}%
\pgfpathlineto{\pgfqpoint{4.855405in}{1.294473in}}%
\pgfpathlineto{\pgfqpoint{4.892643in}{1.301549in}}%
\pgfpathlineto{\pgfqpoint{4.929880in}{1.311594in}}%
\pgfpathlineto{\pgfqpoint{4.967117in}{1.324783in}}%
\pgfpathlineto{\pgfqpoint{5.004354in}{1.341291in}}%
\pgfpathlineto{\pgfqpoint{5.041592in}{1.361300in}}%
\pgfpathlineto{\pgfqpoint{5.078829in}{1.384994in}}%
\pgfpathlineto{\pgfqpoint{5.116066in}{1.412559in}}%
\pgfpathlineto{\pgfqpoint{5.153303in}{1.444188in}}%
\pgfpathlineto{\pgfqpoint{5.190541in}{1.480075in}}%
\pgfpathlineto{\pgfqpoint{5.227778in}{1.520418in}}%
\pgfpathlineto{\pgfqpoint{5.265015in}{1.565419in}}%
\pgfpathlineto{\pgfqpoint{5.302252in}{1.615282in}}%
\pgfpathlineto{\pgfqpoint{5.339489in}{1.670217in}}%
\pgfpathlineto{\pgfqpoint{5.376727in}{1.730435in}}%
\pgfpathlineto{\pgfqpoint{5.400000in}{1.770851in}}%
\pgfpathlineto{\pgfqpoint{5.400000in}{1.770851in}}%
\pgfusepath{stroke}%
\end{pgfscope}%
\begin{pgfscope}%
\pgfpathrectangle{\pgfqpoint{0.750000in}{0.500000in}}{\pgfqpoint{4.650000in}{3.020000in}}%
\pgfusepath{clip}%
\pgfsetrectcap%
\pgfsetroundjoin%
\pgfsetlinewidth{1.505625pt}%
\definecolor{currentstroke}{rgb}{0.839216,0.152941,0.156863}%
\pgfsetstrokecolor{currentstroke}%
\pgfsetdash{}{0pt}%
\pgfpathmoveto{\pgfqpoint{0.813102in}{0.486111in}}%
\pgfpathlineto{\pgfqpoint{0.838438in}{0.604063in}}%
\pgfpathlineto{\pgfqpoint{0.866366in}{0.720063in}}%
\pgfpathlineto{\pgfqpoint{0.894294in}{0.822522in}}%
\pgfpathlineto{\pgfqpoint{0.922222in}{0.912521in}}%
\pgfpathlineto{\pgfqpoint{0.950150in}{0.991083in}}%
\pgfpathlineto{\pgfqpoint{0.978078in}{1.059185in}}%
\pgfpathlineto{\pgfqpoint{1.001351in}{1.108612in}}%
\pgfpathlineto{\pgfqpoint{1.024625in}{1.151927in}}%
\pgfpathlineto{\pgfqpoint{1.047898in}{1.189614in}}%
\pgfpathlineto{\pgfqpoint{1.071171in}{1.222136in}}%
\pgfpathlineto{\pgfqpoint{1.094444in}{1.249936in}}%
\pgfpathlineto{\pgfqpoint{1.117718in}{1.273435in}}%
\pgfpathlineto{\pgfqpoint{1.140991in}{1.293032in}}%
\pgfpathlineto{\pgfqpoint{1.164264in}{1.309107in}}%
\pgfpathlineto{\pgfqpoint{1.192192in}{1.324253in}}%
\pgfpathlineto{\pgfqpoint{1.220120in}{1.335433in}}%
\pgfpathlineto{\pgfqpoint{1.248048in}{1.343195in}}%
\pgfpathlineto{\pgfqpoint{1.280631in}{1.348610in}}%
\pgfpathlineto{\pgfqpoint{1.313213in}{1.350820in}}%
\pgfpathlineto{\pgfqpoint{1.350450in}{1.350301in}}%
\pgfpathlineto{\pgfqpoint{1.396997in}{1.346440in}}%
\pgfpathlineto{\pgfqpoint{1.471471in}{1.336519in}}%
\pgfpathlineto{\pgfqpoint{1.564565in}{1.324658in}}%
\pgfpathlineto{\pgfqpoint{1.620420in}{1.320859in}}%
\pgfpathlineto{\pgfqpoint{1.666967in}{1.320718in}}%
\pgfpathlineto{\pgfqpoint{1.708859in}{1.323404in}}%
\pgfpathlineto{\pgfqpoint{1.750751in}{1.329046in}}%
\pgfpathlineto{\pgfqpoint{1.792643in}{1.337833in}}%
\pgfpathlineto{\pgfqpoint{1.834535in}{1.349875in}}%
\pgfpathlineto{\pgfqpoint{1.876426in}{1.365211in}}%
\pgfpathlineto{\pgfqpoint{1.918318in}{1.383816in}}%
\pgfpathlineto{\pgfqpoint{1.960210in}{1.405603in}}%
\pgfpathlineto{\pgfqpoint{2.006757in}{1.433374in}}%
\pgfpathlineto{\pgfqpoint{2.053303in}{1.464647in}}%
\pgfpathlineto{\pgfqpoint{2.104505in}{1.502711in}}%
\pgfpathlineto{\pgfqpoint{2.160360in}{1.548047in}}%
\pgfpathlineto{\pgfqpoint{2.225526in}{1.605021in}}%
\pgfpathlineto{\pgfqpoint{2.309309in}{1.682780in}}%
\pgfpathlineto{\pgfqpoint{2.551351in}{1.910670in}}%
\pgfpathlineto{\pgfqpoint{2.616517in}{1.966483in}}%
\pgfpathlineto{\pgfqpoint{2.672372in}{2.010592in}}%
\pgfpathlineto{\pgfqpoint{2.723574in}{2.047372in}}%
\pgfpathlineto{\pgfqpoint{2.770120in}{2.077354in}}%
\pgfpathlineto{\pgfqpoint{2.816667in}{2.103712in}}%
\pgfpathlineto{\pgfqpoint{2.858559in}{2.124100in}}%
\pgfpathlineto{\pgfqpoint{2.900450in}{2.141146in}}%
\pgfpathlineto{\pgfqpoint{2.942342in}{2.154702in}}%
\pgfpathlineto{\pgfqpoint{2.979580in}{2.163727in}}%
\pgfpathlineto{\pgfqpoint{3.016817in}{2.169840in}}%
\pgfpathlineto{\pgfqpoint{3.054054in}{2.173001in}}%
\pgfpathlineto{\pgfqpoint{3.091291in}{2.173187in}}%
\pgfpathlineto{\pgfqpoint{3.128529in}{2.170398in}}%
\pgfpathlineto{\pgfqpoint{3.165766in}{2.164651in}}%
\pgfpathlineto{\pgfqpoint{3.203003in}{2.155987in}}%
\pgfpathlineto{\pgfqpoint{3.244895in}{2.142826in}}%
\pgfpathlineto{\pgfqpoint{3.286787in}{2.126162in}}%
\pgfpathlineto{\pgfqpoint{3.328679in}{2.106137in}}%
\pgfpathlineto{\pgfqpoint{3.375225in}{2.080158in}}%
\pgfpathlineto{\pgfqpoint{3.421772in}{2.050524in}}%
\pgfpathlineto{\pgfqpoint{3.472973in}{2.014089in}}%
\pgfpathlineto{\pgfqpoint{3.528829in}{1.970302in}}%
\pgfpathlineto{\pgfqpoint{3.589339in}{1.918889in}}%
\pgfpathlineto{\pgfqpoint{3.663814in}{1.851317in}}%
\pgfpathlineto{\pgfqpoint{3.775526in}{1.745082in}}%
\pgfpathlineto{\pgfqpoint{3.905856in}{1.621944in}}%
\pgfpathlineto{\pgfqpoint{3.980330in}{1.555942in}}%
\pgfpathlineto{\pgfqpoint{4.040841in}{1.506346in}}%
\pgfpathlineto{\pgfqpoint{4.096697in}{1.464647in}}%
\pgfpathlineto{\pgfqpoint{4.147898in}{1.430435in}}%
\pgfpathlineto{\pgfqpoint{4.194444in}{1.403029in}}%
\pgfpathlineto{\pgfqpoint{4.240991in}{1.379403in}}%
\pgfpathlineto{\pgfqpoint{4.282883in}{1.361519in}}%
\pgfpathlineto{\pgfqpoint{4.324775in}{1.346914in}}%
\pgfpathlineto{\pgfqpoint{4.366667in}{1.335601in}}%
\pgfpathlineto{\pgfqpoint{4.408559in}{1.327526in}}%
\pgfpathlineto{\pgfqpoint{4.450450in}{1.322560in}}%
\pgfpathlineto{\pgfqpoint{4.492342in}{1.320496in}}%
\pgfpathlineto{\pgfqpoint{4.538889in}{1.321242in}}%
\pgfpathlineto{\pgfqpoint{4.590090in}{1.325121in}}%
\pgfpathlineto{\pgfqpoint{4.659910in}{1.333860in}}%
\pgfpathlineto{\pgfqpoint{4.771622in}{1.348319in}}%
\pgfpathlineto{\pgfqpoint{4.818168in}{1.350911in}}%
\pgfpathlineto{\pgfqpoint{4.855405in}{1.349908in}}%
\pgfpathlineto{\pgfqpoint{4.887988in}{1.345955in}}%
\pgfpathlineto{\pgfqpoint{4.915916in}{1.339708in}}%
\pgfpathlineto{\pgfqpoint{4.943844in}{1.330304in}}%
\pgfpathlineto{\pgfqpoint{4.971772in}{1.317212in}}%
\pgfpathlineto{\pgfqpoint{4.995045in}{1.303076in}}%
\pgfpathlineto{\pgfqpoint{5.018318in}{1.285637in}}%
\pgfpathlineto{\pgfqpoint{5.041592in}{1.264526in}}%
\pgfpathlineto{\pgfqpoint{5.064865in}{1.239356in}}%
\pgfpathlineto{\pgfqpoint{5.088138in}{1.209718in}}%
\pgfpathlineto{\pgfqpoint{5.111411in}{1.175184in}}%
\pgfpathlineto{\pgfqpoint{5.134685in}{1.135303in}}%
\pgfpathlineto{\pgfqpoint{5.157958in}{1.089603in}}%
\pgfpathlineto{\pgfqpoint{5.181231in}{1.037588in}}%
\pgfpathlineto{\pgfqpoint{5.204505in}{0.978742in}}%
\pgfpathlineto{\pgfqpoint{5.227778in}{0.912521in}}%
\pgfpathlineto{\pgfqpoint{5.255706in}{0.822522in}}%
\pgfpathlineto{\pgfqpoint{5.283634in}{0.720063in}}%
\pgfpathlineto{\pgfqpoint{5.311562in}{0.604063in}}%
\pgfpathlineto{\pgfqpoint{5.336898in}{0.486111in}}%
\pgfpathlineto{\pgfqpoint{5.336898in}{0.486111in}}%
\pgfusepath{stroke}%
\end{pgfscope}%
\begin{pgfscope}%
\pgfpathrectangle{\pgfqpoint{0.750000in}{0.500000in}}{\pgfqpoint{4.650000in}{3.020000in}}%
\pgfusepath{clip}%
\pgfsetrectcap%
\pgfsetroundjoin%
\pgfsetlinewidth{1.505625pt}%
\definecolor{currentstroke}{rgb}{0.580392,0.403922,0.741176}%
\pgfsetstrokecolor{currentstroke}%
\pgfsetdash{}{0pt}%
\pgfpathmoveto{\pgfqpoint{0.780133in}{3.533889in}}%
\pgfpathlineto{\pgfqpoint{0.801201in}{3.210938in}}%
\pgfpathlineto{\pgfqpoint{0.824474in}{2.897653in}}%
\pgfpathlineto{\pgfqpoint{0.847748in}{2.625407in}}%
\pgfpathlineto{\pgfqpoint{0.871021in}{2.390080in}}%
\pgfpathlineto{\pgfqpoint{0.894294in}{2.187858in}}%
\pgfpathlineto{\pgfqpoint{0.917568in}{2.015222in}}%
\pgfpathlineto{\pgfqpoint{0.940841in}{1.868929in}}%
\pgfpathlineto{\pgfqpoint{0.959459in}{1.768847in}}%
\pgfpathlineto{\pgfqpoint{0.978078in}{1.682297in}}%
\pgfpathlineto{\pgfqpoint{0.996697in}{1.607961in}}%
\pgfpathlineto{\pgfqpoint{1.015315in}{1.544613in}}%
\pgfpathlineto{\pgfqpoint{1.033934in}{1.491114in}}%
\pgfpathlineto{\pgfqpoint{1.052553in}{1.446410in}}%
\pgfpathlineto{\pgfqpoint{1.071171in}{1.409522in}}%
\pgfpathlineto{\pgfqpoint{1.089790in}{1.379548in}}%
\pgfpathlineto{\pgfqpoint{1.108408in}{1.355657in}}%
\pgfpathlineto{\pgfqpoint{1.127027in}{1.337085in}}%
\pgfpathlineto{\pgfqpoint{1.145646in}{1.323131in}}%
\pgfpathlineto{\pgfqpoint{1.164264in}{1.313155in}}%
\pgfpathlineto{\pgfqpoint{1.182883in}{1.306573in}}%
\pgfpathlineto{\pgfqpoint{1.201502in}{1.302856in}}%
\pgfpathlineto{\pgfqpoint{1.224775in}{1.301512in}}%
\pgfpathlineto{\pgfqpoint{1.252703in}{1.303656in}}%
\pgfpathlineto{\pgfqpoint{1.285285in}{1.309799in}}%
\pgfpathlineto{\pgfqpoint{1.331832in}{1.322391in}}%
\pgfpathlineto{\pgfqpoint{1.438889in}{1.352641in}}%
\pgfpathlineto{\pgfqpoint{1.485435in}{1.362062in}}%
\pgfpathlineto{\pgfqpoint{1.531982in}{1.368157in}}%
\pgfpathlineto{\pgfqpoint{1.578529in}{1.370978in}}%
\pgfpathlineto{\pgfqpoint{1.629730in}{1.370876in}}%
\pgfpathlineto{\pgfqpoint{1.699550in}{1.367222in}}%
\pgfpathlineto{\pgfqpoint{1.811261in}{1.361000in}}%
\pgfpathlineto{\pgfqpoint{1.862462in}{1.361495in}}%
\pgfpathlineto{\pgfqpoint{1.904354in}{1.364811in}}%
\pgfpathlineto{\pgfqpoint{1.941592in}{1.370517in}}%
\pgfpathlineto{\pgfqpoint{1.978829in}{1.379215in}}%
\pgfpathlineto{\pgfqpoint{2.016066in}{1.391217in}}%
\pgfpathlineto{\pgfqpoint{2.053303in}{1.406758in}}%
\pgfpathlineto{\pgfqpoint{2.090541in}{1.425997in}}%
\pgfpathlineto{\pgfqpoint{2.127778in}{1.449013in}}%
\pgfpathlineto{\pgfqpoint{2.165015in}{1.475810in}}%
\pgfpathlineto{\pgfqpoint{2.202252in}{1.506312in}}%
\pgfpathlineto{\pgfqpoint{2.244144in}{1.544868in}}%
\pgfpathlineto{\pgfqpoint{2.286036in}{1.587608in}}%
\pgfpathlineto{\pgfqpoint{2.332583in}{1.639490in}}%
\pgfpathlineto{\pgfqpoint{2.383784in}{1.701081in}}%
\pgfpathlineto{\pgfqpoint{2.448949in}{1.784602in}}%
\pgfpathlineto{\pgfqpoint{2.569970in}{1.946401in}}%
\pgfpathlineto{\pgfqpoint{2.649099in}{2.049615in}}%
\pgfpathlineto{\pgfqpoint{2.704955in}{2.117855in}}%
\pgfpathlineto{\pgfqpoint{2.751502in}{2.170348in}}%
\pgfpathlineto{\pgfqpoint{2.793393in}{2.213360in}}%
\pgfpathlineto{\pgfqpoint{2.830631in}{2.247699in}}%
\pgfpathlineto{\pgfqpoint{2.867868in}{2.277960in}}%
\pgfpathlineto{\pgfqpoint{2.900450in}{2.300816in}}%
\pgfpathlineto{\pgfqpoint{2.933033in}{2.320078in}}%
\pgfpathlineto{\pgfqpoint{2.965616in}{2.335576in}}%
\pgfpathlineto{\pgfqpoint{2.993544in}{2.345758in}}%
\pgfpathlineto{\pgfqpoint{3.021471in}{2.353006in}}%
\pgfpathlineto{\pgfqpoint{3.049399in}{2.357273in}}%
\pgfpathlineto{\pgfqpoint{3.077327in}{2.358530in}}%
\pgfpathlineto{\pgfqpoint{3.105255in}{2.356770in}}%
\pgfpathlineto{\pgfqpoint{3.133183in}{2.352004in}}%
\pgfpathlineto{\pgfqpoint{3.161111in}{2.344263in}}%
\pgfpathlineto{\pgfqpoint{3.189039in}{2.333598in}}%
\pgfpathlineto{\pgfqpoint{3.221622in}{2.317553in}}%
\pgfpathlineto{\pgfqpoint{3.254204in}{2.297766in}}%
\pgfpathlineto{\pgfqpoint{3.286787in}{2.274413in}}%
\pgfpathlineto{\pgfqpoint{3.324024in}{2.243622in}}%
\pgfpathlineto{\pgfqpoint{3.361261in}{2.208800in}}%
\pgfpathlineto{\pgfqpoint{3.403153in}{2.165308in}}%
\pgfpathlineto{\pgfqpoint{3.449700in}{2.112367in}}%
\pgfpathlineto{\pgfqpoint{3.505556in}{2.043720in}}%
\pgfpathlineto{\pgfqpoint{3.575375in}{1.952606in}}%
\pgfpathlineto{\pgfqpoint{3.761562in}{1.706879in}}%
\pgfpathlineto{\pgfqpoint{3.817417in}{1.639490in}}%
\pgfpathlineto{\pgfqpoint{3.868619in}{1.582664in}}%
\pgfpathlineto{\pgfqpoint{3.915165in}{1.535926in}}%
\pgfpathlineto{\pgfqpoint{3.957057in}{1.498346in}}%
\pgfpathlineto{\pgfqpoint{3.994294in}{1.468759in}}%
\pgfpathlineto{\pgfqpoint{4.031532in}{1.442904in}}%
\pgfpathlineto{\pgfqpoint{4.068769in}{1.420834in}}%
\pgfpathlineto{\pgfqpoint{4.106006in}{1.402531in}}%
\pgfpathlineto{\pgfqpoint{4.143243in}{1.387893in}}%
\pgfpathlineto{\pgfqpoint{4.180480in}{1.376742in}}%
\pgfpathlineto{\pgfqpoint{4.217718in}{1.368823in}}%
\pgfpathlineto{\pgfqpoint{4.259610in}{1.363368in}}%
\pgfpathlineto{\pgfqpoint{4.301502in}{1.361017in}}%
\pgfpathlineto{\pgfqpoint{4.352703in}{1.361388in}}%
\pgfpathlineto{\pgfqpoint{4.427177in}{1.365617in}}%
\pgfpathlineto{\pgfqpoint{4.524925in}{1.371000in}}%
\pgfpathlineto{\pgfqpoint{4.576126in}{1.370832in}}%
\pgfpathlineto{\pgfqpoint{4.622673in}{1.367699in}}%
\pgfpathlineto{\pgfqpoint{4.669219in}{1.361266in}}%
\pgfpathlineto{\pgfqpoint{4.715766in}{1.351530in}}%
\pgfpathlineto{\pgfqpoint{4.771622in}{1.336340in}}%
\pgfpathlineto{\pgfqpoint{4.869369in}{1.308742in}}%
\pgfpathlineto{\pgfqpoint{4.901952in}{1.303064in}}%
\pgfpathlineto{\pgfqpoint{4.929880in}{1.301524in}}%
\pgfpathlineto{\pgfqpoint{4.953153in}{1.303544in}}%
\pgfpathlineto{\pgfqpoint{4.971772in}{1.307930in}}%
\pgfpathlineto{\pgfqpoint{4.990390in}{1.315309in}}%
\pgfpathlineto{\pgfqpoint{5.009009in}{1.326223in}}%
\pgfpathlineto{\pgfqpoint{5.027628in}{1.341269in}}%
\pgfpathlineto{\pgfqpoint{5.046246in}{1.361102in}}%
\pgfpathlineto{\pgfqpoint{5.064865in}{1.386440in}}%
\pgfpathlineto{\pgfqpoint{5.083483in}{1.418061in}}%
\pgfpathlineto{\pgfqpoint{5.102102in}{1.456816in}}%
\pgfpathlineto{\pgfqpoint{5.120721in}{1.503625in}}%
\pgfpathlineto{\pgfqpoint{5.139339in}{1.559483in}}%
\pgfpathlineto{\pgfqpoint{5.157958in}{1.625468in}}%
\pgfpathlineto{\pgfqpoint{5.176577in}{1.702739in}}%
\pgfpathlineto{\pgfqpoint{5.195195in}{1.792545in}}%
\pgfpathlineto{\pgfqpoint{5.213814in}{1.896227in}}%
\pgfpathlineto{\pgfqpoint{5.232432in}{2.015222in}}%
\pgfpathlineto{\pgfqpoint{5.251051in}{2.151072in}}%
\pgfpathlineto{\pgfqpoint{5.274324in}{2.347103in}}%
\pgfpathlineto{\pgfqpoint{5.297598in}{2.575515in}}%
\pgfpathlineto{\pgfqpoint{5.320871in}{2.840057in}}%
\pgfpathlineto{\pgfqpoint{5.344144in}{3.144789in}}%
\pgfpathlineto{\pgfqpoint{5.367417in}{3.494091in}}%
\pgfpathlineto{\pgfqpoint{5.369867in}{3.533889in}}%
\pgfpathlineto{\pgfqpoint{5.369867in}{3.533889in}}%
\pgfusepath{stroke}%
\end{pgfscope}%
\begin{pgfscope}%
\pgfpathrectangle{\pgfqpoint{0.750000in}{0.500000in}}{\pgfqpoint{4.650000in}{3.020000in}}%
\pgfusepath{clip}%
\pgfsetrectcap%
\pgfsetroundjoin%
\pgfsetlinewidth{1.505625pt}%
\definecolor{currentstroke}{rgb}{0.549020,0.337255,0.294118}%
\pgfsetstrokecolor{currentstroke}%
\pgfsetdash{}{0pt}%
\pgfpathmoveto{\pgfqpoint{0.750000in}{1.295811in}}%
\pgfpathlineto{\pgfqpoint{1.001351in}{1.305950in}}%
\pgfpathlineto{\pgfqpoint{1.206156in}{1.317243in}}%
\pgfpathlineto{\pgfqpoint{1.373724in}{1.329474in}}%
\pgfpathlineto{\pgfqpoint{1.518018in}{1.343075in}}%
\pgfpathlineto{\pgfqpoint{1.639039in}{1.357496in}}%
\pgfpathlineto{\pgfqpoint{1.746096in}{1.373329in}}%
\pgfpathlineto{\pgfqpoint{1.839189in}{1.390172in}}%
\pgfpathlineto{\pgfqpoint{1.922973in}{1.408478in}}%
\pgfpathlineto{\pgfqpoint{1.997447in}{1.427912in}}%
\pgfpathlineto{\pgfqpoint{2.062613in}{1.447953in}}%
\pgfpathlineto{\pgfqpoint{2.123123in}{1.469666in}}%
\pgfpathlineto{\pgfqpoint{2.178979in}{1.492916in}}%
\pgfpathlineto{\pgfqpoint{2.230180in}{1.517463in}}%
\pgfpathlineto{\pgfqpoint{2.276727in}{1.542957in}}%
\pgfpathlineto{\pgfqpoint{2.323273in}{1.572003in}}%
\pgfpathlineto{\pgfqpoint{2.365165in}{1.601680in}}%
\pgfpathlineto{\pgfqpoint{2.407057in}{1.635235in}}%
\pgfpathlineto{\pgfqpoint{2.444294in}{1.668793in}}%
\pgfpathlineto{\pgfqpoint{2.481532in}{1.706341in}}%
\pgfpathlineto{\pgfqpoint{2.518769in}{1.748387in}}%
\pgfpathlineto{\pgfqpoint{2.556006in}{1.795477in}}%
\pgfpathlineto{\pgfqpoint{2.593243in}{1.848168in}}%
\pgfpathlineto{\pgfqpoint{2.630480in}{1.907001in}}%
\pgfpathlineto{\pgfqpoint{2.667718in}{1.972439in}}%
\pgfpathlineto{\pgfqpoint{2.704955in}{2.044774in}}%
\pgfpathlineto{\pgfqpoint{2.742192in}{2.123995in}}%
\pgfpathlineto{\pgfqpoint{2.784084in}{2.220704in}}%
\pgfpathlineto{\pgfqpoint{2.839940in}{2.358824in}}%
\pgfpathlineto{\pgfqpoint{2.914414in}{2.543683in}}%
\pgfpathlineto{\pgfqpoint{2.946997in}{2.616442in}}%
\pgfpathlineto{\pgfqpoint{2.970270in}{2.662209in}}%
\pgfpathlineto{\pgfqpoint{2.993544in}{2.701099in}}%
\pgfpathlineto{\pgfqpoint{3.012162in}{2.726310in}}%
\pgfpathlineto{\pgfqpoint{3.030781in}{2.745589in}}%
\pgfpathlineto{\pgfqpoint{3.044745in}{2.755851in}}%
\pgfpathlineto{\pgfqpoint{3.058709in}{2.762336in}}%
\pgfpathlineto{\pgfqpoint{3.072673in}{2.764946in}}%
\pgfpathlineto{\pgfqpoint{3.086637in}{2.763640in}}%
\pgfpathlineto{\pgfqpoint{3.100601in}{2.758438in}}%
\pgfpathlineto{\pgfqpoint{3.114565in}{2.749421in}}%
\pgfpathlineto{\pgfqpoint{3.128529in}{2.736725in}}%
\pgfpathlineto{\pgfqpoint{3.147147in}{2.714409in}}%
\pgfpathlineto{\pgfqpoint{3.165766in}{2.686462in}}%
\pgfpathlineto{\pgfqpoint{3.189039in}{2.644644in}}%
\pgfpathlineto{\pgfqpoint{3.216967in}{2.586306in}}%
\pgfpathlineto{\pgfqpoint{3.249550in}{2.510294in}}%
\pgfpathlineto{\pgfqpoint{3.310060in}{2.358824in}}%
\pgfpathlineto{\pgfqpoint{3.370571in}{2.209604in}}%
\pgfpathlineto{\pgfqpoint{3.412462in}{2.113727in}}%
\pgfpathlineto{\pgfqpoint{3.449700in}{2.035352in}}%
\pgfpathlineto{\pgfqpoint{3.486937in}{1.963886in}}%
\pgfpathlineto{\pgfqpoint{3.524174in}{1.899294in}}%
\pgfpathlineto{\pgfqpoint{3.561411in}{1.841256in}}%
\pgfpathlineto{\pgfqpoint{3.598649in}{1.789296in}}%
\pgfpathlineto{\pgfqpoint{3.635886in}{1.742867in}}%
\pgfpathlineto{\pgfqpoint{3.673123in}{1.701412in}}%
\pgfpathlineto{\pgfqpoint{3.710360in}{1.664390in}}%
\pgfpathlineto{\pgfqpoint{3.752252in}{1.627414in}}%
\pgfpathlineto{\pgfqpoint{3.794144in}{1.594769in}}%
\pgfpathlineto{\pgfqpoint{3.836036in}{1.565882in}}%
\pgfpathlineto{\pgfqpoint{3.882583in}{1.537591in}}%
\pgfpathlineto{\pgfqpoint{3.929129in}{1.512744in}}%
\pgfpathlineto{\pgfqpoint{3.980330in}{1.488802in}}%
\pgfpathlineto{\pgfqpoint{4.036186in}{1.466107in}}%
\pgfpathlineto{\pgfqpoint{4.096697in}{1.444893in}}%
\pgfpathlineto{\pgfqpoint{4.161862in}{1.425296in}}%
\pgfpathlineto{\pgfqpoint{4.236336in}{1.406272in}}%
\pgfpathlineto{\pgfqpoint{4.315465in}{1.389250in}}%
\pgfpathlineto{\pgfqpoint{4.403904in}{1.373329in}}%
\pgfpathlineto{\pgfqpoint{4.506306in}{1.358118in}}%
\pgfpathlineto{\pgfqpoint{4.622673in}{1.344075in}}%
\pgfpathlineto{\pgfqpoint{4.757658in}{1.331048in}}%
\pgfpathlineto{\pgfqpoint{4.911261in}{1.319377in}}%
\pgfpathlineto{\pgfqpoint{5.097447in}{1.308470in}}%
\pgfpathlineto{\pgfqpoint{5.320871in}{1.298653in}}%
\pgfpathlineto{\pgfqpoint{5.400000in}{1.295811in}}%
\pgfpathlineto{\pgfqpoint{5.400000in}{1.295811in}}%
\pgfusepath{stroke}%
\end{pgfscope}%
\begin{pgfscope}%
\pgfsetrectcap%
\pgfsetmiterjoin%
\pgfsetlinewidth{0.803000pt}%
\definecolor{currentstroke}{rgb}{0.000000,0.000000,0.000000}%
\pgfsetstrokecolor{currentstroke}%
\pgfsetdash{}{0pt}%
\pgfpathmoveto{\pgfqpoint{0.750000in}{0.500000in}}%
\pgfpathlineto{\pgfqpoint{0.750000in}{3.520000in}}%
\pgfusepath{stroke}%
\end{pgfscope}%
\begin{pgfscope}%
\pgfsetrectcap%
\pgfsetmiterjoin%
\pgfsetlinewidth{0.803000pt}%
\definecolor{currentstroke}{rgb}{0.000000,0.000000,0.000000}%
\pgfsetstrokecolor{currentstroke}%
\pgfsetdash{}{0pt}%
\pgfpathmoveto{\pgfqpoint{5.400000in}{0.500000in}}%
\pgfpathlineto{\pgfqpoint{5.400000in}{3.520000in}}%
\pgfusepath{stroke}%
\end{pgfscope}%
\begin{pgfscope}%
\pgfsetrectcap%
\pgfsetmiterjoin%
\pgfsetlinewidth{0.803000pt}%
\definecolor{currentstroke}{rgb}{0.000000,0.000000,0.000000}%
\pgfsetstrokecolor{currentstroke}%
\pgfsetdash{}{0pt}%
\pgfpathmoveto{\pgfqpoint{0.750000in}{0.500000in}}%
\pgfpathlineto{\pgfqpoint{5.400000in}{0.500000in}}%
\pgfusepath{stroke}%
\end{pgfscope}%
\begin{pgfscope}%
\pgfsetrectcap%
\pgfsetmiterjoin%
\pgfsetlinewidth{0.803000pt}%
\definecolor{currentstroke}{rgb}{0.000000,0.000000,0.000000}%
\pgfsetstrokecolor{currentstroke}%
\pgfsetdash{}{0pt}%
\pgfpathmoveto{\pgfqpoint{0.750000in}{3.520000in}}%
\pgfpathlineto{\pgfqpoint{5.400000in}{3.520000in}}%
\pgfusepath{stroke}%
\end{pgfscope}%
\begin{pgfscope}%
\pgfsetbuttcap%
\pgfsetmiterjoin%
\definecolor{currentfill}{rgb}{1.000000,1.000000,1.000000}%
\pgfsetfillcolor{currentfill}%
\pgfsetfillopacity{0.800000}%
\pgfsetlinewidth{1.003750pt}%
\definecolor{currentstroke}{rgb}{0.800000,0.800000,0.800000}%
\pgfsetstrokecolor{currentstroke}%
\pgfsetstrokeopacity{0.800000}%
\pgfsetdash{}{0pt}%
\pgfpathmoveto{\pgfqpoint{4.520339in}{2.247161in}}%
\pgfpathlineto{\pgfqpoint{5.302778in}{2.247161in}}%
\pgfpathquadraticcurveto{\pgfqpoint{5.330556in}{2.247161in}}{\pgfqpoint{5.330556in}{2.274939in}}%
\pgfpathlineto{\pgfqpoint{5.330556in}{3.422778in}}%
\pgfpathquadraticcurveto{\pgfqpoint{5.330556in}{3.450556in}}{\pgfqpoint{5.302778in}{3.450556in}}%
\pgfpathlineto{\pgfqpoint{4.520339in}{3.450556in}}%
\pgfpathquadraticcurveto{\pgfqpoint{4.492561in}{3.450556in}}{\pgfqpoint{4.492561in}{3.422778in}}%
\pgfpathlineto{\pgfqpoint{4.492561in}{2.274939in}}%
\pgfpathquadraticcurveto{\pgfqpoint{4.492561in}{2.247161in}}{\pgfqpoint{4.520339in}{2.247161in}}%
\pgfpathclose%
\pgfusepath{stroke,fill}%
\end{pgfscope}%
\begin{pgfscope}%
\pgfsetrectcap%
\pgfsetroundjoin%
\pgfsetlinewidth{1.505625pt}%
\definecolor{currentstroke}{rgb}{0.121569,0.466667,0.705882}%
\pgfsetstrokecolor{currentstroke}%
\pgfsetdash{}{0pt}%
\pgfpathmoveto{\pgfqpoint{4.548117in}{3.346389in}}%
\pgfpathlineto{\pgfqpoint{4.825895in}{3.346389in}}%
\pgfusepath{stroke}%
\end{pgfscope}%
\begin{pgfscope}%
\pgftext[x=4.937006in,y=3.297778in,left,base]{\rmfamily\fontsize{10.000000}{12.000000}\selectfont \(\displaystyle  n = 1 \)}%
\end{pgfscope}%
\begin{pgfscope}%
\pgfsetrectcap%
\pgfsetroundjoin%
\pgfsetlinewidth{1.505625pt}%
\definecolor{currentstroke}{rgb}{1.000000,0.498039,0.054902}%
\pgfsetstrokecolor{currentstroke}%
\pgfsetdash{}{0pt}%
\pgfpathmoveto{\pgfqpoint{4.548117in}{3.152778in}}%
\pgfpathlineto{\pgfqpoint{4.825895in}{3.152778in}}%
\pgfusepath{stroke}%
\end{pgfscope}%
\begin{pgfscope}%
\pgftext[x=4.937006in,y=3.104167in,left,base]{\rmfamily\fontsize{10.000000}{12.000000}\selectfont \(\displaystyle  n = 3 \)}%
\end{pgfscope}%
\begin{pgfscope}%
\pgfsetrectcap%
\pgfsetroundjoin%
\pgfsetlinewidth{1.505625pt}%
\definecolor{currentstroke}{rgb}{0.172549,0.627451,0.172549}%
\pgfsetstrokecolor{currentstroke}%
\pgfsetdash{}{0pt}%
\pgfpathmoveto{\pgfqpoint{4.548117in}{2.959167in}}%
\pgfpathlineto{\pgfqpoint{4.825895in}{2.959167in}}%
\pgfusepath{stroke}%
\end{pgfscope}%
\begin{pgfscope}%
\pgftext[x=4.937006in,y=2.910556in,left,base]{\rmfamily\fontsize{10.000000}{12.000000}\selectfont \(\displaystyle  n = 5 \)}%
\end{pgfscope}%
\begin{pgfscope}%
\pgfsetrectcap%
\pgfsetroundjoin%
\pgfsetlinewidth{1.505625pt}%
\definecolor{currentstroke}{rgb}{0.839216,0.152941,0.156863}%
\pgfsetstrokecolor{currentstroke}%
\pgfsetdash{}{0pt}%
\pgfpathmoveto{\pgfqpoint{4.548117in}{2.765556in}}%
\pgfpathlineto{\pgfqpoint{4.825895in}{2.765556in}}%
\pgfusepath{stroke}%
\end{pgfscope}%
\begin{pgfscope}%
\pgftext[x=4.937006in,y=2.716945in,left,base]{\rmfamily\fontsize{10.000000}{12.000000}\selectfont \(\displaystyle  n = 7 \)}%
\end{pgfscope}%
\begin{pgfscope}%
\pgfsetrectcap%
\pgfsetroundjoin%
\pgfsetlinewidth{1.505625pt}%
\definecolor{currentstroke}{rgb}{0.580392,0.403922,0.741176}%
\pgfsetstrokecolor{currentstroke}%
\pgfsetdash{}{0pt}%
\pgfpathmoveto{\pgfqpoint{4.548117in}{2.571945in}}%
\pgfpathlineto{\pgfqpoint{4.825895in}{2.571945in}}%
\pgfusepath{stroke}%
\end{pgfscope}%
\begin{pgfscope}%
\pgftext[x=4.937006in,y=2.523334in,left,base]{\rmfamily\fontsize{10.000000}{12.000000}\selectfont \(\displaystyle  n = 9 \)}%
\end{pgfscope}%
\begin{pgfscope}%
\pgfsetrectcap%
\pgfsetroundjoin%
\pgfsetlinewidth{1.505625pt}%
\definecolor{currentstroke}{rgb}{0.549020,0.337255,0.294118}%
\pgfsetstrokecolor{currentstroke}%
\pgfsetdash{}{0pt}%
\pgfpathmoveto{\pgfqpoint{4.548117in}{2.378334in}}%
\pgfpathlineto{\pgfqpoint{4.825895in}{2.378334in}}%
\pgfusepath{stroke}%
\end{pgfscope}%
\begin{pgfscope}%
\pgftext[x=4.937006in,y=2.329723in,left,base]{\rmfamily\fontsize{10.000000}{12.000000}\selectfont \(\displaystyle f_1\)}%
\end{pgfscope}%
\end{pgfpicture}%
\makeatother%
\endgroup%
}
\centering \scalebox{0.8}{%% Creator: Matplotlib, PGF backend
%%
%% To include the figure in your LaTeX document, write
%%   \input{<filename>.pgf}
%%
%% Make sure the required packages are loaded in your preamble
%%   \usepackage{pgf}
%%
%% Figures using additional raster images can only be included by \input if
%% they are in the same directory as the main LaTeX file. For loading figures
%% from other directories you can use the `import` package
%%   \usepackage{import}
%% and then include the figures with
%%   \import{<path to file>}{<filename>.pgf}
%%
%% Matplotlib used the following preamble
%%   \usepackage{fontspec}
%%
\begingroup%
\makeatletter%
\begin{pgfpicture}%
\pgfpathrectangle{\pgfpointorigin}{\pgfqpoint{6.000000in}{4.000000in}}%
\pgfusepath{use as bounding box, clip}%
\begin{pgfscope}%
\pgfsetbuttcap%
\pgfsetmiterjoin%
\definecolor{currentfill}{rgb}{1.000000,1.000000,1.000000}%
\pgfsetfillcolor{currentfill}%
\pgfsetlinewidth{0.000000pt}%
\definecolor{currentstroke}{rgb}{1.000000,1.000000,1.000000}%
\pgfsetstrokecolor{currentstroke}%
\pgfsetdash{}{0pt}%
\pgfpathmoveto{\pgfqpoint{0.000000in}{0.000000in}}%
\pgfpathlineto{\pgfqpoint{6.000000in}{0.000000in}}%
\pgfpathlineto{\pgfqpoint{6.000000in}{4.000000in}}%
\pgfpathlineto{\pgfqpoint{0.000000in}{4.000000in}}%
\pgfpathclose%
\pgfusepath{fill}%
\end{pgfscope}%
\begin{pgfscope}%
\pgfsetbuttcap%
\pgfsetmiterjoin%
\definecolor{currentfill}{rgb}{1.000000,1.000000,1.000000}%
\pgfsetfillcolor{currentfill}%
\pgfsetlinewidth{0.000000pt}%
\definecolor{currentstroke}{rgb}{0.000000,0.000000,0.000000}%
\pgfsetstrokecolor{currentstroke}%
\pgfsetstrokeopacity{0.000000}%
\pgfsetdash{}{0pt}%
\pgfpathmoveto{\pgfqpoint{0.750000in}{0.500000in}}%
\pgfpathlineto{\pgfqpoint{5.400000in}{0.500000in}}%
\pgfpathlineto{\pgfqpoint{5.400000in}{3.520000in}}%
\pgfpathlineto{\pgfqpoint{0.750000in}{3.520000in}}%
\pgfpathclose%
\pgfusepath{fill}%
\end{pgfscope}%
\begin{pgfscope}%
\pgfpathrectangle{\pgfqpoint{0.750000in}{0.500000in}}{\pgfqpoint{4.650000in}{3.020000in}}%
\pgfusepath{clip}%
\pgfsetrectcap%
\pgfsetroundjoin%
\pgfsetlinewidth{0.803000pt}%
\definecolor{currentstroke}{rgb}{0.690196,0.690196,0.690196}%
\pgfsetstrokecolor{currentstroke}%
\pgfsetdash{}{0pt}%
\pgfpathmoveto{\pgfqpoint{0.750000in}{0.500000in}}%
\pgfpathlineto{\pgfqpoint{0.750000in}{3.520000in}}%
\pgfusepath{stroke}%
\end{pgfscope}%
\begin{pgfscope}%
\pgfsetbuttcap%
\pgfsetroundjoin%
\definecolor{currentfill}{rgb}{0.000000,0.000000,0.000000}%
\pgfsetfillcolor{currentfill}%
\pgfsetlinewidth{0.803000pt}%
\definecolor{currentstroke}{rgb}{0.000000,0.000000,0.000000}%
\pgfsetstrokecolor{currentstroke}%
\pgfsetdash{}{0pt}%
\pgfsys@defobject{currentmarker}{\pgfqpoint{0.000000in}{-0.048611in}}{\pgfqpoint{0.000000in}{0.000000in}}{%
\pgfpathmoveto{\pgfqpoint{0.000000in}{0.000000in}}%
\pgfpathlineto{\pgfqpoint{0.000000in}{-0.048611in}}%
\pgfusepath{stroke,fill}%
}%
\begin{pgfscope}%
\pgfsys@transformshift{0.750000in}{0.500000in}%
\pgfsys@useobject{currentmarker}{}%
\end{pgfscope}%
\end{pgfscope}%
\begin{pgfscope}%
\pgftext[x=0.750000in,y=0.402778in,,top]{\rmfamily\fontsize{10.000000}{12.000000}\selectfont \(\displaystyle -6\)}%
\end{pgfscope}%
\begin{pgfscope}%
\pgfpathrectangle{\pgfqpoint{0.750000in}{0.500000in}}{\pgfqpoint{4.650000in}{3.020000in}}%
\pgfusepath{clip}%
\pgfsetrectcap%
\pgfsetroundjoin%
\pgfsetlinewidth{0.803000pt}%
\definecolor{currentstroke}{rgb}{0.690196,0.690196,0.690196}%
\pgfsetstrokecolor{currentstroke}%
\pgfsetdash{}{0pt}%
\pgfpathmoveto{\pgfqpoint{1.525000in}{0.500000in}}%
\pgfpathlineto{\pgfqpoint{1.525000in}{3.520000in}}%
\pgfusepath{stroke}%
\end{pgfscope}%
\begin{pgfscope}%
\pgfsetbuttcap%
\pgfsetroundjoin%
\definecolor{currentfill}{rgb}{0.000000,0.000000,0.000000}%
\pgfsetfillcolor{currentfill}%
\pgfsetlinewidth{0.803000pt}%
\definecolor{currentstroke}{rgb}{0.000000,0.000000,0.000000}%
\pgfsetstrokecolor{currentstroke}%
\pgfsetdash{}{0pt}%
\pgfsys@defobject{currentmarker}{\pgfqpoint{0.000000in}{-0.048611in}}{\pgfqpoint{0.000000in}{0.000000in}}{%
\pgfpathmoveto{\pgfqpoint{0.000000in}{0.000000in}}%
\pgfpathlineto{\pgfqpoint{0.000000in}{-0.048611in}}%
\pgfusepath{stroke,fill}%
}%
\begin{pgfscope}%
\pgfsys@transformshift{1.525000in}{0.500000in}%
\pgfsys@useobject{currentmarker}{}%
\end{pgfscope}%
\end{pgfscope}%
\begin{pgfscope}%
\pgftext[x=1.525000in,y=0.402778in,,top]{\rmfamily\fontsize{10.000000}{12.000000}\selectfont \(\displaystyle -4\)}%
\end{pgfscope}%
\begin{pgfscope}%
\pgfpathrectangle{\pgfqpoint{0.750000in}{0.500000in}}{\pgfqpoint{4.650000in}{3.020000in}}%
\pgfusepath{clip}%
\pgfsetrectcap%
\pgfsetroundjoin%
\pgfsetlinewidth{0.803000pt}%
\definecolor{currentstroke}{rgb}{0.690196,0.690196,0.690196}%
\pgfsetstrokecolor{currentstroke}%
\pgfsetdash{}{0pt}%
\pgfpathmoveto{\pgfqpoint{2.300000in}{0.500000in}}%
\pgfpathlineto{\pgfqpoint{2.300000in}{3.520000in}}%
\pgfusepath{stroke}%
\end{pgfscope}%
\begin{pgfscope}%
\pgfsetbuttcap%
\pgfsetroundjoin%
\definecolor{currentfill}{rgb}{0.000000,0.000000,0.000000}%
\pgfsetfillcolor{currentfill}%
\pgfsetlinewidth{0.803000pt}%
\definecolor{currentstroke}{rgb}{0.000000,0.000000,0.000000}%
\pgfsetstrokecolor{currentstroke}%
\pgfsetdash{}{0pt}%
\pgfsys@defobject{currentmarker}{\pgfqpoint{0.000000in}{-0.048611in}}{\pgfqpoint{0.000000in}{0.000000in}}{%
\pgfpathmoveto{\pgfqpoint{0.000000in}{0.000000in}}%
\pgfpathlineto{\pgfqpoint{0.000000in}{-0.048611in}}%
\pgfusepath{stroke,fill}%
}%
\begin{pgfscope}%
\pgfsys@transformshift{2.300000in}{0.500000in}%
\pgfsys@useobject{currentmarker}{}%
\end{pgfscope}%
\end{pgfscope}%
\begin{pgfscope}%
\pgftext[x=2.300000in,y=0.402778in,,top]{\rmfamily\fontsize{10.000000}{12.000000}\selectfont \(\displaystyle -2\)}%
\end{pgfscope}%
\begin{pgfscope}%
\pgfpathrectangle{\pgfqpoint{0.750000in}{0.500000in}}{\pgfqpoint{4.650000in}{3.020000in}}%
\pgfusepath{clip}%
\pgfsetrectcap%
\pgfsetroundjoin%
\pgfsetlinewidth{0.803000pt}%
\definecolor{currentstroke}{rgb}{0.690196,0.690196,0.690196}%
\pgfsetstrokecolor{currentstroke}%
\pgfsetdash{}{0pt}%
\pgfpathmoveto{\pgfqpoint{3.075000in}{0.500000in}}%
\pgfpathlineto{\pgfqpoint{3.075000in}{3.520000in}}%
\pgfusepath{stroke}%
\end{pgfscope}%
\begin{pgfscope}%
\pgfsetbuttcap%
\pgfsetroundjoin%
\definecolor{currentfill}{rgb}{0.000000,0.000000,0.000000}%
\pgfsetfillcolor{currentfill}%
\pgfsetlinewidth{0.803000pt}%
\definecolor{currentstroke}{rgb}{0.000000,0.000000,0.000000}%
\pgfsetstrokecolor{currentstroke}%
\pgfsetdash{}{0pt}%
\pgfsys@defobject{currentmarker}{\pgfqpoint{0.000000in}{-0.048611in}}{\pgfqpoint{0.000000in}{0.000000in}}{%
\pgfpathmoveto{\pgfqpoint{0.000000in}{0.000000in}}%
\pgfpathlineto{\pgfqpoint{0.000000in}{-0.048611in}}%
\pgfusepath{stroke,fill}%
}%
\begin{pgfscope}%
\pgfsys@transformshift{3.075000in}{0.500000in}%
\pgfsys@useobject{currentmarker}{}%
\end{pgfscope}%
\end{pgfscope}%
\begin{pgfscope}%
\pgftext[x=3.075000in,y=0.402778in,,top]{\rmfamily\fontsize{10.000000}{12.000000}\selectfont \(\displaystyle 0\)}%
\end{pgfscope}%
\begin{pgfscope}%
\pgfpathrectangle{\pgfqpoint{0.750000in}{0.500000in}}{\pgfqpoint{4.650000in}{3.020000in}}%
\pgfusepath{clip}%
\pgfsetrectcap%
\pgfsetroundjoin%
\pgfsetlinewidth{0.803000pt}%
\definecolor{currentstroke}{rgb}{0.690196,0.690196,0.690196}%
\pgfsetstrokecolor{currentstroke}%
\pgfsetdash{}{0pt}%
\pgfpathmoveto{\pgfqpoint{3.850000in}{0.500000in}}%
\pgfpathlineto{\pgfqpoint{3.850000in}{3.520000in}}%
\pgfusepath{stroke}%
\end{pgfscope}%
\begin{pgfscope}%
\pgfsetbuttcap%
\pgfsetroundjoin%
\definecolor{currentfill}{rgb}{0.000000,0.000000,0.000000}%
\pgfsetfillcolor{currentfill}%
\pgfsetlinewidth{0.803000pt}%
\definecolor{currentstroke}{rgb}{0.000000,0.000000,0.000000}%
\pgfsetstrokecolor{currentstroke}%
\pgfsetdash{}{0pt}%
\pgfsys@defobject{currentmarker}{\pgfqpoint{0.000000in}{-0.048611in}}{\pgfqpoint{0.000000in}{0.000000in}}{%
\pgfpathmoveto{\pgfqpoint{0.000000in}{0.000000in}}%
\pgfpathlineto{\pgfqpoint{0.000000in}{-0.048611in}}%
\pgfusepath{stroke,fill}%
}%
\begin{pgfscope}%
\pgfsys@transformshift{3.850000in}{0.500000in}%
\pgfsys@useobject{currentmarker}{}%
\end{pgfscope}%
\end{pgfscope}%
\begin{pgfscope}%
\pgftext[x=3.850000in,y=0.402778in,,top]{\rmfamily\fontsize{10.000000}{12.000000}\selectfont \(\displaystyle 2\)}%
\end{pgfscope}%
\begin{pgfscope}%
\pgfpathrectangle{\pgfqpoint{0.750000in}{0.500000in}}{\pgfqpoint{4.650000in}{3.020000in}}%
\pgfusepath{clip}%
\pgfsetrectcap%
\pgfsetroundjoin%
\pgfsetlinewidth{0.803000pt}%
\definecolor{currentstroke}{rgb}{0.690196,0.690196,0.690196}%
\pgfsetstrokecolor{currentstroke}%
\pgfsetdash{}{0pt}%
\pgfpathmoveto{\pgfqpoint{4.625000in}{0.500000in}}%
\pgfpathlineto{\pgfqpoint{4.625000in}{3.520000in}}%
\pgfusepath{stroke}%
\end{pgfscope}%
\begin{pgfscope}%
\pgfsetbuttcap%
\pgfsetroundjoin%
\definecolor{currentfill}{rgb}{0.000000,0.000000,0.000000}%
\pgfsetfillcolor{currentfill}%
\pgfsetlinewidth{0.803000pt}%
\definecolor{currentstroke}{rgb}{0.000000,0.000000,0.000000}%
\pgfsetstrokecolor{currentstroke}%
\pgfsetdash{}{0pt}%
\pgfsys@defobject{currentmarker}{\pgfqpoint{0.000000in}{-0.048611in}}{\pgfqpoint{0.000000in}{0.000000in}}{%
\pgfpathmoveto{\pgfqpoint{0.000000in}{0.000000in}}%
\pgfpathlineto{\pgfqpoint{0.000000in}{-0.048611in}}%
\pgfusepath{stroke,fill}%
}%
\begin{pgfscope}%
\pgfsys@transformshift{4.625000in}{0.500000in}%
\pgfsys@useobject{currentmarker}{}%
\end{pgfscope}%
\end{pgfscope}%
\begin{pgfscope}%
\pgftext[x=4.625000in,y=0.402778in,,top]{\rmfamily\fontsize{10.000000}{12.000000}\selectfont \(\displaystyle 4\)}%
\end{pgfscope}%
\begin{pgfscope}%
\pgfpathrectangle{\pgfqpoint{0.750000in}{0.500000in}}{\pgfqpoint{4.650000in}{3.020000in}}%
\pgfusepath{clip}%
\pgfsetrectcap%
\pgfsetroundjoin%
\pgfsetlinewidth{0.803000pt}%
\definecolor{currentstroke}{rgb}{0.690196,0.690196,0.690196}%
\pgfsetstrokecolor{currentstroke}%
\pgfsetdash{}{0pt}%
\pgfpathmoveto{\pgfqpoint{5.400000in}{0.500000in}}%
\pgfpathlineto{\pgfqpoint{5.400000in}{3.520000in}}%
\pgfusepath{stroke}%
\end{pgfscope}%
\begin{pgfscope}%
\pgfsetbuttcap%
\pgfsetroundjoin%
\definecolor{currentfill}{rgb}{0.000000,0.000000,0.000000}%
\pgfsetfillcolor{currentfill}%
\pgfsetlinewidth{0.803000pt}%
\definecolor{currentstroke}{rgb}{0.000000,0.000000,0.000000}%
\pgfsetstrokecolor{currentstroke}%
\pgfsetdash{}{0pt}%
\pgfsys@defobject{currentmarker}{\pgfqpoint{0.000000in}{-0.048611in}}{\pgfqpoint{0.000000in}{0.000000in}}{%
\pgfpathmoveto{\pgfqpoint{0.000000in}{0.000000in}}%
\pgfpathlineto{\pgfqpoint{0.000000in}{-0.048611in}}%
\pgfusepath{stroke,fill}%
}%
\begin{pgfscope}%
\pgfsys@transformshift{5.400000in}{0.500000in}%
\pgfsys@useobject{currentmarker}{}%
\end{pgfscope}%
\end{pgfscope}%
\begin{pgfscope}%
\pgftext[x=5.400000in,y=0.402778in,,top]{\rmfamily\fontsize{10.000000}{12.000000}\selectfont \(\displaystyle 6\)}%
\end{pgfscope}%
\begin{pgfscope}%
\pgfpathrectangle{\pgfqpoint{0.750000in}{0.500000in}}{\pgfqpoint{4.650000in}{3.020000in}}%
\pgfusepath{clip}%
\pgfsetrectcap%
\pgfsetroundjoin%
\pgfsetlinewidth{0.803000pt}%
\definecolor{currentstroke}{rgb}{0.690196,0.690196,0.690196}%
\pgfsetstrokecolor{currentstroke}%
\pgfsetdash{}{0pt}%
\pgfpathmoveto{\pgfqpoint{0.750000in}{0.500000in}}%
\pgfpathlineto{\pgfqpoint{5.400000in}{0.500000in}}%
\pgfusepath{stroke}%
\end{pgfscope}%
\begin{pgfscope}%
\pgfsetbuttcap%
\pgfsetroundjoin%
\definecolor{currentfill}{rgb}{0.000000,0.000000,0.000000}%
\pgfsetfillcolor{currentfill}%
\pgfsetlinewidth{0.803000pt}%
\definecolor{currentstroke}{rgb}{0.000000,0.000000,0.000000}%
\pgfsetstrokecolor{currentstroke}%
\pgfsetdash{}{0pt}%
\pgfsys@defobject{currentmarker}{\pgfqpoint{-0.048611in}{0.000000in}}{\pgfqpoint{0.000000in}{0.000000in}}{%
\pgfpathmoveto{\pgfqpoint{0.000000in}{0.000000in}}%
\pgfpathlineto{\pgfqpoint{-0.048611in}{0.000000in}}%
\pgfusepath{stroke,fill}%
}%
\begin{pgfscope}%
\pgfsys@transformshift{0.750000in}{0.500000in}%
\pgfsys@useobject{currentmarker}{}%
\end{pgfscope}%
\end{pgfscope}%
\begin{pgfscope}%
\pgftext[x=0.297838in,y=0.451806in,left,base]{\rmfamily\fontsize{10.000000}{12.000000}\selectfont \(\displaystyle -0.50\)}%
\end{pgfscope}%
\begin{pgfscope}%
\pgfpathrectangle{\pgfqpoint{0.750000in}{0.500000in}}{\pgfqpoint{4.650000in}{3.020000in}}%
\pgfusepath{clip}%
\pgfsetrectcap%
\pgfsetroundjoin%
\pgfsetlinewidth{0.803000pt}%
\definecolor{currentstroke}{rgb}{0.690196,0.690196,0.690196}%
\pgfsetstrokecolor{currentstroke}%
\pgfsetdash{}{0pt}%
\pgfpathmoveto{\pgfqpoint{0.750000in}{0.877500in}}%
\pgfpathlineto{\pgfqpoint{5.400000in}{0.877500in}}%
\pgfusepath{stroke}%
\end{pgfscope}%
\begin{pgfscope}%
\pgfsetbuttcap%
\pgfsetroundjoin%
\definecolor{currentfill}{rgb}{0.000000,0.000000,0.000000}%
\pgfsetfillcolor{currentfill}%
\pgfsetlinewidth{0.803000pt}%
\definecolor{currentstroke}{rgb}{0.000000,0.000000,0.000000}%
\pgfsetstrokecolor{currentstroke}%
\pgfsetdash{}{0pt}%
\pgfsys@defobject{currentmarker}{\pgfqpoint{-0.048611in}{0.000000in}}{\pgfqpoint{0.000000in}{0.000000in}}{%
\pgfpathmoveto{\pgfqpoint{0.000000in}{0.000000in}}%
\pgfpathlineto{\pgfqpoint{-0.048611in}{0.000000in}}%
\pgfusepath{stroke,fill}%
}%
\begin{pgfscope}%
\pgfsys@transformshift{0.750000in}{0.877500in}%
\pgfsys@useobject{currentmarker}{}%
\end{pgfscope}%
\end{pgfscope}%
\begin{pgfscope}%
\pgftext[x=0.297838in,y=0.829306in,left,base]{\rmfamily\fontsize{10.000000}{12.000000}\selectfont \(\displaystyle -0.25\)}%
\end{pgfscope}%
\begin{pgfscope}%
\pgfpathrectangle{\pgfqpoint{0.750000in}{0.500000in}}{\pgfqpoint{4.650000in}{3.020000in}}%
\pgfusepath{clip}%
\pgfsetrectcap%
\pgfsetroundjoin%
\pgfsetlinewidth{0.803000pt}%
\definecolor{currentstroke}{rgb}{0.690196,0.690196,0.690196}%
\pgfsetstrokecolor{currentstroke}%
\pgfsetdash{}{0pt}%
\pgfpathmoveto{\pgfqpoint{0.750000in}{1.255000in}}%
\pgfpathlineto{\pgfqpoint{5.400000in}{1.255000in}}%
\pgfusepath{stroke}%
\end{pgfscope}%
\begin{pgfscope}%
\pgfsetbuttcap%
\pgfsetroundjoin%
\definecolor{currentfill}{rgb}{0.000000,0.000000,0.000000}%
\pgfsetfillcolor{currentfill}%
\pgfsetlinewidth{0.803000pt}%
\definecolor{currentstroke}{rgb}{0.000000,0.000000,0.000000}%
\pgfsetstrokecolor{currentstroke}%
\pgfsetdash{}{0pt}%
\pgfsys@defobject{currentmarker}{\pgfqpoint{-0.048611in}{0.000000in}}{\pgfqpoint{0.000000in}{0.000000in}}{%
\pgfpathmoveto{\pgfqpoint{0.000000in}{0.000000in}}%
\pgfpathlineto{\pgfqpoint{-0.048611in}{0.000000in}}%
\pgfusepath{stroke,fill}%
}%
\begin{pgfscope}%
\pgfsys@transformshift{0.750000in}{1.255000in}%
\pgfsys@useobject{currentmarker}{}%
\end{pgfscope}%
\end{pgfscope}%
\begin{pgfscope}%
\pgftext[x=0.405863in,y=1.206806in,left,base]{\rmfamily\fontsize{10.000000}{12.000000}\selectfont \(\displaystyle 0.00\)}%
\end{pgfscope}%
\begin{pgfscope}%
\pgfpathrectangle{\pgfqpoint{0.750000in}{0.500000in}}{\pgfqpoint{4.650000in}{3.020000in}}%
\pgfusepath{clip}%
\pgfsetrectcap%
\pgfsetroundjoin%
\pgfsetlinewidth{0.803000pt}%
\definecolor{currentstroke}{rgb}{0.690196,0.690196,0.690196}%
\pgfsetstrokecolor{currentstroke}%
\pgfsetdash{}{0pt}%
\pgfpathmoveto{\pgfqpoint{0.750000in}{1.632500in}}%
\pgfpathlineto{\pgfqpoint{5.400000in}{1.632500in}}%
\pgfusepath{stroke}%
\end{pgfscope}%
\begin{pgfscope}%
\pgfsetbuttcap%
\pgfsetroundjoin%
\definecolor{currentfill}{rgb}{0.000000,0.000000,0.000000}%
\pgfsetfillcolor{currentfill}%
\pgfsetlinewidth{0.803000pt}%
\definecolor{currentstroke}{rgb}{0.000000,0.000000,0.000000}%
\pgfsetstrokecolor{currentstroke}%
\pgfsetdash{}{0pt}%
\pgfsys@defobject{currentmarker}{\pgfqpoint{-0.048611in}{0.000000in}}{\pgfqpoint{0.000000in}{0.000000in}}{%
\pgfpathmoveto{\pgfqpoint{0.000000in}{0.000000in}}%
\pgfpathlineto{\pgfqpoint{-0.048611in}{0.000000in}}%
\pgfusepath{stroke,fill}%
}%
\begin{pgfscope}%
\pgfsys@transformshift{0.750000in}{1.632500in}%
\pgfsys@useobject{currentmarker}{}%
\end{pgfscope}%
\end{pgfscope}%
\begin{pgfscope}%
\pgftext[x=0.405863in,y=1.584306in,left,base]{\rmfamily\fontsize{10.000000}{12.000000}\selectfont \(\displaystyle 0.25\)}%
\end{pgfscope}%
\begin{pgfscope}%
\pgfpathrectangle{\pgfqpoint{0.750000in}{0.500000in}}{\pgfqpoint{4.650000in}{3.020000in}}%
\pgfusepath{clip}%
\pgfsetrectcap%
\pgfsetroundjoin%
\pgfsetlinewidth{0.803000pt}%
\definecolor{currentstroke}{rgb}{0.690196,0.690196,0.690196}%
\pgfsetstrokecolor{currentstroke}%
\pgfsetdash{}{0pt}%
\pgfpathmoveto{\pgfqpoint{0.750000in}{2.010000in}}%
\pgfpathlineto{\pgfqpoint{5.400000in}{2.010000in}}%
\pgfusepath{stroke}%
\end{pgfscope}%
\begin{pgfscope}%
\pgfsetbuttcap%
\pgfsetroundjoin%
\definecolor{currentfill}{rgb}{0.000000,0.000000,0.000000}%
\pgfsetfillcolor{currentfill}%
\pgfsetlinewidth{0.803000pt}%
\definecolor{currentstroke}{rgb}{0.000000,0.000000,0.000000}%
\pgfsetstrokecolor{currentstroke}%
\pgfsetdash{}{0pt}%
\pgfsys@defobject{currentmarker}{\pgfqpoint{-0.048611in}{0.000000in}}{\pgfqpoint{0.000000in}{0.000000in}}{%
\pgfpathmoveto{\pgfqpoint{0.000000in}{0.000000in}}%
\pgfpathlineto{\pgfqpoint{-0.048611in}{0.000000in}}%
\pgfusepath{stroke,fill}%
}%
\begin{pgfscope}%
\pgfsys@transformshift{0.750000in}{2.010000in}%
\pgfsys@useobject{currentmarker}{}%
\end{pgfscope}%
\end{pgfscope}%
\begin{pgfscope}%
\pgftext[x=0.405863in,y=1.961806in,left,base]{\rmfamily\fontsize{10.000000}{12.000000}\selectfont \(\displaystyle 0.50\)}%
\end{pgfscope}%
\begin{pgfscope}%
\pgfpathrectangle{\pgfqpoint{0.750000in}{0.500000in}}{\pgfqpoint{4.650000in}{3.020000in}}%
\pgfusepath{clip}%
\pgfsetrectcap%
\pgfsetroundjoin%
\pgfsetlinewidth{0.803000pt}%
\definecolor{currentstroke}{rgb}{0.690196,0.690196,0.690196}%
\pgfsetstrokecolor{currentstroke}%
\pgfsetdash{}{0pt}%
\pgfpathmoveto{\pgfqpoint{0.750000in}{2.387500in}}%
\pgfpathlineto{\pgfqpoint{5.400000in}{2.387500in}}%
\pgfusepath{stroke}%
\end{pgfscope}%
\begin{pgfscope}%
\pgfsetbuttcap%
\pgfsetroundjoin%
\definecolor{currentfill}{rgb}{0.000000,0.000000,0.000000}%
\pgfsetfillcolor{currentfill}%
\pgfsetlinewidth{0.803000pt}%
\definecolor{currentstroke}{rgb}{0.000000,0.000000,0.000000}%
\pgfsetstrokecolor{currentstroke}%
\pgfsetdash{}{0pt}%
\pgfsys@defobject{currentmarker}{\pgfqpoint{-0.048611in}{0.000000in}}{\pgfqpoint{0.000000in}{0.000000in}}{%
\pgfpathmoveto{\pgfqpoint{0.000000in}{0.000000in}}%
\pgfpathlineto{\pgfqpoint{-0.048611in}{0.000000in}}%
\pgfusepath{stroke,fill}%
}%
\begin{pgfscope}%
\pgfsys@transformshift{0.750000in}{2.387500in}%
\pgfsys@useobject{currentmarker}{}%
\end{pgfscope}%
\end{pgfscope}%
\begin{pgfscope}%
\pgftext[x=0.405863in,y=2.339306in,left,base]{\rmfamily\fontsize{10.000000}{12.000000}\selectfont \(\displaystyle 0.75\)}%
\end{pgfscope}%
\begin{pgfscope}%
\pgfpathrectangle{\pgfqpoint{0.750000in}{0.500000in}}{\pgfqpoint{4.650000in}{3.020000in}}%
\pgfusepath{clip}%
\pgfsetrectcap%
\pgfsetroundjoin%
\pgfsetlinewidth{0.803000pt}%
\definecolor{currentstroke}{rgb}{0.690196,0.690196,0.690196}%
\pgfsetstrokecolor{currentstroke}%
\pgfsetdash{}{0pt}%
\pgfpathmoveto{\pgfqpoint{0.750000in}{2.765000in}}%
\pgfpathlineto{\pgfqpoint{5.400000in}{2.765000in}}%
\pgfusepath{stroke}%
\end{pgfscope}%
\begin{pgfscope}%
\pgfsetbuttcap%
\pgfsetroundjoin%
\definecolor{currentfill}{rgb}{0.000000,0.000000,0.000000}%
\pgfsetfillcolor{currentfill}%
\pgfsetlinewidth{0.803000pt}%
\definecolor{currentstroke}{rgb}{0.000000,0.000000,0.000000}%
\pgfsetstrokecolor{currentstroke}%
\pgfsetdash{}{0pt}%
\pgfsys@defobject{currentmarker}{\pgfqpoint{-0.048611in}{0.000000in}}{\pgfqpoint{0.000000in}{0.000000in}}{%
\pgfpathmoveto{\pgfqpoint{0.000000in}{0.000000in}}%
\pgfpathlineto{\pgfqpoint{-0.048611in}{0.000000in}}%
\pgfusepath{stroke,fill}%
}%
\begin{pgfscope}%
\pgfsys@transformshift{0.750000in}{2.765000in}%
\pgfsys@useobject{currentmarker}{}%
\end{pgfscope}%
\end{pgfscope}%
\begin{pgfscope}%
\pgftext[x=0.405863in,y=2.716806in,left,base]{\rmfamily\fontsize{10.000000}{12.000000}\selectfont \(\displaystyle 1.00\)}%
\end{pgfscope}%
\begin{pgfscope}%
\pgfpathrectangle{\pgfqpoint{0.750000in}{0.500000in}}{\pgfqpoint{4.650000in}{3.020000in}}%
\pgfusepath{clip}%
\pgfsetrectcap%
\pgfsetroundjoin%
\pgfsetlinewidth{0.803000pt}%
\definecolor{currentstroke}{rgb}{0.690196,0.690196,0.690196}%
\pgfsetstrokecolor{currentstroke}%
\pgfsetdash{}{0pt}%
\pgfpathmoveto{\pgfqpoint{0.750000in}{3.142500in}}%
\pgfpathlineto{\pgfqpoint{5.400000in}{3.142500in}}%
\pgfusepath{stroke}%
\end{pgfscope}%
\begin{pgfscope}%
\pgfsetbuttcap%
\pgfsetroundjoin%
\definecolor{currentfill}{rgb}{0.000000,0.000000,0.000000}%
\pgfsetfillcolor{currentfill}%
\pgfsetlinewidth{0.803000pt}%
\definecolor{currentstroke}{rgb}{0.000000,0.000000,0.000000}%
\pgfsetstrokecolor{currentstroke}%
\pgfsetdash{}{0pt}%
\pgfsys@defobject{currentmarker}{\pgfqpoint{-0.048611in}{0.000000in}}{\pgfqpoint{0.000000in}{0.000000in}}{%
\pgfpathmoveto{\pgfqpoint{0.000000in}{0.000000in}}%
\pgfpathlineto{\pgfqpoint{-0.048611in}{0.000000in}}%
\pgfusepath{stroke,fill}%
}%
\begin{pgfscope}%
\pgfsys@transformshift{0.750000in}{3.142500in}%
\pgfsys@useobject{currentmarker}{}%
\end{pgfscope}%
\end{pgfscope}%
\begin{pgfscope}%
\pgftext[x=0.405863in,y=3.094306in,left,base]{\rmfamily\fontsize{10.000000}{12.000000}\selectfont \(\displaystyle 1.25\)}%
\end{pgfscope}%
\begin{pgfscope}%
\pgfpathrectangle{\pgfqpoint{0.750000in}{0.500000in}}{\pgfqpoint{4.650000in}{3.020000in}}%
\pgfusepath{clip}%
\pgfsetrectcap%
\pgfsetroundjoin%
\pgfsetlinewidth{0.803000pt}%
\definecolor{currentstroke}{rgb}{0.690196,0.690196,0.690196}%
\pgfsetstrokecolor{currentstroke}%
\pgfsetdash{}{0pt}%
\pgfpathmoveto{\pgfqpoint{0.750000in}{3.520000in}}%
\pgfpathlineto{\pgfqpoint{5.400000in}{3.520000in}}%
\pgfusepath{stroke}%
\end{pgfscope}%
\begin{pgfscope}%
\pgfsetbuttcap%
\pgfsetroundjoin%
\definecolor{currentfill}{rgb}{0.000000,0.000000,0.000000}%
\pgfsetfillcolor{currentfill}%
\pgfsetlinewidth{0.803000pt}%
\definecolor{currentstroke}{rgb}{0.000000,0.000000,0.000000}%
\pgfsetstrokecolor{currentstroke}%
\pgfsetdash{}{0pt}%
\pgfsys@defobject{currentmarker}{\pgfqpoint{-0.048611in}{0.000000in}}{\pgfqpoint{0.000000in}{0.000000in}}{%
\pgfpathmoveto{\pgfqpoint{0.000000in}{0.000000in}}%
\pgfpathlineto{\pgfqpoint{-0.048611in}{0.000000in}}%
\pgfusepath{stroke,fill}%
}%
\begin{pgfscope}%
\pgfsys@transformshift{0.750000in}{3.520000in}%
\pgfsys@useobject{currentmarker}{}%
\end{pgfscope}%
\end{pgfscope}%
\begin{pgfscope}%
\pgftext[x=0.405863in,y=3.471806in,left,base]{\rmfamily\fontsize{10.000000}{12.000000}\selectfont \(\displaystyle 1.50\)}%
\end{pgfscope}%
\begin{pgfscope}%
\pgfpathrectangle{\pgfqpoint{0.750000in}{0.500000in}}{\pgfqpoint{4.650000in}{3.020000in}}%
\pgfusepath{clip}%
\pgfsetrectcap%
\pgfsetroundjoin%
\pgfsetlinewidth{1.505625pt}%
\definecolor{currentstroke}{rgb}{0.121569,0.466667,0.705882}%
\pgfsetstrokecolor{currentstroke}%
\pgfsetdash{}{0pt}%
\pgfpathmoveto{\pgfqpoint{0.959425in}{0.486111in}}%
\pgfpathlineto{\pgfqpoint{1.038589in}{0.653469in}}%
\pgfpathlineto{\pgfqpoint{1.122372in}{0.823644in}}%
\pgfpathlineto{\pgfqpoint{1.206156in}{0.986670in}}%
\pgfpathlineto{\pgfqpoint{1.285285in}{1.134075in}}%
\pgfpathlineto{\pgfqpoint{1.364414in}{1.275104in}}%
\pgfpathlineto{\pgfqpoint{1.443544in}{1.409756in}}%
\pgfpathlineto{\pgfqpoint{1.518018in}{1.530663in}}%
\pgfpathlineto{\pgfqpoint{1.592492in}{1.645922in}}%
\pgfpathlineto{\pgfqpoint{1.666967in}{1.755533in}}%
\pgfpathlineto{\pgfqpoint{1.741441in}{1.859495in}}%
\pgfpathlineto{\pgfqpoint{1.811261in}{1.951831in}}%
\pgfpathlineto{\pgfqpoint{1.881081in}{2.039201in}}%
\pgfpathlineto{\pgfqpoint{1.950901in}{2.121608in}}%
\pgfpathlineto{\pgfqpoint{2.020721in}{2.199050in}}%
\pgfpathlineto{\pgfqpoint{2.085886in}{2.266851in}}%
\pgfpathlineto{\pgfqpoint{2.151051in}{2.330327in}}%
\pgfpathlineto{\pgfqpoint{2.216216in}{2.389479in}}%
\pgfpathlineto{\pgfqpoint{2.281381in}{2.444307in}}%
\pgfpathlineto{\pgfqpoint{2.341892in}{2.491346in}}%
\pgfpathlineto{\pgfqpoint{2.402402in}{2.534656in}}%
\pgfpathlineto{\pgfqpoint{2.462913in}{2.574238in}}%
\pgfpathlineto{\pgfqpoint{2.523423in}{2.610090in}}%
\pgfpathlineto{\pgfqpoint{2.583934in}{2.642215in}}%
\pgfpathlineto{\pgfqpoint{2.644444in}{2.670610in}}%
\pgfpathlineto{\pgfqpoint{2.700300in}{2.693512in}}%
\pgfpathlineto{\pgfqpoint{2.756156in}{2.713237in}}%
\pgfpathlineto{\pgfqpoint{2.812012in}{2.729784in}}%
\pgfpathlineto{\pgfqpoint{2.867868in}{2.743155in}}%
\pgfpathlineto{\pgfqpoint{2.923724in}{2.753348in}}%
\pgfpathlineto{\pgfqpoint{2.979580in}{2.760364in}}%
\pgfpathlineto{\pgfqpoint{3.035435in}{2.764203in}}%
\pgfpathlineto{\pgfqpoint{3.091291in}{2.764865in}}%
\pgfpathlineto{\pgfqpoint{3.147147in}{2.762350in}}%
\pgfpathlineto{\pgfqpoint{3.203003in}{2.756657in}}%
\pgfpathlineto{\pgfqpoint{3.258859in}{2.747788in}}%
\pgfpathlineto{\pgfqpoint{3.314715in}{2.735741in}}%
\pgfpathlineto{\pgfqpoint{3.370571in}{2.720518in}}%
\pgfpathlineto{\pgfqpoint{3.426426in}{2.702117in}}%
\pgfpathlineto{\pgfqpoint{3.482282in}{2.680539in}}%
\pgfpathlineto{\pgfqpoint{3.538138in}{2.655784in}}%
\pgfpathlineto{\pgfqpoint{3.593994in}{2.627851in}}%
\pgfpathlineto{\pgfqpoint{3.654505in}{2.594006in}}%
\pgfpathlineto{\pgfqpoint{3.715015in}{2.556432in}}%
\pgfpathlineto{\pgfqpoint{3.775526in}{2.515130in}}%
\pgfpathlineto{\pgfqpoint{3.836036in}{2.470099in}}%
\pgfpathlineto{\pgfqpoint{3.896547in}{2.421339in}}%
\pgfpathlineto{\pgfqpoint{3.961712in}{2.364658in}}%
\pgfpathlineto{\pgfqpoint{4.026877in}{2.303653in}}%
\pgfpathlineto{\pgfqpoint{4.092042in}{2.238323in}}%
\pgfpathlineto{\pgfqpoint{4.157207in}{2.168669in}}%
\pgfpathlineto{\pgfqpoint{4.222372in}{2.094691in}}%
\pgfpathlineto{\pgfqpoint{4.292192in}{2.010629in}}%
\pgfpathlineto{\pgfqpoint{4.362012in}{1.921604in}}%
\pgfpathlineto{\pgfqpoint{4.431832in}{1.827614in}}%
\pgfpathlineto{\pgfqpoint{4.501652in}{1.728660in}}%
\pgfpathlineto{\pgfqpoint{4.576126in}{1.617637in}}%
\pgfpathlineto{\pgfqpoint{4.650601in}{1.500966in}}%
\pgfpathlineto{\pgfqpoint{4.725075in}{1.378647in}}%
\pgfpathlineto{\pgfqpoint{4.804204in}{1.242494in}}%
\pgfpathlineto{\pgfqpoint{4.883333in}{1.099965in}}%
\pgfpathlineto{\pgfqpoint{4.962462in}{0.951059in}}%
\pgfpathlineto{\pgfqpoint{5.041592in}{0.795777in}}%
\pgfpathlineto{\pgfqpoint{5.125375in}{0.624411in}}%
\pgfpathlineto{\pgfqpoint{5.190575in}{0.486111in}}%
\pgfpathlineto{\pgfqpoint{5.190575in}{0.486111in}}%
\pgfusepath{stroke}%
\end{pgfscope}%
\begin{pgfscope}%
\pgfpathrectangle{\pgfqpoint{0.750000in}{0.500000in}}{\pgfqpoint{4.650000in}{3.020000in}}%
\pgfusepath{clip}%
\pgfsetrectcap%
\pgfsetroundjoin%
\pgfsetlinewidth{1.505625pt}%
\definecolor{currentstroke}{rgb}{1.000000,0.498039,0.054902}%
\pgfsetstrokecolor{currentstroke}%
\pgfsetdash{}{0pt}%
\pgfpathmoveto{\pgfqpoint{0.765789in}{3.533889in}}%
\pgfpathlineto{\pgfqpoint{0.805856in}{3.229419in}}%
\pgfpathlineto{\pgfqpoint{0.843093in}{2.968586in}}%
\pgfpathlineto{\pgfqpoint{0.880330in}{2.728195in}}%
\pgfpathlineto{\pgfqpoint{0.917568in}{2.507440in}}%
\pgfpathlineto{\pgfqpoint{0.954805in}{2.305527in}}%
\pgfpathlineto{\pgfqpoint{0.992042in}{2.121675in}}%
\pgfpathlineto{\pgfqpoint{1.024625in}{1.975018in}}%
\pgfpathlineto{\pgfqpoint{1.057207in}{1.841098in}}%
\pgfpathlineto{\pgfqpoint{1.089790in}{1.719418in}}%
\pgfpathlineto{\pgfqpoint{1.122372in}{1.609488in}}%
\pgfpathlineto{\pgfqpoint{1.154955in}{1.510827in}}%
\pgfpathlineto{\pgfqpoint{1.187538in}{1.422963in}}%
\pgfpathlineto{\pgfqpoint{1.215465in}{1.355892in}}%
\pgfpathlineto{\pgfqpoint{1.243393in}{1.296122in}}%
\pgfpathlineto{\pgfqpoint{1.271321in}{1.243371in}}%
\pgfpathlineto{\pgfqpoint{1.299249in}{1.197358in}}%
\pgfpathlineto{\pgfqpoint{1.327177in}{1.157808in}}%
\pgfpathlineto{\pgfqpoint{1.350450in}{1.129591in}}%
\pgfpathlineto{\pgfqpoint{1.373724in}{1.105519in}}%
\pgfpathlineto{\pgfqpoint{1.396997in}{1.085440in}}%
\pgfpathlineto{\pgfqpoint{1.420270in}{1.069204in}}%
\pgfpathlineto{\pgfqpoint{1.443544in}{1.056661in}}%
\pgfpathlineto{\pgfqpoint{1.466817in}{1.047667in}}%
\pgfpathlineto{\pgfqpoint{1.490090in}{1.042075in}}%
\pgfpathlineto{\pgfqpoint{1.513363in}{1.039746in}}%
\pgfpathlineto{\pgfqpoint{1.536637in}{1.040537in}}%
\pgfpathlineto{\pgfqpoint{1.559910in}{1.044311in}}%
\pgfpathlineto{\pgfqpoint{1.583183in}{1.050933in}}%
\pgfpathlineto{\pgfqpoint{1.611111in}{1.062450in}}%
\pgfpathlineto{\pgfqpoint{1.639039in}{1.077646in}}%
\pgfpathlineto{\pgfqpoint{1.666967in}{1.096299in}}%
\pgfpathlineto{\pgfqpoint{1.699550in}{1.122139in}}%
\pgfpathlineto{\pgfqpoint{1.732132in}{1.152047in}}%
\pgfpathlineto{\pgfqpoint{1.769369in}{1.190783in}}%
\pgfpathlineto{\pgfqpoint{1.806607in}{1.233915in}}%
\pgfpathlineto{\pgfqpoint{1.848498in}{1.287110in}}%
\pgfpathlineto{\pgfqpoint{1.895045in}{1.351258in}}%
\pgfpathlineto{\pgfqpoint{1.946246in}{1.426964in}}%
\pgfpathlineto{\pgfqpoint{2.006757in}{1.521931in}}%
\pgfpathlineto{\pgfqpoint{2.085886in}{1.652395in}}%
\pgfpathlineto{\pgfqpoint{2.253453in}{1.937173in}}%
\pgfpathlineto{\pgfqpoint{2.351201in}{2.099154in}}%
\pgfpathlineto{\pgfqpoint{2.421021in}{2.209309in}}%
\pgfpathlineto{\pgfqpoint{2.481532in}{2.299560in}}%
\pgfpathlineto{\pgfqpoint{2.537387in}{2.377676in}}%
\pgfpathlineto{\pgfqpoint{2.588589in}{2.444288in}}%
\pgfpathlineto{\pgfqpoint{2.635135in}{2.500276in}}%
\pgfpathlineto{\pgfqpoint{2.681682in}{2.551572in}}%
\pgfpathlineto{\pgfqpoint{2.723574in}{2.593481in}}%
\pgfpathlineto{\pgfqpoint{2.765465in}{2.631155in}}%
\pgfpathlineto{\pgfqpoint{2.807357in}{2.664422in}}%
\pgfpathlineto{\pgfqpoint{2.844595in}{2.690173in}}%
\pgfpathlineto{\pgfqpoint{2.881832in}{2.712232in}}%
\pgfpathlineto{\pgfqpoint{2.919069in}{2.730522in}}%
\pgfpathlineto{\pgfqpoint{2.951652in}{2.743385in}}%
\pgfpathlineto{\pgfqpoint{2.984234in}{2.753279in}}%
\pgfpathlineto{\pgfqpoint{3.016817in}{2.760179in}}%
\pgfpathlineto{\pgfqpoint{3.049399in}{2.764066in}}%
\pgfpathlineto{\pgfqpoint{3.081982in}{2.764931in}}%
\pgfpathlineto{\pgfqpoint{3.114565in}{2.762770in}}%
\pgfpathlineto{\pgfqpoint{3.147147in}{2.757590in}}%
\pgfpathlineto{\pgfqpoint{3.179730in}{2.749404in}}%
\pgfpathlineto{\pgfqpoint{3.212312in}{2.738234in}}%
\pgfpathlineto{\pgfqpoint{3.244895in}{2.724109in}}%
\pgfpathlineto{\pgfqpoint{3.282132in}{2.704397in}}%
\pgfpathlineto{\pgfqpoint{3.319369in}{2.680944in}}%
\pgfpathlineto{\pgfqpoint{3.356607in}{2.653833in}}%
\pgfpathlineto{\pgfqpoint{3.393844in}{2.623158in}}%
\pgfpathlineto{\pgfqpoint{3.435736in}{2.584528in}}%
\pgfpathlineto{\pgfqpoint{3.477628in}{2.541703in}}%
\pgfpathlineto{\pgfqpoint{3.524174in}{2.489444in}}%
\pgfpathlineto{\pgfqpoint{3.570721in}{2.432555in}}%
\pgfpathlineto{\pgfqpoint{3.621922in}{2.365034in}}%
\pgfpathlineto{\pgfqpoint{3.677778in}{2.286035in}}%
\pgfpathlineto{\pgfqpoint{3.738288in}{2.194964in}}%
\pgfpathlineto{\pgfqpoint{3.808108in}{2.084062in}}%
\pgfpathlineto{\pgfqpoint{3.896547in}{1.937173in}}%
\pgfpathlineto{\pgfqpoint{4.157207in}{1.499570in}}%
\pgfpathlineto{\pgfqpoint{4.222372in}{1.398877in}}%
\pgfpathlineto{\pgfqpoint{4.273574in}{1.325014in}}%
\pgfpathlineto{\pgfqpoint{4.320120in}{1.262896in}}%
\pgfpathlineto{\pgfqpoint{4.362012in}{1.211829in}}%
\pgfpathlineto{\pgfqpoint{4.399249in}{1.170835in}}%
\pgfpathlineto{\pgfqpoint{4.436486in}{1.134478in}}%
\pgfpathlineto{\pgfqpoint{4.469069in}{1.106854in}}%
\pgfpathlineto{\pgfqpoint{4.501652in}{1.083491in}}%
\pgfpathlineto{\pgfqpoint{4.529580in}{1.067117in}}%
\pgfpathlineto{\pgfqpoint{4.557508in}{1.054349in}}%
\pgfpathlineto{\pgfqpoint{4.585435in}{1.045412in}}%
\pgfpathlineto{\pgfqpoint{4.608709in}{1.041057in}}%
\pgfpathlineto{\pgfqpoint{4.631982in}{1.039659in}}%
\pgfpathlineto{\pgfqpoint{4.655255in}{1.041353in}}%
\pgfpathlineto{\pgfqpoint{4.678529in}{1.046281in}}%
\pgfpathlineto{\pgfqpoint{4.701802in}{1.054583in}}%
\pgfpathlineto{\pgfqpoint{4.725075in}{1.066405in}}%
\pgfpathlineto{\pgfqpoint{4.748348in}{1.081890in}}%
\pgfpathlineto{\pgfqpoint{4.771622in}{1.101189in}}%
\pgfpathlineto{\pgfqpoint{4.794895in}{1.124450in}}%
\pgfpathlineto{\pgfqpoint{4.818168in}{1.151825in}}%
\pgfpathlineto{\pgfqpoint{4.841441in}{1.183470in}}%
\pgfpathlineto{\pgfqpoint{4.869369in}{1.227298in}}%
\pgfpathlineto{\pgfqpoint{4.897297in}{1.277773in}}%
\pgfpathlineto{\pgfqpoint{4.925225in}{1.335171in}}%
\pgfpathlineto{\pgfqpoint{4.953153in}{1.399777in}}%
\pgfpathlineto{\pgfqpoint{4.981081in}{1.471877in}}%
\pgfpathlineto{\pgfqpoint{5.009009in}{1.551761in}}%
\pgfpathlineto{\pgfqpoint{5.041592in}{1.655193in}}%
\pgfpathlineto{\pgfqpoint{5.074174in}{1.770099in}}%
\pgfpathlineto{\pgfqpoint{5.106757in}{1.896965in}}%
\pgfpathlineto{\pgfqpoint{5.139339in}{2.036282in}}%
\pgfpathlineto{\pgfqpoint{5.171922in}{2.188552in}}%
\pgfpathlineto{\pgfqpoint{5.204505in}{2.354282in}}%
\pgfpathlineto{\pgfqpoint{5.241742in}{2.560832in}}%
\pgfpathlineto{\pgfqpoint{5.278979in}{2.786421in}}%
\pgfpathlineto{\pgfqpoint{5.316216in}{3.031846in}}%
\pgfpathlineto{\pgfqpoint{5.353453in}{3.297918in}}%
\pgfpathlineto{\pgfqpoint{5.384211in}{3.533889in}}%
\pgfpathlineto{\pgfqpoint{5.384211in}{3.533889in}}%
\pgfusepath{stroke}%
\end{pgfscope}%
\begin{pgfscope}%
\pgfpathrectangle{\pgfqpoint{0.750000in}{0.500000in}}{\pgfqpoint{4.650000in}{3.020000in}}%
\pgfusepath{clip}%
\pgfsetrectcap%
\pgfsetroundjoin%
\pgfsetlinewidth{1.505625pt}%
\definecolor{currentstroke}{rgb}{0.172549,0.627451,0.172549}%
\pgfsetstrokecolor{currentstroke}%
\pgfsetdash{}{0pt}%
\pgfpathmoveto{\pgfqpoint{1.021355in}{0.486111in}}%
\pgfpathlineto{\pgfqpoint{1.043243in}{0.650887in}}%
\pgfpathlineto{\pgfqpoint{1.066517in}{0.805124in}}%
\pgfpathlineto{\pgfqpoint{1.089790in}{0.939445in}}%
\pgfpathlineto{\pgfqpoint{1.113063in}{1.055399in}}%
\pgfpathlineto{\pgfqpoint{1.136336in}{1.154463in}}%
\pgfpathlineto{\pgfqpoint{1.159610in}{1.238045in}}%
\pgfpathlineto{\pgfqpoint{1.178228in}{1.294664in}}%
\pgfpathlineto{\pgfqpoint{1.196847in}{1.342888in}}%
\pgfpathlineto{\pgfqpoint{1.215465in}{1.383345in}}%
\pgfpathlineto{\pgfqpoint{1.234084in}{1.416639in}}%
\pgfpathlineto{\pgfqpoint{1.252703in}{1.443347in}}%
\pgfpathlineto{\pgfqpoint{1.271321in}{1.464022in}}%
\pgfpathlineto{\pgfqpoint{1.289940in}{1.479193in}}%
\pgfpathlineto{\pgfqpoint{1.308559in}{1.489367in}}%
\pgfpathlineto{\pgfqpoint{1.327177in}{1.495024in}}%
\pgfpathlineto{\pgfqpoint{1.345796in}{1.496626in}}%
\pgfpathlineto{\pgfqpoint{1.364414in}{1.494610in}}%
\pgfpathlineto{\pgfqpoint{1.383033in}{1.489394in}}%
\pgfpathlineto{\pgfqpoint{1.406306in}{1.478974in}}%
\pgfpathlineto{\pgfqpoint{1.429580in}{1.464900in}}%
\pgfpathlineto{\pgfqpoint{1.457508in}{1.444145in}}%
\pgfpathlineto{\pgfqpoint{1.490090in}{1.415966in}}%
\pgfpathlineto{\pgfqpoint{1.536637in}{1.371249in}}%
\pgfpathlineto{\pgfqpoint{1.634384in}{1.275905in}}%
\pgfpathlineto{\pgfqpoint{1.676276in}{1.239995in}}%
\pgfpathlineto{\pgfqpoint{1.708859in}{1.215797in}}%
\pgfpathlineto{\pgfqpoint{1.741441in}{1.195564in}}%
\pgfpathlineto{\pgfqpoint{1.769369in}{1.181776in}}%
\pgfpathlineto{\pgfqpoint{1.797297in}{1.171552in}}%
\pgfpathlineto{\pgfqpoint{1.825225in}{1.165104in}}%
\pgfpathlineto{\pgfqpoint{1.848498in}{1.162735in}}%
\pgfpathlineto{\pgfqpoint{1.871772in}{1.163170in}}%
\pgfpathlineto{\pgfqpoint{1.895045in}{1.166458in}}%
\pgfpathlineto{\pgfqpoint{1.918318in}{1.172626in}}%
\pgfpathlineto{\pgfqpoint{1.941592in}{1.181679in}}%
\pgfpathlineto{\pgfqpoint{1.964865in}{1.193608in}}%
\pgfpathlineto{\pgfqpoint{1.992793in}{1.211678in}}%
\pgfpathlineto{\pgfqpoint{2.020721in}{1.233769in}}%
\pgfpathlineto{\pgfqpoint{2.048649in}{1.259772in}}%
\pgfpathlineto{\pgfqpoint{2.081231in}{1.294866in}}%
\pgfpathlineto{\pgfqpoint{2.113814in}{1.334834in}}%
\pgfpathlineto{\pgfqpoint{2.151051in}{1.386086in}}%
\pgfpathlineto{\pgfqpoint{2.188288in}{1.442788in}}%
\pgfpathlineto{\pgfqpoint{2.230180in}{1.512383in}}%
\pgfpathlineto{\pgfqpoint{2.276727in}{1.595887in}}%
\pgfpathlineto{\pgfqpoint{2.332583in}{1.702908in}}%
\pgfpathlineto{\pgfqpoint{2.407057in}{1.853320in}}%
\pgfpathlineto{\pgfqpoint{2.588589in}{2.223504in}}%
\pgfpathlineto{\pgfqpoint{2.644444in}{2.329135in}}%
\pgfpathlineto{\pgfqpoint{2.695646in}{2.419243in}}%
\pgfpathlineto{\pgfqpoint{2.737538in}{2.487106in}}%
\pgfpathlineto{\pgfqpoint{2.779429in}{2.548890in}}%
\pgfpathlineto{\pgfqpoint{2.816667in}{2.598161in}}%
\pgfpathlineto{\pgfqpoint{2.849249in}{2.636564in}}%
\pgfpathlineto{\pgfqpoint{2.881832in}{2.670308in}}%
\pgfpathlineto{\pgfqpoint{2.914414in}{2.699175in}}%
\pgfpathlineto{\pgfqpoint{2.942342in}{2.719893in}}%
\pgfpathlineto{\pgfqpoint{2.970270in}{2.736792in}}%
\pgfpathlineto{\pgfqpoint{2.998198in}{2.749791in}}%
\pgfpathlineto{\pgfqpoint{3.021471in}{2.757601in}}%
\pgfpathlineto{\pgfqpoint{3.044745in}{2.762634in}}%
\pgfpathlineto{\pgfqpoint{3.068018in}{2.764874in}}%
\pgfpathlineto{\pgfqpoint{3.091291in}{2.764314in}}%
\pgfpathlineto{\pgfqpoint{3.114565in}{2.760955in}}%
\pgfpathlineto{\pgfqpoint{3.137838in}{2.754809in}}%
\pgfpathlineto{\pgfqpoint{3.161111in}{2.745896in}}%
\pgfpathlineto{\pgfqpoint{3.189039in}{2.731589in}}%
\pgfpathlineto{\pgfqpoint{3.216967in}{2.713407in}}%
\pgfpathlineto{\pgfqpoint{3.244895in}{2.691436in}}%
\pgfpathlineto{\pgfqpoint{3.272823in}{2.665781in}}%
\pgfpathlineto{\pgfqpoint{3.305405in}{2.631357in}}%
\pgfpathlineto{\pgfqpoint{3.337988in}{2.592309in}}%
\pgfpathlineto{\pgfqpoint{3.375225in}{2.542347in}}%
\pgfpathlineto{\pgfqpoint{3.412462in}{2.487106in}}%
\pgfpathlineto{\pgfqpoint{3.454354in}{2.419243in}}%
\pgfpathlineto{\pgfqpoint{3.500901in}{2.337617in}}%
\pgfpathlineto{\pgfqpoint{3.552102in}{2.241557in}}%
\pgfpathlineto{\pgfqpoint{3.617267in}{2.112273in}}%
\pgfpathlineto{\pgfqpoint{3.733634in}{1.872490in}}%
\pgfpathlineto{\pgfqpoint{3.822072in}{1.693757in}}%
\pgfpathlineto{\pgfqpoint{3.877928in}{1.587278in}}%
\pgfpathlineto{\pgfqpoint{3.924474in}{1.504372in}}%
\pgfpathlineto{\pgfqpoint{3.966366in}{1.435421in}}%
\pgfpathlineto{\pgfqpoint{4.003604in}{1.379370in}}%
\pgfpathlineto{\pgfqpoint{4.040841in}{1.328837in}}%
\pgfpathlineto{\pgfqpoint{4.073423in}{1.289548in}}%
\pgfpathlineto{\pgfqpoint{4.106006in}{1.255172in}}%
\pgfpathlineto{\pgfqpoint{4.133934in}{1.229812in}}%
\pgfpathlineto{\pgfqpoint{4.161862in}{1.208385in}}%
\pgfpathlineto{\pgfqpoint{4.189790in}{1.190993in}}%
\pgfpathlineto{\pgfqpoint{4.217718in}{1.177712in}}%
\pgfpathlineto{\pgfqpoint{4.240991in}{1.169812in}}%
\pgfpathlineto{\pgfqpoint{4.264264in}{1.164799in}}%
\pgfpathlineto{\pgfqpoint{4.287538in}{1.162656in}}%
\pgfpathlineto{\pgfqpoint{4.310811in}{1.163349in}}%
\pgfpathlineto{\pgfqpoint{4.334084in}{1.166823in}}%
\pgfpathlineto{\pgfqpoint{4.357357in}{1.172999in}}%
\pgfpathlineto{\pgfqpoint{4.385285in}{1.183833in}}%
\pgfpathlineto{\pgfqpoint{4.413213in}{1.198191in}}%
\pgfpathlineto{\pgfqpoint{4.441141in}{1.215797in}}%
\pgfpathlineto{\pgfqpoint{4.473724in}{1.239995in}}%
\pgfpathlineto{\pgfqpoint{4.510961in}{1.271691in}}%
\pgfpathlineto{\pgfqpoint{4.557508in}{1.315681in}}%
\pgfpathlineto{\pgfqpoint{4.678529in}{1.432506in}}%
\pgfpathlineto{\pgfqpoint{4.711111in}{1.458402in}}%
\pgfpathlineto{\pgfqpoint{4.739039in}{1.476429in}}%
\pgfpathlineto{\pgfqpoint{4.762312in}{1.487636in}}%
\pgfpathlineto{\pgfqpoint{4.780931in}{1.493590in}}%
\pgfpathlineto{\pgfqpoint{4.799550in}{1.496444in}}%
\pgfpathlineto{\pgfqpoint{4.818168in}{1.495787in}}%
\pgfpathlineto{\pgfqpoint{4.836787in}{1.491186in}}%
\pgfpathlineto{\pgfqpoint{4.850751in}{1.484874in}}%
\pgfpathlineto{\pgfqpoint{4.864715in}{1.475889in}}%
\pgfpathlineto{\pgfqpoint{4.878679in}{1.464022in}}%
\pgfpathlineto{\pgfqpoint{4.897297in}{1.443347in}}%
\pgfpathlineto{\pgfqpoint{4.915916in}{1.416639in}}%
\pgfpathlineto{\pgfqpoint{4.934535in}{1.383345in}}%
\pgfpathlineto{\pgfqpoint{4.953153in}{1.342888in}}%
\pgfpathlineto{\pgfqpoint{4.971772in}{1.294664in}}%
\pgfpathlineto{\pgfqpoint{4.990390in}{1.238045in}}%
\pgfpathlineto{\pgfqpoint{5.009009in}{1.172375in}}%
\pgfpathlineto{\pgfqpoint{5.027628in}{1.096971in}}%
\pgfpathlineto{\pgfqpoint{5.050901in}{0.987946in}}%
\pgfpathlineto{\pgfqpoint{5.074174in}{0.861154in}}%
\pgfpathlineto{\pgfqpoint{5.097447in}{0.715074in}}%
\pgfpathlineto{\pgfqpoint{5.120721in}{0.548115in}}%
\pgfpathlineto{\pgfqpoint{5.128645in}{0.486111in}}%
\pgfpathlineto{\pgfqpoint{5.128645in}{0.486111in}}%
\pgfusepath{stroke}%
\end{pgfscope}%
\begin{pgfscope}%
\pgfpathrectangle{\pgfqpoint{0.750000in}{0.500000in}}{\pgfqpoint{4.650000in}{3.020000in}}%
\pgfusepath{clip}%
\pgfsetrectcap%
\pgfsetroundjoin%
\pgfsetlinewidth{1.505625pt}%
\definecolor{currentstroke}{rgb}{0.839216,0.152941,0.156863}%
\pgfsetstrokecolor{currentstroke}%
\pgfsetdash{}{0pt}%
\pgfpathmoveto{\pgfqpoint{0.954468in}{3.533889in}}%
\pgfpathlineto{\pgfqpoint{0.973423in}{3.139575in}}%
\pgfpathlineto{\pgfqpoint{0.992042in}{2.802369in}}%
\pgfpathlineto{\pgfqpoint{1.010661in}{2.510112in}}%
\pgfpathlineto{\pgfqpoint{1.029279in}{2.258536in}}%
\pgfpathlineto{\pgfqpoint{1.047898in}{2.043664in}}%
\pgfpathlineto{\pgfqpoint{1.066517in}{1.861790in}}%
\pgfpathlineto{\pgfqpoint{1.085135in}{1.709475in}}%
\pgfpathlineto{\pgfqpoint{1.103754in}{1.583525in}}%
\pgfpathlineto{\pgfqpoint{1.122372in}{1.480983in}}%
\pgfpathlineto{\pgfqpoint{1.140991in}{1.399118in}}%
\pgfpathlineto{\pgfqpoint{1.154955in}{1.349766in}}%
\pgfpathlineto{\pgfqpoint{1.168919in}{1.309636in}}%
\pgfpathlineto{\pgfqpoint{1.182883in}{1.277802in}}%
\pgfpathlineto{\pgfqpoint{1.196847in}{1.253396in}}%
\pgfpathlineto{\pgfqpoint{1.210811in}{1.235608in}}%
\pgfpathlineto{\pgfqpoint{1.224775in}{1.223679in}}%
\pgfpathlineto{\pgfqpoint{1.238739in}{1.216905in}}%
\pgfpathlineto{\pgfqpoint{1.252703in}{1.214629in}}%
\pgfpathlineto{\pgfqpoint{1.266667in}{1.216242in}}%
\pgfpathlineto{\pgfqpoint{1.280631in}{1.221180in}}%
\pgfpathlineto{\pgfqpoint{1.299249in}{1.232042in}}%
\pgfpathlineto{\pgfqpoint{1.317868in}{1.246773in}}%
\pgfpathlineto{\pgfqpoint{1.341141in}{1.269113in}}%
\pgfpathlineto{\pgfqpoint{1.378378in}{1.310032in}}%
\pgfpathlineto{\pgfqpoint{1.448198in}{1.387573in}}%
\pgfpathlineto{\pgfqpoint{1.480781in}{1.418550in}}%
\pgfpathlineto{\pgfqpoint{1.508709in}{1.440762in}}%
\pgfpathlineto{\pgfqpoint{1.531982in}{1.455762in}}%
\pgfpathlineto{\pgfqpoint{1.555255in}{1.467366in}}%
\pgfpathlineto{\pgfqpoint{1.578529in}{1.475491in}}%
\pgfpathlineto{\pgfqpoint{1.601802in}{1.480154in}}%
\pgfpathlineto{\pgfqpoint{1.625075in}{1.481451in}}%
\pgfpathlineto{\pgfqpoint{1.648348in}{1.479550in}}%
\pgfpathlineto{\pgfqpoint{1.671622in}{1.474680in}}%
\pgfpathlineto{\pgfqpoint{1.699550in}{1.465311in}}%
\pgfpathlineto{\pgfqpoint{1.727477in}{1.452623in}}%
\pgfpathlineto{\pgfqpoint{1.760060in}{1.434453in}}%
\pgfpathlineto{\pgfqpoint{1.801952in}{1.407388in}}%
\pgfpathlineto{\pgfqpoint{1.927628in}{1.323371in}}%
\pgfpathlineto{\pgfqpoint{1.960210in}{1.306096in}}%
\pgfpathlineto{\pgfqpoint{1.988138in}{1.294190in}}%
\pgfpathlineto{\pgfqpoint{2.016066in}{1.285487in}}%
\pgfpathlineto{\pgfqpoint{2.043994in}{1.280428in}}%
\pgfpathlineto{\pgfqpoint{2.067267in}{1.279270in}}%
\pgfpathlineto{\pgfqpoint{2.090541in}{1.281091in}}%
\pgfpathlineto{\pgfqpoint{2.113814in}{1.286044in}}%
\pgfpathlineto{\pgfqpoint{2.137087in}{1.294250in}}%
\pgfpathlineto{\pgfqpoint{2.160360in}{1.305797in}}%
\pgfpathlineto{\pgfqpoint{2.183634in}{1.320743in}}%
\pgfpathlineto{\pgfqpoint{2.206907in}{1.339111in}}%
\pgfpathlineto{\pgfqpoint{2.234835in}{1.365659in}}%
\pgfpathlineto{\pgfqpoint{2.262763in}{1.397054in}}%
\pgfpathlineto{\pgfqpoint{2.290691in}{1.433168in}}%
\pgfpathlineto{\pgfqpoint{2.323273in}{1.481009in}}%
\pgfpathlineto{\pgfqpoint{2.355856in}{1.534623in}}%
\pgfpathlineto{\pgfqpoint{2.393093in}{1.602342in}}%
\pgfpathlineto{\pgfqpoint{2.434985in}{1.685714in}}%
\pgfpathlineto{\pgfqpoint{2.481532in}{1.785628in}}%
\pgfpathlineto{\pgfqpoint{2.542042in}{1.923525in}}%
\pgfpathlineto{\pgfqpoint{2.714264in}{2.321406in}}%
\pgfpathlineto{\pgfqpoint{2.760811in}{2.418880in}}%
\pgfpathlineto{\pgfqpoint{2.802703in}{2.499264in}}%
\pgfpathlineto{\pgfqpoint{2.839940in}{2.563621in}}%
\pgfpathlineto{\pgfqpoint{2.872523in}{2.613681in}}%
\pgfpathlineto{\pgfqpoint{2.900450in}{2.651494in}}%
\pgfpathlineto{\pgfqpoint{2.928378in}{2.684274in}}%
\pgfpathlineto{\pgfqpoint{2.956306in}{2.711751in}}%
\pgfpathlineto{\pgfqpoint{2.979580in}{2.730435in}}%
\pgfpathlineto{\pgfqpoint{3.002853in}{2.745172in}}%
\pgfpathlineto{\pgfqpoint{3.026126in}{2.755878in}}%
\pgfpathlineto{\pgfqpoint{3.044745in}{2.761500in}}%
\pgfpathlineto{\pgfqpoint{3.063363in}{2.764482in}}%
\pgfpathlineto{\pgfqpoint{3.081982in}{2.764813in}}%
\pgfpathlineto{\pgfqpoint{3.100601in}{2.762493in}}%
\pgfpathlineto{\pgfqpoint{3.119219in}{2.757530in}}%
\pgfpathlineto{\pgfqpoint{3.142492in}{2.747638in}}%
\pgfpathlineto{\pgfqpoint{3.165766in}{2.733701in}}%
\pgfpathlineto{\pgfqpoint{3.189039in}{2.715799in}}%
\pgfpathlineto{\pgfqpoint{3.212312in}{2.694034in}}%
\pgfpathlineto{\pgfqpoint{3.240240in}{2.662994in}}%
\pgfpathlineto{\pgfqpoint{3.268168in}{2.626826in}}%
\pgfpathlineto{\pgfqpoint{3.296096in}{2.585828in}}%
\pgfpathlineto{\pgfqpoint{3.328679in}{2.532344in}}%
\pgfpathlineto{\pgfqpoint{3.365916in}{2.464502in}}%
\pgfpathlineto{\pgfqpoint{3.407808in}{2.380809in}}%
\pgfpathlineto{\pgfqpoint{3.454354in}{2.280484in}}%
\pgfpathlineto{\pgfqpoint{3.514865in}{2.142188in}}%
\pgfpathlineto{\pgfqpoint{3.677778in}{1.765127in}}%
\pgfpathlineto{\pgfqpoint{3.724324in}{1.666594in}}%
\pgfpathlineto{\pgfqpoint{3.766216in}{1.584809in}}%
\pgfpathlineto{\pgfqpoint{3.803453in}{1.518743in}}%
\pgfpathlineto{\pgfqpoint{3.836036in}{1.466732in}}%
\pgfpathlineto{\pgfqpoint{3.868619in}{1.420616in}}%
\pgfpathlineto{\pgfqpoint{3.896547in}{1.386057in}}%
\pgfpathlineto{\pgfqpoint{3.924474in}{1.356266in}}%
\pgfpathlineto{\pgfqpoint{3.952402in}{1.331353in}}%
\pgfpathlineto{\pgfqpoint{3.975676in}{1.314355in}}%
\pgfpathlineto{\pgfqpoint{3.998949in}{1.300773in}}%
\pgfpathlineto{\pgfqpoint{4.022222in}{1.290571in}}%
\pgfpathlineto{\pgfqpoint{4.045495in}{1.283678in}}%
\pgfpathlineto{\pgfqpoint{4.068769in}{1.279995in}}%
\pgfpathlineto{\pgfqpoint{4.092042in}{1.279385in}}%
\pgfpathlineto{\pgfqpoint{4.115315in}{1.281685in}}%
\pgfpathlineto{\pgfqpoint{4.138589in}{1.286696in}}%
\pgfpathlineto{\pgfqpoint{4.166517in}{1.295964in}}%
\pgfpathlineto{\pgfqpoint{4.194444in}{1.308358in}}%
\pgfpathlineto{\pgfqpoint{4.227027in}{1.326089in}}%
\pgfpathlineto{\pgfqpoint{4.268919in}{1.352673in}}%
\pgfpathlineto{\pgfqpoint{4.408559in}{1.445226in}}%
\pgfpathlineto{\pgfqpoint{4.441141in}{1.461421in}}%
\pgfpathlineto{\pgfqpoint{4.469069in}{1.471962in}}%
\pgfpathlineto{\pgfqpoint{4.492342in}{1.477944in}}%
\pgfpathlineto{\pgfqpoint{4.515616in}{1.481063in}}%
\pgfpathlineto{\pgfqpoint{4.538889in}{1.481068in}}%
\pgfpathlineto{\pgfqpoint{4.562162in}{1.477768in}}%
\pgfpathlineto{\pgfqpoint{4.585435in}{1.471034in}}%
\pgfpathlineto{\pgfqpoint{4.608709in}{1.460818in}}%
\pgfpathlineto{\pgfqpoint{4.631982in}{1.447160in}}%
\pgfpathlineto{\pgfqpoint{4.655255in}{1.430204in}}%
\pgfpathlineto{\pgfqpoint{4.683183in}{1.405880in}}%
\pgfpathlineto{\pgfqpoint{4.715766in}{1.372938in}}%
\pgfpathlineto{\pgfqpoint{4.762312in}{1.320691in}}%
\pgfpathlineto{\pgfqpoint{4.808859in}{1.269113in}}%
\pgfpathlineto{\pgfqpoint{4.836787in}{1.242784in}}%
\pgfpathlineto{\pgfqpoint{4.855405in}{1.228922in}}%
\pgfpathlineto{\pgfqpoint{4.874024in}{1.219197in}}%
\pgfpathlineto{\pgfqpoint{4.887988in}{1.215308in}}%
\pgfpathlineto{\pgfqpoint{4.901952in}{1.214926in}}%
\pgfpathlineto{\pgfqpoint{4.915916in}{1.218632in}}%
\pgfpathlineto{\pgfqpoint{4.929880in}{1.227049in}}%
\pgfpathlineto{\pgfqpoint{4.943844in}{1.240850in}}%
\pgfpathlineto{\pgfqpoint{4.957808in}{1.260757in}}%
\pgfpathlineto{\pgfqpoint{4.971772in}{1.287546in}}%
\pgfpathlineto{\pgfqpoint{4.985736in}{1.322046in}}%
\pgfpathlineto{\pgfqpoint{4.999700in}{1.365145in}}%
\pgfpathlineto{\pgfqpoint{5.013664in}{1.417787in}}%
\pgfpathlineto{\pgfqpoint{5.027628in}{1.480983in}}%
\pgfpathlineto{\pgfqpoint{5.041592in}{1.555805in}}%
\pgfpathlineto{\pgfqpoint{5.060210in}{1.675635in}}%
\pgfpathlineto{\pgfqpoint{5.078829in}{1.821069in}}%
\pgfpathlineto{\pgfqpoint{5.097447in}{1.995240in}}%
\pgfpathlineto{\pgfqpoint{5.116066in}{2.201526in}}%
\pgfpathlineto{\pgfqpoint{5.134685in}{2.443564in}}%
\pgfpathlineto{\pgfqpoint{5.153303in}{2.725263in}}%
\pgfpathlineto{\pgfqpoint{5.171922in}{3.050815in}}%
\pgfpathlineto{\pgfqpoint{5.190541in}{3.424715in}}%
\pgfpathlineto{\pgfqpoint{5.195532in}{3.533889in}}%
\pgfpathlineto{\pgfqpoint{5.195532in}{3.533889in}}%
\pgfusepath{stroke}%
\end{pgfscope}%
\begin{pgfscope}%
\pgfpathrectangle{\pgfqpoint{0.750000in}{0.500000in}}{\pgfqpoint{4.650000in}{3.020000in}}%
\pgfusepath{clip}%
\pgfsetrectcap%
\pgfsetroundjoin%
\pgfsetlinewidth{1.505625pt}%
\definecolor{currentstroke}{rgb}{0.580392,0.403922,0.741176}%
\pgfsetstrokecolor{currentstroke}%
\pgfsetdash{}{0pt}%
\pgfpathmoveto{\pgfqpoint{1.052209in}{0.486111in}}%
\pgfpathlineto{\pgfqpoint{1.066517in}{0.685134in}}%
\pgfpathlineto{\pgfqpoint{1.080480in}{0.846543in}}%
\pgfpathlineto{\pgfqpoint{1.094444in}{0.979561in}}%
\pgfpathlineto{\pgfqpoint{1.108408in}{1.087726in}}%
\pgfpathlineto{\pgfqpoint{1.122372in}{1.174254in}}%
\pgfpathlineto{\pgfqpoint{1.136336in}{1.242066in}}%
\pgfpathlineto{\pgfqpoint{1.150300in}{1.293802in}}%
\pgfpathlineto{\pgfqpoint{1.164264in}{1.331848in}}%
\pgfpathlineto{\pgfqpoint{1.178228in}{1.358346in}}%
\pgfpathlineto{\pgfqpoint{1.187538in}{1.370552in}}%
\pgfpathlineto{\pgfqpoint{1.196847in}{1.379002in}}%
\pgfpathlineto{\pgfqpoint{1.206156in}{1.384172in}}%
\pgfpathlineto{\pgfqpoint{1.215465in}{1.386505in}}%
\pgfpathlineto{\pgfqpoint{1.224775in}{1.386408in}}%
\pgfpathlineto{\pgfqpoint{1.238739in}{1.382515in}}%
\pgfpathlineto{\pgfqpoint{1.252703in}{1.375119in}}%
\pgfpathlineto{\pgfqpoint{1.271321in}{1.361496in}}%
\pgfpathlineto{\pgfqpoint{1.303904in}{1.332584in}}%
\pgfpathlineto{\pgfqpoint{1.341141in}{1.299944in}}%
\pgfpathlineto{\pgfqpoint{1.364414in}{1.283005in}}%
\pgfpathlineto{\pgfqpoint{1.387688in}{1.270092in}}%
\pgfpathlineto{\pgfqpoint{1.406306in}{1.263076in}}%
\pgfpathlineto{\pgfqpoint{1.424925in}{1.259158in}}%
\pgfpathlineto{\pgfqpoint{1.443544in}{1.258343in}}%
\pgfpathlineto{\pgfqpoint{1.462162in}{1.260532in}}%
\pgfpathlineto{\pgfqpoint{1.480781in}{1.265541in}}%
\pgfpathlineto{\pgfqpoint{1.504054in}{1.275384in}}%
\pgfpathlineto{\pgfqpoint{1.527327in}{1.288651in}}%
\pgfpathlineto{\pgfqpoint{1.555255in}{1.308150in}}%
\pgfpathlineto{\pgfqpoint{1.592492in}{1.338206in}}%
\pgfpathlineto{\pgfqpoint{1.694895in}{1.423270in}}%
\pgfpathlineto{\pgfqpoint{1.727477in}{1.445378in}}%
\pgfpathlineto{\pgfqpoint{1.755405in}{1.460925in}}%
\pgfpathlineto{\pgfqpoint{1.783333in}{1.472897in}}%
\pgfpathlineto{\pgfqpoint{1.811261in}{1.481069in}}%
\pgfpathlineto{\pgfqpoint{1.834535in}{1.484935in}}%
\pgfpathlineto{\pgfqpoint{1.862462in}{1.486124in}}%
\pgfpathlineto{\pgfqpoint{1.890390in}{1.483780in}}%
\pgfpathlineto{\pgfqpoint{1.918318in}{1.478271in}}%
\pgfpathlineto{\pgfqpoint{1.950901in}{1.468499in}}%
\pgfpathlineto{\pgfqpoint{1.992793in}{1.452104in}}%
\pgfpathlineto{\pgfqpoint{2.113814in}{1.401451in}}%
\pgfpathlineto{\pgfqpoint{2.146396in}{1.392571in}}%
\pgfpathlineto{\pgfqpoint{2.174324in}{1.388245in}}%
\pgfpathlineto{\pgfqpoint{2.197598in}{1.387440in}}%
\pgfpathlineto{\pgfqpoint{2.220871in}{1.389530in}}%
\pgfpathlineto{\pgfqpoint{2.244144in}{1.394804in}}%
\pgfpathlineto{\pgfqpoint{2.267417in}{1.403508in}}%
\pgfpathlineto{\pgfqpoint{2.290691in}{1.415846in}}%
\pgfpathlineto{\pgfqpoint{2.313964in}{1.431970in}}%
\pgfpathlineto{\pgfqpoint{2.337237in}{1.451989in}}%
\pgfpathlineto{\pgfqpoint{2.360511in}{1.475956in}}%
\pgfpathlineto{\pgfqpoint{2.383784in}{1.503877in}}%
\pgfpathlineto{\pgfqpoint{2.411712in}{1.542535in}}%
\pgfpathlineto{\pgfqpoint{2.439640in}{1.586646in}}%
\pgfpathlineto{\pgfqpoint{2.472222in}{1.644633in}}%
\pgfpathlineto{\pgfqpoint{2.504805in}{1.709075in}}%
\pgfpathlineto{\pgfqpoint{2.542042in}{1.789649in}}%
\pgfpathlineto{\pgfqpoint{2.588589in}{1.898740in}}%
\pgfpathlineto{\pgfqpoint{2.649099in}{2.049607in}}%
\pgfpathlineto{\pgfqpoint{2.765465in}{2.341327in}}%
\pgfpathlineto{\pgfqpoint{2.812012in}{2.448153in}}%
\pgfpathlineto{\pgfqpoint{2.849249in}{2.525835in}}%
\pgfpathlineto{\pgfqpoint{2.881832in}{2.586705in}}%
\pgfpathlineto{\pgfqpoint{2.909760in}{2.632801in}}%
\pgfpathlineto{\pgfqpoint{2.937688in}{2.672690in}}%
\pgfpathlineto{\pgfqpoint{2.960961in}{2.700838in}}%
\pgfpathlineto{\pgfqpoint{2.984234in}{2.724099in}}%
\pgfpathlineto{\pgfqpoint{3.002853in}{2.739055in}}%
\pgfpathlineto{\pgfqpoint{3.021471in}{2.750674in}}%
\pgfpathlineto{\pgfqpoint{3.040090in}{2.758894in}}%
\pgfpathlineto{\pgfqpoint{3.058709in}{2.763668in}}%
\pgfpathlineto{\pgfqpoint{3.077327in}{2.764973in}}%
\pgfpathlineto{\pgfqpoint{3.095946in}{2.762799in}}%
\pgfpathlineto{\pgfqpoint{3.114565in}{2.757160in}}%
\pgfpathlineto{\pgfqpoint{3.133183in}{2.748086in}}%
\pgfpathlineto{\pgfqpoint{3.151802in}{2.735626in}}%
\pgfpathlineto{\pgfqpoint{3.170420in}{2.719848in}}%
\pgfpathlineto{\pgfqpoint{3.193694in}{2.695592in}}%
\pgfpathlineto{\pgfqpoint{3.216967in}{2.666494in}}%
\pgfpathlineto{\pgfqpoint{3.240240in}{2.632801in}}%
\pgfpathlineto{\pgfqpoint{3.268168in}{2.586705in}}%
\pgfpathlineto{\pgfqpoint{3.300751in}{2.525835in}}%
\pgfpathlineto{\pgfqpoint{3.333333in}{2.458285in}}%
\pgfpathlineto{\pgfqpoint{3.375225in}{2.363408in}}%
\pgfpathlineto{\pgfqpoint{3.426426in}{2.238667in}}%
\pgfpathlineto{\pgfqpoint{3.598649in}{1.810801in}}%
\pgfpathlineto{\pgfqpoint{3.640541in}{1.718765in}}%
\pgfpathlineto{\pgfqpoint{3.677778in}{1.644633in}}%
\pgfpathlineto{\pgfqpoint{3.710360in}{1.586646in}}%
\pgfpathlineto{\pgfqpoint{3.738288in}{1.542535in}}%
\pgfpathlineto{\pgfqpoint{3.766216in}{1.503877in}}%
\pgfpathlineto{\pgfqpoint{3.794144in}{1.470846in}}%
\pgfpathlineto{\pgfqpoint{3.817417in}{1.447670in}}%
\pgfpathlineto{\pgfqpoint{3.840691in}{1.428437in}}%
\pgfpathlineto{\pgfqpoint{3.863964in}{1.413079in}}%
\pgfpathlineto{\pgfqpoint{3.887237in}{1.401482in}}%
\pgfpathlineto{\pgfqpoint{3.910511in}{1.393482in}}%
\pgfpathlineto{\pgfqpoint{3.933784in}{1.388866in}}%
\pgfpathlineto{\pgfqpoint{3.957057in}{1.387379in}}%
\pgfpathlineto{\pgfqpoint{3.980330in}{1.388726in}}%
\pgfpathlineto{\pgfqpoint{4.008258in}{1.393609in}}%
\pgfpathlineto{\pgfqpoint{4.040841in}{1.402998in}}%
\pgfpathlineto{\pgfqpoint{4.078078in}{1.417239in}}%
\pgfpathlineto{\pgfqpoint{4.217718in}{1.474478in}}%
\pgfpathlineto{\pgfqpoint{4.250300in}{1.482273in}}%
\pgfpathlineto{\pgfqpoint{4.278228in}{1.485720in}}%
\pgfpathlineto{\pgfqpoint{4.306156in}{1.485741in}}%
\pgfpathlineto{\pgfqpoint{4.329429in}{1.482935in}}%
\pgfpathlineto{\pgfqpoint{4.352703in}{1.477465in}}%
\pgfpathlineto{\pgfqpoint{4.380631in}{1.467376in}}%
\pgfpathlineto{\pgfqpoint{4.408559in}{1.453578in}}%
\pgfpathlineto{\pgfqpoint{4.436486in}{1.436383in}}%
\pgfpathlineto{\pgfqpoint{4.469069in}{1.412701in}}%
\pgfpathlineto{\pgfqpoint{4.510961in}{1.378249in}}%
\pgfpathlineto{\pgfqpoint{4.594745in}{1.308150in}}%
\pgfpathlineto{\pgfqpoint{4.627327in}{1.285755in}}%
\pgfpathlineto{\pgfqpoint{4.650601in}{1.273121in}}%
\pgfpathlineto{\pgfqpoint{4.673874in}{1.264037in}}%
\pgfpathlineto{\pgfqpoint{4.692492in}{1.259712in}}%
\pgfpathlineto{\pgfqpoint{4.711111in}{1.258260in}}%
\pgfpathlineto{\pgfqpoint{4.729730in}{1.259845in}}%
\pgfpathlineto{\pgfqpoint{4.748348in}{1.264542in}}%
\pgfpathlineto{\pgfqpoint{4.766967in}{1.272318in}}%
\pgfpathlineto{\pgfqpoint{4.790240in}{1.286097in}}%
\pgfpathlineto{\pgfqpoint{4.813514in}{1.303731in}}%
\pgfpathlineto{\pgfqpoint{4.850751in}{1.336862in}}%
\pgfpathlineto{\pgfqpoint{4.883333in}{1.365208in}}%
\pgfpathlineto{\pgfqpoint{4.901952in}{1.377910in}}%
\pgfpathlineto{\pgfqpoint{4.915916in}{1.384254in}}%
\pgfpathlineto{\pgfqpoint{4.929880in}{1.386736in}}%
\pgfpathlineto{\pgfqpoint{4.939189in}{1.385667in}}%
\pgfpathlineto{\pgfqpoint{4.948498in}{1.381968in}}%
\pgfpathlineto{\pgfqpoint{4.957808in}{1.375216in}}%
\pgfpathlineto{\pgfqpoint{4.967117in}{1.364950in}}%
\pgfpathlineto{\pgfqpoint{4.976426in}{1.350673in}}%
\pgfpathlineto{\pgfqpoint{4.985736in}{1.331848in}}%
\pgfpathlineto{\pgfqpoint{4.995045in}{1.307895in}}%
\pgfpathlineto{\pgfqpoint{5.009009in}{1.260974in}}%
\pgfpathlineto{\pgfqpoint{5.022973in}{1.198802in}}%
\pgfpathlineto{\pgfqpoint{5.036937in}{1.118822in}}%
\pgfpathlineto{\pgfqpoint{5.050901in}{1.018212in}}%
\pgfpathlineto{\pgfqpoint{5.064865in}{0.893855in}}%
\pgfpathlineto{\pgfqpoint{5.078829in}{0.742324in}}%
\pgfpathlineto{\pgfqpoint{5.092793in}{0.559857in}}%
\pgfpathlineto{\pgfqpoint{5.097791in}{0.486111in}}%
\pgfpathlineto{\pgfqpoint{5.097791in}{0.486111in}}%
\pgfusepath{stroke}%
\end{pgfscope}%
\begin{pgfscope}%
\pgfpathrectangle{\pgfqpoint{0.750000in}{0.500000in}}{\pgfqpoint{4.650000in}{3.020000in}}%
\pgfusepath{clip}%
\pgfsetrectcap%
\pgfsetroundjoin%
\pgfsetlinewidth{1.505625pt}%
\definecolor{currentstroke}{rgb}{0.549020,0.337255,0.294118}%
\pgfsetstrokecolor{currentstroke}%
\pgfsetdash{}{0pt}%
\pgfpathmoveto{\pgfqpoint{0.750000in}{1.295811in}}%
\pgfpathlineto{\pgfqpoint{1.001351in}{1.305950in}}%
\pgfpathlineto{\pgfqpoint{1.206156in}{1.317243in}}%
\pgfpathlineto{\pgfqpoint{1.373724in}{1.329474in}}%
\pgfpathlineto{\pgfqpoint{1.518018in}{1.343075in}}%
\pgfpathlineto{\pgfqpoint{1.639039in}{1.357496in}}%
\pgfpathlineto{\pgfqpoint{1.746096in}{1.373329in}}%
\pgfpathlineto{\pgfqpoint{1.839189in}{1.390172in}}%
\pgfpathlineto{\pgfqpoint{1.922973in}{1.408478in}}%
\pgfpathlineto{\pgfqpoint{1.997447in}{1.427912in}}%
\pgfpathlineto{\pgfqpoint{2.062613in}{1.447953in}}%
\pgfpathlineto{\pgfqpoint{2.123123in}{1.469666in}}%
\pgfpathlineto{\pgfqpoint{2.178979in}{1.492916in}}%
\pgfpathlineto{\pgfqpoint{2.230180in}{1.517463in}}%
\pgfpathlineto{\pgfqpoint{2.276727in}{1.542957in}}%
\pgfpathlineto{\pgfqpoint{2.323273in}{1.572003in}}%
\pgfpathlineto{\pgfqpoint{2.365165in}{1.601680in}}%
\pgfpathlineto{\pgfqpoint{2.407057in}{1.635235in}}%
\pgfpathlineto{\pgfqpoint{2.444294in}{1.668793in}}%
\pgfpathlineto{\pgfqpoint{2.481532in}{1.706341in}}%
\pgfpathlineto{\pgfqpoint{2.518769in}{1.748387in}}%
\pgfpathlineto{\pgfqpoint{2.556006in}{1.795477in}}%
\pgfpathlineto{\pgfqpoint{2.593243in}{1.848168in}}%
\pgfpathlineto{\pgfqpoint{2.630480in}{1.907001in}}%
\pgfpathlineto{\pgfqpoint{2.667718in}{1.972439in}}%
\pgfpathlineto{\pgfqpoint{2.704955in}{2.044774in}}%
\pgfpathlineto{\pgfqpoint{2.742192in}{2.123995in}}%
\pgfpathlineto{\pgfqpoint{2.784084in}{2.220704in}}%
\pgfpathlineto{\pgfqpoint{2.839940in}{2.358824in}}%
\pgfpathlineto{\pgfqpoint{2.914414in}{2.543683in}}%
\pgfpathlineto{\pgfqpoint{2.946997in}{2.616442in}}%
\pgfpathlineto{\pgfqpoint{2.970270in}{2.662209in}}%
\pgfpathlineto{\pgfqpoint{2.993544in}{2.701099in}}%
\pgfpathlineto{\pgfqpoint{3.012162in}{2.726310in}}%
\pgfpathlineto{\pgfqpoint{3.030781in}{2.745589in}}%
\pgfpathlineto{\pgfqpoint{3.044745in}{2.755851in}}%
\pgfpathlineto{\pgfqpoint{3.058709in}{2.762336in}}%
\pgfpathlineto{\pgfqpoint{3.072673in}{2.764946in}}%
\pgfpathlineto{\pgfqpoint{3.086637in}{2.763640in}}%
\pgfpathlineto{\pgfqpoint{3.100601in}{2.758438in}}%
\pgfpathlineto{\pgfqpoint{3.114565in}{2.749421in}}%
\pgfpathlineto{\pgfqpoint{3.128529in}{2.736725in}}%
\pgfpathlineto{\pgfqpoint{3.147147in}{2.714409in}}%
\pgfpathlineto{\pgfqpoint{3.165766in}{2.686462in}}%
\pgfpathlineto{\pgfqpoint{3.189039in}{2.644644in}}%
\pgfpathlineto{\pgfqpoint{3.216967in}{2.586306in}}%
\pgfpathlineto{\pgfqpoint{3.249550in}{2.510294in}}%
\pgfpathlineto{\pgfqpoint{3.310060in}{2.358824in}}%
\pgfpathlineto{\pgfqpoint{3.370571in}{2.209604in}}%
\pgfpathlineto{\pgfqpoint{3.412462in}{2.113727in}}%
\pgfpathlineto{\pgfqpoint{3.449700in}{2.035352in}}%
\pgfpathlineto{\pgfqpoint{3.486937in}{1.963886in}}%
\pgfpathlineto{\pgfqpoint{3.524174in}{1.899294in}}%
\pgfpathlineto{\pgfqpoint{3.561411in}{1.841256in}}%
\pgfpathlineto{\pgfqpoint{3.598649in}{1.789296in}}%
\pgfpathlineto{\pgfqpoint{3.635886in}{1.742867in}}%
\pgfpathlineto{\pgfqpoint{3.673123in}{1.701412in}}%
\pgfpathlineto{\pgfqpoint{3.710360in}{1.664390in}}%
\pgfpathlineto{\pgfqpoint{3.752252in}{1.627414in}}%
\pgfpathlineto{\pgfqpoint{3.794144in}{1.594769in}}%
\pgfpathlineto{\pgfqpoint{3.836036in}{1.565882in}}%
\pgfpathlineto{\pgfqpoint{3.882583in}{1.537591in}}%
\pgfpathlineto{\pgfqpoint{3.929129in}{1.512744in}}%
\pgfpathlineto{\pgfqpoint{3.980330in}{1.488802in}}%
\pgfpathlineto{\pgfqpoint{4.036186in}{1.466107in}}%
\pgfpathlineto{\pgfqpoint{4.096697in}{1.444893in}}%
\pgfpathlineto{\pgfqpoint{4.161862in}{1.425296in}}%
\pgfpathlineto{\pgfqpoint{4.236336in}{1.406272in}}%
\pgfpathlineto{\pgfqpoint{4.315465in}{1.389250in}}%
\pgfpathlineto{\pgfqpoint{4.403904in}{1.373329in}}%
\pgfpathlineto{\pgfqpoint{4.506306in}{1.358118in}}%
\pgfpathlineto{\pgfqpoint{4.622673in}{1.344075in}}%
\pgfpathlineto{\pgfqpoint{4.757658in}{1.331048in}}%
\pgfpathlineto{\pgfqpoint{4.911261in}{1.319377in}}%
\pgfpathlineto{\pgfqpoint{5.097447in}{1.308470in}}%
\pgfpathlineto{\pgfqpoint{5.320871in}{1.298653in}}%
\pgfpathlineto{\pgfqpoint{5.400000in}{1.295811in}}%
\pgfpathlineto{\pgfqpoint{5.400000in}{1.295811in}}%
\pgfusepath{stroke}%
\end{pgfscope}%
\begin{pgfscope}%
\pgfsetrectcap%
\pgfsetmiterjoin%
\pgfsetlinewidth{0.803000pt}%
\definecolor{currentstroke}{rgb}{0.000000,0.000000,0.000000}%
\pgfsetstrokecolor{currentstroke}%
\pgfsetdash{}{0pt}%
\pgfpathmoveto{\pgfqpoint{0.750000in}{0.500000in}}%
\pgfpathlineto{\pgfqpoint{0.750000in}{3.520000in}}%
\pgfusepath{stroke}%
\end{pgfscope}%
\begin{pgfscope}%
\pgfsetrectcap%
\pgfsetmiterjoin%
\pgfsetlinewidth{0.803000pt}%
\definecolor{currentstroke}{rgb}{0.000000,0.000000,0.000000}%
\pgfsetstrokecolor{currentstroke}%
\pgfsetdash{}{0pt}%
\pgfpathmoveto{\pgfqpoint{5.400000in}{0.500000in}}%
\pgfpathlineto{\pgfqpoint{5.400000in}{3.520000in}}%
\pgfusepath{stroke}%
\end{pgfscope}%
\begin{pgfscope}%
\pgfsetrectcap%
\pgfsetmiterjoin%
\pgfsetlinewidth{0.803000pt}%
\definecolor{currentstroke}{rgb}{0.000000,0.000000,0.000000}%
\pgfsetstrokecolor{currentstroke}%
\pgfsetdash{}{0pt}%
\pgfpathmoveto{\pgfqpoint{0.750000in}{0.500000in}}%
\pgfpathlineto{\pgfqpoint{5.400000in}{0.500000in}}%
\pgfusepath{stroke}%
\end{pgfscope}%
\begin{pgfscope}%
\pgfsetrectcap%
\pgfsetmiterjoin%
\pgfsetlinewidth{0.803000pt}%
\definecolor{currentstroke}{rgb}{0.000000,0.000000,0.000000}%
\pgfsetstrokecolor{currentstroke}%
\pgfsetdash{}{0pt}%
\pgfpathmoveto{\pgfqpoint{0.750000in}{3.520000in}}%
\pgfpathlineto{\pgfqpoint{5.400000in}{3.520000in}}%
\pgfusepath{stroke}%
\end{pgfscope}%
\begin{pgfscope}%
\pgfsetbuttcap%
\pgfsetmiterjoin%
\definecolor{currentfill}{rgb}{1.000000,1.000000,1.000000}%
\pgfsetfillcolor{currentfill}%
\pgfsetfillopacity{0.800000}%
\pgfsetlinewidth{1.003750pt}%
\definecolor{currentstroke}{rgb}{0.800000,0.800000,0.800000}%
\pgfsetstrokecolor{currentstroke}%
\pgfsetstrokeopacity{0.800000}%
\pgfsetdash{}{0pt}%
\pgfpathmoveto{\pgfqpoint{2.649058in}{0.569444in}}%
\pgfpathlineto{\pgfqpoint{3.500942in}{0.569444in}}%
\pgfpathquadraticcurveto{\pgfqpoint{3.528719in}{0.569444in}}{\pgfqpoint{3.528719in}{0.597222in}}%
\pgfpathlineto{\pgfqpoint{3.528719in}{1.745061in}}%
\pgfpathquadraticcurveto{\pgfqpoint{3.528719in}{1.772839in}}{\pgfqpoint{3.500942in}{1.772839in}}%
\pgfpathlineto{\pgfqpoint{2.649058in}{1.772839in}}%
\pgfpathquadraticcurveto{\pgfqpoint{2.621281in}{1.772839in}}{\pgfqpoint{2.621281in}{1.745061in}}%
\pgfpathlineto{\pgfqpoint{2.621281in}{0.597222in}}%
\pgfpathquadraticcurveto{\pgfqpoint{2.621281in}{0.569444in}}{\pgfqpoint{2.649058in}{0.569444in}}%
\pgfpathclose%
\pgfusepath{stroke,fill}%
\end{pgfscope}%
\begin{pgfscope}%
\pgfsetrectcap%
\pgfsetroundjoin%
\pgfsetlinewidth{1.505625pt}%
\definecolor{currentstroke}{rgb}{0.121569,0.466667,0.705882}%
\pgfsetstrokecolor{currentstroke}%
\pgfsetdash{}{0pt}%
\pgfpathmoveto{\pgfqpoint{2.676836in}{1.668672in}}%
\pgfpathlineto{\pgfqpoint{2.954614in}{1.668672in}}%
\pgfusepath{stroke}%
\end{pgfscope}%
\begin{pgfscope}%
\pgftext[x=3.065725in,y=1.620061in,left,base]{\rmfamily\fontsize{10.000000}{12.000000}\selectfont \(\displaystyle  n = 2 \)}%
\end{pgfscope}%
\begin{pgfscope}%
\pgfsetrectcap%
\pgfsetroundjoin%
\pgfsetlinewidth{1.505625pt}%
\definecolor{currentstroke}{rgb}{1.000000,0.498039,0.054902}%
\pgfsetstrokecolor{currentstroke}%
\pgfsetdash{}{0pt}%
\pgfpathmoveto{\pgfqpoint{2.676836in}{1.475061in}}%
\pgfpathlineto{\pgfqpoint{2.954614in}{1.475061in}}%
\pgfusepath{stroke}%
\end{pgfscope}%
\begin{pgfscope}%
\pgftext[x=3.065725in,y=1.426450in,left,base]{\rmfamily\fontsize{10.000000}{12.000000}\selectfont \(\displaystyle  n = 4 \)}%
\end{pgfscope}%
\begin{pgfscope}%
\pgfsetrectcap%
\pgfsetroundjoin%
\pgfsetlinewidth{1.505625pt}%
\definecolor{currentstroke}{rgb}{0.172549,0.627451,0.172549}%
\pgfsetstrokecolor{currentstroke}%
\pgfsetdash{}{0pt}%
\pgfpathmoveto{\pgfqpoint{2.676836in}{1.281450in}}%
\pgfpathlineto{\pgfqpoint{2.954614in}{1.281450in}}%
\pgfusepath{stroke}%
\end{pgfscope}%
\begin{pgfscope}%
\pgftext[x=3.065725in,y=1.232839in,left,base]{\rmfamily\fontsize{10.000000}{12.000000}\selectfont \(\displaystyle  n = 6 \)}%
\end{pgfscope}%
\begin{pgfscope}%
\pgfsetrectcap%
\pgfsetroundjoin%
\pgfsetlinewidth{1.505625pt}%
\definecolor{currentstroke}{rgb}{0.839216,0.152941,0.156863}%
\pgfsetstrokecolor{currentstroke}%
\pgfsetdash{}{0pt}%
\pgfpathmoveto{\pgfqpoint{2.676836in}{1.087839in}}%
\pgfpathlineto{\pgfqpoint{2.954614in}{1.087839in}}%
\pgfusepath{stroke}%
\end{pgfscope}%
\begin{pgfscope}%
\pgftext[x=3.065725in,y=1.039228in,left,base]{\rmfamily\fontsize{10.000000}{12.000000}\selectfont \(\displaystyle  n = 8 \)}%
\end{pgfscope}%
\begin{pgfscope}%
\pgfsetrectcap%
\pgfsetroundjoin%
\pgfsetlinewidth{1.505625pt}%
\definecolor{currentstroke}{rgb}{0.580392,0.403922,0.741176}%
\pgfsetstrokecolor{currentstroke}%
\pgfsetdash{}{0pt}%
\pgfpathmoveto{\pgfqpoint{2.676836in}{0.894228in}}%
\pgfpathlineto{\pgfqpoint{2.954614in}{0.894228in}}%
\pgfusepath{stroke}%
\end{pgfscope}%
\begin{pgfscope}%
\pgftext[x=3.065725in,y=0.845617in,left,base]{\rmfamily\fontsize{10.000000}{12.000000}\selectfont \(\displaystyle  n = 10 \)}%
\end{pgfscope}%
\begin{pgfscope}%
\pgfsetrectcap%
\pgfsetroundjoin%
\pgfsetlinewidth{1.505625pt}%
\definecolor{currentstroke}{rgb}{0.549020,0.337255,0.294118}%
\pgfsetstrokecolor{currentstroke}%
\pgfsetdash{}{0pt}%
\pgfpathmoveto{\pgfqpoint{2.676836in}{0.700617in}}%
\pgfpathlineto{\pgfqpoint{2.954614in}{0.700617in}}%
\pgfusepath{stroke}%
\end{pgfscope}%
\begin{pgfscope}%
\pgftext[x=3.065725in,y=0.652006in,left,base]{\rmfamily\fontsize{10.000000}{12.000000}\selectfont \(\displaystyle f_1\)}%
\end{pgfscope}%
\end{pgfpicture}%
\makeatother%
\endgroup%
}
\caption{Interpolating polynomials of degree $n$ to $f_1$ using zeros of $T_n$} \label{Fig:CheTan}
\end{figure}
The graph for interpolations of $ f_2 \rbr{x} = \se^{-x^2} $ using zeros of Chebyshev polynomials is shown in Figure \ref{Fig:CheExp}.
\begin{figure}[htbp]
\centering \scalebox{0.8}{%% Creator: Matplotlib, PGF backend
%%
%% To include the figure in your LaTeX document, write
%%   \input{<filename>.pgf}
%%
%% Make sure the required packages are loaded in your preamble
%%   \usepackage{pgf}
%%
%% Figures using additional raster images can only be included by \input if
%% they are in the same directory as the main LaTeX file. For loading figures
%% from other directories you can use the `import` package
%%   \usepackage{import}
%% and then include the figures with
%%   \import{<path to file>}{<filename>.pgf}
%%
%% Matplotlib used the following preamble
%%   \usepackage{fontspec}
%%
\begingroup%
\makeatletter%
\begin{pgfpicture}%
\pgfpathrectangle{\pgfpointorigin}{\pgfqpoint{6.000000in}{4.000000in}}%
\pgfusepath{use as bounding box, clip}%
\begin{pgfscope}%
\pgfsetbuttcap%
\pgfsetmiterjoin%
\definecolor{currentfill}{rgb}{1.000000,1.000000,1.000000}%
\pgfsetfillcolor{currentfill}%
\pgfsetlinewidth{0.000000pt}%
\definecolor{currentstroke}{rgb}{1.000000,1.000000,1.000000}%
\pgfsetstrokecolor{currentstroke}%
\pgfsetdash{}{0pt}%
\pgfpathmoveto{\pgfqpoint{0.000000in}{0.000000in}}%
\pgfpathlineto{\pgfqpoint{6.000000in}{0.000000in}}%
\pgfpathlineto{\pgfqpoint{6.000000in}{4.000000in}}%
\pgfpathlineto{\pgfqpoint{0.000000in}{4.000000in}}%
\pgfpathclose%
\pgfusepath{fill}%
\end{pgfscope}%
\begin{pgfscope}%
\pgfsetbuttcap%
\pgfsetmiterjoin%
\definecolor{currentfill}{rgb}{1.000000,1.000000,1.000000}%
\pgfsetfillcolor{currentfill}%
\pgfsetlinewidth{0.000000pt}%
\definecolor{currentstroke}{rgb}{0.000000,0.000000,0.000000}%
\pgfsetstrokecolor{currentstroke}%
\pgfsetstrokeopacity{0.000000}%
\pgfsetdash{}{0pt}%
\pgfpathmoveto{\pgfqpoint{0.750000in}{0.500000in}}%
\pgfpathlineto{\pgfqpoint{5.400000in}{0.500000in}}%
\pgfpathlineto{\pgfqpoint{5.400000in}{3.520000in}}%
\pgfpathlineto{\pgfqpoint{0.750000in}{3.520000in}}%
\pgfpathclose%
\pgfusepath{fill}%
\end{pgfscope}%
\begin{pgfscope}%
\pgfpathrectangle{\pgfqpoint{0.750000in}{0.500000in}}{\pgfqpoint{4.650000in}{3.020000in}}%
\pgfusepath{clip}%
\pgfsetrectcap%
\pgfsetroundjoin%
\pgfsetlinewidth{0.803000pt}%
\definecolor{currentstroke}{rgb}{0.690196,0.690196,0.690196}%
\pgfsetstrokecolor{currentstroke}%
\pgfsetdash{}{0pt}%
\pgfpathmoveto{\pgfqpoint{0.750000in}{0.500000in}}%
\pgfpathlineto{\pgfqpoint{0.750000in}{3.520000in}}%
\pgfusepath{stroke}%
\end{pgfscope}%
\begin{pgfscope}%
\pgfsetbuttcap%
\pgfsetroundjoin%
\definecolor{currentfill}{rgb}{0.000000,0.000000,0.000000}%
\pgfsetfillcolor{currentfill}%
\pgfsetlinewidth{0.803000pt}%
\definecolor{currentstroke}{rgb}{0.000000,0.000000,0.000000}%
\pgfsetstrokecolor{currentstroke}%
\pgfsetdash{}{0pt}%
\pgfsys@defobject{currentmarker}{\pgfqpoint{0.000000in}{-0.048611in}}{\pgfqpoint{0.000000in}{0.000000in}}{%
\pgfpathmoveto{\pgfqpoint{0.000000in}{0.000000in}}%
\pgfpathlineto{\pgfqpoint{0.000000in}{-0.048611in}}%
\pgfusepath{stroke,fill}%
}%
\begin{pgfscope}%
\pgfsys@transformshift{0.750000in}{0.500000in}%
\pgfsys@useobject{currentmarker}{}%
\end{pgfscope}%
\end{pgfscope}%
\begin{pgfscope}%
\pgftext[x=0.750000in,y=0.402778in,,top]{\rmfamily\fontsize{10.000000}{12.000000}\selectfont \(\displaystyle -6\)}%
\end{pgfscope}%
\begin{pgfscope}%
\pgfpathrectangle{\pgfqpoint{0.750000in}{0.500000in}}{\pgfqpoint{4.650000in}{3.020000in}}%
\pgfusepath{clip}%
\pgfsetrectcap%
\pgfsetroundjoin%
\pgfsetlinewidth{0.803000pt}%
\definecolor{currentstroke}{rgb}{0.690196,0.690196,0.690196}%
\pgfsetstrokecolor{currentstroke}%
\pgfsetdash{}{0pt}%
\pgfpathmoveto{\pgfqpoint{1.525000in}{0.500000in}}%
\pgfpathlineto{\pgfqpoint{1.525000in}{3.520000in}}%
\pgfusepath{stroke}%
\end{pgfscope}%
\begin{pgfscope}%
\pgfsetbuttcap%
\pgfsetroundjoin%
\definecolor{currentfill}{rgb}{0.000000,0.000000,0.000000}%
\pgfsetfillcolor{currentfill}%
\pgfsetlinewidth{0.803000pt}%
\definecolor{currentstroke}{rgb}{0.000000,0.000000,0.000000}%
\pgfsetstrokecolor{currentstroke}%
\pgfsetdash{}{0pt}%
\pgfsys@defobject{currentmarker}{\pgfqpoint{0.000000in}{-0.048611in}}{\pgfqpoint{0.000000in}{0.000000in}}{%
\pgfpathmoveto{\pgfqpoint{0.000000in}{0.000000in}}%
\pgfpathlineto{\pgfqpoint{0.000000in}{-0.048611in}}%
\pgfusepath{stroke,fill}%
}%
\begin{pgfscope}%
\pgfsys@transformshift{1.525000in}{0.500000in}%
\pgfsys@useobject{currentmarker}{}%
\end{pgfscope}%
\end{pgfscope}%
\begin{pgfscope}%
\pgftext[x=1.525000in,y=0.402778in,,top]{\rmfamily\fontsize{10.000000}{12.000000}\selectfont \(\displaystyle -4\)}%
\end{pgfscope}%
\begin{pgfscope}%
\pgfpathrectangle{\pgfqpoint{0.750000in}{0.500000in}}{\pgfqpoint{4.650000in}{3.020000in}}%
\pgfusepath{clip}%
\pgfsetrectcap%
\pgfsetroundjoin%
\pgfsetlinewidth{0.803000pt}%
\definecolor{currentstroke}{rgb}{0.690196,0.690196,0.690196}%
\pgfsetstrokecolor{currentstroke}%
\pgfsetdash{}{0pt}%
\pgfpathmoveto{\pgfqpoint{2.300000in}{0.500000in}}%
\pgfpathlineto{\pgfqpoint{2.300000in}{3.520000in}}%
\pgfusepath{stroke}%
\end{pgfscope}%
\begin{pgfscope}%
\pgfsetbuttcap%
\pgfsetroundjoin%
\definecolor{currentfill}{rgb}{0.000000,0.000000,0.000000}%
\pgfsetfillcolor{currentfill}%
\pgfsetlinewidth{0.803000pt}%
\definecolor{currentstroke}{rgb}{0.000000,0.000000,0.000000}%
\pgfsetstrokecolor{currentstroke}%
\pgfsetdash{}{0pt}%
\pgfsys@defobject{currentmarker}{\pgfqpoint{0.000000in}{-0.048611in}}{\pgfqpoint{0.000000in}{0.000000in}}{%
\pgfpathmoveto{\pgfqpoint{0.000000in}{0.000000in}}%
\pgfpathlineto{\pgfqpoint{0.000000in}{-0.048611in}}%
\pgfusepath{stroke,fill}%
}%
\begin{pgfscope}%
\pgfsys@transformshift{2.300000in}{0.500000in}%
\pgfsys@useobject{currentmarker}{}%
\end{pgfscope}%
\end{pgfscope}%
\begin{pgfscope}%
\pgftext[x=2.300000in,y=0.402778in,,top]{\rmfamily\fontsize{10.000000}{12.000000}\selectfont \(\displaystyle -2\)}%
\end{pgfscope}%
\begin{pgfscope}%
\pgfpathrectangle{\pgfqpoint{0.750000in}{0.500000in}}{\pgfqpoint{4.650000in}{3.020000in}}%
\pgfusepath{clip}%
\pgfsetrectcap%
\pgfsetroundjoin%
\pgfsetlinewidth{0.803000pt}%
\definecolor{currentstroke}{rgb}{0.690196,0.690196,0.690196}%
\pgfsetstrokecolor{currentstroke}%
\pgfsetdash{}{0pt}%
\pgfpathmoveto{\pgfqpoint{3.075000in}{0.500000in}}%
\pgfpathlineto{\pgfqpoint{3.075000in}{3.520000in}}%
\pgfusepath{stroke}%
\end{pgfscope}%
\begin{pgfscope}%
\pgfsetbuttcap%
\pgfsetroundjoin%
\definecolor{currentfill}{rgb}{0.000000,0.000000,0.000000}%
\pgfsetfillcolor{currentfill}%
\pgfsetlinewidth{0.803000pt}%
\definecolor{currentstroke}{rgb}{0.000000,0.000000,0.000000}%
\pgfsetstrokecolor{currentstroke}%
\pgfsetdash{}{0pt}%
\pgfsys@defobject{currentmarker}{\pgfqpoint{0.000000in}{-0.048611in}}{\pgfqpoint{0.000000in}{0.000000in}}{%
\pgfpathmoveto{\pgfqpoint{0.000000in}{0.000000in}}%
\pgfpathlineto{\pgfqpoint{0.000000in}{-0.048611in}}%
\pgfusepath{stroke,fill}%
}%
\begin{pgfscope}%
\pgfsys@transformshift{3.075000in}{0.500000in}%
\pgfsys@useobject{currentmarker}{}%
\end{pgfscope}%
\end{pgfscope}%
\begin{pgfscope}%
\pgftext[x=3.075000in,y=0.402778in,,top]{\rmfamily\fontsize{10.000000}{12.000000}\selectfont \(\displaystyle 0\)}%
\end{pgfscope}%
\begin{pgfscope}%
\pgfpathrectangle{\pgfqpoint{0.750000in}{0.500000in}}{\pgfqpoint{4.650000in}{3.020000in}}%
\pgfusepath{clip}%
\pgfsetrectcap%
\pgfsetroundjoin%
\pgfsetlinewidth{0.803000pt}%
\definecolor{currentstroke}{rgb}{0.690196,0.690196,0.690196}%
\pgfsetstrokecolor{currentstroke}%
\pgfsetdash{}{0pt}%
\pgfpathmoveto{\pgfqpoint{3.850000in}{0.500000in}}%
\pgfpathlineto{\pgfqpoint{3.850000in}{3.520000in}}%
\pgfusepath{stroke}%
\end{pgfscope}%
\begin{pgfscope}%
\pgfsetbuttcap%
\pgfsetroundjoin%
\definecolor{currentfill}{rgb}{0.000000,0.000000,0.000000}%
\pgfsetfillcolor{currentfill}%
\pgfsetlinewidth{0.803000pt}%
\definecolor{currentstroke}{rgb}{0.000000,0.000000,0.000000}%
\pgfsetstrokecolor{currentstroke}%
\pgfsetdash{}{0pt}%
\pgfsys@defobject{currentmarker}{\pgfqpoint{0.000000in}{-0.048611in}}{\pgfqpoint{0.000000in}{0.000000in}}{%
\pgfpathmoveto{\pgfqpoint{0.000000in}{0.000000in}}%
\pgfpathlineto{\pgfqpoint{0.000000in}{-0.048611in}}%
\pgfusepath{stroke,fill}%
}%
\begin{pgfscope}%
\pgfsys@transformshift{3.850000in}{0.500000in}%
\pgfsys@useobject{currentmarker}{}%
\end{pgfscope}%
\end{pgfscope}%
\begin{pgfscope}%
\pgftext[x=3.850000in,y=0.402778in,,top]{\rmfamily\fontsize{10.000000}{12.000000}\selectfont \(\displaystyle 2\)}%
\end{pgfscope}%
\begin{pgfscope}%
\pgfpathrectangle{\pgfqpoint{0.750000in}{0.500000in}}{\pgfqpoint{4.650000in}{3.020000in}}%
\pgfusepath{clip}%
\pgfsetrectcap%
\pgfsetroundjoin%
\pgfsetlinewidth{0.803000pt}%
\definecolor{currentstroke}{rgb}{0.690196,0.690196,0.690196}%
\pgfsetstrokecolor{currentstroke}%
\pgfsetdash{}{0pt}%
\pgfpathmoveto{\pgfqpoint{4.625000in}{0.500000in}}%
\pgfpathlineto{\pgfqpoint{4.625000in}{3.520000in}}%
\pgfusepath{stroke}%
\end{pgfscope}%
\begin{pgfscope}%
\pgfsetbuttcap%
\pgfsetroundjoin%
\definecolor{currentfill}{rgb}{0.000000,0.000000,0.000000}%
\pgfsetfillcolor{currentfill}%
\pgfsetlinewidth{0.803000pt}%
\definecolor{currentstroke}{rgb}{0.000000,0.000000,0.000000}%
\pgfsetstrokecolor{currentstroke}%
\pgfsetdash{}{0pt}%
\pgfsys@defobject{currentmarker}{\pgfqpoint{0.000000in}{-0.048611in}}{\pgfqpoint{0.000000in}{0.000000in}}{%
\pgfpathmoveto{\pgfqpoint{0.000000in}{0.000000in}}%
\pgfpathlineto{\pgfqpoint{0.000000in}{-0.048611in}}%
\pgfusepath{stroke,fill}%
}%
\begin{pgfscope}%
\pgfsys@transformshift{4.625000in}{0.500000in}%
\pgfsys@useobject{currentmarker}{}%
\end{pgfscope}%
\end{pgfscope}%
\begin{pgfscope}%
\pgftext[x=4.625000in,y=0.402778in,,top]{\rmfamily\fontsize{10.000000}{12.000000}\selectfont \(\displaystyle 4\)}%
\end{pgfscope}%
\begin{pgfscope}%
\pgfpathrectangle{\pgfqpoint{0.750000in}{0.500000in}}{\pgfqpoint{4.650000in}{3.020000in}}%
\pgfusepath{clip}%
\pgfsetrectcap%
\pgfsetroundjoin%
\pgfsetlinewidth{0.803000pt}%
\definecolor{currentstroke}{rgb}{0.690196,0.690196,0.690196}%
\pgfsetstrokecolor{currentstroke}%
\pgfsetdash{}{0pt}%
\pgfpathmoveto{\pgfqpoint{5.400000in}{0.500000in}}%
\pgfpathlineto{\pgfqpoint{5.400000in}{3.520000in}}%
\pgfusepath{stroke}%
\end{pgfscope}%
\begin{pgfscope}%
\pgfsetbuttcap%
\pgfsetroundjoin%
\definecolor{currentfill}{rgb}{0.000000,0.000000,0.000000}%
\pgfsetfillcolor{currentfill}%
\pgfsetlinewidth{0.803000pt}%
\definecolor{currentstroke}{rgb}{0.000000,0.000000,0.000000}%
\pgfsetstrokecolor{currentstroke}%
\pgfsetdash{}{0pt}%
\pgfsys@defobject{currentmarker}{\pgfqpoint{0.000000in}{-0.048611in}}{\pgfqpoint{0.000000in}{0.000000in}}{%
\pgfpathmoveto{\pgfqpoint{0.000000in}{0.000000in}}%
\pgfpathlineto{\pgfqpoint{0.000000in}{-0.048611in}}%
\pgfusepath{stroke,fill}%
}%
\begin{pgfscope}%
\pgfsys@transformshift{5.400000in}{0.500000in}%
\pgfsys@useobject{currentmarker}{}%
\end{pgfscope}%
\end{pgfscope}%
\begin{pgfscope}%
\pgftext[x=5.400000in,y=0.402778in,,top]{\rmfamily\fontsize{10.000000}{12.000000}\selectfont \(\displaystyle 6\)}%
\end{pgfscope}%
\begin{pgfscope}%
\pgfpathrectangle{\pgfqpoint{0.750000in}{0.500000in}}{\pgfqpoint{4.650000in}{3.020000in}}%
\pgfusepath{clip}%
\pgfsetrectcap%
\pgfsetroundjoin%
\pgfsetlinewidth{0.803000pt}%
\definecolor{currentstroke}{rgb}{0.690196,0.690196,0.690196}%
\pgfsetstrokecolor{currentstroke}%
\pgfsetdash{}{0pt}%
\pgfpathmoveto{\pgfqpoint{0.750000in}{0.500000in}}%
\pgfpathlineto{\pgfqpoint{5.400000in}{0.500000in}}%
\pgfusepath{stroke}%
\end{pgfscope}%
\begin{pgfscope}%
\pgfsetbuttcap%
\pgfsetroundjoin%
\definecolor{currentfill}{rgb}{0.000000,0.000000,0.000000}%
\pgfsetfillcolor{currentfill}%
\pgfsetlinewidth{0.803000pt}%
\definecolor{currentstroke}{rgb}{0.000000,0.000000,0.000000}%
\pgfsetstrokecolor{currentstroke}%
\pgfsetdash{}{0pt}%
\pgfsys@defobject{currentmarker}{\pgfqpoint{-0.048611in}{0.000000in}}{\pgfqpoint{0.000000in}{0.000000in}}{%
\pgfpathmoveto{\pgfqpoint{0.000000in}{0.000000in}}%
\pgfpathlineto{\pgfqpoint{-0.048611in}{0.000000in}}%
\pgfusepath{stroke,fill}%
}%
\begin{pgfscope}%
\pgfsys@transformshift{0.750000in}{0.500000in}%
\pgfsys@useobject{currentmarker}{}%
\end{pgfscope}%
\end{pgfscope}%
\begin{pgfscope}%
\pgftext[x=0.297838in,y=0.451806in,left,base]{\rmfamily\fontsize{10.000000}{12.000000}\selectfont \(\displaystyle -0.50\)}%
\end{pgfscope}%
\begin{pgfscope}%
\pgfpathrectangle{\pgfqpoint{0.750000in}{0.500000in}}{\pgfqpoint{4.650000in}{3.020000in}}%
\pgfusepath{clip}%
\pgfsetrectcap%
\pgfsetroundjoin%
\pgfsetlinewidth{0.803000pt}%
\definecolor{currentstroke}{rgb}{0.690196,0.690196,0.690196}%
\pgfsetstrokecolor{currentstroke}%
\pgfsetdash{}{0pt}%
\pgfpathmoveto{\pgfqpoint{0.750000in}{0.877500in}}%
\pgfpathlineto{\pgfqpoint{5.400000in}{0.877500in}}%
\pgfusepath{stroke}%
\end{pgfscope}%
\begin{pgfscope}%
\pgfsetbuttcap%
\pgfsetroundjoin%
\definecolor{currentfill}{rgb}{0.000000,0.000000,0.000000}%
\pgfsetfillcolor{currentfill}%
\pgfsetlinewidth{0.803000pt}%
\definecolor{currentstroke}{rgb}{0.000000,0.000000,0.000000}%
\pgfsetstrokecolor{currentstroke}%
\pgfsetdash{}{0pt}%
\pgfsys@defobject{currentmarker}{\pgfqpoint{-0.048611in}{0.000000in}}{\pgfqpoint{0.000000in}{0.000000in}}{%
\pgfpathmoveto{\pgfqpoint{0.000000in}{0.000000in}}%
\pgfpathlineto{\pgfqpoint{-0.048611in}{0.000000in}}%
\pgfusepath{stroke,fill}%
}%
\begin{pgfscope}%
\pgfsys@transformshift{0.750000in}{0.877500in}%
\pgfsys@useobject{currentmarker}{}%
\end{pgfscope}%
\end{pgfscope}%
\begin{pgfscope}%
\pgftext[x=0.297838in,y=0.829306in,left,base]{\rmfamily\fontsize{10.000000}{12.000000}\selectfont \(\displaystyle -0.25\)}%
\end{pgfscope}%
\begin{pgfscope}%
\pgfpathrectangle{\pgfqpoint{0.750000in}{0.500000in}}{\pgfqpoint{4.650000in}{3.020000in}}%
\pgfusepath{clip}%
\pgfsetrectcap%
\pgfsetroundjoin%
\pgfsetlinewidth{0.803000pt}%
\definecolor{currentstroke}{rgb}{0.690196,0.690196,0.690196}%
\pgfsetstrokecolor{currentstroke}%
\pgfsetdash{}{0pt}%
\pgfpathmoveto{\pgfqpoint{0.750000in}{1.255000in}}%
\pgfpathlineto{\pgfqpoint{5.400000in}{1.255000in}}%
\pgfusepath{stroke}%
\end{pgfscope}%
\begin{pgfscope}%
\pgfsetbuttcap%
\pgfsetroundjoin%
\definecolor{currentfill}{rgb}{0.000000,0.000000,0.000000}%
\pgfsetfillcolor{currentfill}%
\pgfsetlinewidth{0.803000pt}%
\definecolor{currentstroke}{rgb}{0.000000,0.000000,0.000000}%
\pgfsetstrokecolor{currentstroke}%
\pgfsetdash{}{0pt}%
\pgfsys@defobject{currentmarker}{\pgfqpoint{-0.048611in}{0.000000in}}{\pgfqpoint{0.000000in}{0.000000in}}{%
\pgfpathmoveto{\pgfqpoint{0.000000in}{0.000000in}}%
\pgfpathlineto{\pgfqpoint{-0.048611in}{0.000000in}}%
\pgfusepath{stroke,fill}%
}%
\begin{pgfscope}%
\pgfsys@transformshift{0.750000in}{1.255000in}%
\pgfsys@useobject{currentmarker}{}%
\end{pgfscope}%
\end{pgfscope}%
\begin{pgfscope}%
\pgftext[x=0.405863in,y=1.206806in,left,base]{\rmfamily\fontsize{10.000000}{12.000000}\selectfont \(\displaystyle 0.00\)}%
\end{pgfscope}%
\begin{pgfscope}%
\pgfpathrectangle{\pgfqpoint{0.750000in}{0.500000in}}{\pgfqpoint{4.650000in}{3.020000in}}%
\pgfusepath{clip}%
\pgfsetrectcap%
\pgfsetroundjoin%
\pgfsetlinewidth{0.803000pt}%
\definecolor{currentstroke}{rgb}{0.690196,0.690196,0.690196}%
\pgfsetstrokecolor{currentstroke}%
\pgfsetdash{}{0pt}%
\pgfpathmoveto{\pgfqpoint{0.750000in}{1.632500in}}%
\pgfpathlineto{\pgfqpoint{5.400000in}{1.632500in}}%
\pgfusepath{stroke}%
\end{pgfscope}%
\begin{pgfscope}%
\pgfsetbuttcap%
\pgfsetroundjoin%
\definecolor{currentfill}{rgb}{0.000000,0.000000,0.000000}%
\pgfsetfillcolor{currentfill}%
\pgfsetlinewidth{0.803000pt}%
\definecolor{currentstroke}{rgb}{0.000000,0.000000,0.000000}%
\pgfsetstrokecolor{currentstroke}%
\pgfsetdash{}{0pt}%
\pgfsys@defobject{currentmarker}{\pgfqpoint{-0.048611in}{0.000000in}}{\pgfqpoint{0.000000in}{0.000000in}}{%
\pgfpathmoveto{\pgfqpoint{0.000000in}{0.000000in}}%
\pgfpathlineto{\pgfqpoint{-0.048611in}{0.000000in}}%
\pgfusepath{stroke,fill}%
}%
\begin{pgfscope}%
\pgfsys@transformshift{0.750000in}{1.632500in}%
\pgfsys@useobject{currentmarker}{}%
\end{pgfscope}%
\end{pgfscope}%
\begin{pgfscope}%
\pgftext[x=0.405863in,y=1.584306in,left,base]{\rmfamily\fontsize{10.000000}{12.000000}\selectfont \(\displaystyle 0.25\)}%
\end{pgfscope}%
\begin{pgfscope}%
\pgfpathrectangle{\pgfqpoint{0.750000in}{0.500000in}}{\pgfqpoint{4.650000in}{3.020000in}}%
\pgfusepath{clip}%
\pgfsetrectcap%
\pgfsetroundjoin%
\pgfsetlinewidth{0.803000pt}%
\definecolor{currentstroke}{rgb}{0.690196,0.690196,0.690196}%
\pgfsetstrokecolor{currentstroke}%
\pgfsetdash{}{0pt}%
\pgfpathmoveto{\pgfqpoint{0.750000in}{2.010000in}}%
\pgfpathlineto{\pgfqpoint{5.400000in}{2.010000in}}%
\pgfusepath{stroke}%
\end{pgfscope}%
\begin{pgfscope}%
\pgfsetbuttcap%
\pgfsetroundjoin%
\definecolor{currentfill}{rgb}{0.000000,0.000000,0.000000}%
\pgfsetfillcolor{currentfill}%
\pgfsetlinewidth{0.803000pt}%
\definecolor{currentstroke}{rgb}{0.000000,0.000000,0.000000}%
\pgfsetstrokecolor{currentstroke}%
\pgfsetdash{}{0pt}%
\pgfsys@defobject{currentmarker}{\pgfqpoint{-0.048611in}{0.000000in}}{\pgfqpoint{0.000000in}{0.000000in}}{%
\pgfpathmoveto{\pgfqpoint{0.000000in}{0.000000in}}%
\pgfpathlineto{\pgfqpoint{-0.048611in}{0.000000in}}%
\pgfusepath{stroke,fill}%
}%
\begin{pgfscope}%
\pgfsys@transformshift{0.750000in}{2.010000in}%
\pgfsys@useobject{currentmarker}{}%
\end{pgfscope}%
\end{pgfscope}%
\begin{pgfscope}%
\pgftext[x=0.405863in,y=1.961806in,left,base]{\rmfamily\fontsize{10.000000}{12.000000}\selectfont \(\displaystyle 0.50\)}%
\end{pgfscope}%
\begin{pgfscope}%
\pgfpathrectangle{\pgfqpoint{0.750000in}{0.500000in}}{\pgfqpoint{4.650000in}{3.020000in}}%
\pgfusepath{clip}%
\pgfsetrectcap%
\pgfsetroundjoin%
\pgfsetlinewidth{0.803000pt}%
\definecolor{currentstroke}{rgb}{0.690196,0.690196,0.690196}%
\pgfsetstrokecolor{currentstroke}%
\pgfsetdash{}{0pt}%
\pgfpathmoveto{\pgfqpoint{0.750000in}{2.387500in}}%
\pgfpathlineto{\pgfqpoint{5.400000in}{2.387500in}}%
\pgfusepath{stroke}%
\end{pgfscope}%
\begin{pgfscope}%
\pgfsetbuttcap%
\pgfsetroundjoin%
\definecolor{currentfill}{rgb}{0.000000,0.000000,0.000000}%
\pgfsetfillcolor{currentfill}%
\pgfsetlinewidth{0.803000pt}%
\definecolor{currentstroke}{rgb}{0.000000,0.000000,0.000000}%
\pgfsetstrokecolor{currentstroke}%
\pgfsetdash{}{0pt}%
\pgfsys@defobject{currentmarker}{\pgfqpoint{-0.048611in}{0.000000in}}{\pgfqpoint{0.000000in}{0.000000in}}{%
\pgfpathmoveto{\pgfqpoint{0.000000in}{0.000000in}}%
\pgfpathlineto{\pgfqpoint{-0.048611in}{0.000000in}}%
\pgfusepath{stroke,fill}%
}%
\begin{pgfscope}%
\pgfsys@transformshift{0.750000in}{2.387500in}%
\pgfsys@useobject{currentmarker}{}%
\end{pgfscope}%
\end{pgfscope}%
\begin{pgfscope}%
\pgftext[x=0.405863in,y=2.339306in,left,base]{\rmfamily\fontsize{10.000000}{12.000000}\selectfont \(\displaystyle 0.75\)}%
\end{pgfscope}%
\begin{pgfscope}%
\pgfpathrectangle{\pgfqpoint{0.750000in}{0.500000in}}{\pgfqpoint{4.650000in}{3.020000in}}%
\pgfusepath{clip}%
\pgfsetrectcap%
\pgfsetroundjoin%
\pgfsetlinewidth{0.803000pt}%
\definecolor{currentstroke}{rgb}{0.690196,0.690196,0.690196}%
\pgfsetstrokecolor{currentstroke}%
\pgfsetdash{}{0pt}%
\pgfpathmoveto{\pgfqpoint{0.750000in}{2.765000in}}%
\pgfpathlineto{\pgfqpoint{5.400000in}{2.765000in}}%
\pgfusepath{stroke}%
\end{pgfscope}%
\begin{pgfscope}%
\pgfsetbuttcap%
\pgfsetroundjoin%
\definecolor{currentfill}{rgb}{0.000000,0.000000,0.000000}%
\pgfsetfillcolor{currentfill}%
\pgfsetlinewidth{0.803000pt}%
\definecolor{currentstroke}{rgb}{0.000000,0.000000,0.000000}%
\pgfsetstrokecolor{currentstroke}%
\pgfsetdash{}{0pt}%
\pgfsys@defobject{currentmarker}{\pgfqpoint{-0.048611in}{0.000000in}}{\pgfqpoint{0.000000in}{0.000000in}}{%
\pgfpathmoveto{\pgfqpoint{0.000000in}{0.000000in}}%
\pgfpathlineto{\pgfqpoint{-0.048611in}{0.000000in}}%
\pgfusepath{stroke,fill}%
}%
\begin{pgfscope}%
\pgfsys@transformshift{0.750000in}{2.765000in}%
\pgfsys@useobject{currentmarker}{}%
\end{pgfscope}%
\end{pgfscope}%
\begin{pgfscope}%
\pgftext[x=0.405863in,y=2.716806in,left,base]{\rmfamily\fontsize{10.000000}{12.000000}\selectfont \(\displaystyle 1.00\)}%
\end{pgfscope}%
\begin{pgfscope}%
\pgfpathrectangle{\pgfqpoint{0.750000in}{0.500000in}}{\pgfqpoint{4.650000in}{3.020000in}}%
\pgfusepath{clip}%
\pgfsetrectcap%
\pgfsetroundjoin%
\pgfsetlinewidth{0.803000pt}%
\definecolor{currentstroke}{rgb}{0.690196,0.690196,0.690196}%
\pgfsetstrokecolor{currentstroke}%
\pgfsetdash{}{0pt}%
\pgfpathmoveto{\pgfqpoint{0.750000in}{3.142500in}}%
\pgfpathlineto{\pgfqpoint{5.400000in}{3.142500in}}%
\pgfusepath{stroke}%
\end{pgfscope}%
\begin{pgfscope}%
\pgfsetbuttcap%
\pgfsetroundjoin%
\definecolor{currentfill}{rgb}{0.000000,0.000000,0.000000}%
\pgfsetfillcolor{currentfill}%
\pgfsetlinewidth{0.803000pt}%
\definecolor{currentstroke}{rgb}{0.000000,0.000000,0.000000}%
\pgfsetstrokecolor{currentstroke}%
\pgfsetdash{}{0pt}%
\pgfsys@defobject{currentmarker}{\pgfqpoint{-0.048611in}{0.000000in}}{\pgfqpoint{0.000000in}{0.000000in}}{%
\pgfpathmoveto{\pgfqpoint{0.000000in}{0.000000in}}%
\pgfpathlineto{\pgfqpoint{-0.048611in}{0.000000in}}%
\pgfusepath{stroke,fill}%
}%
\begin{pgfscope}%
\pgfsys@transformshift{0.750000in}{3.142500in}%
\pgfsys@useobject{currentmarker}{}%
\end{pgfscope}%
\end{pgfscope}%
\begin{pgfscope}%
\pgftext[x=0.405863in,y=3.094306in,left,base]{\rmfamily\fontsize{10.000000}{12.000000}\selectfont \(\displaystyle 1.25\)}%
\end{pgfscope}%
\begin{pgfscope}%
\pgfpathrectangle{\pgfqpoint{0.750000in}{0.500000in}}{\pgfqpoint{4.650000in}{3.020000in}}%
\pgfusepath{clip}%
\pgfsetrectcap%
\pgfsetroundjoin%
\pgfsetlinewidth{0.803000pt}%
\definecolor{currentstroke}{rgb}{0.690196,0.690196,0.690196}%
\pgfsetstrokecolor{currentstroke}%
\pgfsetdash{}{0pt}%
\pgfpathmoveto{\pgfqpoint{0.750000in}{3.520000in}}%
\pgfpathlineto{\pgfqpoint{5.400000in}{3.520000in}}%
\pgfusepath{stroke}%
\end{pgfscope}%
\begin{pgfscope}%
\pgfsetbuttcap%
\pgfsetroundjoin%
\definecolor{currentfill}{rgb}{0.000000,0.000000,0.000000}%
\pgfsetfillcolor{currentfill}%
\pgfsetlinewidth{0.803000pt}%
\definecolor{currentstroke}{rgb}{0.000000,0.000000,0.000000}%
\pgfsetstrokecolor{currentstroke}%
\pgfsetdash{}{0pt}%
\pgfsys@defobject{currentmarker}{\pgfqpoint{-0.048611in}{0.000000in}}{\pgfqpoint{0.000000in}{0.000000in}}{%
\pgfpathmoveto{\pgfqpoint{0.000000in}{0.000000in}}%
\pgfpathlineto{\pgfqpoint{-0.048611in}{0.000000in}}%
\pgfusepath{stroke,fill}%
}%
\begin{pgfscope}%
\pgfsys@transformshift{0.750000in}{3.520000in}%
\pgfsys@useobject{currentmarker}{}%
\end{pgfscope}%
\end{pgfscope}%
\begin{pgfscope}%
\pgftext[x=0.405863in,y=3.471806in,left,base]{\rmfamily\fontsize{10.000000}{12.000000}\selectfont \(\displaystyle 1.50\)}%
\end{pgfscope}%
\begin{pgfscope}%
\pgfpathrectangle{\pgfqpoint{0.750000in}{0.500000in}}{\pgfqpoint{4.650000in}{3.020000in}}%
\pgfusepath{clip}%
\pgfsetrectcap%
\pgfsetroundjoin%
\pgfsetlinewidth{1.505625pt}%
\definecolor{currentstroke}{rgb}{0.121569,0.466667,0.705882}%
\pgfsetstrokecolor{currentstroke}%
\pgfsetdash{}{0pt}%
\pgfpathmoveto{\pgfqpoint{0.750000in}{1.255006in}}%
\pgfpathlineto{\pgfqpoint{5.400000in}{1.255006in}}%
\pgfpathlineto{\pgfqpoint{5.400000in}{1.255006in}}%
\pgfusepath{stroke}%
\end{pgfscope}%
\begin{pgfscope}%
\pgfpathrectangle{\pgfqpoint{0.750000in}{0.500000in}}{\pgfqpoint{4.650000in}{3.020000in}}%
\pgfusepath{clip}%
\pgfsetrectcap%
\pgfsetroundjoin%
\pgfsetlinewidth{1.505625pt}%
\definecolor{currentstroke}{rgb}{1.000000,0.498039,0.054902}%
\pgfsetstrokecolor{currentstroke}%
\pgfsetdash{}{0pt}%
\pgfpathmoveto{\pgfqpoint{0.750000in}{1.222812in}}%
\pgfpathlineto{\pgfqpoint{1.075826in}{1.243412in}}%
\pgfpathlineto{\pgfqpoint{1.401652in}{1.260908in}}%
\pgfpathlineto{\pgfqpoint{1.727477in}{1.275299in}}%
\pgfpathlineto{\pgfqpoint{2.053303in}{1.286586in}}%
\pgfpathlineto{\pgfqpoint{2.379129in}{1.294769in}}%
\pgfpathlineto{\pgfqpoint{2.704955in}{1.299847in}}%
\pgfpathlineto{\pgfqpoint{3.030781in}{1.301820in}}%
\pgfpathlineto{\pgfqpoint{3.356607in}{1.300689in}}%
\pgfpathlineto{\pgfqpoint{3.682432in}{1.296454in}}%
\pgfpathlineto{\pgfqpoint{4.008258in}{1.289114in}}%
\pgfpathlineto{\pgfqpoint{4.334084in}{1.278670in}}%
\pgfpathlineto{\pgfqpoint{4.659910in}{1.265121in}}%
\pgfpathlineto{\pgfqpoint{4.985736in}{1.248468in}}%
\pgfpathlineto{\pgfqpoint{5.311562in}{1.228710in}}%
\pgfpathlineto{\pgfqpoint{5.400000in}{1.222812in}}%
\pgfpathlineto{\pgfqpoint{5.400000in}{1.222812in}}%
\pgfusepath{stroke}%
\end{pgfscope}%
\begin{pgfscope}%
\pgfpathrectangle{\pgfqpoint{0.750000in}{0.500000in}}{\pgfqpoint{4.650000in}{3.020000in}}%
\pgfusepath{clip}%
\pgfsetrectcap%
\pgfsetroundjoin%
\pgfsetlinewidth{1.505625pt}%
\definecolor{currentstroke}{rgb}{0.172549,0.627451,0.172549}%
\pgfsetstrokecolor{currentstroke}%
\pgfsetdash{}{0pt}%
\pgfpathmoveto{\pgfqpoint{0.750000in}{1.614534in}}%
\pgfpathlineto{\pgfqpoint{0.791892in}{1.560392in}}%
\pgfpathlineto{\pgfqpoint{0.833784in}{1.511116in}}%
\pgfpathlineto{\pgfqpoint{0.875676in}{1.466493in}}%
\pgfpathlineto{\pgfqpoint{0.917568in}{1.426313in}}%
\pgfpathlineto{\pgfqpoint{0.959459in}{1.390370in}}%
\pgfpathlineto{\pgfqpoint{1.001351in}{1.358462in}}%
\pgfpathlineto{\pgfqpoint{1.043243in}{1.330392in}}%
\pgfpathlineto{\pgfqpoint{1.085135in}{1.305966in}}%
\pgfpathlineto{\pgfqpoint{1.127027in}{1.284993in}}%
\pgfpathlineto{\pgfqpoint{1.168919in}{1.267288in}}%
\pgfpathlineto{\pgfqpoint{1.210811in}{1.252669in}}%
\pgfpathlineto{\pgfqpoint{1.257357in}{1.239828in}}%
\pgfpathlineto{\pgfqpoint{1.303904in}{1.230338in}}%
\pgfpathlineto{\pgfqpoint{1.350450in}{1.223967in}}%
\pgfpathlineto{\pgfqpoint{1.401652in}{1.220291in}}%
\pgfpathlineto{\pgfqpoint{1.452853in}{1.219821in}}%
\pgfpathlineto{\pgfqpoint{1.508709in}{1.222632in}}%
\pgfpathlineto{\pgfqpoint{1.564565in}{1.228564in}}%
\pgfpathlineto{\pgfqpoint{1.625075in}{1.238114in}}%
\pgfpathlineto{\pgfqpoint{1.690240in}{1.251560in}}%
\pgfpathlineto{\pgfqpoint{1.760060in}{1.269047in}}%
\pgfpathlineto{\pgfqpoint{1.839189in}{1.291989in}}%
\pgfpathlineto{\pgfqpoint{1.936937in}{1.323755in}}%
\pgfpathlineto{\pgfqpoint{2.067267in}{1.369701in}}%
\pgfpathlineto{\pgfqpoint{2.351201in}{1.470776in}}%
\pgfpathlineto{\pgfqpoint{2.458258in}{1.505160in}}%
\pgfpathlineto{\pgfqpoint{2.551351in}{1.532015in}}%
\pgfpathlineto{\pgfqpoint{2.639790in}{1.554333in}}%
\pgfpathlineto{\pgfqpoint{2.723574in}{1.572210in}}%
\pgfpathlineto{\pgfqpoint{2.802703in}{1.585902in}}%
\pgfpathlineto{\pgfqpoint{2.877177in}{1.595773in}}%
\pgfpathlineto{\pgfqpoint{2.951652in}{1.602592in}}%
\pgfpathlineto{\pgfqpoint{3.026126in}{1.606274in}}%
\pgfpathlineto{\pgfqpoint{3.100601in}{1.606773in}}%
\pgfpathlineto{\pgfqpoint{3.175075in}{1.604083in}}%
\pgfpathlineto{\pgfqpoint{3.249550in}{1.598237in}}%
\pgfpathlineto{\pgfqpoint{3.324024in}{1.589308in}}%
\pgfpathlineto{\pgfqpoint{3.403153in}{1.576569in}}%
\pgfpathlineto{\pgfqpoint{3.482282in}{1.560662in}}%
\pgfpathlineto{\pgfqpoint{3.566066in}{1.540625in}}%
\pgfpathlineto{\pgfqpoint{3.654505in}{1.516281in}}%
\pgfpathlineto{\pgfqpoint{3.752252in}{1.486127in}}%
\pgfpathlineto{\pgfqpoint{3.868619in}{1.446807in}}%
\pgfpathlineto{\pgfqpoint{4.045495in}{1.383197in}}%
\pgfpathlineto{\pgfqpoint{4.217718in}{1.322176in}}%
\pgfpathlineto{\pgfqpoint{4.320120in}{1.289143in}}%
\pgfpathlineto{\pgfqpoint{4.403904in}{1.265322in}}%
\pgfpathlineto{\pgfqpoint{4.478378in}{1.247412in}}%
\pgfpathlineto{\pgfqpoint{4.543544in}{1.234854in}}%
\pgfpathlineto{\pgfqpoint{4.604054in}{1.226262in}}%
\pgfpathlineto{\pgfqpoint{4.659910in}{1.221331in}}%
\pgfpathlineto{\pgfqpoint{4.715766in}{1.219639in}}%
\pgfpathlineto{\pgfqpoint{4.766967in}{1.221241in}}%
\pgfpathlineto{\pgfqpoint{4.813514in}{1.225565in}}%
\pgfpathlineto{\pgfqpoint{4.860060in}{1.232847in}}%
\pgfpathlineto{\pgfqpoint{4.906607in}{1.243318in}}%
\pgfpathlineto{\pgfqpoint{4.953153in}{1.257211in}}%
\pgfpathlineto{\pgfqpoint{4.995045in}{1.272838in}}%
\pgfpathlineto{\pgfqpoint{5.036937in}{1.291612in}}%
\pgfpathlineto{\pgfqpoint{5.078829in}{1.313715in}}%
\pgfpathlineto{\pgfqpoint{5.120721in}{1.339335in}}%
\pgfpathlineto{\pgfqpoint{5.162613in}{1.368662in}}%
\pgfpathlineto{\pgfqpoint{5.204505in}{1.401893in}}%
\pgfpathlineto{\pgfqpoint{5.246396in}{1.439225in}}%
\pgfpathlineto{\pgfqpoint{5.288288in}{1.480863in}}%
\pgfpathlineto{\pgfqpoint{5.330180in}{1.527014in}}%
\pgfpathlineto{\pgfqpoint{5.372072in}{1.577888in}}%
\pgfpathlineto{\pgfqpoint{5.400000in}{1.614534in}}%
\pgfpathlineto{\pgfqpoint{5.400000in}{1.614534in}}%
\pgfusepath{stroke}%
\end{pgfscope}%
\begin{pgfscope}%
\pgfpathrectangle{\pgfqpoint{0.750000in}{0.500000in}}{\pgfqpoint{4.650000in}{3.020000in}}%
\pgfusepath{clip}%
\pgfsetrectcap%
\pgfsetroundjoin%
\pgfsetlinewidth{1.505625pt}%
\definecolor{currentstroke}{rgb}{0.839216,0.152941,0.156863}%
\pgfsetstrokecolor{currentstroke}%
\pgfsetdash{}{0pt}%
\pgfpathmoveto{\pgfqpoint{0.857071in}{0.486111in}}%
\pgfpathlineto{\pgfqpoint{0.884985in}{0.616914in}}%
\pgfpathlineto{\pgfqpoint{0.912913in}{0.732071in}}%
\pgfpathlineto{\pgfqpoint{0.940841in}{0.832764in}}%
\pgfpathlineto{\pgfqpoint{0.968769in}{0.920181in}}%
\pgfpathlineto{\pgfqpoint{0.992042in}{0.983703in}}%
\pgfpathlineto{\pgfqpoint{1.015315in}{1.039409in}}%
\pgfpathlineto{\pgfqpoint{1.038589in}{1.087896in}}%
\pgfpathlineto{\pgfqpoint{1.061862in}{1.129729in}}%
\pgfpathlineto{\pgfqpoint{1.085135in}{1.165452in}}%
\pgfpathlineto{\pgfqpoint{1.108408in}{1.195579in}}%
\pgfpathlineto{\pgfqpoint{1.131682in}{1.220602in}}%
\pgfpathlineto{\pgfqpoint{1.154955in}{1.240989in}}%
\pgfpathlineto{\pgfqpoint{1.178228in}{1.257184in}}%
\pgfpathlineto{\pgfqpoint{1.201502in}{1.269609in}}%
\pgfpathlineto{\pgfqpoint{1.224775in}{1.278662in}}%
\pgfpathlineto{\pgfqpoint{1.248048in}{1.284721in}}%
\pgfpathlineto{\pgfqpoint{1.275976in}{1.288542in}}%
\pgfpathlineto{\pgfqpoint{1.303904in}{1.289144in}}%
\pgfpathlineto{\pgfqpoint{1.336486in}{1.286493in}}%
\pgfpathlineto{\pgfqpoint{1.373724in}{1.280001in}}%
\pgfpathlineto{\pgfqpoint{1.420270in}{1.268236in}}%
\pgfpathlineto{\pgfqpoint{1.490090in}{1.246679in}}%
\pgfpathlineto{\pgfqpoint{1.583183in}{1.218317in}}%
\pgfpathlineto{\pgfqpoint{1.634384in}{1.205898in}}%
\pgfpathlineto{\pgfqpoint{1.680931in}{1.197798in}}%
\pgfpathlineto{\pgfqpoint{1.722823in}{1.193644in}}%
\pgfpathlineto{\pgfqpoint{1.764715in}{1.192812in}}%
\pgfpathlineto{\pgfqpoint{1.801952in}{1.195057in}}%
\pgfpathlineto{\pgfqpoint{1.839189in}{1.200221in}}%
\pgfpathlineto{\pgfqpoint{1.876426in}{1.208357in}}%
\pgfpathlineto{\pgfqpoint{1.913664in}{1.219471in}}%
\pgfpathlineto{\pgfqpoint{1.950901in}{1.233522in}}%
\pgfpathlineto{\pgfqpoint{1.992793in}{1.252741in}}%
\pgfpathlineto{\pgfqpoint{2.034685in}{1.275408in}}%
\pgfpathlineto{\pgfqpoint{2.081231in}{1.304374in}}%
\pgfpathlineto{\pgfqpoint{2.127778in}{1.336964in}}%
\pgfpathlineto{\pgfqpoint{2.178979in}{1.376500in}}%
\pgfpathlineto{\pgfqpoint{2.239489in}{1.427371in}}%
\pgfpathlineto{\pgfqpoint{2.309309in}{1.490256in}}%
\pgfpathlineto{\pgfqpoint{2.421021in}{1.595788in}}%
\pgfpathlineto{\pgfqpoint{2.542042in}{1.709107in}}%
\pgfpathlineto{\pgfqpoint{2.611862in}{1.770401in}}%
\pgfpathlineto{\pgfqpoint{2.672372in}{1.819455in}}%
\pgfpathlineto{\pgfqpoint{2.723574in}{1.857230in}}%
\pgfpathlineto{\pgfqpoint{2.770120in}{1.888114in}}%
\pgfpathlineto{\pgfqpoint{2.816667in}{1.915326in}}%
\pgfpathlineto{\pgfqpoint{2.858559in}{1.936414in}}%
\pgfpathlineto{\pgfqpoint{2.900450in}{1.954068in}}%
\pgfpathlineto{\pgfqpoint{2.942342in}{1.968124in}}%
\pgfpathlineto{\pgfqpoint{2.979580in}{1.977490in}}%
\pgfpathlineto{\pgfqpoint{3.016817in}{1.983837in}}%
\pgfpathlineto{\pgfqpoint{3.054054in}{1.987120in}}%
\pgfpathlineto{\pgfqpoint{3.091291in}{1.987313in}}%
\pgfpathlineto{\pgfqpoint{3.128529in}{1.984416in}}%
\pgfpathlineto{\pgfqpoint{3.165766in}{1.978449in}}%
\pgfpathlineto{\pgfqpoint{3.203003in}{1.969458in}}%
\pgfpathlineto{\pgfqpoint{3.240240in}{1.957508in}}%
\pgfpathlineto{\pgfqpoint{3.282132in}{1.940640in}}%
\pgfpathlineto{\pgfqpoint{3.324024in}{1.920299in}}%
\pgfpathlineto{\pgfqpoint{3.365916in}{1.896678in}}%
\pgfpathlineto{\pgfqpoint{3.412462in}{1.866860in}}%
\pgfpathlineto{\pgfqpoint{3.463664in}{1.830124in}}%
\pgfpathlineto{\pgfqpoint{3.519520in}{1.785950in}}%
\pgfpathlineto{\pgfqpoint{3.580030in}{1.734132in}}%
\pgfpathlineto{\pgfqpoint{3.654505in}{1.666234in}}%
\pgfpathlineto{\pgfqpoint{3.929129in}{1.411300in}}%
\pgfpathlineto{\pgfqpoint{3.989640in}{1.361712in}}%
\pgfpathlineto{\pgfqpoint{4.040841in}{1.323518in}}%
\pgfpathlineto{\pgfqpoint{4.087387in}{1.292331in}}%
\pgfpathlineto{\pgfqpoint{4.133934in}{1.264921in}}%
\pgfpathlineto{\pgfqpoint{4.175826in}{1.243762in}}%
\pgfpathlineto{\pgfqpoint{4.217718in}{1.226133in}}%
\pgfpathlineto{\pgfqpoint{4.254955in}{1.213543in}}%
\pgfpathlineto{\pgfqpoint{4.292192in}{1.203916in}}%
\pgfpathlineto{\pgfqpoint{4.329429in}{1.197269in}}%
\pgfpathlineto{\pgfqpoint{4.366667in}{1.193575in}}%
\pgfpathlineto{\pgfqpoint{4.403904in}{1.192754in}}%
\pgfpathlineto{\pgfqpoint{4.445796in}{1.195097in}}%
\pgfpathlineto{\pgfqpoint{4.487688in}{1.200627in}}%
\pgfpathlineto{\pgfqpoint{4.534234in}{1.210036in}}%
\pgfpathlineto{\pgfqpoint{4.585435in}{1.223550in}}%
\pgfpathlineto{\pgfqpoint{4.664565in}{1.248174in}}%
\pgfpathlineto{\pgfqpoint{4.743694in}{1.272091in}}%
\pgfpathlineto{\pgfqpoint{4.790240in}{1.282804in}}%
\pgfpathlineto{\pgfqpoint{4.827477in}{1.288024in}}%
\pgfpathlineto{\pgfqpoint{4.860060in}{1.289211in}}%
\pgfpathlineto{\pgfqpoint{4.887988in}{1.287069in}}%
\pgfpathlineto{\pgfqpoint{4.915916in}{1.281424in}}%
\pgfpathlineto{\pgfqpoint{4.939189in}{1.273613in}}%
\pgfpathlineto{\pgfqpoint{4.962462in}{1.262584in}}%
\pgfpathlineto{\pgfqpoint{4.985736in}{1.247946in}}%
\pgfpathlineto{\pgfqpoint{5.009009in}{1.229288in}}%
\pgfpathlineto{\pgfqpoint{5.032282in}{1.206174in}}%
\pgfpathlineto{\pgfqpoint{5.055556in}{1.178146in}}%
\pgfpathlineto{\pgfqpoint{5.078829in}{1.144722in}}%
\pgfpathlineto{\pgfqpoint{5.102102in}{1.105397in}}%
\pgfpathlineto{\pgfqpoint{5.125375in}{1.059638in}}%
\pgfpathlineto{\pgfqpoint{5.148649in}{1.006889in}}%
\pgfpathlineto{\pgfqpoint{5.171922in}{0.946567in}}%
\pgfpathlineto{\pgfqpoint{5.195195in}{0.878061in}}%
\pgfpathlineto{\pgfqpoint{5.218468in}{0.800732in}}%
\pgfpathlineto{\pgfqpoint{5.241742in}{0.713915in}}%
\pgfpathlineto{\pgfqpoint{5.269670in}{0.596228in}}%
\pgfpathlineto{\pgfqpoint{5.292929in}{0.486111in}}%
\pgfpathlineto{\pgfqpoint{5.292929in}{0.486111in}}%
\pgfusepath{stroke}%
\end{pgfscope}%
\begin{pgfscope}%
\pgfpathrectangle{\pgfqpoint{0.750000in}{0.500000in}}{\pgfqpoint{4.650000in}{3.020000in}}%
\pgfusepath{clip}%
\pgfsetrectcap%
\pgfsetroundjoin%
\pgfsetlinewidth{1.505625pt}%
\definecolor{currentstroke}{rgb}{0.580392,0.403922,0.741176}%
\pgfsetstrokecolor{currentstroke}%
\pgfsetdash{}{0pt}%
\pgfpathmoveto{\pgfqpoint{0.837472in}{3.533889in}}%
\pgfpathlineto{\pgfqpoint{0.857057in}{3.196135in}}%
\pgfpathlineto{\pgfqpoint{0.880330in}{2.845582in}}%
\pgfpathlineto{\pgfqpoint{0.903604in}{2.544547in}}%
\pgfpathlineto{\pgfqpoint{0.926877in}{2.287738in}}%
\pgfpathlineto{\pgfqpoint{0.950150in}{2.070281in}}%
\pgfpathlineto{\pgfqpoint{0.968769in}{1.921623in}}%
\pgfpathlineto{\pgfqpoint{0.987387in}{1.793149in}}%
\pgfpathlineto{\pgfqpoint{1.006006in}{1.682877in}}%
\pgfpathlineto{\pgfqpoint{1.024625in}{1.588966in}}%
\pgfpathlineto{\pgfqpoint{1.043243in}{1.509702in}}%
\pgfpathlineto{\pgfqpoint{1.061862in}{1.443498in}}%
\pgfpathlineto{\pgfqpoint{1.080480in}{1.388885in}}%
\pgfpathlineto{\pgfqpoint{1.099099in}{1.344504in}}%
\pgfpathlineto{\pgfqpoint{1.117718in}{1.309106in}}%
\pgfpathlineto{\pgfqpoint{1.136336in}{1.281541in}}%
\pgfpathlineto{\pgfqpoint{1.154955in}{1.260755in}}%
\pgfpathlineto{\pgfqpoint{1.173574in}{1.245784in}}%
\pgfpathlineto{\pgfqpoint{1.192192in}{1.235749in}}%
\pgfpathlineto{\pgfqpoint{1.210811in}{1.229853in}}%
\pgfpathlineto{\pgfqpoint{1.229429in}{1.227374in}}%
\pgfpathlineto{\pgfqpoint{1.252703in}{1.228095in}}%
\pgfpathlineto{\pgfqpoint{1.275976in}{1.232020in}}%
\pgfpathlineto{\pgfqpoint{1.308559in}{1.241069in}}%
\pgfpathlineto{\pgfqpoint{1.369069in}{1.262302in}}%
\pgfpathlineto{\pgfqpoint{1.420270in}{1.278762in}}%
\pgfpathlineto{\pgfqpoint{1.457508in}{1.287538in}}%
\pgfpathlineto{\pgfqpoint{1.490090in}{1.292283in}}%
\pgfpathlineto{\pgfqpoint{1.522673in}{1.294029in}}%
\pgfpathlineto{\pgfqpoint{1.555255in}{1.292745in}}%
\pgfpathlineto{\pgfqpoint{1.587838in}{1.288571in}}%
\pgfpathlineto{\pgfqpoint{1.625075in}{1.280624in}}%
\pgfpathlineto{\pgfqpoint{1.666967in}{1.268353in}}%
\pgfpathlineto{\pgfqpoint{1.722823in}{1.248270in}}%
\pgfpathlineto{\pgfqpoint{1.848498in}{1.201483in}}%
\pgfpathlineto{\pgfqpoint{1.890390in}{1.189726in}}%
\pgfpathlineto{\pgfqpoint{1.927628in}{1.182419in}}%
\pgfpathlineto{\pgfqpoint{1.960210in}{1.179003in}}%
\pgfpathlineto{\pgfqpoint{1.992793in}{1.178774in}}%
\pgfpathlineto{\pgfqpoint{2.020721in}{1.181364in}}%
\pgfpathlineto{\pgfqpoint{2.048649in}{1.186706in}}%
\pgfpathlineto{\pgfqpoint{2.076577in}{1.194932in}}%
\pgfpathlineto{\pgfqpoint{2.104505in}{1.206140in}}%
\pgfpathlineto{\pgfqpoint{2.132432in}{1.220389in}}%
\pgfpathlineto{\pgfqpoint{2.160360in}{1.237703in}}%
\pgfpathlineto{\pgfqpoint{2.192943in}{1.261758in}}%
\pgfpathlineto{\pgfqpoint{2.225526in}{1.289888in}}%
\pgfpathlineto{\pgfqpoint{2.258108in}{1.321951in}}%
\pgfpathlineto{\pgfqpoint{2.295345in}{1.363153in}}%
\pgfpathlineto{\pgfqpoint{2.332583in}{1.408845in}}%
\pgfpathlineto{\pgfqpoint{2.374474in}{1.465027in}}%
\pgfpathlineto{\pgfqpoint{2.421021in}{1.532465in}}%
\pgfpathlineto{\pgfqpoint{2.481532in}{1.626065in}}%
\pgfpathlineto{\pgfqpoint{2.583934in}{1.791814in}}%
\pgfpathlineto{\pgfqpoint{2.667718in}{1.925017in}}%
\pgfpathlineto{\pgfqpoint{2.723574in}{2.008045in}}%
\pgfpathlineto{\pgfqpoint{2.770120in}{2.071666in}}%
\pgfpathlineto{\pgfqpoint{2.812012in}{2.123467in}}%
\pgfpathlineto{\pgfqpoint{2.849249in}{2.164462in}}%
\pgfpathlineto{\pgfqpoint{2.881832in}{2.196000in}}%
\pgfpathlineto{\pgfqpoint{2.914414in}{2.223167in}}%
\pgfpathlineto{\pgfqpoint{2.942342in}{2.242771in}}%
\pgfpathlineto{\pgfqpoint{2.970270in}{2.258826in}}%
\pgfpathlineto{\pgfqpoint{2.998198in}{2.271215in}}%
\pgfpathlineto{\pgfqpoint{3.026126in}{2.279849in}}%
\pgfpathlineto{\pgfqpoint{3.049399in}{2.284131in}}%
\pgfpathlineto{\pgfqpoint{3.072673in}{2.285739in}}%
\pgfpathlineto{\pgfqpoint{3.095946in}{2.284666in}}%
\pgfpathlineto{\pgfqpoint{3.119219in}{2.280919in}}%
\pgfpathlineto{\pgfqpoint{3.142492in}{2.274514in}}%
\pgfpathlineto{\pgfqpoint{3.170420in}{2.263368in}}%
\pgfpathlineto{\pgfqpoint{3.198348in}{2.248523in}}%
\pgfpathlineto{\pgfqpoint{3.226276in}{2.230088in}}%
\pgfpathlineto{\pgfqpoint{3.254204in}{2.208195in}}%
\pgfpathlineto{\pgfqpoint{3.286787in}{2.178495in}}%
\pgfpathlineto{\pgfqpoint{3.319369in}{2.144596in}}%
\pgfpathlineto{\pgfqpoint{3.356607in}{2.101143in}}%
\pgfpathlineto{\pgfqpoint{3.398498in}{2.046920in}}%
\pgfpathlineto{\pgfqpoint{3.445045in}{1.981079in}}%
\pgfpathlineto{\pgfqpoint{3.500901in}{1.896100in}}%
\pgfpathlineto{\pgfqpoint{3.580030in}{1.769108in}}%
\pgfpathlineto{\pgfqpoint{3.701051in}{1.574979in}}%
\pgfpathlineto{\pgfqpoint{3.756907in}{1.491430in}}%
\pgfpathlineto{\pgfqpoint{3.803453in}{1.427043in}}%
\pgfpathlineto{\pgfqpoint{3.845345in}{1.374172in}}%
\pgfpathlineto{\pgfqpoint{3.882583in}{1.331808in}}%
\pgfpathlineto{\pgfqpoint{3.919820in}{1.294232in}}%
\pgfpathlineto{\pgfqpoint{3.952402in}{1.265530in}}%
\pgfpathlineto{\pgfqpoint{3.984985in}{1.240887in}}%
\pgfpathlineto{\pgfqpoint{4.017568in}{1.220389in}}%
\pgfpathlineto{\pgfqpoint{4.045495in}{1.206140in}}%
\pgfpathlineto{\pgfqpoint{4.073423in}{1.194932in}}%
\pgfpathlineto{\pgfqpoint{4.101351in}{1.186706in}}%
\pgfpathlineto{\pgfqpoint{4.129279in}{1.181364in}}%
\pgfpathlineto{\pgfqpoint{4.157207in}{1.178774in}}%
\pgfpathlineto{\pgfqpoint{4.189790in}{1.179003in}}%
\pgfpathlineto{\pgfqpoint{4.222372in}{1.182419in}}%
\pgfpathlineto{\pgfqpoint{4.259610in}{1.189726in}}%
\pgfpathlineto{\pgfqpoint{4.301502in}{1.201483in}}%
\pgfpathlineto{\pgfqpoint{4.352703in}{1.219416in}}%
\pgfpathlineto{\pgfqpoint{4.501652in}{1.274184in}}%
\pgfpathlineto{\pgfqpoint{4.543544in}{1.284990in}}%
\pgfpathlineto{\pgfqpoint{4.580781in}{1.291297in}}%
\pgfpathlineto{\pgfqpoint{4.613363in}{1.293846in}}%
\pgfpathlineto{\pgfqpoint{4.645946in}{1.293405in}}%
\pgfpathlineto{\pgfqpoint{4.678529in}{1.289929in}}%
\pgfpathlineto{\pgfqpoint{4.711111in}{1.283563in}}%
\pgfpathlineto{\pgfqpoint{4.748348in}{1.273243in}}%
\pgfpathlineto{\pgfqpoint{4.804204in}{1.253963in}}%
\pgfpathlineto{\pgfqpoint{4.860060in}{1.235501in}}%
\pgfpathlineto{\pgfqpoint{4.887988in}{1.229338in}}%
\pgfpathlineto{\pgfqpoint{4.911261in}{1.227210in}}%
\pgfpathlineto{\pgfqpoint{4.929880in}{1.228229in}}%
\pgfpathlineto{\pgfqpoint{4.948498in}{1.232330in}}%
\pgfpathlineto{\pgfqpoint{4.967117in}{1.240201in}}%
\pgfpathlineto{\pgfqpoint{4.981081in}{1.249031in}}%
\pgfpathlineto{\pgfqpoint{4.995045in}{1.260755in}}%
\pgfpathlineto{\pgfqpoint{5.009009in}{1.275749in}}%
\pgfpathlineto{\pgfqpoint{5.022973in}{1.294413in}}%
\pgfpathlineto{\pgfqpoint{5.036937in}{1.317178in}}%
\pgfpathlineto{\pgfqpoint{5.055556in}{1.354710in}}%
\pgfpathlineto{\pgfqpoint{5.074174in}{1.401527in}}%
\pgfpathlineto{\pgfqpoint{5.092793in}{1.458907in}}%
\pgfpathlineto{\pgfqpoint{5.111411in}{1.528233in}}%
\pgfpathlineto{\pgfqpoint{5.130030in}{1.611006in}}%
\pgfpathlineto{\pgfqpoint{5.148649in}{1.708841in}}%
\pgfpathlineto{\pgfqpoint{5.167267in}{1.823486in}}%
\pgfpathlineto{\pgfqpoint{5.185886in}{1.956815in}}%
\pgfpathlineto{\pgfqpoint{5.204505in}{2.110845in}}%
\pgfpathlineto{\pgfqpoint{5.223123in}{2.287738in}}%
\pgfpathlineto{\pgfqpoint{5.241742in}{2.489808in}}%
\pgfpathlineto{\pgfqpoint{5.265015in}{2.781589in}}%
\pgfpathlineto{\pgfqpoint{5.288288in}{3.121795in}}%
\pgfpathlineto{\pgfqpoint{5.311562in}{3.516064in}}%
\pgfpathlineto{\pgfqpoint{5.312528in}{3.533889in}}%
\pgfpathlineto{\pgfqpoint{5.312528in}{3.533889in}}%
\pgfusepath{stroke}%
\end{pgfscope}%
\begin{pgfscope}%
\pgfpathrectangle{\pgfqpoint{0.750000in}{0.500000in}}{\pgfqpoint{4.650000in}{3.020000in}}%
\pgfusepath{clip}%
\pgfsetrectcap%
\pgfsetroundjoin%
\pgfsetlinewidth{1.505625pt}%
\definecolor{currentstroke}{rgb}{0.549020,0.337255,0.294118}%
\pgfsetstrokecolor{currentstroke}%
\pgfsetdash{}{0pt}%
\pgfpathmoveto{\pgfqpoint{0.750000in}{1.255000in}}%
\pgfpathlineto{\pgfqpoint{2.057958in}{1.256539in}}%
\pgfpathlineto{\pgfqpoint{2.146396in}{1.259841in}}%
\pgfpathlineto{\pgfqpoint{2.206907in}{1.264986in}}%
\pgfpathlineto{\pgfqpoint{2.253453in}{1.271860in}}%
\pgfpathlineto{\pgfqpoint{2.290691in}{1.280108in}}%
\pgfpathlineto{\pgfqpoint{2.323273in}{1.290040in}}%
\pgfpathlineto{\pgfqpoint{2.355856in}{1.303215in}}%
\pgfpathlineto{\pgfqpoint{2.383784in}{1.317676in}}%
\pgfpathlineto{\pgfqpoint{2.411712in}{1.335633in}}%
\pgfpathlineto{\pgfqpoint{2.439640in}{1.357662in}}%
\pgfpathlineto{\pgfqpoint{2.467568in}{1.384360in}}%
\pgfpathlineto{\pgfqpoint{2.490841in}{1.410600in}}%
\pgfpathlineto{\pgfqpoint{2.514114in}{1.440817in}}%
\pgfpathlineto{\pgfqpoint{2.542042in}{1.482741in}}%
\pgfpathlineto{\pgfqpoint{2.569970in}{1.531239in}}%
\pgfpathlineto{\pgfqpoint{2.597898in}{1.586602in}}%
\pgfpathlineto{\pgfqpoint{2.625826in}{1.648946in}}%
\pgfpathlineto{\pgfqpoint{2.658408in}{1.730363in}}%
\pgfpathlineto{\pgfqpoint{2.690991in}{1.820552in}}%
\pgfpathlineto{\pgfqpoint{2.728228in}{1.932919in}}%
\pgfpathlineto{\pgfqpoint{2.779429in}{2.098920in}}%
\pgfpathlineto{\pgfqpoint{2.872523in}{2.404216in}}%
\pgfpathlineto{\pgfqpoint{2.905105in}{2.500929in}}%
\pgfpathlineto{\pgfqpoint{2.933033in}{2.575335in}}%
\pgfpathlineto{\pgfqpoint{2.956306in}{2.629770in}}%
\pgfpathlineto{\pgfqpoint{2.979580in}{2.676159in}}%
\pgfpathlineto{\pgfqpoint{2.998198in}{2.706833in}}%
\pgfpathlineto{\pgfqpoint{3.016817in}{2.731338in}}%
\pgfpathlineto{\pgfqpoint{3.030781in}{2.745464in}}%
\pgfpathlineto{\pgfqpoint{3.044745in}{2.755823in}}%
\pgfpathlineto{\pgfqpoint{3.058709in}{2.762333in}}%
\pgfpathlineto{\pgfqpoint{3.072673in}{2.764946in}}%
\pgfpathlineto{\pgfqpoint{3.086637in}{2.763639in}}%
\pgfpathlineto{\pgfqpoint{3.100601in}{2.758424in}}%
\pgfpathlineto{\pgfqpoint{3.114565in}{2.749340in}}%
\pgfpathlineto{\pgfqpoint{3.128529in}{2.736459in}}%
\pgfpathlineto{\pgfqpoint{3.142492in}{2.719880in}}%
\pgfpathlineto{\pgfqpoint{3.161111in}{2.692243in}}%
\pgfpathlineto{\pgfqpoint{3.179730in}{2.658633in}}%
\pgfpathlineto{\pgfqpoint{3.203003in}{2.608903in}}%
\pgfpathlineto{\pgfqpoint{3.226276in}{2.551548in}}%
\pgfpathlineto{\pgfqpoint{3.254204in}{2.474253in}}%
\pgfpathlineto{\pgfqpoint{3.286787in}{2.375076in}}%
\pgfpathlineto{\pgfqpoint{3.337988in}{2.207662in}}%
\pgfpathlineto{\pgfqpoint{3.421772in}{1.932919in}}%
\pgfpathlineto{\pgfqpoint{3.463664in}{1.807167in}}%
\pgfpathlineto{\pgfqpoint{3.496246in}{1.718176in}}%
\pgfpathlineto{\pgfqpoint{3.528829in}{1.638072in}}%
\pgfpathlineto{\pgfqpoint{3.556757in}{1.576890in}}%
\pgfpathlineto{\pgfqpoint{3.584685in}{1.522685in}}%
\pgfpathlineto{\pgfqpoint{3.612613in}{1.475307in}}%
\pgfpathlineto{\pgfqpoint{3.640541in}{1.434440in}}%
\pgfpathlineto{\pgfqpoint{3.668468in}{1.399644in}}%
\pgfpathlineto{\pgfqpoint{3.696396in}{1.370390in}}%
\pgfpathlineto{\pgfqpoint{3.724324in}{1.346102in}}%
\pgfpathlineto{\pgfqpoint{3.752252in}{1.326182in}}%
\pgfpathlineto{\pgfqpoint{3.780180in}{1.310043in}}%
\pgfpathlineto{\pgfqpoint{3.808108in}{1.297124in}}%
\pgfpathlineto{\pgfqpoint{3.840691in}{1.285429in}}%
\pgfpathlineto{\pgfqpoint{3.877928in}{1.275622in}}%
\pgfpathlineto{\pgfqpoint{3.919820in}{1.268023in}}%
\pgfpathlineto{\pgfqpoint{3.971021in}{1.262193in}}%
\pgfpathlineto{\pgfqpoint{4.036186in}{1.258213in}}%
\pgfpathlineto{\pgfqpoint{4.129279in}{1.255921in}}%
\pgfpathlineto{\pgfqpoint{4.320120in}{1.255050in}}%
\pgfpathlineto{\pgfqpoint{5.400000in}{1.255000in}}%
\pgfpathlineto{\pgfqpoint{5.400000in}{1.255000in}}%
\pgfusepath{stroke}%
\end{pgfscope}%
\begin{pgfscope}%
\pgfsetrectcap%
\pgfsetmiterjoin%
\pgfsetlinewidth{0.803000pt}%
\definecolor{currentstroke}{rgb}{0.000000,0.000000,0.000000}%
\pgfsetstrokecolor{currentstroke}%
\pgfsetdash{}{0pt}%
\pgfpathmoveto{\pgfqpoint{0.750000in}{0.500000in}}%
\pgfpathlineto{\pgfqpoint{0.750000in}{3.520000in}}%
\pgfusepath{stroke}%
\end{pgfscope}%
\begin{pgfscope}%
\pgfsetrectcap%
\pgfsetmiterjoin%
\pgfsetlinewidth{0.803000pt}%
\definecolor{currentstroke}{rgb}{0.000000,0.000000,0.000000}%
\pgfsetstrokecolor{currentstroke}%
\pgfsetdash{}{0pt}%
\pgfpathmoveto{\pgfqpoint{5.400000in}{0.500000in}}%
\pgfpathlineto{\pgfqpoint{5.400000in}{3.520000in}}%
\pgfusepath{stroke}%
\end{pgfscope}%
\begin{pgfscope}%
\pgfsetrectcap%
\pgfsetmiterjoin%
\pgfsetlinewidth{0.803000pt}%
\definecolor{currentstroke}{rgb}{0.000000,0.000000,0.000000}%
\pgfsetstrokecolor{currentstroke}%
\pgfsetdash{}{0pt}%
\pgfpathmoveto{\pgfqpoint{0.750000in}{0.500000in}}%
\pgfpathlineto{\pgfqpoint{5.400000in}{0.500000in}}%
\pgfusepath{stroke}%
\end{pgfscope}%
\begin{pgfscope}%
\pgfsetrectcap%
\pgfsetmiterjoin%
\pgfsetlinewidth{0.803000pt}%
\definecolor{currentstroke}{rgb}{0.000000,0.000000,0.000000}%
\pgfsetstrokecolor{currentstroke}%
\pgfsetdash{}{0pt}%
\pgfpathmoveto{\pgfqpoint{0.750000in}{3.520000in}}%
\pgfpathlineto{\pgfqpoint{5.400000in}{3.520000in}}%
\pgfusepath{stroke}%
\end{pgfscope}%
\begin{pgfscope}%
\pgfsetbuttcap%
\pgfsetmiterjoin%
\definecolor{currentfill}{rgb}{1.000000,1.000000,1.000000}%
\pgfsetfillcolor{currentfill}%
\pgfsetfillopacity{0.800000}%
\pgfsetlinewidth{1.003750pt}%
\definecolor{currentstroke}{rgb}{0.800000,0.800000,0.800000}%
\pgfsetstrokecolor{currentstroke}%
\pgfsetstrokeopacity{0.800000}%
\pgfsetdash{}{0pt}%
\pgfpathmoveto{\pgfqpoint{4.520339in}{2.247161in}}%
\pgfpathlineto{\pgfqpoint{5.302778in}{2.247161in}}%
\pgfpathquadraticcurveto{\pgfqpoint{5.330556in}{2.247161in}}{\pgfqpoint{5.330556in}{2.274939in}}%
\pgfpathlineto{\pgfqpoint{5.330556in}{3.422778in}}%
\pgfpathquadraticcurveto{\pgfqpoint{5.330556in}{3.450556in}}{\pgfqpoint{5.302778in}{3.450556in}}%
\pgfpathlineto{\pgfqpoint{4.520339in}{3.450556in}}%
\pgfpathquadraticcurveto{\pgfqpoint{4.492561in}{3.450556in}}{\pgfqpoint{4.492561in}{3.422778in}}%
\pgfpathlineto{\pgfqpoint{4.492561in}{2.274939in}}%
\pgfpathquadraticcurveto{\pgfqpoint{4.492561in}{2.247161in}}{\pgfqpoint{4.520339in}{2.247161in}}%
\pgfpathclose%
\pgfusepath{stroke,fill}%
\end{pgfscope}%
\begin{pgfscope}%
\pgfsetrectcap%
\pgfsetroundjoin%
\pgfsetlinewidth{1.505625pt}%
\definecolor{currentstroke}{rgb}{0.121569,0.466667,0.705882}%
\pgfsetstrokecolor{currentstroke}%
\pgfsetdash{}{0pt}%
\pgfpathmoveto{\pgfqpoint{4.548117in}{3.346389in}}%
\pgfpathlineto{\pgfqpoint{4.825895in}{3.346389in}}%
\pgfusepath{stroke}%
\end{pgfscope}%
\begin{pgfscope}%
\pgftext[x=4.937006in,y=3.297778in,left,base]{\rmfamily\fontsize{10.000000}{12.000000}\selectfont \(\displaystyle  n = 1 \)}%
\end{pgfscope}%
\begin{pgfscope}%
\pgfsetrectcap%
\pgfsetroundjoin%
\pgfsetlinewidth{1.505625pt}%
\definecolor{currentstroke}{rgb}{1.000000,0.498039,0.054902}%
\pgfsetstrokecolor{currentstroke}%
\pgfsetdash{}{0pt}%
\pgfpathmoveto{\pgfqpoint{4.548117in}{3.152778in}}%
\pgfpathlineto{\pgfqpoint{4.825895in}{3.152778in}}%
\pgfusepath{stroke}%
\end{pgfscope}%
\begin{pgfscope}%
\pgftext[x=4.937006in,y=3.104167in,left,base]{\rmfamily\fontsize{10.000000}{12.000000}\selectfont \(\displaystyle  n = 3 \)}%
\end{pgfscope}%
\begin{pgfscope}%
\pgfsetrectcap%
\pgfsetroundjoin%
\pgfsetlinewidth{1.505625pt}%
\definecolor{currentstroke}{rgb}{0.172549,0.627451,0.172549}%
\pgfsetstrokecolor{currentstroke}%
\pgfsetdash{}{0pt}%
\pgfpathmoveto{\pgfqpoint{4.548117in}{2.959167in}}%
\pgfpathlineto{\pgfqpoint{4.825895in}{2.959167in}}%
\pgfusepath{stroke}%
\end{pgfscope}%
\begin{pgfscope}%
\pgftext[x=4.937006in,y=2.910556in,left,base]{\rmfamily\fontsize{10.000000}{12.000000}\selectfont \(\displaystyle  n = 5 \)}%
\end{pgfscope}%
\begin{pgfscope}%
\pgfsetrectcap%
\pgfsetroundjoin%
\pgfsetlinewidth{1.505625pt}%
\definecolor{currentstroke}{rgb}{0.839216,0.152941,0.156863}%
\pgfsetstrokecolor{currentstroke}%
\pgfsetdash{}{0pt}%
\pgfpathmoveto{\pgfqpoint{4.548117in}{2.765556in}}%
\pgfpathlineto{\pgfqpoint{4.825895in}{2.765556in}}%
\pgfusepath{stroke}%
\end{pgfscope}%
\begin{pgfscope}%
\pgftext[x=4.937006in,y=2.716945in,left,base]{\rmfamily\fontsize{10.000000}{12.000000}\selectfont \(\displaystyle  n = 7 \)}%
\end{pgfscope}%
\begin{pgfscope}%
\pgfsetrectcap%
\pgfsetroundjoin%
\pgfsetlinewidth{1.505625pt}%
\definecolor{currentstroke}{rgb}{0.580392,0.403922,0.741176}%
\pgfsetstrokecolor{currentstroke}%
\pgfsetdash{}{0pt}%
\pgfpathmoveto{\pgfqpoint{4.548117in}{2.571945in}}%
\pgfpathlineto{\pgfqpoint{4.825895in}{2.571945in}}%
\pgfusepath{stroke}%
\end{pgfscope}%
\begin{pgfscope}%
\pgftext[x=4.937006in,y=2.523334in,left,base]{\rmfamily\fontsize{10.000000}{12.000000}\selectfont \(\displaystyle  n = 9 \)}%
\end{pgfscope}%
\begin{pgfscope}%
\pgfsetrectcap%
\pgfsetroundjoin%
\pgfsetlinewidth{1.505625pt}%
\definecolor{currentstroke}{rgb}{0.549020,0.337255,0.294118}%
\pgfsetstrokecolor{currentstroke}%
\pgfsetdash{}{0pt}%
\pgfpathmoveto{\pgfqpoint{4.548117in}{2.378334in}}%
\pgfpathlineto{\pgfqpoint{4.825895in}{2.378334in}}%
\pgfusepath{stroke}%
\end{pgfscope}%
\begin{pgfscope}%
\pgftext[x=4.937006in,y=2.329723in,left,base]{\rmfamily\fontsize{10.000000}{12.000000}\selectfont \(\displaystyle f_2\)}%
\end{pgfscope}%
\end{pgfpicture}%
\makeatother%
\endgroup%
}
\centering \scalebox{0.8}{%% Creator: Matplotlib, PGF backend
%%
%% To include the figure in your LaTeX document, write
%%   \input{<filename>.pgf}
%%
%% Make sure the required packages are loaded in your preamble
%%   \usepackage{pgf}
%%
%% Figures using additional raster images can only be included by \input if
%% they are in the same directory as the main LaTeX file. For loading figures
%% from other directories you can use the `import` package
%%   \usepackage{import}
%% and then include the figures with
%%   \import{<path to file>}{<filename>.pgf}
%%
%% Matplotlib used the following preamble
%%   \usepackage{fontspec}
%%
\begingroup%
\makeatletter%
\begin{pgfpicture}%
\pgfpathrectangle{\pgfpointorigin}{\pgfqpoint{6.000000in}{4.000000in}}%
\pgfusepath{use as bounding box, clip}%
\begin{pgfscope}%
\pgfsetbuttcap%
\pgfsetmiterjoin%
\definecolor{currentfill}{rgb}{1.000000,1.000000,1.000000}%
\pgfsetfillcolor{currentfill}%
\pgfsetlinewidth{0.000000pt}%
\definecolor{currentstroke}{rgb}{1.000000,1.000000,1.000000}%
\pgfsetstrokecolor{currentstroke}%
\pgfsetdash{}{0pt}%
\pgfpathmoveto{\pgfqpoint{0.000000in}{0.000000in}}%
\pgfpathlineto{\pgfqpoint{6.000000in}{0.000000in}}%
\pgfpathlineto{\pgfqpoint{6.000000in}{4.000000in}}%
\pgfpathlineto{\pgfqpoint{0.000000in}{4.000000in}}%
\pgfpathclose%
\pgfusepath{fill}%
\end{pgfscope}%
\begin{pgfscope}%
\pgfsetbuttcap%
\pgfsetmiterjoin%
\definecolor{currentfill}{rgb}{1.000000,1.000000,1.000000}%
\pgfsetfillcolor{currentfill}%
\pgfsetlinewidth{0.000000pt}%
\definecolor{currentstroke}{rgb}{0.000000,0.000000,0.000000}%
\pgfsetstrokecolor{currentstroke}%
\pgfsetstrokeopacity{0.000000}%
\pgfsetdash{}{0pt}%
\pgfpathmoveto{\pgfqpoint{0.750000in}{0.500000in}}%
\pgfpathlineto{\pgfqpoint{5.400000in}{0.500000in}}%
\pgfpathlineto{\pgfqpoint{5.400000in}{3.520000in}}%
\pgfpathlineto{\pgfqpoint{0.750000in}{3.520000in}}%
\pgfpathclose%
\pgfusepath{fill}%
\end{pgfscope}%
\begin{pgfscope}%
\pgfpathrectangle{\pgfqpoint{0.750000in}{0.500000in}}{\pgfqpoint{4.650000in}{3.020000in}}%
\pgfusepath{clip}%
\pgfsetrectcap%
\pgfsetroundjoin%
\pgfsetlinewidth{0.803000pt}%
\definecolor{currentstroke}{rgb}{0.690196,0.690196,0.690196}%
\pgfsetstrokecolor{currentstroke}%
\pgfsetdash{}{0pt}%
\pgfpathmoveto{\pgfqpoint{0.750000in}{0.500000in}}%
\pgfpathlineto{\pgfqpoint{0.750000in}{3.520000in}}%
\pgfusepath{stroke}%
\end{pgfscope}%
\begin{pgfscope}%
\pgfsetbuttcap%
\pgfsetroundjoin%
\definecolor{currentfill}{rgb}{0.000000,0.000000,0.000000}%
\pgfsetfillcolor{currentfill}%
\pgfsetlinewidth{0.803000pt}%
\definecolor{currentstroke}{rgb}{0.000000,0.000000,0.000000}%
\pgfsetstrokecolor{currentstroke}%
\pgfsetdash{}{0pt}%
\pgfsys@defobject{currentmarker}{\pgfqpoint{0.000000in}{-0.048611in}}{\pgfqpoint{0.000000in}{0.000000in}}{%
\pgfpathmoveto{\pgfqpoint{0.000000in}{0.000000in}}%
\pgfpathlineto{\pgfqpoint{0.000000in}{-0.048611in}}%
\pgfusepath{stroke,fill}%
}%
\begin{pgfscope}%
\pgfsys@transformshift{0.750000in}{0.500000in}%
\pgfsys@useobject{currentmarker}{}%
\end{pgfscope}%
\end{pgfscope}%
\begin{pgfscope}%
\pgftext[x=0.750000in,y=0.402778in,,top]{\rmfamily\fontsize{10.000000}{12.000000}\selectfont \(\displaystyle -6\)}%
\end{pgfscope}%
\begin{pgfscope}%
\pgfpathrectangle{\pgfqpoint{0.750000in}{0.500000in}}{\pgfqpoint{4.650000in}{3.020000in}}%
\pgfusepath{clip}%
\pgfsetrectcap%
\pgfsetroundjoin%
\pgfsetlinewidth{0.803000pt}%
\definecolor{currentstroke}{rgb}{0.690196,0.690196,0.690196}%
\pgfsetstrokecolor{currentstroke}%
\pgfsetdash{}{0pt}%
\pgfpathmoveto{\pgfqpoint{1.525000in}{0.500000in}}%
\pgfpathlineto{\pgfqpoint{1.525000in}{3.520000in}}%
\pgfusepath{stroke}%
\end{pgfscope}%
\begin{pgfscope}%
\pgfsetbuttcap%
\pgfsetroundjoin%
\definecolor{currentfill}{rgb}{0.000000,0.000000,0.000000}%
\pgfsetfillcolor{currentfill}%
\pgfsetlinewidth{0.803000pt}%
\definecolor{currentstroke}{rgb}{0.000000,0.000000,0.000000}%
\pgfsetstrokecolor{currentstroke}%
\pgfsetdash{}{0pt}%
\pgfsys@defobject{currentmarker}{\pgfqpoint{0.000000in}{-0.048611in}}{\pgfqpoint{0.000000in}{0.000000in}}{%
\pgfpathmoveto{\pgfqpoint{0.000000in}{0.000000in}}%
\pgfpathlineto{\pgfqpoint{0.000000in}{-0.048611in}}%
\pgfusepath{stroke,fill}%
}%
\begin{pgfscope}%
\pgfsys@transformshift{1.525000in}{0.500000in}%
\pgfsys@useobject{currentmarker}{}%
\end{pgfscope}%
\end{pgfscope}%
\begin{pgfscope}%
\pgftext[x=1.525000in,y=0.402778in,,top]{\rmfamily\fontsize{10.000000}{12.000000}\selectfont \(\displaystyle -4\)}%
\end{pgfscope}%
\begin{pgfscope}%
\pgfpathrectangle{\pgfqpoint{0.750000in}{0.500000in}}{\pgfqpoint{4.650000in}{3.020000in}}%
\pgfusepath{clip}%
\pgfsetrectcap%
\pgfsetroundjoin%
\pgfsetlinewidth{0.803000pt}%
\definecolor{currentstroke}{rgb}{0.690196,0.690196,0.690196}%
\pgfsetstrokecolor{currentstroke}%
\pgfsetdash{}{0pt}%
\pgfpathmoveto{\pgfqpoint{2.300000in}{0.500000in}}%
\pgfpathlineto{\pgfqpoint{2.300000in}{3.520000in}}%
\pgfusepath{stroke}%
\end{pgfscope}%
\begin{pgfscope}%
\pgfsetbuttcap%
\pgfsetroundjoin%
\definecolor{currentfill}{rgb}{0.000000,0.000000,0.000000}%
\pgfsetfillcolor{currentfill}%
\pgfsetlinewidth{0.803000pt}%
\definecolor{currentstroke}{rgb}{0.000000,0.000000,0.000000}%
\pgfsetstrokecolor{currentstroke}%
\pgfsetdash{}{0pt}%
\pgfsys@defobject{currentmarker}{\pgfqpoint{0.000000in}{-0.048611in}}{\pgfqpoint{0.000000in}{0.000000in}}{%
\pgfpathmoveto{\pgfqpoint{0.000000in}{0.000000in}}%
\pgfpathlineto{\pgfqpoint{0.000000in}{-0.048611in}}%
\pgfusepath{stroke,fill}%
}%
\begin{pgfscope}%
\pgfsys@transformshift{2.300000in}{0.500000in}%
\pgfsys@useobject{currentmarker}{}%
\end{pgfscope}%
\end{pgfscope}%
\begin{pgfscope}%
\pgftext[x=2.300000in,y=0.402778in,,top]{\rmfamily\fontsize{10.000000}{12.000000}\selectfont \(\displaystyle -2\)}%
\end{pgfscope}%
\begin{pgfscope}%
\pgfpathrectangle{\pgfqpoint{0.750000in}{0.500000in}}{\pgfqpoint{4.650000in}{3.020000in}}%
\pgfusepath{clip}%
\pgfsetrectcap%
\pgfsetroundjoin%
\pgfsetlinewidth{0.803000pt}%
\definecolor{currentstroke}{rgb}{0.690196,0.690196,0.690196}%
\pgfsetstrokecolor{currentstroke}%
\pgfsetdash{}{0pt}%
\pgfpathmoveto{\pgfqpoint{3.075000in}{0.500000in}}%
\pgfpathlineto{\pgfqpoint{3.075000in}{3.520000in}}%
\pgfusepath{stroke}%
\end{pgfscope}%
\begin{pgfscope}%
\pgfsetbuttcap%
\pgfsetroundjoin%
\definecolor{currentfill}{rgb}{0.000000,0.000000,0.000000}%
\pgfsetfillcolor{currentfill}%
\pgfsetlinewidth{0.803000pt}%
\definecolor{currentstroke}{rgb}{0.000000,0.000000,0.000000}%
\pgfsetstrokecolor{currentstroke}%
\pgfsetdash{}{0pt}%
\pgfsys@defobject{currentmarker}{\pgfqpoint{0.000000in}{-0.048611in}}{\pgfqpoint{0.000000in}{0.000000in}}{%
\pgfpathmoveto{\pgfqpoint{0.000000in}{0.000000in}}%
\pgfpathlineto{\pgfqpoint{0.000000in}{-0.048611in}}%
\pgfusepath{stroke,fill}%
}%
\begin{pgfscope}%
\pgfsys@transformshift{3.075000in}{0.500000in}%
\pgfsys@useobject{currentmarker}{}%
\end{pgfscope}%
\end{pgfscope}%
\begin{pgfscope}%
\pgftext[x=3.075000in,y=0.402778in,,top]{\rmfamily\fontsize{10.000000}{12.000000}\selectfont \(\displaystyle 0\)}%
\end{pgfscope}%
\begin{pgfscope}%
\pgfpathrectangle{\pgfqpoint{0.750000in}{0.500000in}}{\pgfqpoint{4.650000in}{3.020000in}}%
\pgfusepath{clip}%
\pgfsetrectcap%
\pgfsetroundjoin%
\pgfsetlinewidth{0.803000pt}%
\definecolor{currentstroke}{rgb}{0.690196,0.690196,0.690196}%
\pgfsetstrokecolor{currentstroke}%
\pgfsetdash{}{0pt}%
\pgfpathmoveto{\pgfqpoint{3.850000in}{0.500000in}}%
\pgfpathlineto{\pgfqpoint{3.850000in}{3.520000in}}%
\pgfusepath{stroke}%
\end{pgfscope}%
\begin{pgfscope}%
\pgfsetbuttcap%
\pgfsetroundjoin%
\definecolor{currentfill}{rgb}{0.000000,0.000000,0.000000}%
\pgfsetfillcolor{currentfill}%
\pgfsetlinewidth{0.803000pt}%
\definecolor{currentstroke}{rgb}{0.000000,0.000000,0.000000}%
\pgfsetstrokecolor{currentstroke}%
\pgfsetdash{}{0pt}%
\pgfsys@defobject{currentmarker}{\pgfqpoint{0.000000in}{-0.048611in}}{\pgfqpoint{0.000000in}{0.000000in}}{%
\pgfpathmoveto{\pgfqpoint{0.000000in}{0.000000in}}%
\pgfpathlineto{\pgfqpoint{0.000000in}{-0.048611in}}%
\pgfusepath{stroke,fill}%
}%
\begin{pgfscope}%
\pgfsys@transformshift{3.850000in}{0.500000in}%
\pgfsys@useobject{currentmarker}{}%
\end{pgfscope}%
\end{pgfscope}%
\begin{pgfscope}%
\pgftext[x=3.850000in,y=0.402778in,,top]{\rmfamily\fontsize{10.000000}{12.000000}\selectfont \(\displaystyle 2\)}%
\end{pgfscope}%
\begin{pgfscope}%
\pgfpathrectangle{\pgfqpoint{0.750000in}{0.500000in}}{\pgfqpoint{4.650000in}{3.020000in}}%
\pgfusepath{clip}%
\pgfsetrectcap%
\pgfsetroundjoin%
\pgfsetlinewidth{0.803000pt}%
\definecolor{currentstroke}{rgb}{0.690196,0.690196,0.690196}%
\pgfsetstrokecolor{currentstroke}%
\pgfsetdash{}{0pt}%
\pgfpathmoveto{\pgfqpoint{4.625000in}{0.500000in}}%
\pgfpathlineto{\pgfqpoint{4.625000in}{3.520000in}}%
\pgfusepath{stroke}%
\end{pgfscope}%
\begin{pgfscope}%
\pgfsetbuttcap%
\pgfsetroundjoin%
\definecolor{currentfill}{rgb}{0.000000,0.000000,0.000000}%
\pgfsetfillcolor{currentfill}%
\pgfsetlinewidth{0.803000pt}%
\definecolor{currentstroke}{rgb}{0.000000,0.000000,0.000000}%
\pgfsetstrokecolor{currentstroke}%
\pgfsetdash{}{0pt}%
\pgfsys@defobject{currentmarker}{\pgfqpoint{0.000000in}{-0.048611in}}{\pgfqpoint{0.000000in}{0.000000in}}{%
\pgfpathmoveto{\pgfqpoint{0.000000in}{0.000000in}}%
\pgfpathlineto{\pgfqpoint{0.000000in}{-0.048611in}}%
\pgfusepath{stroke,fill}%
}%
\begin{pgfscope}%
\pgfsys@transformshift{4.625000in}{0.500000in}%
\pgfsys@useobject{currentmarker}{}%
\end{pgfscope}%
\end{pgfscope}%
\begin{pgfscope}%
\pgftext[x=4.625000in,y=0.402778in,,top]{\rmfamily\fontsize{10.000000}{12.000000}\selectfont \(\displaystyle 4\)}%
\end{pgfscope}%
\begin{pgfscope}%
\pgfpathrectangle{\pgfqpoint{0.750000in}{0.500000in}}{\pgfqpoint{4.650000in}{3.020000in}}%
\pgfusepath{clip}%
\pgfsetrectcap%
\pgfsetroundjoin%
\pgfsetlinewidth{0.803000pt}%
\definecolor{currentstroke}{rgb}{0.690196,0.690196,0.690196}%
\pgfsetstrokecolor{currentstroke}%
\pgfsetdash{}{0pt}%
\pgfpathmoveto{\pgfqpoint{5.400000in}{0.500000in}}%
\pgfpathlineto{\pgfqpoint{5.400000in}{3.520000in}}%
\pgfusepath{stroke}%
\end{pgfscope}%
\begin{pgfscope}%
\pgfsetbuttcap%
\pgfsetroundjoin%
\definecolor{currentfill}{rgb}{0.000000,0.000000,0.000000}%
\pgfsetfillcolor{currentfill}%
\pgfsetlinewidth{0.803000pt}%
\definecolor{currentstroke}{rgb}{0.000000,0.000000,0.000000}%
\pgfsetstrokecolor{currentstroke}%
\pgfsetdash{}{0pt}%
\pgfsys@defobject{currentmarker}{\pgfqpoint{0.000000in}{-0.048611in}}{\pgfqpoint{0.000000in}{0.000000in}}{%
\pgfpathmoveto{\pgfqpoint{0.000000in}{0.000000in}}%
\pgfpathlineto{\pgfqpoint{0.000000in}{-0.048611in}}%
\pgfusepath{stroke,fill}%
}%
\begin{pgfscope}%
\pgfsys@transformshift{5.400000in}{0.500000in}%
\pgfsys@useobject{currentmarker}{}%
\end{pgfscope}%
\end{pgfscope}%
\begin{pgfscope}%
\pgftext[x=5.400000in,y=0.402778in,,top]{\rmfamily\fontsize{10.000000}{12.000000}\selectfont \(\displaystyle 6\)}%
\end{pgfscope}%
\begin{pgfscope}%
\pgfpathrectangle{\pgfqpoint{0.750000in}{0.500000in}}{\pgfqpoint{4.650000in}{3.020000in}}%
\pgfusepath{clip}%
\pgfsetrectcap%
\pgfsetroundjoin%
\pgfsetlinewidth{0.803000pt}%
\definecolor{currentstroke}{rgb}{0.690196,0.690196,0.690196}%
\pgfsetstrokecolor{currentstroke}%
\pgfsetdash{}{0pt}%
\pgfpathmoveto{\pgfqpoint{0.750000in}{0.500000in}}%
\pgfpathlineto{\pgfqpoint{5.400000in}{0.500000in}}%
\pgfusepath{stroke}%
\end{pgfscope}%
\begin{pgfscope}%
\pgfsetbuttcap%
\pgfsetroundjoin%
\definecolor{currentfill}{rgb}{0.000000,0.000000,0.000000}%
\pgfsetfillcolor{currentfill}%
\pgfsetlinewidth{0.803000pt}%
\definecolor{currentstroke}{rgb}{0.000000,0.000000,0.000000}%
\pgfsetstrokecolor{currentstroke}%
\pgfsetdash{}{0pt}%
\pgfsys@defobject{currentmarker}{\pgfqpoint{-0.048611in}{0.000000in}}{\pgfqpoint{0.000000in}{0.000000in}}{%
\pgfpathmoveto{\pgfqpoint{0.000000in}{0.000000in}}%
\pgfpathlineto{\pgfqpoint{-0.048611in}{0.000000in}}%
\pgfusepath{stroke,fill}%
}%
\begin{pgfscope}%
\pgfsys@transformshift{0.750000in}{0.500000in}%
\pgfsys@useobject{currentmarker}{}%
\end{pgfscope}%
\end{pgfscope}%
\begin{pgfscope}%
\pgftext[x=0.297838in,y=0.451806in,left,base]{\rmfamily\fontsize{10.000000}{12.000000}\selectfont \(\displaystyle -0.50\)}%
\end{pgfscope}%
\begin{pgfscope}%
\pgfpathrectangle{\pgfqpoint{0.750000in}{0.500000in}}{\pgfqpoint{4.650000in}{3.020000in}}%
\pgfusepath{clip}%
\pgfsetrectcap%
\pgfsetroundjoin%
\pgfsetlinewidth{0.803000pt}%
\definecolor{currentstroke}{rgb}{0.690196,0.690196,0.690196}%
\pgfsetstrokecolor{currentstroke}%
\pgfsetdash{}{0pt}%
\pgfpathmoveto{\pgfqpoint{0.750000in}{0.877500in}}%
\pgfpathlineto{\pgfqpoint{5.400000in}{0.877500in}}%
\pgfusepath{stroke}%
\end{pgfscope}%
\begin{pgfscope}%
\pgfsetbuttcap%
\pgfsetroundjoin%
\definecolor{currentfill}{rgb}{0.000000,0.000000,0.000000}%
\pgfsetfillcolor{currentfill}%
\pgfsetlinewidth{0.803000pt}%
\definecolor{currentstroke}{rgb}{0.000000,0.000000,0.000000}%
\pgfsetstrokecolor{currentstroke}%
\pgfsetdash{}{0pt}%
\pgfsys@defobject{currentmarker}{\pgfqpoint{-0.048611in}{0.000000in}}{\pgfqpoint{0.000000in}{0.000000in}}{%
\pgfpathmoveto{\pgfqpoint{0.000000in}{0.000000in}}%
\pgfpathlineto{\pgfqpoint{-0.048611in}{0.000000in}}%
\pgfusepath{stroke,fill}%
}%
\begin{pgfscope}%
\pgfsys@transformshift{0.750000in}{0.877500in}%
\pgfsys@useobject{currentmarker}{}%
\end{pgfscope}%
\end{pgfscope}%
\begin{pgfscope}%
\pgftext[x=0.297838in,y=0.829306in,left,base]{\rmfamily\fontsize{10.000000}{12.000000}\selectfont \(\displaystyle -0.25\)}%
\end{pgfscope}%
\begin{pgfscope}%
\pgfpathrectangle{\pgfqpoint{0.750000in}{0.500000in}}{\pgfqpoint{4.650000in}{3.020000in}}%
\pgfusepath{clip}%
\pgfsetrectcap%
\pgfsetroundjoin%
\pgfsetlinewidth{0.803000pt}%
\definecolor{currentstroke}{rgb}{0.690196,0.690196,0.690196}%
\pgfsetstrokecolor{currentstroke}%
\pgfsetdash{}{0pt}%
\pgfpathmoveto{\pgfqpoint{0.750000in}{1.255000in}}%
\pgfpathlineto{\pgfqpoint{5.400000in}{1.255000in}}%
\pgfusepath{stroke}%
\end{pgfscope}%
\begin{pgfscope}%
\pgfsetbuttcap%
\pgfsetroundjoin%
\definecolor{currentfill}{rgb}{0.000000,0.000000,0.000000}%
\pgfsetfillcolor{currentfill}%
\pgfsetlinewidth{0.803000pt}%
\definecolor{currentstroke}{rgb}{0.000000,0.000000,0.000000}%
\pgfsetstrokecolor{currentstroke}%
\pgfsetdash{}{0pt}%
\pgfsys@defobject{currentmarker}{\pgfqpoint{-0.048611in}{0.000000in}}{\pgfqpoint{0.000000in}{0.000000in}}{%
\pgfpathmoveto{\pgfqpoint{0.000000in}{0.000000in}}%
\pgfpathlineto{\pgfqpoint{-0.048611in}{0.000000in}}%
\pgfusepath{stroke,fill}%
}%
\begin{pgfscope}%
\pgfsys@transformshift{0.750000in}{1.255000in}%
\pgfsys@useobject{currentmarker}{}%
\end{pgfscope}%
\end{pgfscope}%
\begin{pgfscope}%
\pgftext[x=0.405863in,y=1.206806in,left,base]{\rmfamily\fontsize{10.000000}{12.000000}\selectfont \(\displaystyle 0.00\)}%
\end{pgfscope}%
\begin{pgfscope}%
\pgfpathrectangle{\pgfqpoint{0.750000in}{0.500000in}}{\pgfqpoint{4.650000in}{3.020000in}}%
\pgfusepath{clip}%
\pgfsetrectcap%
\pgfsetroundjoin%
\pgfsetlinewidth{0.803000pt}%
\definecolor{currentstroke}{rgb}{0.690196,0.690196,0.690196}%
\pgfsetstrokecolor{currentstroke}%
\pgfsetdash{}{0pt}%
\pgfpathmoveto{\pgfqpoint{0.750000in}{1.632500in}}%
\pgfpathlineto{\pgfqpoint{5.400000in}{1.632500in}}%
\pgfusepath{stroke}%
\end{pgfscope}%
\begin{pgfscope}%
\pgfsetbuttcap%
\pgfsetroundjoin%
\definecolor{currentfill}{rgb}{0.000000,0.000000,0.000000}%
\pgfsetfillcolor{currentfill}%
\pgfsetlinewidth{0.803000pt}%
\definecolor{currentstroke}{rgb}{0.000000,0.000000,0.000000}%
\pgfsetstrokecolor{currentstroke}%
\pgfsetdash{}{0pt}%
\pgfsys@defobject{currentmarker}{\pgfqpoint{-0.048611in}{0.000000in}}{\pgfqpoint{0.000000in}{0.000000in}}{%
\pgfpathmoveto{\pgfqpoint{0.000000in}{0.000000in}}%
\pgfpathlineto{\pgfqpoint{-0.048611in}{0.000000in}}%
\pgfusepath{stroke,fill}%
}%
\begin{pgfscope}%
\pgfsys@transformshift{0.750000in}{1.632500in}%
\pgfsys@useobject{currentmarker}{}%
\end{pgfscope}%
\end{pgfscope}%
\begin{pgfscope}%
\pgftext[x=0.405863in,y=1.584306in,left,base]{\rmfamily\fontsize{10.000000}{12.000000}\selectfont \(\displaystyle 0.25\)}%
\end{pgfscope}%
\begin{pgfscope}%
\pgfpathrectangle{\pgfqpoint{0.750000in}{0.500000in}}{\pgfqpoint{4.650000in}{3.020000in}}%
\pgfusepath{clip}%
\pgfsetrectcap%
\pgfsetroundjoin%
\pgfsetlinewidth{0.803000pt}%
\definecolor{currentstroke}{rgb}{0.690196,0.690196,0.690196}%
\pgfsetstrokecolor{currentstroke}%
\pgfsetdash{}{0pt}%
\pgfpathmoveto{\pgfqpoint{0.750000in}{2.010000in}}%
\pgfpathlineto{\pgfqpoint{5.400000in}{2.010000in}}%
\pgfusepath{stroke}%
\end{pgfscope}%
\begin{pgfscope}%
\pgfsetbuttcap%
\pgfsetroundjoin%
\definecolor{currentfill}{rgb}{0.000000,0.000000,0.000000}%
\pgfsetfillcolor{currentfill}%
\pgfsetlinewidth{0.803000pt}%
\definecolor{currentstroke}{rgb}{0.000000,0.000000,0.000000}%
\pgfsetstrokecolor{currentstroke}%
\pgfsetdash{}{0pt}%
\pgfsys@defobject{currentmarker}{\pgfqpoint{-0.048611in}{0.000000in}}{\pgfqpoint{0.000000in}{0.000000in}}{%
\pgfpathmoveto{\pgfqpoint{0.000000in}{0.000000in}}%
\pgfpathlineto{\pgfqpoint{-0.048611in}{0.000000in}}%
\pgfusepath{stroke,fill}%
}%
\begin{pgfscope}%
\pgfsys@transformshift{0.750000in}{2.010000in}%
\pgfsys@useobject{currentmarker}{}%
\end{pgfscope}%
\end{pgfscope}%
\begin{pgfscope}%
\pgftext[x=0.405863in,y=1.961806in,left,base]{\rmfamily\fontsize{10.000000}{12.000000}\selectfont \(\displaystyle 0.50\)}%
\end{pgfscope}%
\begin{pgfscope}%
\pgfpathrectangle{\pgfqpoint{0.750000in}{0.500000in}}{\pgfqpoint{4.650000in}{3.020000in}}%
\pgfusepath{clip}%
\pgfsetrectcap%
\pgfsetroundjoin%
\pgfsetlinewidth{0.803000pt}%
\definecolor{currentstroke}{rgb}{0.690196,0.690196,0.690196}%
\pgfsetstrokecolor{currentstroke}%
\pgfsetdash{}{0pt}%
\pgfpathmoveto{\pgfqpoint{0.750000in}{2.387500in}}%
\pgfpathlineto{\pgfqpoint{5.400000in}{2.387500in}}%
\pgfusepath{stroke}%
\end{pgfscope}%
\begin{pgfscope}%
\pgfsetbuttcap%
\pgfsetroundjoin%
\definecolor{currentfill}{rgb}{0.000000,0.000000,0.000000}%
\pgfsetfillcolor{currentfill}%
\pgfsetlinewidth{0.803000pt}%
\definecolor{currentstroke}{rgb}{0.000000,0.000000,0.000000}%
\pgfsetstrokecolor{currentstroke}%
\pgfsetdash{}{0pt}%
\pgfsys@defobject{currentmarker}{\pgfqpoint{-0.048611in}{0.000000in}}{\pgfqpoint{0.000000in}{0.000000in}}{%
\pgfpathmoveto{\pgfqpoint{0.000000in}{0.000000in}}%
\pgfpathlineto{\pgfqpoint{-0.048611in}{0.000000in}}%
\pgfusepath{stroke,fill}%
}%
\begin{pgfscope}%
\pgfsys@transformshift{0.750000in}{2.387500in}%
\pgfsys@useobject{currentmarker}{}%
\end{pgfscope}%
\end{pgfscope}%
\begin{pgfscope}%
\pgftext[x=0.405863in,y=2.339306in,left,base]{\rmfamily\fontsize{10.000000}{12.000000}\selectfont \(\displaystyle 0.75\)}%
\end{pgfscope}%
\begin{pgfscope}%
\pgfpathrectangle{\pgfqpoint{0.750000in}{0.500000in}}{\pgfqpoint{4.650000in}{3.020000in}}%
\pgfusepath{clip}%
\pgfsetrectcap%
\pgfsetroundjoin%
\pgfsetlinewidth{0.803000pt}%
\definecolor{currentstroke}{rgb}{0.690196,0.690196,0.690196}%
\pgfsetstrokecolor{currentstroke}%
\pgfsetdash{}{0pt}%
\pgfpathmoveto{\pgfqpoint{0.750000in}{2.765000in}}%
\pgfpathlineto{\pgfqpoint{5.400000in}{2.765000in}}%
\pgfusepath{stroke}%
\end{pgfscope}%
\begin{pgfscope}%
\pgfsetbuttcap%
\pgfsetroundjoin%
\definecolor{currentfill}{rgb}{0.000000,0.000000,0.000000}%
\pgfsetfillcolor{currentfill}%
\pgfsetlinewidth{0.803000pt}%
\definecolor{currentstroke}{rgb}{0.000000,0.000000,0.000000}%
\pgfsetstrokecolor{currentstroke}%
\pgfsetdash{}{0pt}%
\pgfsys@defobject{currentmarker}{\pgfqpoint{-0.048611in}{0.000000in}}{\pgfqpoint{0.000000in}{0.000000in}}{%
\pgfpathmoveto{\pgfqpoint{0.000000in}{0.000000in}}%
\pgfpathlineto{\pgfqpoint{-0.048611in}{0.000000in}}%
\pgfusepath{stroke,fill}%
}%
\begin{pgfscope}%
\pgfsys@transformshift{0.750000in}{2.765000in}%
\pgfsys@useobject{currentmarker}{}%
\end{pgfscope}%
\end{pgfscope}%
\begin{pgfscope}%
\pgftext[x=0.405863in,y=2.716806in,left,base]{\rmfamily\fontsize{10.000000}{12.000000}\selectfont \(\displaystyle 1.00\)}%
\end{pgfscope}%
\begin{pgfscope}%
\pgfpathrectangle{\pgfqpoint{0.750000in}{0.500000in}}{\pgfqpoint{4.650000in}{3.020000in}}%
\pgfusepath{clip}%
\pgfsetrectcap%
\pgfsetroundjoin%
\pgfsetlinewidth{0.803000pt}%
\definecolor{currentstroke}{rgb}{0.690196,0.690196,0.690196}%
\pgfsetstrokecolor{currentstroke}%
\pgfsetdash{}{0pt}%
\pgfpathmoveto{\pgfqpoint{0.750000in}{3.142500in}}%
\pgfpathlineto{\pgfqpoint{5.400000in}{3.142500in}}%
\pgfusepath{stroke}%
\end{pgfscope}%
\begin{pgfscope}%
\pgfsetbuttcap%
\pgfsetroundjoin%
\definecolor{currentfill}{rgb}{0.000000,0.000000,0.000000}%
\pgfsetfillcolor{currentfill}%
\pgfsetlinewidth{0.803000pt}%
\definecolor{currentstroke}{rgb}{0.000000,0.000000,0.000000}%
\pgfsetstrokecolor{currentstroke}%
\pgfsetdash{}{0pt}%
\pgfsys@defobject{currentmarker}{\pgfqpoint{-0.048611in}{0.000000in}}{\pgfqpoint{0.000000in}{0.000000in}}{%
\pgfpathmoveto{\pgfqpoint{0.000000in}{0.000000in}}%
\pgfpathlineto{\pgfqpoint{-0.048611in}{0.000000in}}%
\pgfusepath{stroke,fill}%
}%
\begin{pgfscope}%
\pgfsys@transformshift{0.750000in}{3.142500in}%
\pgfsys@useobject{currentmarker}{}%
\end{pgfscope}%
\end{pgfscope}%
\begin{pgfscope}%
\pgftext[x=0.405863in,y=3.094306in,left,base]{\rmfamily\fontsize{10.000000}{12.000000}\selectfont \(\displaystyle 1.25\)}%
\end{pgfscope}%
\begin{pgfscope}%
\pgfpathrectangle{\pgfqpoint{0.750000in}{0.500000in}}{\pgfqpoint{4.650000in}{3.020000in}}%
\pgfusepath{clip}%
\pgfsetrectcap%
\pgfsetroundjoin%
\pgfsetlinewidth{0.803000pt}%
\definecolor{currentstroke}{rgb}{0.690196,0.690196,0.690196}%
\pgfsetstrokecolor{currentstroke}%
\pgfsetdash{}{0pt}%
\pgfpathmoveto{\pgfqpoint{0.750000in}{3.520000in}}%
\pgfpathlineto{\pgfqpoint{5.400000in}{3.520000in}}%
\pgfusepath{stroke}%
\end{pgfscope}%
\begin{pgfscope}%
\pgfsetbuttcap%
\pgfsetroundjoin%
\definecolor{currentfill}{rgb}{0.000000,0.000000,0.000000}%
\pgfsetfillcolor{currentfill}%
\pgfsetlinewidth{0.803000pt}%
\definecolor{currentstroke}{rgb}{0.000000,0.000000,0.000000}%
\pgfsetstrokecolor{currentstroke}%
\pgfsetdash{}{0pt}%
\pgfsys@defobject{currentmarker}{\pgfqpoint{-0.048611in}{0.000000in}}{\pgfqpoint{0.000000in}{0.000000in}}{%
\pgfpathmoveto{\pgfqpoint{0.000000in}{0.000000in}}%
\pgfpathlineto{\pgfqpoint{-0.048611in}{0.000000in}}%
\pgfusepath{stroke,fill}%
}%
\begin{pgfscope}%
\pgfsys@transformshift{0.750000in}{3.520000in}%
\pgfsys@useobject{currentmarker}{}%
\end{pgfscope}%
\end{pgfscope}%
\begin{pgfscope}%
\pgftext[x=0.405863in,y=3.471806in,left,base]{\rmfamily\fontsize{10.000000}{12.000000}\selectfont \(\displaystyle 1.50\)}%
\end{pgfscope}%
\begin{pgfscope}%
\pgfpathrectangle{\pgfqpoint{0.750000in}{0.500000in}}{\pgfqpoint{4.650000in}{3.020000in}}%
\pgfusepath{clip}%
\pgfsetrectcap%
\pgfsetroundjoin%
\pgfsetlinewidth{1.505625pt}%
\definecolor{currentstroke}{rgb}{0.121569,0.466667,0.705882}%
\pgfsetstrokecolor{currentstroke}%
\pgfsetdash{}{0pt}%
\pgfpathmoveto{\pgfqpoint{1.013681in}{0.486111in}}%
\pgfpathlineto{\pgfqpoint{1.094444in}{0.661191in}}%
\pgfpathlineto{\pgfqpoint{1.178228in}{0.835422in}}%
\pgfpathlineto{\pgfqpoint{1.257357in}{0.993059in}}%
\pgfpathlineto{\pgfqpoint{1.336486in}{1.143980in}}%
\pgfpathlineto{\pgfqpoint{1.415616in}{1.288184in}}%
\pgfpathlineto{\pgfqpoint{1.490090in}{1.417771in}}%
\pgfpathlineto{\pgfqpoint{1.564565in}{1.541408in}}%
\pgfpathlineto{\pgfqpoint{1.639039in}{1.659096in}}%
\pgfpathlineto{\pgfqpoint{1.713514in}{1.770834in}}%
\pgfpathlineto{\pgfqpoint{1.783333in}{1.870185in}}%
\pgfpathlineto{\pgfqpoint{1.853153in}{1.964307in}}%
\pgfpathlineto{\pgfqpoint{1.922973in}{2.053201in}}%
\pgfpathlineto{\pgfqpoint{1.992793in}{2.136865in}}%
\pgfpathlineto{\pgfqpoint{2.057958in}{2.210234in}}%
\pgfpathlineto{\pgfqpoint{2.123123in}{2.279048in}}%
\pgfpathlineto{\pgfqpoint{2.188288in}{2.343306in}}%
\pgfpathlineto{\pgfqpoint{2.253453in}{2.403010in}}%
\pgfpathlineto{\pgfqpoint{2.313964in}{2.454370in}}%
\pgfpathlineto{\pgfqpoint{2.374474in}{2.501803in}}%
\pgfpathlineto{\pgfqpoint{2.434985in}{2.545309in}}%
\pgfpathlineto{\pgfqpoint{2.495495in}{2.584887in}}%
\pgfpathlineto{\pgfqpoint{2.556006in}{2.620537in}}%
\pgfpathlineto{\pgfqpoint{2.611862in}{2.649959in}}%
\pgfpathlineto{\pgfqpoint{2.667718in}{2.676034in}}%
\pgfpathlineto{\pgfqpoint{2.723574in}{2.698763in}}%
\pgfpathlineto{\pgfqpoint{2.779429in}{2.718145in}}%
\pgfpathlineto{\pgfqpoint{2.835285in}{2.734181in}}%
\pgfpathlineto{\pgfqpoint{2.891141in}{2.746870in}}%
\pgfpathlineto{\pgfqpoint{2.946997in}{2.756212in}}%
\pgfpathlineto{\pgfqpoint{3.002853in}{2.762208in}}%
\pgfpathlineto{\pgfqpoint{3.058709in}{2.764858in}}%
\pgfpathlineto{\pgfqpoint{3.114565in}{2.764160in}}%
\pgfpathlineto{\pgfqpoint{3.170420in}{2.760117in}}%
\pgfpathlineto{\pgfqpoint{3.226276in}{2.752726in}}%
\pgfpathlineto{\pgfqpoint{3.282132in}{2.741989in}}%
\pgfpathlineto{\pgfqpoint{3.337988in}{2.727906in}}%
\pgfpathlineto{\pgfqpoint{3.393844in}{2.710476in}}%
\pgfpathlineto{\pgfqpoint{3.449700in}{2.689699in}}%
\pgfpathlineto{\pgfqpoint{3.505556in}{2.665576in}}%
\pgfpathlineto{\pgfqpoint{3.561411in}{2.638106in}}%
\pgfpathlineto{\pgfqpoint{3.617267in}{2.607290in}}%
\pgfpathlineto{\pgfqpoint{3.673123in}{2.573127in}}%
\pgfpathlineto{\pgfqpoint{3.733634in}{2.532341in}}%
\pgfpathlineto{\pgfqpoint{3.794144in}{2.487627in}}%
\pgfpathlineto{\pgfqpoint{3.854655in}{2.438986in}}%
\pgfpathlineto{\pgfqpoint{3.915165in}{2.386417in}}%
\pgfpathlineto{\pgfqpoint{3.975676in}{2.329920in}}%
\pgfpathlineto{\pgfqpoint{4.040841in}{2.264685in}}%
\pgfpathlineto{\pgfqpoint{4.106006in}{2.194895in}}%
\pgfpathlineto{\pgfqpoint{4.171171in}{2.120550in}}%
\pgfpathlineto{\pgfqpoint{4.236336in}{2.041650in}}%
\pgfpathlineto{\pgfqpoint{4.306156in}{1.952060in}}%
\pgfpathlineto{\pgfqpoint{4.375976in}{1.857240in}}%
\pgfpathlineto{\pgfqpoint{4.445796in}{1.757192in}}%
\pgfpathlineto{\pgfqpoint{4.515616in}{1.651915in}}%
\pgfpathlineto{\pgfqpoint{4.590090in}{1.533855in}}%
\pgfpathlineto{\pgfqpoint{4.664565in}{1.409846in}}%
\pgfpathlineto{\pgfqpoint{4.739039in}{1.279888in}}%
\pgfpathlineto{\pgfqpoint{4.818168in}{1.135288in}}%
\pgfpathlineto{\pgfqpoint{4.897297in}{0.983972in}}%
\pgfpathlineto{\pgfqpoint{4.976426in}{0.825940in}}%
\pgfpathlineto{\pgfqpoint{5.055556in}{0.661191in}}%
\pgfpathlineto{\pgfqpoint{5.136319in}{0.486111in}}%
\pgfpathlineto{\pgfqpoint{5.136319in}{0.486111in}}%
\pgfusepath{stroke}%
\end{pgfscope}%
\begin{pgfscope}%
\pgfpathrectangle{\pgfqpoint{0.750000in}{0.500000in}}{\pgfqpoint{4.650000in}{3.020000in}}%
\pgfusepath{clip}%
\pgfsetrectcap%
\pgfsetroundjoin%
\pgfsetlinewidth{1.505625pt}%
\definecolor{currentstroke}{rgb}{1.000000,0.498039,0.054902}%
\pgfsetstrokecolor{currentstroke}%
\pgfsetdash{}{0pt}%
\pgfpathmoveto{\pgfqpoint{0.810055in}{3.533889in}}%
\pgfpathlineto{\pgfqpoint{0.843093in}{3.257906in}}%
\pgfpathlineto{\pgfqpoint{0.880330in}{2.969439in}}%
\pgfpathlineto{\pgfqpoint{0.917568in}{2.703986in}}%
\pgfpathlineto{\pgfqpoint{0.954805in}{2.460621in}}%
\pgfpathlineto{\pgfqpoint{0.992042in}{2.238437in}}%
\pgfpathlineto{\pgfqpoint{1.029279in}{2.036541in}}%
\pgfpathlineto{\pgfqpoint{1.061862in}{1.875834in}}%
\pgfpathlineto{\pgfqpoint{1.094444in}{1.729411in}}%
\pgfpathlineto{\pgfqpoint{1.127027in}{1.596703in}}%
\pgfpathlineto{\pgfqpoint{1.159610in}{1.477152in}}%
\pgfpathlineto{\pgfqpoint{1.192192in}{1.370207in}}%
\pgfpathlineto{\pgfqpoint{1.220120in}{1.288164in}}%
\pgfpathlineto{\pgfqpoint{1.248048in}{1.214651in}}%
\pgfpathlineto{\pgfqpoint{1.275976in}{1.149338in}}%
\pgfpathlineto{\pgfqpoint{1.303904in}{1.091900in}}%
\pgfpathlineto{\pgfqpoint{1.331832in}{1.042017in}}%
\pgfpathlineto{\pgfqpoint{1.355105in}{1.005992in}}%
\pgfpathlineto{\pgfqpoint{1.378378in}{0.974815in}}%
\pgfpathlineto{\pgfqpoint{1.401652in}{0.948310in}}%
\pgfpathlineto{\pgfqpoint{1.424925in}{0.926301in}}%
\pgfpathlineto{\pgfqpoint{1.448198in}{0.908616in}}%
\pgfpathlineto{\pgfqpoint{1.471471in}{0.895085in}}%
\pgfpathlineto{\pgfqpoint{1.494745in}{0.885541in}}%
\pgfpathlineto{\pgfqpoint{1.518018in}{0.879818in}}%
\pgfpathlineto{\pgfqpoint{1.541291in}{0.877754in}}%
\pgfpathlineto{\pgfqpoint{1.564565in}{0.879189in}}%
\pgfpathlineto{\pgfqpoint{1.587838in}{0.883964in}}%
\pgfpathlineto{\pgfqpoint{1.611111in}{0.891925in}}%
\pgfpathlineto{\pgfqpoint{1.634384in}{0.902918in}}%
\pgfpathlineto{\pgfqpoint{1.662312in}{0.919901in}}%
\pgfpathlineto{\pgfqpoint{1.690240in}{0.940777in}}%
\pgfpathlineto{\pgfqpoint{1.718168in}{0.965297in}}%
\pgfpathlineto{\pgfqpoint{1.750751in}{0.998181in}}%
\pgfpathlineto{\pgfqpoint{1.783333in}{1.035308in}}%
\pgfpathlineto{\pgfqpoint{1.820571in}{1.082458in}}%
\pgfpathlineto{\pgfqpoint{1.862462in}{1.140874in}}%
\pgfpathlineto{\pgfqpoint{1.909009in}{1.211577in}}%
\pgfpathlineto{\pgfqpoint{1.960210in}{1.295261in}}%
\pgfpathlineto{\pgfqpoint{2.020721in}{1.400475in}}%
\pgfpathlineto{\pgfqpoint{2.095195in}{1.536594in}}%
\pgfpathlineto{\pgfqpoint{2.230180in}{1.791488in}}%
\pgfpathlineto{\pgfqpoint{2.341892in}{1.999448in}}%
\pgfpathlineto{\pgfqpoint{2.416366in}{2.131675in}}%
\pgfpathlineto{\pgfqpoint{2.481532in}{2.240947in}}%
\pgfpathlineto{\pgfqpoint{2.537387in}{2.328690in}}%
\pgfpathlineto{\pgfqpoint{2.588589in}{2.403587in}}%
\pgfpathlineto{\pgfqpoint{2.635135in}{2.466588in}}%
\pgfpathlineto{\pgfqpoint{2.681682in}{2.524346in}}%
\pgfpathlineto{\pgfqpoint{2.723574in}{2.571559in}}%
\pgfpathlineto{\pgfqpoint{2.765465in}{2.614019in}}%
\pgfpathlineto{\pgfqpoint{2.802703in}{2.647609in}}%
\pgfpathlineto{\pgfqpoint{2.839940in}{2.677163in}}%
\pgfpathlineto{\pgfqpoint{2.877177in}{2.702572in}}%
\pgfpathlineto{\pgfqpoint{2.909760in}{2.721332in}}%
\pgfpathlineto{\pgfqpoint{2.942342in}{2.736797in}}%
\pgfpathlineto{\pgfqpoint{2.974925in}{2.748924in}}%
\pgfpathlineto{\pgfqpoint{3.007508in}{2.757679in}}%
\pgfpathlineto{\pgfqpoint{3.040090in}{2.763040in}}%
\pgfpathlineto{\pgfqpoint{3.072673in}{2.764991in}}%
\pgfpathlineto{\pgfqpoint{3.105255in}{2.763528in}}%
\pgfpathlineto{\pgfqpoint{3.137838in}{2.758653in}}%
\pgfpathlineto{\pgfqpoint{3.170420in}{2.750382in}}%
\pgfpathlineto{\pgfqpoint{3.203003in}{2.738735in}}%
\pgfpathlineto{\pgfqpoint{3.235586in}{2.723744in}}%
\pgfpathlineto{\pgfqpoint{3.268168in}{2.705452in}}%
\pgfpathlineto{\pgfqpoint{3.300751in}{2.683908in}}%
\pgfpathlineto{\pgfqpoint{3.337988in}{2.655381in}}%
\pgfpathlineto{\pgfqpoint{3.375225in}{2.622788in}}%
\pgfpathlineto{\pgfqpoint{3.412462in}{2.586250in}}%
\pgfpathlineto{\pgfqpoint{3.454354in}{2.540598in}}%
\pgfpathlineto{\pgfqpoint{3.496246in}{2.490340in}}%
\pgfpathlineto{\pgfqpoint{3.542793in}{2.429393in}}%
\pgfpathlineto{\pgfqpoint{3.589339in}{2.363428in}}%
\pgfpathlineto{\pgfqpoint{3.640541in}{2.285560in}}%
\pgfpathlineto{\pgfqpoint{3.696396in}{2.194954in}}%
\pgfpathlineto{\pgfqpoint{3.761562in}{2.082897in}}%
\pgfpathlineto{\pgfqpoint{3.836036in}{1.948294in}}%
\pgfpathlineto{\pgfqpoint{3.938438in}{1.756234in}}%
\pgfpathlineto{\pgfqpoint{4.096697in}{1.459309in}}%
\pgfpathlineto{\pgfqpoint{4.166517in}{1.335016in}}%
\pgfpathlineto{\pgfqpoint{4.222372in}{1.241354in}}%
\pgfpathlineto{\pgfqpoint{4.273574in}{1.161486in}}%
\pgfpathlineto{\pgfqpoint{4.315465in}{1.101332in}}%
\pgfpathlineto{\pgfqpoint{4.357357in}{1.046643in}}%
\pgfpathlineto{\pgfqpoint{4.394595in}{1.003234in}}%
\pgfpathlineto{\pgfqpoint{4.427177in}{0.969721in}}%
\pgfpathlineto{\pgfqpoint{4.459760in}{0.940777in}}%
\pgfpathlineto{\pgfqpoint{4.487688in}{0.919901in}}%
\pgfpathlineto{\pgfqpoint{4.515616in}{0.902918in}}%
\pgfpathlineto{\pgfqpoint{4.538889in}{0.891925in}}%
\pgfpathlineto{\pgfqpoint{4.562162in}{0.883964in}}%
\pgfpathlineto{\pgfqpoint{4.585435in}{0.879189in}}%
\pgfpathlineto{\pgfqpoint{4.608709in}{0.877754in}}%
\pgfpathlineto{\pgfqpoint{4.631982in}{0.879818in}}%
\pgfpathlineto{\pgfqpoint{4.655255in}{0.885541in}}%
\pgfpathlineto{\pgfqpoint{4.678529in}{0.895085in}}%
\pgfpathlineto{\pgfqpoint{4.701802in}{0.908616in}}%
\pgfpathlineto{\pgfqpoint{4.725075in}{0.926301in}}%
\pgfpathlineto{\pgfqpoint{4.748348in}{0.948310in}}%
\pgfpathlineto{\pgfqpoint{4.771622in}{0.974815in}}%
\pgfpathlineto{\pgfqpoint{4.794895in}{1.005992in}}%
\pgfpathlineto{\pgfqpoint{4.818168in}{1.042017in}}%
\pgfpathlineto{\pgfqpoint{4.841441in}{1.083070in}}%
\pgfpathlineto{\pgfqpoint{4.869369in}{1.139227in}}%
\pgfpathlineto{\pgfqpoint{4.897297in}{1.203205in}}%
\pgfpathlineto{\pgfqpoint{4.925225in}{1.275328in}}%
\pgfpathlineto{\pgfqpoint{4.953153in}{1.355927in}}%
\pgfpathlineto{\pgfqpoint{4.981081in}{1.445333in}}%
\pgfpathlineto{\pgfqpoint{5.013664in}{1.561227in}}%
\pgfpathlineto{\pgfqpoint{5.046246in}{1.690119in}}%
\pgfpathlineto{\pgfqpoint{5.078829in}{1.832566in}}%
\pgfpathlineto{\pgfqpoint{5.111411in}{1.989133in}}%
\pgfpathlineto{\pgfqpoint{5.143994in}{2.160396in}}%
\pgfpathlineto{\pgfqpoint{5.176577in}{2.346938in}}%
\pgfpathlineto{\pgfqpoint{5.213814in}{2.579599in}}%
\pgfpathlineto{\pgfqpoint{5.251051in}{2.833894in}}%
\pgfpathlineto{\pgfqpoint{5.288288in}{3.110738in}}%
\pgfpathlineto{\pgfqpoint{5.325526in}{3.411063in}}%
\pgfpathlineto{\pgfqpoint{5.339945in}{3.533889in}}%
\pgfpathlineto{\pgfqpoint{5.339945in}{3.533889in}}%
\pgfusepath{stroke}%
\end{pgfscope}%
\begin{pgfscope}%
\pgfpathrectangle{\pgfqpoint{0.750000in}{0.500000in}}{\pgfqpoint{4.650000in}{3.020000in}}%
\pgfusepath{clip}%
\pgfsetrectcap%
\pgfsetroundjoin%
\pgfsetlinewidth{1.505625pt}%
\definecolor{currentstroke}{rgb}{0.172549,0.627451,0.172549}%
\pgfsetstrokecolor{currentstroke}%
\pgfsetdash{}{0pt}%
\pgfpathmoveto{\pgfqpoint{1.057345in}{0.486111in}}%
\pgfpathlineto{\pgfqpoint{1.080480in}{0.677254in}}%
\pgfpathlineto{\pgfqpoint{1.103754in}{0.843815in}}%
\pgfpathlineto{\pgfqpoint{1.127027in}{0.986601in}}%
\pgfpathlineto{\pgfqpoint{1.150300in}{1.107533in}}%
\pgfpathlineto{\pgfqpoint{1.168919in}{1.189778in}}%
\pgfpathlineto{\pgfqpoint{1.187538in}{1.260106in}}%
\pgfpathlineto{\pgfqpoint{1.206156in}{1.319377in}}%
\pgfpathlineto{\pgfqpoint{1.224775in}{1.368417in}}%
\pgfpathlineto{\pgfqpoint{1.243393in}{1.408019in}}%
\pgfpathlineto{\pgfqpoint{1.262012in}{1.438941in}}%
\pgfpathlineto{\pgfqpoint{1.275976in}{1.456875in}}%
\pgfpathlineto{\pgfqpoint{1.289940in}{1.470630in}}%
\pgfpathlineto{\pgfqpoint{1.303904in}{1.480491in}}%
\pgfpathlineto{\pgfqpoint{1.317868in}{1.486735in}}%
\pgfpathlineto{\pgfqpoint{1.331832in}{1.489626in}}%
\pgfpathlineto{\pgfqpoint{1.345796in}{1.489421in}}%
\pgfpathlineto{\pgfqpoint{1.359760in}{1.486368in}}%
\pgfpathlineto{\pgfqpoint{1.373724in}{1.480704in}}%
\pgfpathlineto{\pgfqpoint{1.392342in}{1.469488in}}%
\pgfpathlineto{\pgfqpoint{1.410961in}{1.454559in}}%
\pgfpathlineto{\pgfqpoint{1.434234in}{1.431426in}}%
\pgfpathlineto{\pgfqpoint{1.462162in}{1.398299in}}%
\pgfpathlineto{\pgfqpoint{1.494745in}{1.354072in}}%
\pgfpathlineto{\pgfqpoint{1.536637in}{1.291577in}}%
\pgfpathlineto{\pgfqpoint{1.657658in}{1.107391in}}%
\pgfpathlineto{\pgfqpoint{1.694895in}{1.057470in}}%
\pgfpathlineto{\pgfqpoint{1.727477in}{1.018631in}}%
\pgfpathlineto{\pgfqpoint{1.755405in}{0.989574in}}%
\pgfpathlineto{\pgfqpoint{1.783333in}{0.964868in}}%
\pgfpathlineto{\pgfqpoint{1.806607in}{0.947850in}}%
\pgfpathlineto{\pgfqpoint{1.829880in}{0.934250in}}%
\pgfpathlineto{\pgfqpoint{1.853153in}{0.924199in}}%
\pgfpathlineto{\pgfqpoint{1.876426in}{0.917793in}}%
\pgfpathlineto{\pgfqpoint{1.899700in}{0.915104in}}%
\pgfpathlineto{\pgfqpoint{1.922973in}{0.916173in}}%
\pgfpathlineto{\pgfqpoint{1.946246in}{0.921019in}}%
\pgfpathlineto{\pgfqpoint{1.969520in}{0.929634in}}%
\pgfpathlineto{\pgfqpoint{1.992793in}{0.941989in}}%
\pgfpathlineto{\pgfqpoint{2.016066in}{0.958034in}}%
\pgfpathlineto{\pgfqpoint{2.039339in}{0.977697in}}%
\pgfpathlineto{\pgfqpoint{2.067267in}{1.005943in}}%
\pgfpathlineto{\pgfqpoint{2.095195in}{1.039085in}}%
\pgfpathlineto{\pgfqpoint{2.123123in}{1.076904in}}%
\pgfpathlineto{\pgfqpoint{2.155706in}{1.126604in}}%
\pgfpathlineto{\pgfqpoint{2.188288in}{1.181893in}}%
\pgfpathlineto{\pgfqpoint{2.225526in}{1.251303in}}%
\pgfpathlineto{\pgfqpoint{2.267417in}{1.336394in}}%
\pgfpathlineto{\pgfqpoint{2.313964in}{1.438250in}}%
\pgfpathlineto{\pgfqpoint{2.369820in}{1.568302in}}%
\pgfpathlineto{\pgfqpoint{2.448949in}{1.761469in}}%
\pgfpathlineto{\pgfqpoint{2.583934in}{2.092038in}}%
\pgfpathlineto{\pgfqpoint{2.644444in}{2.231333in}}%
\pgfpathlineto{\pgfqpoint{2.695646in}{2.341134in}}%
\pgfpathlineto{\pgfqpoint{2.737538in}{2.424028in}}%
\pgfpathlineto{\pgfqpoint{2.774775in}{2.491622in}}%
\pgfpathlineto{\pgfqpoint{2.812012in}{2.552839in}}%
\pgfpathlineto{\pgfqpoint{2.844595in}{2.600738in}}%
\pgfpathlineto{\pgfqpoint{2.877177in}{2.643002in}}%
\pgfpathlineto{\pgfqpoint{2.905105in}{2.674521in}}%
\pgfpathlineto{\pgfqpoint{2.933033in}{2.701528in}}%
\pgfpathlineto{\pgfqpoint{2.960961in}{2.723888in}}%
\pgfpathlineto{\pgfqpoint{2.984234in}{2.738890in}}%
\pgfpathlineto{\pgfqpoint{3.007508in}{2.750535in}}%
\pgfpathlineto{\pgfqpoint{3.030781in}{2.758782in}}%
\pgfpathlineto{\pgfqpoint{3.054054in}{2.763604in}}%
\pgfpathlineto{\pgfqpoint{3.077327in}{2.764983in}}%
\pgfpathlineto{\pgfqpoint{3.100601in}{2.762914in}}%
\pgfpathlineto{\pgfqpoint{3.123874in}{2.757406in}}%
\pgfpathlineto{\pgfqpoint{3.147147in}{2.748477in}}%
\pgfpathlineto{\pgfqpoint{3.170420in}{2.736157in}}%
\pgfpathlineto{\pgfqpoint{3.193694in}{2.720489in}}%
\pgfpathlineto{\pgfqpoint{3.216967in}{2.701528in}}%
\pgfpathlineto{\pgfqpoint{3.244895in}{2.674521in}}%
\pgfpathlineto{\pgfqpoint{3.272823in}{2.643002in}}%
\pgfpathlineto{\pgfqpoint{3.300751in}{2.607128in}}%
\pgfpathlineto{\pgfqpoint{3.333333in}{2.560015in}}%
\pgfpathlineto{\pgfqpoint{3.365916in}{2.507546in}}%
\pgfpathlineto{\pgfqpoint{3.403153in}{2.441492in}}%
\pgfpathlineto{\pgfqpoint{3.445045in}{2.360137in}}%
\pgfpathlineto{\pgfqpoint{3.491592in}{2.262109in}}%
\pgfpathlineto{\pgfqpoint{3.542793in}{2.146640in}}%
\pgfpathlineto{\pgfqpoint{3.607958in}{1.991266in}}%
\pgfpathlineto{\pgfqpoint{3.836036in}{1.438250in}}%
\pgfpathlineto{\pgfqpoint{3.887237in}{1.326607in}}%
\pgfpathlineto{\pgfqpoint{3.929129in}{1.242288in}}%
\pgfpathlineto{\pgfqpoint{3.966366in}{1.173670in}}%
\pgfpathlineto{\pgfqpoint{4.003604in}{1.111814in}}%
\pgfpathlineto{\pgfqpoint{4.036186in}{1.063793in}}%
\pgfpathlineto{\pgfqpoint{4.064114in}{1.027507in}}%
\pgfpathlineto{\pgfqpoint{4.092042in}{0.995974in}}%
\pgfpathlineto{\pgfqpoint{4.119970in}{0.969403in}}%
\pgfpathlineto{\pgfqpoint{4.143243in}{0.951177in}}%
\pgfpathlineto{\pgfqpoint{4.166517in}{0.936601in}}%
\pgfpathlineto{\pgfqpoint{4.189790in}{0.925737in}}%
\pgfpathlineto{\pgfqpoint{4.213063in}{0.918627in}}%
\pgfpathlineto{\pgfqpoint{4.236336in}{0.915293in}}%
\pgfpathlineto{\pgfqpoint{4.259610in}{0.915730in}}%
\pgfpathlineto{\pgfqpoint{4.282883in}{0.919913in}}%
\pgfpathlineto{\pgfqpoint{4.306156in}{0.927787in}}%
\pgfpathlineto{\pgfqpoint{4.329429in}{0.939271in}}%
\pgfpathlineto{\pgfqpoint{4.352703in}{0.954255in}}%
\pgfpathlineto{\pgfqpoint{4.375976in}{0.972597in}}%
\pgfpathlineto{\pgfqpoint{4.403904in}{0.998795in}}%
\pgfpathlineto{\pgfqpoint{4.431832in}{1.029213in}}%
\pgfpathlineto{\pgfqpoint{4.464414in}{1.069443in}}%
\pgfpathlineto{\pgfqpoint{4.501652in}{1.120641in}}%
\pgfpathlineto{\pgfqpoint{4.548198in}{1.190239in}}%
\pgfpathlineto{\pgfqpoint{4.664565in}{1.367206in}}%
\pgfpathlineto{\pgfqpoint{4.697147in}{1.409909in}}%
\pgfpathlineto{\pgfqpoint{4.725075in}{1.441223in}}%
\pgfpathlineto{\pgfqpoint{4.748348in}{1.462456in}}%
\pgfpathlineto{\pgfqpoint{4.766967in}{1.475592in}}%
\pgfpathlineto{\pgfqpoint{4.785586in}{1.484759in}}%
\pgfpathlineto{\pgfqpoint{4.799550in}{1.488708in}}%
\pgfpathlineto{\pgfqpoint{4.813514in}{1.489889in}}%
\pgfpathlineto{\pgfqpoint{4.827477in}{1.488058in}}%
\pgfpathlineto{\pgfqpoint{4.841441in}{1.482961in}}%
\pgfpathlineto{\pgfqpoint{4.855405in}{1.474336in}}%
\pgfpathlineto{\pgfqpoint{4.869369in}{1.461910in}}%
\pgfpathlineto{\pgfqpoint{4.883333in}{1.445401in}}%
\pgfpathlineto{\pgfqpoint{4.897297in}{1.424519in}}%
\pgfpathlineto{\pgfqpoint{4.911261in}{1.398961in}}%
\pgfpathlineto{\pgfqpoint{4.929880in}{1.357072in}}%
\pgfpathlineto{\pgfqpoint{4.948498in}{1.305550in}}%
\pgfpathlineto{\pgfqpoint{4.967117in}{1.243593in}}%
\pgfpathlineto{\pgfqpoint{4.985736in}{1.170368in}}%
\pgfpathlineto{\pgfqpoint{5.004354in}{1.085006in}}%
\pgfpathlineto{\pgfqpoint{5.022973in}{0.986601in}}%
\pgfpathlineto{\pgfqpoint{5.041592in}{0.874211in}}%
\pgfpathlineto{\pgfqpoint{5.064865in}{0.712564in}}%
\pgfpathlineto{\pgfqpoint{5.088138in}{0.525538in}}%
\pgfpathlineto{\pgfqpoint{5.092655in}{0.486111in}}%
\pgfpathlineto{\pgfqpoint{5.092655in}{0.486111in}}%
\pgfusepath{stroke}%
\end{pgfscope}%
\begin{pgfscope}%
\pgfpathrectangle{\pgfqpoint{0.750000in}{0.500000in}}{\pgfqpoint{4.650000in}{3.020000in}}%
\pgfusepath{clip}%
\pgfsetrectcap%
\pgfsetroundjoin%
\pgfsetlinewidth{1.505625pt}%
\definecolor{currentstroke}{rgb}{0.839216,0.152941,0.156863}%
\pgfsetstrokecolor{currentstroke}%
\pgfsetdash{}{0pt}%
\pgfpathmoveto{\pgfqpoint{0.989712in}{3.533889in}}%
\pgfpathlineto{\pgfqpoint{1.006006in}{3.143114in}}%
\pgfpathlineto{\pgfqpoint{1.024625in}{2.754665in}}%
\pgfpathlineto{\pgfqpoint{1.043243in}{2.421827in}}%
\pgfpathlineto{\pgfqpoint{1.061862in}{2.139053in}}%
\pgfpathlineto{\pgfqpoint{1.080480in}{1.901182in}}%
\pgfpathlineto{\pgfqpoint{1.099099in}{1.703424in}}%
\pgfpathlineto{\pgfqpoint{1.117718in}{1.541338in}}%
\pgfpathlineto{\pgfqpoint{1.136336in}{1.410816in}}%
\pgfpathlineto{\pgfqpoint{1.150300in}{1.331347in}}%
\pgfpathlineto{\pgfqpoint{1.164264in}{1.266005in}}%
\pgfpathlineto{\pgfqpoint{1.178228in}{1.213395in}}%
\pgfpathlineto{\pgfqpoint{1.192192in}{1.172205in}}%
\pgfpathlineto{\pgfqpoint{1.206156in}{1.141211in}}%
\pgfpathlineto{\pgfqpoint{1.220120in}{1.119265in}}%
\pgfpathlineto{\pgfqpoint{1.229429in}{1.109131in}}%
\pgfpathlineto{\pgfqpoint{1.238739in}{1.102244in}}%
\pgfpathlineto{\pgfqpoint{1.248048in}{1.098319in}}%
\pgfpathlineto{\pgfqpoint{1.257357in}{1.097086in}}%
\pgfpathlineto{\pgfqpoint{1.266667in}{1.098288in}}%
\pgfpathlineto{\pgfqpoint{1.280631in}{1.104122in}}%
\pgfpathlineto{\pgfqpoint{1.294595in}{1.114114in}}%
\pgfpathlineto{\pgfqpoint{1.313213in}{1.132680in}}%
\pgfpathlineto{\pgfqpoint{1.331832in}{1.155844in}}%
\pgfpathlineto{\pgfqpoint{1.359760in}{1.196277in}}%
\pgfpathlineto{\pgfqpoint{1.462162in}{1.351083in}}%
\pgfpathlineto{\pgfqpoint{1.490090in}{1.384555in}}%
\pgfpathlineto{\pgfqpoint{1.513363in}{1.407474in}}%
\pgfpathlineto{\pgfqpoint{1.531982in}{1.422218in}}%
\pgfpathlineto{\pgfqpoint{1.550601in}{1.433606in}}%
\pgfpathlineto{\pgfqpoint{1.569219in}{1.441559in}}%
\pgfpathlineto{\pgfqpoint{1.587838in}{1.446054in}}%
\pgfpathlineto{\pgfqpoint{1.606456in}{1.447126in}}%
\pgfpathlineto{\pgfqpoint{1.625075in}{1.444854in}}%
\pgfpathlineto{\pgfqpoint{1.643694in}{1.439360in}}%
\pgfpathlineto{\pgfqpoint{1.662312in}{1.430804in}}%
\pgfpathlineto{\pgfqpoint{1.685586in}{1.416097in}}%
\pgfpathlineto{\pgfqpoint{1.708859in}{1.397337in}}%
\pgfpathlineto{\pgfqpoint{1.736787in}{1.370164in}}%
\pgfpathlineto{\pgfqpoint{1.769369in}{1.333200in}}%
\pgfpathlineto{\pgfqpoint{1.806607in}{1.285924in}}%
\pgfpathlineto{\pgfqpoint{1.881081in}{1.184584in}}%
\pgfpathlineto{\pgfqpoint{1.932282in}{1.117649in}}%
\pgfpathlineto{\pgfqpoint{1.969520in}{1.074182in}}%
\pgfpathlineto{\pgfqpoint{1.997447in}{1.045807in}}%
\pgfpathlineto{\pgfqpoint{2.025375in}{1.021868in}}%
\pgfpathlineto{\pgfqpoint{2.048649in}{1.005775in}}%
\pgfpathlineto{\pgfqpoint{2.071922in}{0.993538in}}%
\pgfpathlineto{\pgfqpoint{2.090541in}{0.986715in}}%
\pgfpathlineto{\pgfqpoint{2.109159in}{0.982664in}}%
\pgfpathlineto{\pgfqpoint{2.127778in}{0.981490in}}%
\pgfpathlineto{\pgfqpoint{2.146396in}{0.983278in}}%
\pgfpathlineto{\pgfqpoint{2.165015in}{0.988095in}}%
\pgfpathlineto{\pgfqpoint{2.183634in}{0.995989in}}%
\pgfpathlineto{\pgfqpoint{2.202252in}{1.006988in}}%
\pgfpathlineto{\pgfqpoint{2.220871in}{1.021104in}}%
\pgfpathlineto{\pgfqpoint{2.244144in}{1.043119in}}%
\pgfpathlineto{\pgfqpoint{2.267417in}{1.069937in}}%
\pgfpathlineto{\pgfqpoint{2.290691in}{1.101464in}}%
\pgfpathlineto{\pgfqpoint{2.318619in}{1.145319in}}%
\pgfpathlineto{\pgfqpoint{2.346547in}{1.195454in}}%
\pgfpathlineto{\pgfqpoint{2.379129in}{1.261374in}}%
\pgfpathlineto{\pgfqpoint{2.411712in}{1.334609in}}%
\pgfpathlineto{\pgfqpoint{2.448949in}{1.426202in}}%
\pgfpathlineto{\pgfqpoint{2.490841in}{1.537642in}}%
\pgfpathlineto{\pgfqpoint{2.542042in}{1.682842in}}%
\pgfpathlineto{\pgfqpoint{2.635135in}{1.958729in}}%
\pgfpathlineto{\pgfqpoint{2.709610in}{2.175197in}}%
\pgfpathlineto{\pgfqpoint{2.756156in}{2.301904in}}%
\pgfpathlineto{\pgfqpoint{2.798048in}{2.407154in}}%
\pgfpathlineto{\pgfqpoint{2.835285in}{2.491957in}}%
\pgfpathlineto{\pgfqpoint{2.867868in}{2.558304in}}%
\pgfpathlineto{\pgfqpoint{2.895796in}{2.608698in}}%
\pgfpathlineto{\pgfqpoint{2.923724in}{2.652646in}}%
\pgfpathlineto{\pgfqpoint{2.946997in}{2.684063in}}%
\pgfpathlineto{\pgfqpoint{2.970270in}{2.710543in}}%
\pgfpathlineto{\pgfqpoint{2.993544in}{2.731922in}}%
\pgfpathlineto{\pgfqpoint{3.012162in}{2.745266in}}%
\pgfpathlineto{\pgfqpoint{3.030781in}{2.755209in}}%
\pgfpathlineto{\pgfqpoint{3.049399in}{2.761714in}}%
\pgfpathlineto{\pgfqpoint{3.068018in}{2.764755in}}%
\pgfpathlineto{\pgfqpoint{3.086637in}{2.764321in}}%
\pgfpathlineto{\pgfqpoint{3.105255in}{2.760412in}}%
\pgfpathlineto{\pgfqpoint{3.123874in}{2.753044in}}%
\pgfpathlineto{\pgfqpoint{3.142492in}{2.742247in}}%
\pgfpathlineto{\pgfqpoint{3.161111in}{2.728061in}}%
\pgfpathlineto{\pgfqpoint{3.179730in}{2.710543in}}%
\pgfpathlineto{\pgfqpoint{3.203003in}{2.684063in}}%
\pgfpathlineto{\pgfqpoint{3.226276in}{2.652646in}}%
\pgfpathlineto{\pgfqpoint{3.249550in}{2.616482in}}%
\pgfpathlineto{\pgfqpoint{3.277477in}{2.567133in}}%
\pgfpathlineto{\pgfqpoint{3.305405in}{2.511702in}}%
\pgfpathlineto{\pgfqpoint{3.337988in}{2.440002in}}%
\pgfpathlineto{\pgfqpoint{3.375225in}{2.349848in}}%
\pgfpathlineto{\pgfqpoint{3.417117in}{2.239658in}}%
\pgfpathlineto{\pgfqpoint{3.468318in}{2.095480in}}%
\pgfpathlineto{\pgfqpoint{3.547447in}{1.861651in}}%
\pgfpathlineto{\pgfqpoint{3.635886in}{1.602636in}}%
\pgfpathlineto{\pgfqpoint{3.687087in}{1.462456in}}%
\pgfpathlineto{\pgfqpoint{3.728979in}{1.356765in}}%
\pgfpathlineto{\pgfqpoint{3.766216in}{1.271406in}}%
\pgfpathlineto{\pgfqpoint{3.798799in}{1.204395in}}%
\pgfpathlineto{\pgfqpoint{3.826727in}{1.153248in}}%
\pgfpathlineto{\pgfqpoint{3.854655in}{1.108323in}}%
\pgfpathlineto{\pgfqpoint{3.877928in}{1.075870in}}%
\pgfpathlineto{\pgfqpoint{3.901201in}{1.048101in}}%
\pgfpathlineto{\pgfqpoint{3.924474in}{1.025120in}}%
\pgfpathlineto{\pgfqpoint{3.947748in}{1.006988in}}%
\pgfpathlineto{\pgfqpoint{3.966366in}{0.995989in}}%
\pgfpathlineto{\pgfqpoint{3.984985in}{0.988095in}}%
\pgfpathlineto{\pgfqpoint{4.003604in}{0.983278in}}%
\pgfpathlineto{\pgfqpoint{4.022222in}{0.981490in}}%
\pgfpathlineto{\pgfqpoint{4.040841in}{0.982664in}}%
\pgfpathlineto{\pgfqpoint{4.059459in}{0.986715in}}%
\pgfpathlineto{\pgfqpoint{4.078078in}{0.993538in}}%
\pgfpathlineto{\pgfqpoint{4.101351in}{1.005775in}}%
\pgfpathlineto{\pgfqpoint{4.124625in}{1.021868in}}%
\pgfpathlineto{\pgfqpoint{4.147898in}{1.041490in}}%
\pgfpathlineto{\pgfqpoint{4.175826in}{1.069175in}}%
\pgfpathlineto{\pgfqpoint{4.208408in}{1.106267in}}%
\pgfpathlineto{\pgfqpoint{4.245646in}{1.153417in}}%
\pgfpathlineto{\pgfqpoint{4.310811in}{1.241962in}}%
\pgfpathlineto{\pgfqpoint{4.366667in}{1.315984in}}%
\pgfpathlineto{\pgfqpoint{4.403904in}{1.360122in}}%
\pgfpathlineto{\pgfqpoint{4.431832in}{1.388806in}}%
\pgfpathlineto{\pgfqpoint{4.455105in}{1.409053in}}%
\pgfpathlineto{\pgfqpoint{4.478378in}{1.425436in}}%
\pgfpathlineto{\pgfqpoint{4.501652in}{1.437501in}}%
\pgfpathlineto{\pgfqpoint{4.520270in}{1.443777in}}%
\pgfpathlineto{\pgfqpoint{4.538889in}{1.446867in}}%
\pgfpathlineto{\pgfqpoint{4.557508in}{1.446641in}}%
\pgfpathlineto{\pgfqpoint{4.576126in}{1.443006in}}%
\pgfpathlineto{\pgfqpoint{4.594745in}{1.435918in}}%
\pgfpathlineto{\pgfqpoint{4.613363in}{1.425384in}}%
\pgfpathlineto{\pgfqpoint{4.631982in}{1.411468in}}%
\pgfpathlineto{\pgfqpoint{4.655255in}{1.389521in}}%
\pgfpathlineto{\pgfqpoint{4.678529in}{1.362907in}}%
\pgfpathlineto{\pgfqpoint{4.706456in}{1.325678in}}%
\pgfpathlineto{\pgfqpoint{4.743694in}{1.269569in}}%
\pgfpathlineto{\pgfqpoint{4.813514in}{1.162191in}}%
\pgfpathlineto{\pgfqpoint{4.836787in}{1.132680in}}%
\pgfpathlineto{\pgfqpoint{4.855405in}{1.114114in}}%
\pgfpathlineto{\pgfqpoint{4.869369in}{1.104122in}}%
\pgfpathlineto{\pgfqpoint{4.883333in}{1.098288in}}%
\pgfpathlineto{\pgfqpoint{4.892643in}{1.097086in}}%
\pgfpathlineto{\pgfqpoint{4.901952in}{1.098319in}}%
\pgfpathlineto{\pgfqpoint{4.911261in}{1.102244in}}%
\pgfpathlineto{\pgfqpoint{4.920571in}{1.109131in}}%
\pgfpathlineto{\pgfqpoint{4.929880in}{1.119265in}}%
\pgfpathlineto{\pgfqpoint{4.939189in}{1.132945in}}%
\pgfpathlineto{\pgfqpoint{4.953153in}{1.160799in}}%
\pgfpathlineto{\pgfqpoint{4.967117in}{1.198458in}}%
\pgfpathlineto{\pgfqpoint{4.981081in}{1.247120in}}%
\pgfpathlineto{\pgfqpoint{4.995045in}{1.308067in}}%
\pgfpathlineto{\pgfqpoint{5.009009in}{1.382667in}}%
\pgfpathlineto{\pgfqpoint{5.022973in}{1.472376in}}%
\pgfpathlineto{\pgfqpoint{5.036937in}{1.578745in}}%
\pgfpathlineto{\pgfqpoint{5.055556in}{1.749352in}}%
\pgfpathlineto{\pgfqpoint{5.074174in}{1.956708in}}%
\pgfpathlineto{\pgfqpoint{5.092793in}{2.205342in}}%
\pgfpathlineto{\pgfqpoint{5.111411in}{2.500133in}}%
\pgfpathlineto{\pgfqpoint{5.130030in}{2.846338in}}%
\pgfpathlineto{\pgfqpoint{5.148649in}{3.249604in}}%
\pgfpathlineto{\pgfqpoint{5.160288in}{3.533889in}}%
\pgfpathlineto{\pgfqpoint{5.160288in}{3.533889in}}%
\pgfusepath{stroke}%
\end{pgfscope}%
\begin{pgfscope}%
\pgfpathrectangle{\pgfqpoint{0.750000in}{0.500000in}}{\pgfqpoint{4.650000in}{3.020000in}}%
\pgfusepath{clip}%
\pgfsetrectcap%
\pgfsetroundjoin%
\pgfsetlinewidth{1.505625pt}%
\definecolor{currentstroke}{rgb}{0.580392,0.403922,0.741176}%
\pgfsetstrokecolor{currentstroke}%
\pgfsetdash{}{0pt}%
\pgfpathmoveto{\pgfqpoint{1.075434in}{0.486111in}}%
\pgfpathlineto{\pgfqpoint{1.085135in}{0.637144in}}%
\pgfpathlineto{\pgfqpoint{1.099099in}{0.820554in}}%
\pgfpathlineto{\pgfqpoint{1.113063in}{0.968720in}}%
\pgfpathlineto{\pgfqpoint{1.127027in}{1.086257in}}%
\pgfpathlineto{\pgfqpoint{1.140991in}{1.177353in}}%
\pgfpathlineto{\pgfqpoint{1.154955in}{1.245793in}}%
\pgfpathlineto{\pgfqpoint{1.168919in}{1.294989in}}%
\pgfpathlineto{\pgfqpoint{1.178228in}{1.318622in}}%
\pgfpathlineto{\pgfqpoint{1.187538in}{1.335898in}}%
\pgfpathlineto{\pgfqpoint{1.196847in}{1.347583in}}%
\pgfpathlineto{\pgfqpoint{1.206156in}{1.354389in}}%
\pgfpathlineto{\pgfqpoint{1.215465in}{1.356969in}}%
\pgfpathlineto{\pgfqpoint{1.224775in}{1.355927in}}%
\pgfpathlineto{\pgfqpoint{1.234084in}{1.351818in}}%
\pgfpathlineto{\pgfqpoint{1.248048in}{1.341004in}}%
\pgfpathlineto{\pgfqpoint{1.262012in}{1.325954in}}%
\pgfpathlineto{\pgfqpoint{1.285285in}{1.294989in}}%
\pgfpathlineto{\pgfqpoint{1.350450in}{1.203241in}}%
\pgfpathlineto{\pgfqpoint{1.373724in}{1.177485in}}%
\pgfpathlineto{\pgfqpoint{1.392342in}{1.161376in}}%
\pgfpathlineto{\pgfqpoint{1.410961in}{1.149670in}}%
\pgfpathlineto{\pgfqpoint{1.424925in}{1.143873in}}%
\pgfpathlineto{\pgfqpoint{1.438889in}{1.140630in}}%
\pgfpathlineto{\pgfqpoint{1.452853in}{1.139882in}}%
\pgfpathlineto{\pgfqpoint{1.466817in}{1.141533in}}%
\pgfpathlineto{\pgfqpoint{1.485435in}{1.147229in}}%
\pgfpathlineto{\pgfqpoint{1.504054in}{1.156553in}}%
\pgfpathlineto{\pgfqpoint{1.522673in}{1.169046in}}%
\pgfpathlineto{\pgfqpoint{1.545946in}{1.188336in}}%
\pgfpathlineto{\pgfqpoint{1.573874in}{1.215481in}}%
\pgfpathlineto{\pgfqpoint{1.625075in}{1.270708in}}%
\pgfpathlineto{\pgfqpoint{1.671622in}{1.319476in}}%
\pgfpathlineto{\pgfqpoint{1.704204in}{1.348963in}}%
\pgfpathlineto{\pgfqpoint{1.732132in}{1.369568in}}%
\pgfpathlineto{\pgfqpoint{1.755405in}{1.382779in}}%
\pgfpathlineto{\pgfqpoint{1.778679in}{1.392013in}}%
\pgfpathlineto{\pgfqpoint{1.797297in}{1.396383in}}%
\pgfpathlineto{\pgfqpoint{1.815916in}{1.398009in}}%
\pgfpathlineto{\pgfqpoint{1.834535in}{1.396886in}}%
\pgfpathlineto{\pgfqpoint{1.853153in}{1.393058in}}%
\pgfpathlineto{\pgfqpoint{1.876426in}{1.384605in}}%
\pgfpathlineto{\pgfqpoint{1.899700in}{1.372326in}}%
\pgfpathlineto{\pgfqpoint{1.922973in}{1.356574in}}%
\pgfpathlineto{\pgfqpoint{1.950901in}{1.333703in}}%
\pgfpathlineto{\pgfqpoint{1.983483in}{1.302670in}}%
\pgfpathlineto{\pgfqpoint{2.030030in}{1.253252in}}%
\pgfpathlineto{\pgfqpoint{2.104505in}{1.173442in}}%
\pgfpathlineto{\pgfqpoint{2.137087in}{1.142866in}}%
\pgfpathlineto{\pgfqpoint{2.165015in}{1.120653in}}%
\pgfpathlineto{\pgfqpoint{2.188288in}{1.105722in}}%
\pgfpathlineto{\pgfqpoint{2.211562in}{1.094604in}}%
\pgfpathlineto{\pgfqpoint{2.230180in}{1.088777in}}%
\pgfpathlineto{\pgfqpoint{2.248799in}{1.085910in}}%
\pgfpathlineto{\pgfqpoint{2.267417in}{1.086194in}}%
\pgfpathlineto{\pgfqpoint{2.286036in}{1.089793in}}%
\pgfpathlineto{\pgfqpoint{2.304655in}{1.096843in}}%
\pgfpathlineto{\pgfqpoint{2.323273in}{1.107449in}}%
\pgfpathlineto{\pgfqpoint{2.341892in}{1.121690in}}%
\pgfpathlineto{\pgfqpoint{2.360511in}{1.139613in}}%
\pgfpathlineto{\pgfqpoint{2.379129in}{1.161234in}}%
\pgfpathlineto{\pgfqpoint{2.402402in}{1.193441in}}%
\pgfpathlineto{\pgfqpoint{2.425676in}{1.231314in}}%
\pgfpathlineto{\pgfqpoint{2.448949in}{1.274693in}}%
\pgfpathlineto{\pgfqpoint{2.476877in}{1.333685in}}%
\pgfpathlineto{\pgfqpoint{2.504805in}{1.399748in}}%
\pgfpathlineto{\pgfqpoint{2.537387in}{1.484888in}}%
\pgfpathlineto{\pgfqpoint{2.574625in}{1.591307in}}%
\pgfpathlineto{\pgfqpoint{2.621171in}{1.734864in}}%
\pgfpathlineto{\pgfqpoint{2.690991in}{1.962317in}}%
\pgfpathlineto{\pgfqpoint{2.770120in}{2.217972in}}%
\pgfpathlineto{\pgfqpoint{2.812012in}{2.343981in}}%
\pgfpathlineto{\pgfqpoint{2.849249in}{2.446676in}}%
\pgfpathlineto{\pgfqpoint{2.881832in}{2.527404in}}%
\pgfpathlineto{\pgfqpoint{2.909760in}{2.588675in}}%
\pgfpathlineto{\pgfqpoint{2.933033in}{2.633533in}}%
\pgfpathlineto{\pgfqpoint{2.956306in}{2.672319in}}%
\pgfpathlineto{\pgfqpoint{2.979580in}{2.704682in}}%
\pgfpathlineto{\pgfqpoint{2.998198in}{2.725748in}}%
\pgfpathlineto{\pgfqpoint{3.016817in}{2.742393in}}%
\pgfpathlineto{\pgfqpoint{3.035435in}{2.754520in}}%
\pgfpathlineto{\pgfqpoint{3.049399in}{2.760607in}}%
\pgfpathlineto{\pgfqpoint{3.063363in}{2.764092in}}%
\pgfpathlineto{\pgfqpoint{3.077327in}{2.764964in}}%
\pgfpathlineto{\pgfqpoint{3.091291in}{2.763220in}}%
\pgfpathlineto{\pgfqpoint{3.105255in}{2.758866in}}%
\pgfpathlineto{\pgfqpoint{3.119219in}{2.751916in}}%
\pgfpathlineto{\pgfqpoint{3.137838in}{2.738652in}}%
\pgfpathlineto{\pgfqpoint{3.156456in}{2.720892in}}%
\pgfpathlineto{\pgfqpoint{3.175075in}{2.698738in}}%
\pgfpathlineto{\pgfqpoint{3.193694in}{2.672319in}}%
\pgfpathlineto{\pgfqpoint{3.216967in}{2.633533in}}%
\pgfpathlineto{\pgfqpoint{3.240240in}{2.588675in}}%
\pgfpathlineto{\pgfqpoint{3.268168in}{2.527404in}}%
\pgfpathlineto{\pgfqpoint{3.296096in}{2.458770in}}%
\pgfpathlineto{\pgfqpoint{3.328679in}{2.370582in}}%
\pgfpathlineto{\pgfqpoint{3.365916in}{2.261003in}}%
\pgfpathlineto{\pgfqpoint{3.417117in}{2.099472in}}%
\pgfpathlineto{\pgfqpoint{3.570721in}{1.605196in}}%
\pgfpathlineto{\pgfqpoint{3.612613in}{1.484888in}}%
\pgfpathlineto{\pgfqpoint{3.645195in}{1.399748in}}%
\pgfpathlineto{\pgfqpoint{3.677778in}{1.323346in}}%
\pgfpathlineto{\pgfqpoint{3.705706in}{1.265587in}}%
\pgfpathlineto{\pgfqpoint{3.733634in}{1.215493in}}%
\pgfpathlineto{\pgfqpoint{3.756907in}{1.179871in}}%
\pgfpathlineto{\pgfqpoint{3.780180in}{1.149961in}}%
\pgfpathlineto{\pgfqpoint{3.803453in}{1.125824in}}%
\pgfpathlineto{\pgfqpoint{3.822072in}{1.110666in}}%
\pgfpathlineto{\pgfqpoint{3.840691in}{1.099157in}}%
\pgfpathlineto{\pgfqpoint{3.859309in}{1.091227in}}%
\pgfpathlineto{\pgfqpoint{3.877928in}{1.086778in}}%
\pgfpathlineto{\pgfqpoint{3.896547in}{1.085679in}}%
\pgfpathlineto{\pgfqpoint{3.915165in}{1.087775in}}%
\pgfpathlineto{\pgfqpoint{3.933784in}{1.092881in}}%
\pgfpathlineto{\pgfqpoint{3.952402in}{1.100789in}}%
\pgfpathlineto{\pgfqpoint{3.975676in}{1.114255in}}%
\pgfpathlineto{\pgfqpoint{3.998949in}{1.131227in}}%
\pgfpathlineto{\pgfqpoint{4.026877in}{1.155442in}}%
\pgfpathlineto{\pgfqpoint{4.064114in}{1.192556in}}%
\pgfpathlineto{\pgfqpoint{4.133934in}{1.268462in}}%
\pgfpathlineto{\pgfqpoint{4.180480in}{1.316455in}}%
\pgfpathlineto{\pgfqpoint{4.213063in}{1.345633in}}%
\pgfpathlineto{\pgfqpoint{4.240991in}{1.366419in}}%
\pgfpathlineto{\pgfqpoint{4.264264in}{1.380134in}}%
\pgfpathlineto{\pgfqpoint{4.287538in}{1.390155in}}%
\pgfpathlineto{\pgfqpoint{4.306156in}{1.395306in}}%
\pgfpathlineto{\pgfqpoint{4.324775in}{1.397790in}}%
\pgfpathlineto{\pgfqpoint{4.343393in}{1.397540in}}%
\pgfpathlineto{\pgfqpoint{4.362012in}{1.394539in}}%
\pgfpathlineto{\pgfqpoint{4.380631in}{1.388814in}}%
\pgfpathlineto{\pgfqpoint{4.403904in}{1.377955in}}%
\pgfpathlineto{\pgfqpoint{4.427177in}{1.363243in}}%
\pgfpathlineto{\pgfqpoint{4.450450in}{1.345077in}}%
\pgfpathlineto{\pgfqpoint{4.478378in}{1.319476in}}%
\pgfpathlineto{\pgfqpoint{4.515616in}{1.280831in}}%
\pgfpathlineto{\pgfqpoint{4.594745in}{1.196981in}}%
\pgfpathlineto{\pgfqpoint{4.622673in}{1.172605in}}%
\pgfpathlineto{\pgfqpoint{4.645946in}{1.156553in}}%
\pgfpathlineto{\pgfqpoint{4.664565in}{1.147229in}}%
\pgfpathlineto{\pgfqpoint{4.683183in}{1.141533in}}%
\pgfpathlineto{\pgfqpoint{4.701802in}{1.139859in}}%
\pgfpathlineto{\pgfqpoint{4.715766in}{1.141430in}}%
\pgfpathlineto{\pgfqpoint{4.729730in}{1.145521in}}%
\pgfpathlineto{\pgfqpoint{4.743694in}{1.152173in}}%
\pgfpathlineto{\pgfqpoint{4.762312in}{1.165001in}}%
\pgfpathlineto{\pgfqpoint{4.780931in}{1.182163in}}%
\pgfpathlineto{\pgfqpoint{4.799550in}{1.203241in}}%
\pgfpathlineto{\pgfqpoint{4.822823in}{1.234019in}}%
\pgfpathlineto{\pgfqpoint{4.892643in}{1.331355in}}%
\pgfpathlineto{\pgfqpoint{4.906607in}{1.345151in}}%
\pgfpathlineto{\pgfqpoint{4.920571in}{1.354223in}}%
\pgfpathlineto{\pgfqpoint{4.929880in}{1.356865in}}%
\pgfpathlineto{\pgfqpoint{4.939189in}{1.356168in}}%
\pgfpathlineto{\pgfqpoint{4.948498in}{1.351554in}}%
\pgfpathlineto{\pgfqpoint{4.957808in}{1.342394in}}%
\pgfpathlineto{\pgfqpoint{4.967117in}{1.328005in}}%
\pgfpathlineto{\pgfqpoint{4.976426in}{1.307650in}}%
\pgfpathlineto{\pgfqpoint{4.985736in}{1.280533in}}%
\pgfpathlineto{\pgfqpoint{4.995045in}{1.245793in}}%
\pgfpathlineto{\pgfqpoint{5.009009in}{1.177353in}}%
\pgfpathlineto{\pgfqpoint{5.022973in}{1.086257in}}%
\pgfpathlineto{\pgfqpoint{5.036937in}{0.968720in}}%
\pgfpathlineto{\pgfqpoint{5.050901in}{0.820554in}}%
\pgfpathlineto{\pgfqpoint{5.064865in}{0.637144in}}%
\pgfpathlineto{\pgfqpoint{5.074566in}{0.486111in}}%
\pgfpathlineto{\pgfqpoint{5.074566in}{0.486111in}}%
\pgfusepath{stroke}%
\end{pgfscope}%
\begin{pgfscope}%
\pgfpathrectangle{\pgfqpoint{0.750000in}{0.500000in}}{\pgfqpoint{4.650000in}{3.020000in}}%
\pgfusepath{clip}%
\pgfsetrectcap%
\pgfsetroundjoin%
\pgfsetlinewidth{1.505625pt}%
\definecolor{currentstroke}{rgb}{0.549020,0.337255,0.294118}%
\pgfsetstrokecolor{currentstroke}%
\pgfsetdash{}{0pt}%
\pgfpathmoveto{\pgfqpoint{0.750000in}{1.255000in}}%
\pgfpathlineto{\pgfqpoint{2.057958in}{1.256539in}}%
\pgfpathlineto{\pgfqpoint{2.146396in}{1.259841in}}%
\pgfpathlineto{\pgfqpoint{2.206907in}{1.264986in}}%
\pgfpathlineto{\pgfqpoint{2.253453in}{1.271860in}}%
\pgfpathlineto{\pgfqpoint{2.290691in}{1.280108in}}%
\pgfpathlineto{\pgfqpoint{2.323273in}{1.290040in}}%
\pgfpathlineto{\pgfqpoint{2.355856in}{1.303215in}}%
\pgfpathlineto{\pgfqpoint{2.383784in}{1.317676in}}%
\pgfpathlineto{\pgfqpoint{2.411712in}{1.335633in}}%
\pgfpathlineto{\pgfqpoint{2.439640in}{1.357662in}}%
\pgfpathlineto{\pgfqpoint{2.467568in}{1.384360in}}%
\pgfpathlineto{\pgfqpoint{2.490841in}{1.410600in}}%
\pgfpathlineto{\pgfqpoint{2.514114in}{1.440817in}}%
\pgfpathlineto{\pgfqpoint{2.542042in}{1.482741in}}%
\pgfpathlineto{\pgfqpoint{2.569970in}{1.531239in}}%
\pgfpathlineto{\pgfqpoint{2.597898in}{1.586602in}}%
\pgfpathlineto{\pgfqpoint{2.625826in}{1.648946in}}%
\pgfpathlineto{\pgfqpoint{2.658408in}{1.730363in}}%
\pgfpathlineto{\pgfqpoint{2.690991in}{1.820552in}}%
\pgfpathlineto{\pgfqpoint{2.728228in}{1.932919in}}%
\pgfpathlineto{\pgfqpoint{2.779429in}{2.098920in}}%
\pgfpathlineto{\pgfqpoint{2.872523in}{2.404216in}}%
\pgfpathlineto{\pgfqpoint{2.905105in}{2.500929in}}%
\pgfpathlineto{\pgfqpoint{2.933033in}{2.575335in}}%
\pgfpathlineto{\pgfqpoint{2.956306in}{2.629770in}}%
\pgfpathlineto{\pgfqpoint{2.979580in}{2.676159in}}%
\pgfpathlineto{\pgfqpoint{2.998198in}{2.706833in}}%
\pgfpathlineto{\pgfqpoint{3.016817in}{2.731338in}}%
\pgfpathlineto{\pgfqpoint{3.030781in}{2.745464in}}%
\pgfpathlineto{\pgfqpoint{3.044745in}{2.755823in}}%
\pgfpathlineto{\pgfqpoint{3.058709in}{2.762333in}}%
\pgfpathlineto{\pgfqpoint{3.072673in}{2.764946in}}%
\pgfpathlineto{\pgfqpoint{3.086637in}{2.763639in}}%
\pgfpathlineto{\pgfqpoint{3.100601in}{2.758424in}}%
\pgfpathlineto{\pgfqpoint{3.114565in}{2.749340in}}%
\pgfpathlineto{\pgfqpoint{3.128529in}{2.736459in}}%
\pgfpathlineto{\pgfqpoint{3.142492in}{2.719880in}}%
\pgfpathlineto{\pgfqpoint{3.161111in}{2.692243in}}%
\pgfpathlineto{\pgfqpoint{3.179730in}{2.658633in}}%
\pgfpathlineto{\pgfqpoint{3.203003in}{2.608903in}}%
\pgfpathlineto{\pgfqpoint{3.226276in}{2.551548in}}%
\pgfpathlineto{\pgfqpoint{3.254204in}{2.474253in}}%
\pgfpathlineto{\pgfqpoint{3.286787in}{2.375076in}}%
\pgfpathlineto{\pgfqpoint{3.337988in}{2.207662in}}%
\pgfpathlineto{\pgfqpoint{3.421772in}{1.932919in}}%
\pgfpathlineto{\pgfqpoint{3.463664in}{1.807167in}}%
\pgfpathlineto{\pgfqpoint{3.496246in}{1.718176in}}%
\pgfpathlineto{\pgfqpoint{3.528829in}{1.638072in}}%
\pgfpathlineto{\pgfqpoint{3.556757in}{1.576890in}}%
\pgfpathlineto{\pgfqpoint{3.584685in}{1.522685in}}%
\pgfpathlineto{\pgfqpoint{3.612613in}{1.475307in}}%
\pgfpathlineto{\pgfqpoint{3.640541in}{1.434440in}}%
\pgfpathlineto{\pgfqpoint{3.668468in}{1.399644in}}%
\pgfpathlineto{\pgfqpoint{3.696396in}{1.370390in}}%
\pgfpathlineto{\pgfqpoint{3.724324in}{1.346102in}}%
\pgfpathlineto{\pgfqpoint{3.752252in}{1.326182in}}%
\pgfpathlineto{\pgfqpoint{3.780180in}{1.310043in}}%
\pgfpathlineto{\pgfqpoint{3.808108in}{1.297124in}}%
\pgfpathlineto{\pgfqpoint{3.840691in}{1.285429in}}%
\pgfpathlineto{\pgfqpoint{3.877928in}{1.275622in}}%
\pgfpathlineto{\pgfqpoint{3.919820in}{1.268023in}}%
\pgfpathlineto{\pgfqpoint{3.971021in}{1.262193in}}%
\pgfpathlineto{\pgfqpoint{4.036186in}{1.258213in}}%
\pgfpathlineto{\pgfqpoint{4.129279in}{1.255921in}}%
\pgfpathlineto{\pgfqpoint{4.320120in}{1.255050in}}%
\pgfpathlineto{\pgfqpoint{5.400000in}{1.255000in}}%
\pgfpathlineto{\pgfqpoint{5.400000in}{1.255000in}}%
\pgfusepath{stroke}%
\end{pgfscope}%
\begin{pgfscope}%
\pgfsetrectcap%
\pgfsetmiterjoin%
\pgfsetlinewidth{0.803000pt}%
\definecolor{currentstroke}{rgb}{0.000000,0.000000,0.000000}%
\pgfsetstrokecolor{currentstroke}%
\pgfsetdash{}{0pt}%
\pgfpathmoveto{\pgfqpoint{0.750000in}{0.500000in}}%
\pgfpathlineto{\pgfqpoint{0.750000in}{3.520000in}}%
\pgfusepath{stroke}%
\end{pgfscope}%
\begin{pgfscope}%
\pgfsetrectcap%
\pgfsetmiterjoin%
\pgfsetlinewidth{0.803000pt}%
\definecolor{currentstroke}{rgb}{0.000000,0.000000,0.000000}%
\pgfsetstrokecolor{currentstroke}%
\pgfsetdash{}{0pt}%
\pgfpathmoveto{\pgfqpoint{5.400000in}{0.500000in}}%
\pgfpathlineto{\pgfqpoint{5.400000in}{3.520000in}}%
\pgfusepath{stroke}%
\end{pgfscope}%
\begin{pgfscope}%
\pgfsetrectcap%
\pgfsetmiterjoin%
\pgfsetlinewidth{0.803000pt}%
\definecolor{currentstroke}{rgb}{0.000000,0.000000,0.000000}%
\pgfsetstrokecolor{currentstroke}%
\pgfsetdash{}{0pt}%
\pgfpathmoveto{\pgfqpoint{0.750000in}{0.500000in}}%
\pgfpathlineto{\pgfqpoint{5.400000in}{0.500000in}}%
\pgfusepath{stroke}%
\end{pgfscope}%
\begin{pgfscope}%
\pgfsetrectcap%
\pgfsetmiterjoin%
\pgfsetlinewidth{0.803000pt}%
\definecolor{currentstroke}{rgb}{0.000000,0.000000,0.000000}%
\pgfsetstrokecolor{currentstroke}%
\pgfsetdash{}{0pt}%
\pgfpathmoveto{\pgfqpoint{0.750000in}{3.520000in}}%
\pgfpathlineto{\pgfqpoint{5.400000in}{3.520000in}}%
\pgfusepath{stroke}%
\end{pgfscope}%
\begin{pgfscope}%
\pgfsetbuttcap%
\pgfsetmiterjoin%
\definecolor{currentfill}{rgb}{1.000000,1.000000,1.000000}%
\pgfsetfillcolor{currentfill}%
\pgfsetfillopacity{0.800000}%
\pgfsetlinewidth{1.003750pt}%
\definecolor{currentstroke}{rgb}{0.800000,0.800000,0.800000}%
\pgfsetstrokecolor{currentstroke}%
\pgfsetstrokeopacity{0.800000}%
\pgfsetdash{}{0pt}%
\pgfpathmoveto{\pgfqpoint{2.649058in}{0.569444in}}%
\pgfpathlineto{\pgfqpoint{3.500942in}{0.569444in}}%
\pgfpathquadraticcurveto{\pgfqpoint{3.528719in}{0.569444in}}{\pgfqpoint{3.528719in}{0.597222in}}%
\pgfpathlineto{\pgfqpoint{3.528719in}{1.745061in}}%
\pgfpathquadraticcurveto{\pgfqpoint{3.528719in}{1.772839in}}{\pgfqpoint{3.500942in}{1.772839in}}%
\pgfpathlineto{\pgfqpoint{2.649058in}{1.772839in}}%
\pgfpathquadraticcurveto{\pgfqpoint{2.621281in}{1.772839in}}{\pgfqpoint{2.621281in}{1.745061in}}%
\pgfpathlineto{\pgfqpoint{2.621281in}{0.597222in}}%
\pgfpathquadraticcurveto{\pgfqpoint{2.621281in}{0.569444in}}{\pgfqpoint{2.649058in}{0.569444in}}%
\pgfpathclose%
\pgfusepath{stroke,fill}%
\end{pgfscope}%
\begin{pgfscope}%
\pgfsetrectcap%
\pgfsetroundjoin%
\pgfsetlinewidth{1.505625pt}%
\definecolor{currentstroke}{rgb}{0.121569,0.466667,0.705882}%
\pgfsetstrokecolor{currentstroke}%
\pgfsetdash{}{0pt}%
\pgfpathmoveto{\pgfqpoint{2.676836in}{1.668672in}}%
\pgfpathlineto{\pgfqpoint{2.954614in}{1.668672in}}%
\pgfusepath{stroke}%
\end{pgfscope}%
\begin{pgfscope}%
\pgftext[x=3.065725in,y=1.620061in,left,base]{\rmfamily\fontsize{10.000000}{12.000000}\selectfont \(\displaystyle  n = 2 \)}%
\end{pgfscope}%
\begin{pgfscope}%
\pgfsetrectcap%
\pgfsetroundjoin%
\pgfsetlinewidth{1.505625pt}%
\definecolor{currentstroke}{rgb}{1.000000,0.498039,0.054902}%
\pgfsetstrokecolor{currentstroke}%
\pgfsetdash{}{0pt}%
\pgfpathmoveto{\pgfqpoint{2.676836in}{1.475061in}}%
\pgfpathlineto{\pgfqpoint{2.954614in}{1.475061in}}%
\pgfusepath{stroke}%
\end{pgfscope}%
\begin{pgfscope}%
\pgftext[x=3.065725in,y=1.426450in,left,base]{\rmfamily\fontsize{10.000000}{12.000000}\selectfont \(\displaystyle  n = 4 \)}%
\end{pgfscope}%
\begin{pgfscope}%
\pgfsetrectcap%
\pgfsetroundjoin%
\pgfsetlinewidth{1.505625pt}%
\definecolor{currentstroke}{rgb}{0.172549,0.627451,0.172549}%
\pgfsetstrokecolor{currentstroke}%
\pgfsetdash{}{0pt}%
\pgfpathmoveto{\pgfqpoint{2.676836in}{1.281450in}}%
\pgfpathlineto{\pgfqpoint{2.954614in}{1.281450in}}%
\pgfusepath{stroke}%
\end{pgfscope}%
\begin{pgfscope}%
\pgftext[x=3.065725in,y=1.232839in,left,base]{\rmfamily\fontsize{10.000000}{12.000000}\selectfont \(\displaystyle  n = 6 \)}%
\end{pgfscope}%
\begin{pgfscope}%
\pgfsetrectcap%
\pgfsetroundjoin%
\pgfsetlinewidth{1.505625pt}%
\definecolor{currentstroke}{rgb}{0.839216,0.152941,0.156863}%
\pgfsetstrokecolor{currentstroke}%
\pgfsetdash{}{0pt}%
\pgfpathmoveto{\pgfqpoint{2.676836in}{1.087839in}}%
\pgfpathlineto{\pgfqpoint{2.954614in}{1.087839in}}%
\pgfusepath{stroke}%
\end{pgfscope}%
\begin{pgfscope}%
\pgftext[x=3.065725in,y=1.039228in,left,base]{\rmfamily\fontsize{10.000000}{12.000000}\selectfont \(\displaystyle  n = 8 \)}%
\end{pgfscope}%
\begin{pgfscope}%
\pgfsetrectcap%
\pgfsetroundjoin%
\pgfsetlinewidth{1.505625pt}%
\definecolor{currentstroke}{rgb}{0.580392,0.403922,0.741176}%
\pgfsetstrokecolor{currentstroke}%
\pgfsetdash{}{0pt}%
\pgfpathmoveto{\pgfqpoint{2.676836in}{0.894228in}}%
\pgfpathlineto{\pgfqpoint{2.954614in}{0.894228in}}%
\pgfusepath{stroke}%
\end{pgfscope}%
\begin{pgfscope}%
\pgftext[x=3.065725in,y=0.845617in,left,base]{\rmfamily\fontsize{10.000000}{12.000000}\selectfont \(\displaystyle  n = 10 \)}%
\end{pgfscope}%
\begin{pgfscope}%
\pgfsetrectcap%
\pgfsetroundjoin%
\pgfsetlinewidth{1.505625pt}%
\definecolor{currentstroke}{rgb}{0.549020,0.337255,0.294118}%
\pgfsetstrokecolor{currentstroke}%
\pgfsetdash{}{0pt}%
\pgfpathmoveto{\pgfqpoint{2.676836in}{0.700617in}}%
\pgfpathlineto{\pgfqpoint{2.954614in}{0.700617in}}%
\pgfusepath{stroke}%
\end{pgfscope}%
\begin{pgfscope}%
\pgftext[x=3.065725in,y=0.652006in,left,base]{\rmfamily\fontsize{10.000000}{12.000000}\selectfont \(\displaystyle f_2\)}%
\end{pgfscope}%
\end{pgfpicture}%
\makeatother%
\endgroup%
}
\caption{Interpolation polynomials of degree $n$ to $f_2$ using zeros of $T_n$} \label{Fig:CheExp}
\end{figure}
Note that polynomials of odd and even orders behave differently, and therefore plots are given separately.
\end{thmanswer}
\end{thmquestion}

\begin{thmquestion}
\
\begin{thmanswer}
The matrix is
\begin{equation}
A = \msbr{ u_1 & h_1 & & & & & \\ h_1 & u_2 & h_2 & & & \\ & h_2 & u_3 & h_3 & & \\ & & \ddots & \ddots & \ddots & \\ & & & h_{ n - 3 } & u_{ n - 2 } & h_{ n - 2 } \\ & & & & h_{ n - 2 } & u_{ n - 1 } }.
\end{equation}
Note that because $ u_1 = 2 \rbr{ h_0 + h_1 } > h_1 $, $ u_i = 2 \rbr{ h_i + h_{ i - 1 } } > h_i + h_{ i - 1 } \crbr{ i = 2, 3, \cdots, n - 2 } $, $ u_{ n - 1 } = 2 \rbr{ h_{ n - 2 } + h_{ n - 1 } } > h_{ n - 2 } $, therefore $A$ is strictly diagonally dominant. From the proof on the slide, one knows LU-decomposition can be perform on $A$. That is, $ A = L U $ with lower triangular $L$ and upper triangular $U$. From the slide, one knows the diagonal entries of $U$ can be set to $1$ and that of $L$ are all non-zero. Therefore, $L$ and $U$ are non-singular and consequently $A$ is invertible.

\sqed
\end{thmanswer}
\end{thmquestion}

\begin{thmquestion}
\
\begin{thmanswer}
Note that there are $ 3 n $ variables for a quadratic spline with $ n + 1 $ nodes, while the number of constraints are $ 2 \rbr{ n - 1 } + \rbr{ n - 1 } + 2 = 3 n - 1 $, where $ 2 \rbr{ n - 1 } $ stands for values at internal nodes, $ \rbr{ n - 1 } $ stands for the continuity of first-order derivative and $2$ stands for values at end points. Therefore, there are (at least) one degree of freedom and one specific constraint should be forced for uniqueness. For example, such constraint may be that the first-order derivative at one end point is fixed, or that first-order derivatives at both end points coincides. Additionally, one may also consider the spline to be an optimization problem. For example, one may try to minimize
\begin{equation}
E = \intb{t_0}{t_{n}}{ \rbr{ S' \rbr{x} }^2 \sd x },
\end{equation}
which stands for the elastic energy if the spline is thought to be a thin stick.
\end{thmanswer}
\end{thmquestion}

\begin{thmquestion}
\
\begin{thmanswer}
Assume $u$ to be $C^4$. Therefore, we have
\begin{align}
u_{ i - 1 } &= u \rbr{ x - h } = u \rbr{x} - h u' \rbr{x} + \frac{h^2}{2} u'' \rbr{x} - \frac{h^3}{6} u^{\rbr{3}} \rbr{x} + O \rbr{h^4}, \\
u_{ i + 1 } &= u \rbr{ x + h } = u \rbr{x} + h u' \rbr{x} + \frac{h^2}{2} u'' \rbr{x} + \frac{h^3}{6} u^{\rbr{3}} \rbr{x} + O \rbr{h^4}, \\
\end{align}
and therefore
\begin{equation}
u_{ i - 1 } - 2 u_i + u_{ i + 1 } = h^2 u'' \rbr{x} + O \rbr{h^4},
\end{equation}
which yields
\begin{equation}
\frac{ \sd[2] \rbr{u} }{ \sd x^2 } - \frac{ u_{ i - 1 } - 2 u_i + u_{ i + 1 } }{h^2} = O \rbr{h^2}.
\end{equation}

\sqed
\end{thmanswer}
\end{thmquestion}

\end{document}
