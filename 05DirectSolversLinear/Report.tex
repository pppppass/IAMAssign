% !TeX encoding = UTF-8
% !TeX program = LuaLaTeX
% !TeX spellcheck = en_US

% Author : Zhihan Li
% Description : Report for Lecture 5 --- Direct Solvers to Linear Systems

\documentclass[english, nochinese]{../textmpls/pkupaper}

\usepackage[paper, cmrgreekup]{../textmpls/def}

\newcommand{\cuniversity}{Peking University}
\newcommand{\cthesisname}{Introduction to Applied Mathematics}
\newcommand{\titlemark}{Assignment for Lecture 5}

\DeclareRobustCommand{\authoring}%
{%
\begin{tabular}{c}%
Zhihan Li \\%
1600010653%
\end{tabular}%
}

\title{\titlemark}
\author{\authoring}
\begin{document}

\maketitle

\begin{thmquestion}
\ 
\begin{thmproof}
Because the first row of $A$ and $A^{\rbr{1}}$ are identical, therefore
\begin{equation}
\abs{A^{\rbr{1}}_{11}} \ge \sume{k}{2}{n}{\abs{A^{\rbr{1}}_{1k}}}
\end{equation}
can be established.

For $ u = 2, 3, \cdots, n $, we have
\begin{equation}
A^{\rbr{1}}_{ u k } = A_{ u k } - \frac{ A_{ u 1 } A_{ 1 k } }{A_{ 1 1 }}
\end{equation}
with $ A^{\rbr{1}}_{ u 1 } = 0 $. Because
\begin{equation}
\begin{split}
\abs{A^{\rbr{1}}_{ u u }} &= \abs{ A_{ u u } - \frac{ A_{ u 1 } A_{ 1 u } }{A_{ 1 1 }} } \ge \abs{ A_{ u u } } - \abs{\frac{ A_{ u 1 } A_{ 1 u } }{A_{ 1 1 }}} \\
&\ge \sumb{\sarr{c}{ k = 1 \\ k \neq u }}{n}{\abs{A_{ u k }}} + \sumb{\sarr{c}{ k = 2 \\ k \neq u }}{n}{\abs{\frac{ A_{ u 1 } A_{ 1 k } }{A_{ 1 1 }}}} - \abs{A_{ u 1 }} \\
&= \sumb{\sarr{c}{ k = 2 \\ k \neq u }}{n}{\rbr{ \abs{A_{ u k }} + \abs{\frac{ A_{ u 1 } A_{ 1 k } }{A_{ 1 1 }}} }} \ge \sumb{\sarr{c}{ k = 2 \\ k \neq u }}{n}{ \abs{ A_{ u k } - \frac{ A_{ u 1 } A_{ 1 k } }{A_{ 1 1 }} } } \\
&= \sumb{\sarr{c}{ k = 1 \\ k \neq u }}{n}{\abs{A^{\rbr{1}}_{ u k }}}.
\end{split}
\end{equation}

From the argument above, we can conclude that $A^{\rbr{1}}$ is diagonally dominant.

\sqed
\end{thmproof}
\end{thmquestion}

\begin{thmquestion}
\ 
\begin{thmanswer}
The results are shown in Table \ref{Tbl:Gauss}.

\begin{table}[htbp]
\centering
\begin{tabular}{|c|c|c|}
\hline
$n$ & $ \norm{ x^{\ast} - x }_2 $ & $ \norm{ x^{\ast} - x }_{\infty} $ \\
\hline
2 & 2.22045e-16 & 2.22045e-16 \\
\hline
12 & 1.53738e-13 & 1.13687e-13 \\
\hline
24 & 6.29651e-10 & 4.65633e-10 \\
\hline
48 & 1.05638e-02 & 7.81202e-03 \\
\hline
84 & 7.25938e+08 & 5.36838e+08 \\
\hline
\end{tabular}
\caption{Errors between $x^{\ast}$ and $x$ for different $n$ using Gaussian elimination}
\label{Tbl:Gauss}
\end{table}
Source codes are given in \verb"Problem2.ipynb".

It can be seen that the error goes large for increasing $n$.
\end{thmanswer}
\end{thmquestion}

\begin{thmquestion}
\ 
\begin{thmproof}
From properties of Gaussian elimination with full pivoting, the $i$-th row of $U$ is identical to that of
\begin{equation}
L_i P_i \cdots L_1 P_1 A Q_1 \cdots Q_i.
\end{equation}
Therefore, followed by properties of Gaussian transform, it is also identical to
\begin{equation}
P_i \cdots L_1 P_1 A Q_1 \cdots Q_i =: \widetilde{U}^{\rbr{i}}.
\end{equation}
Because $P_i$ and $Q_i$ swap the element of maximum absolute value to the $ \rbr{ i, i } $ entry, therefore
\begin{equation}
\abs{\widetilde{U}^{\rbr{i}}_{ i i }} \ge \abs{\widetilde{U}^{\rbr{i}}_{ i j }}
\end{equation}
for $ j > i $ and consequently
\begin{equation}
\abs{U_{ i i }} \ge \abs{U_{ i j }}.
\end{equation}

\sqed
\end{thmproof}
\end{thmquestion}

\begin{thmquestion}
\ 
\begin{thmproof}
From the definition of Gaussian elimination, there exists lower triangular $ k \times k $ matrix $ L_{ 1 1 } $ and $ \rbr{ n - k } \times k $ matrix such that
\begin{equation}
\msbr{ L_{ 1 1 } & \\ L_{ 2 1 } & I } \msbr{ A_{ 1 1 } & A_{ 1 2 } \\ A_{ 2 1 } & A_{ 2 2 } } = \msbr{ A^{\rbr{k}}_{ 1 1 } & A^{\rbr{k}}_{ 1 2 } \\ & A^{\rbr{k}}_{ 2 2 } }.
\end{equation}
Consequently,
\begin{gather}
L_{ 2 1 } A_{ 1 1 } + A_{ 2 1 } = 0, \\
L_{ 2 1 } A_{ 1 2 } + A_{ 2 2 } = A^{\rbr{k}}_{ 2 2 },
\end{gather}
which means (note that $ A_{ 1 1 } $ is invertible because Gaussian elimination can be conducted)
\begin{gather}
L_{ 2 1 } = -A_{ 2 1 } A_{ 1 1 }^{-1},
A^{\rbr{k}}_{ 2 2 } = A_{ 2 2 } - A_{ 2 1 } A_{ 1 1 }^{-1} A_{ 1 2 }
\end{gather}
as desired.

\sqed
\end{thmproof}
\end{thmquestion}

\begin{thmquestion}
\ 
\begin{thmproof}
Suppose there exists upper triangular matrices $U_1$, $U_2$ and lower triangular matrices with diagonal $1$ $L_1$, $L_2$ such that
\begin{equation}
A = L_1 U_1 = L_2 U_2.
\end{equation}
Because principle submatrices of $A$ are non-vanishing, diagonal entries of $U_1$ and $U_2$ are all non-zero.
Therefore
\begin{equation}
L_2^{-1} L_1 = U_2 U_1^{-1},
\end{equation}
where $ L_2^{-1} L_1 $ are lower triangular matrices with diagonal $1$ and $ U_2 U_1^{-1} $ are upper triangular matrices. Comparing matrix shape of two sides, we have
\begin{equation}
L_2^{-1} L_1 = U_2 U_1^{-1} = I,
\end{equation}
which means exactly $ L_1 = L_2 $, $ U_1 = U_2 $, the uniqueness.

\sqed
\end{thmproof}
\end{thmquestion}

\end{document}
