% !TeX encoding = UTF-8
% !TeX program = LuaLaTeX
% !TeX spellcheck = en_US

% Author : Zhihan Li
% Description : Report for Lecture 7 --- Numerical Integrals

\documentclass[english, nochinese]{../textmpls/pkupaper}

\usepackage[paper]{../textmpls/def}

\newcommand{\cuniversity}{Peking University}
\newcommand{\cthesisname}{Introduction to Applied Mathematics}
\newcommand{\titlemark}{Assignment for Lecture 7}

\DeclareRobustCommand{\authoring}%
{%
\begin{tabular}{c}%
Zhihan Li \\%
1600010653%
\end{tabular}%
}

\title{\titlemark}
\author{\authoring}
\begin{document}

\maketitle

\begin{thmquestion}
\ 
\begin{thmproof}
Suppose $ f \in C^4 \sbr{ a, b } $.

Assume $a$ and $b$ to be $-1$ and $1$ respectively, and it remains to prove the existence of $ \xi \in \sbr{ -1, 1 } $ such that
\begin{equation}
\intb{-1}{1}{ f \rbr{x} \sd x } - \frac{1}{3} \rbr{ f \rbr{-1} + 4 f \rbr{0} + f \rbr{1} } = - \frac{1}{90} f^{\rbr{4}} \rbr{\xi}.
\end{equation}

Consider
\begin{equation}
\begin{split}
P \rbr{x} &= \frac{1}{2} f \rbr{-1} x \rbr{ x - 1 } - f \rbr{0} \rbr{ x + 1 } \rbr{ x - 1 } + \frac{1}{2} f \rbr{1} \rbr{ x + 1 } x \\
&+ \rbr{ -\frac{1}{2} f \rbr{-1} - \frac{1}{2} f \rbr{1} - f^{\rbr{1}} \rbr{0} } \rbr{ x + 1 } x \rbr{ x - 1 }.
\end{split}
\end{equation}
and $ d \rbr{x} := f \rbr{x} - P \rbr{x} $. It can be verified that $d$ satisfies $ d \rbr{-1} = d \rbr{0} = d \rbr{1} = d' \rbr{0} = 0 $. For $ x \in \rbr{ -1, 0 } \cup \rbr{ 0, 1 } $, there exists (proof postponed) $ \xi_x \in \sbr{ -1, 1 } $, such that
\begin{equation}
f \rbr{x} - P \rbr{x} = \frac{1}{24} x^2 \rbr{ x - 1 } \rbr{ x + 1 } f^{\rbr{4}} \rbr{\xi_x}.
\end{equation}
Therefore from continuity of $ f^{\rbr{4}} $ there exists $\xi_3$, such that
\begin{equation}
\begin{split}
&\ptrel{=} \intb{-1}{1}{ f \rbr{x} \sd x } - \frac{1}{3} \rbr{ f \rbr{-1} + 4 f \rbr{0} + f \rbr{1} } \\
&= \intb{-1}{1}{ f \rbr{x} \sd x } - \intb{-1}{1}{ P \rbr{x} \sd x } \\
&= \intb{-1}{1}{ \frac{1}{24} x^2 \rbr{ x - 1 } \rbr{ x + 1 } f^{\rbr{4}} \rbr{\xi_x} \sd x } \\
&= \frac{1}{24} \intb{-1}{1}{ x^2 \rbr{ x - 1 } \rbr{ x + 1 } \sd x } f^{\rbr{4}} \rbr{\xi} \\
&= -\frac{1}{90} f^{\rbr{4}} \rbr{\xi}
\end{split}
\end{equation}
as desired.

Existence of $\xi_x$: Without loss of generality, assume $ x \in \rbr{ 0, 1 } $. Consider
\begin{equation}
Q \rbr{y} = f \rbr{y} - P \rbr{y} - \frac{ f \rbr{x} - P \rbr{x} }{ x^2 \rbr{ x - 1 } \rbr{ x + 1 } } y^2 \rbr{ y - 1 } \rbr{ y + 1 },
\end{equation}
which satisfies $ Q \rbr{-1} = Q \rbr{0} = Q \rbr{1} = Q' \rbr{0} = Q \rbr{x} = 0 $. Applying Rolle's mean value theorem, there exists $ \xi_x \in \sbr{ -1, 1 } $ such that $ Q^{\rbr{4}} \rbr{\xi_x} = 0 $, which is equivalent to
\begin{equation}
f \rbr{x} - P \rbr{x} = \frac{1}{24} f^{\rbr{4}} \rbr{\xi} x^2 \rbr{ x - 1 } \rbr{ x + 1 }.
\end{equation}

\sqed
\end{thmproof}
\end{thmquestion}

\begin{thmquestion}
\ 
\begin{thmproof}
Let the (equally spaced) interpolation nodes be $ x_0 = a, x_1, \cdots, x_n = b $ respectively,
\begin{equation}
\phi_k \rbr{x} = \prodb{\sarr{c}{ j = 0 \\ j \neq k }}{n}{\frac{ x - x_j }{ x_k - x_j }},
\end{equation}
and
\begin{equation}
P_n \rbr{x} = \sume{k}{0}{n}{ f \rbr{x_k} \phi_k \rbr{x} }
\end{equation}
and therefore the Newton-Cotes formula can be written as
\begin{equation}
I = \intb{a}{b}{ P_n \rbr{x} \sd x } = \sume{k}{0}{n}{ f \rbr{x_k} \intb{a}{b}{ \phi_k \rbr{x} \sd x } }.
\end{equation}

When $ f \rbr{x} = x^n $, Lagrangian interpolation yields $ f = P_n $, and therefore
\begin{equation}
I = \intb{a}{b}{ P_n \rbr{x} \sd x } = \intb{a}{b}{ f \rbr{x} \sd x }
\end{equation}
exactly.

If $n$ is even, then $ n + 1 $ is odd. It suffices to prove that for the case $ f \rbr{x} = \rbr{ x - \frac{ a + b }{2} }^{ n + 1 } $,
\begin{equation}
I = \intb{a}{b}{ P_n \rbr{x} \sd x } = \intb{a}{b}{ f \rbr{x} \sd x } = 0.
\end{equation}
Note that $ \phi_k \rbr{x} = \phi_{ n - k } \rbr{ a + b - x } $ follows from definition, and therefore
\begin{equation}
\begin{split}
2 I &= \sume{k}{0}{n}{ f \rbr{x_k} \rbr{ \intb{a}{b}{ \phi_k \rbr{x} \sd x } + \intb{a}{b}{ \phi_{ n - k } \rbr{ a + b - k } \sd x } } } \\
&= \sume{k}{0}{n}{ \rbr{ f \rbr{x_k} + f \rbr{x_{ n - k }} } \intb{a}{b}{ \phi_k \rbr{x} \sd x } } = 0
\end{split}
\end{equation}
as desired.

\sqed
\end{thmproof}
\end{thmquestion}

\begin{thmquestion}
\ 
\begin{thmanswer}
The result is shown in table \ref{Tbl:Res}. Error corresponds to $ \abs{ I - \pi } $ here.

\begin{table}[htbp]
\centering
\caption{Errors of different integration methods}
\label{Tbl:Res}
\begin{tabular}{|c|c|c|c|c|c|c|}
\hline
$ 1 / h $ & Trapezoid & Simpson & Romberg 3 & Romberg 4 & Romberg 5 & Gauss 2 \\
\hline
1 & 1.41593e-01 & 8.25932e-03 & 5.24993e-04 & 6.86983e-06 & 1.16879e-08 & 5.94833e-03 \\
\hline
2 & 4.15927e-02 & 2.40261e-05 & 1.44054e-06 & 1.51930e-08 & 5.98170e-11 & 1.72394e-05 \\
\hline
3 & 1.85157e-02 & 8.72654e-07 & 4.40069e-08 & 2.38405e-11 & 9.85878e-14 & 6.20794e-07 \\
\hline
4 & 1.04162e-02 & 1.51131e-07 & 7.55277e-09 & 2.35811e-13 & 8.88178e-16 & 1.07469e-07 \\
\hline
5 & 6.66654e-03 & 3.96506e-08 & 1.98203e-09 & 3.10862e-14 & 0.00000e+00 & 2.81956e-08 \\
\hline
8 & 2.60416e-03 & 2.36497e-09 & 1.18244e-10 & 8.88178e-16 & 4.44089e-16 & 1.68175e-09 \\
\hline
10 & 1.66666e-03 & 6.20008e-10 & 3.09996e-11 & 4.44089e-16 & 0.00000e+00 & 4.40895e-10 \\
\hline
20 & 4.16667e-04 & 9.68825e-12 & 4.84501e-13 & 0.00000e+00 & 4.44089e-16 & 6.88960e-12 \\
\hline
30 & 1.85185e-04 & 8.49987e-13 & 4.21885e-14 & 4.44089e-16 & 0.00000e+00 & 6.04850e-13 \\
\hline
40 & 1.04167e-04 & 1.51879e-13 & 7.99361e-15 & 0.00000e+00 & 0.00000e+00 & 1.07914e-13 \\
\hline
50 & 6.66667e-05 & 3.95239e-14 & 2.22045e-15 & 0.00000e+00 & 0.00000e+00 & 2.84217e-14 \\
\hline
\end{tabular}
\end{table}
\end{thmanswer}
\end{thmquestion}

\begin{thmquestion}
\ 
\begin{thmproof}
Suppose zeros of $\phi_i$ are $ x_1 < x_2 < \cdots < x_k $ with $ k < i $, and without loss of generality assume $ \phi_i, -\phi_i, \cdots, \rbr{-1}^k \phi_i $ are positive on $ \rsbr{ x_k, b }, \rbr{ x_{ k - 1 }, x_k }, \cdots, \rbr{ x_1, x_2 }, \srbr{ a , x_1 } $ respectively. (We may change the sign of $\phi_i$ by multiplying $-1$, and ignore repeated zeros away if signs do not change around $\phi_i$) Consider
\begin{equation}
\psi \rbr{x} = \rbr{ x - x_1 } \rbr{ x - x_2 } \cdots \rbr{ x - x_k }.
\end{equation}
Because $ \deg g = k < i $ and consequently
\begin{equation}
\psi \in \opspan \pbr{ \phi_0, \phi_1, \cdots, \phi_{ i - 1 } },
\end{equation}
therefore
\begin{equation}
\intb{a}{b}{ \rho \rbr{x} \phi_i \rbr{x} \psi \rbr{x} \sd x } = 0.
\end{equation}
However, this contradicts $ \rho \phi_i \psi $ being continuous and non-negative, with finite points reaching equality.

As a result, $ k \ge i $. Note that $ k > i $ can be ruled out because of the degree constraint, and therefore we can conclude $ k = i $.

\sqed
\end{thmproof}
\end{thmquestion}

\begin{thmquestion}
\ 
\begin{thmproof}
Because $L_n$ is $n$-th Legendre polynomial, therefore
\begin{equation}
f_i \rbr{x} = \prodb{\sarr{c}{ j = 1 \\ j \neq i }}{n}{\frac{ x - x_j }{ x_i - x_j }}
\end{equation}
satisfies $ \deg f_i = 2 n - 2 \le 2 n - 1 $ and the quadrature is exact for $f_i$. Note that the quadrature gives
\begin{equation}
\begin{split}
I &= \sume{k}{1}{n}{ A_k l_i \rbr{x_k}} = \sume{k}{1}{n}{ A_k \delta_{ i k } } = A_i \\
&= \intb{a}{b}{ f_i \rbr{x} \sd x } > 0
\end{split}
\end{equation}
as desired.

\sqed
\end{thmproof}
\end{thmquestion}

\begin{thmquestion}
\ 
\begin{thmproof}
It suffices to prove the case $ a = 0 $, $ b = 1 $.

Because
\begin{gather}
\intb{a}{b}{ 1 \sd x } = 1 = \frac{1}{6} \rbr{ 1 + 4 + 1 }, \\
\intb{a}{b}{ x \sd x } = \frac{1}{2} = \frac{1}{6} \rbr{ 0 + 4 \cdot \frac{1}{2} + 1 }, \\
\intb{a}{b}{ x^2 \sd x } = \frac{1}{3} = \frac{1}{6}{ 0 + 4 \cdot \frac{1}{4} + 1 }, \\
\intb{a}{b}{ x^3 \sd x } = \frac{1}{4} = \frac{1}{6}{ 0 + 4 \cdot \frac{1}{8} + 1 }, \\
\intb{a}{b}{ x^4 \sd x } = \frac{1}{5} \neq \frac{5}{24} = \frac{1}{6}{ 0 + 4 \cdot \frac{1}{16} + 1 },
\end{gather}
therefore the quadrature applies for polynomials of degree no greater than $3$ exactly. That is, algebraic precision of the quadrature is $3$.

\sqed
\end{thmproof}
\end{thmquestion}

\end{document}
