% !TeX encoding = UTF-8
% !TeX program = lualatex
% !TeX spellcheck = en_US

% Author : Zhihan Li
% Description : Report for Lecture 4 --- Best Approximation

\documentclass[english, nochinese]{../TeXTemplate/pkupaper}

\usepackage[paper, cmrgreekup]{../TeXTemplate/def}

\newcommand{\cuniversity}{Peking University}
\newcommand{\cthesisname}{Introduction to Applied Mathematics}
\newcommand{\titlemark}{Assignment for Lecture 4}

\DeclareRobustCommand{\authoring}%
{%
\begin{tabular}{c}%
Zhihan Li \\%
1600010653%
\end{tabular}%
}

\title{\titlemark}
\author{\authoring}
\begin{document}

\maketitle

\begin{thmquestion}
\
\begin{thmproof}
Prove by contradiction. Otherwise, there exists $ c = \msbr{ c_1 & c_2 & \cdots & c_m }^{\rmut} $, such that $ A c = 0 $, and therefore $ c^{\dag} A c = 0 $.

Note that
\begin{equation}
\begin{split}
c^{\dag} A c &= \sume{i}{1}{m}{\sume{j}{1}{m}{ \overline{c_j} \pbr{ g_j, g_i } c_i }} \\
&= \pbr{ \sume{j}{1}{m}{ \overline{c_j} g_j }, \sume{i}{1}{m}{ \overline{c_i} g_i } },
\end{split}
\end{equation}
and consequently from the positively definiteness
\begin{equation}
\sume{i}{1}{m}{ \overline{c_i} g_i } = 0.
\end{equation}
However this contradicts the given linear independence, which shows $A$ is non-singular.

\sqed
\end{thmproof}
\end{thmquestion}

\begin{thmquestion}
\ 
\begin{thmanswer}
Let the best approximation be $ g = \alpha x + \beta x^3 + \gamma x^5 $. From
\begin{equation}
f - g \perp \opspan \cbr{ x, x^3, x^5 },
\end{equation}
it can be derived that
\begin{gather}
\pbr{ f - g, x } = \pbr{ f - g, x^3 } = \pbr{ f - g, x^5 } = 0.
\end{gather}
Direct computation leads to the system
\begin{equation}
\left[\begin{matrix}\frac{2}{3} & \frac{2}{5} & \frac{2}{7}\\\frac{2}{5} & \frac{2}{7} & \frac{2}{9}\\\frac{2}{7} & \frac{2}{9} & \frac{2}{11}\end{matrix}\right] \msbr{ \alpha \\ \beta \\ \gamma } = \left[\begin{matrix}- 2 \cos{\left (1 \right )} + 2 \sin{\left (1 \right )}\\- 6 \sin{\left (1 \right )} + 10 \cos{\left (1 \right )}\\- 202 \cos{\left (1 \right )} + 130 \sin{\left (1 \right )}\end{matrix}\right],
\end{equation}
whose solution is
\begin{equation}
\msbr{ \alpha \\ \beta \\ \gamma } = \left[\begin{matrix}- \frac{139965}{8} \cos{\left (1 \right )} + 11235 \sin{\left (1 \right )}\\- \frac{104265}{2} \sin{\left (1 \right )} + \frac{324765}{4} \cos{\left (1 \right )}\\- \frac{582813}{8} \cos{\left (1 \right )} + \frac{93555}{2} \sin{\left (1 \right )}\end{matrix}\right].
\end{equation}
Therefore, the final solution is
\begin{equation}
\begin{split}
g &= x^{5} \left(- \frac{582813}{8} \cos{\left (1 \right )} + \frac{93555}{2} \sin{\left (1 \right )}\right) \\
&+ x^{3} \left(- \frac{104265}{2} \sin{\left (1 \right )} + \frac{324765}{4} \cos{\left (1 \right )}\right) \\
&+ x \left(- \frac{139965}{8} \cos{\left (1 \right )} + 11235 \sin{\left (1 \right )}\right).
\end{split}
\end{equation}
\end{thmanswer}
\end{thmquestion}

\begin{thmquestion}
\
\begin{thmproof}
Because
\begin{equation}
\norm{u_i} = \frac{\norm{ v_i - \sume{j}{1}{ i - 1 }{ \pbr{ v_i, u_j } u_j } }}{\norm{ v_i - \sume{j}{1}{ i - 1 }{ \pbr{ v_i, u_j } u_j } }} = 1
\end{equation}
for $ 1 \le i \le n $, it remains to prove $u_i$ are pairwise perpendicular.

Perform mathematical induction to prove that for a given $i$ with $ 1 \le i \le n $, $ u_j \perp u_k $ is always true for $ 1 \le j < k \le i $. Suppose the case $ i = m - 1  $ is done and then consider the case $ i = m $, where $ 1 < m \le n $. It is sufficient to prove $ u_m \perp u_k $ for $ 1 \le k < m $, which directly follows from
\begin{equation}
\pbr{ u_m, u_k } = \pbr{ v_m, u_k  } - \sume{j}{1}{ m - 1 }{ \pbr{ v_m, u_j } \pbr{ u_j, u_k } } = \pbr{ v_m, u_k } - \pbr{ v_m, u_k } \norm{u_k} = 0.
\end{equation}
Therefore, the induction is finished and we obtain $ u_j \perp u_k $ for $ 1 \le j < k \le n $.

As shown above, $u_i$ are pairwise orthogonal and normalized, which means they form a orthonormal basis.

\sqed
\end{thmproof}
\end{thmquestion}

\begin{thmquestion}
\
\begin{thmproof}
The leading coefficient of $L_n$ is
\begin{equation}
\frac{ \rbr{ 2 n } ! }{ 2^n n ! n ! } = \frac{ \rbr{ 2 n - 1 } !! }{ n ! },
\end{equation}
and therefore
\begin{equation}
g = \frac{ n ! }{ \rbr{ 2 n - 1 } !! } L_n
\end{equation}
is monic. Note that $ L_0, L_1, \cdots, L_{ n - 1 } $ form a basis of $ \opspan \cbr{ 1, x, \cdots, x^{ n - 1 } } $. Therefore, for any monic polynomial $f$ of degree $n$, there exists $ c_i \crbr{ 0 \le i \le n - 1 } $ such that
\begin{equation}
f - g = \sume{i}{0}{ n - 1 }{ c_i L_i }.
\end{equation}
Therefore,
\begin{equation}
\begin{split}
\norm{f}^2 &= \norm{ g + \sume{i}{0}{ n - 1 }{ c_i L_i }}^2 \\
&= \norm{g}^2 + \sume{i}{0}{ n - 1 }{ c_i^2 \norm{L_i}^2 }
\end{split}
\end{equation}
where the last equality follows from the orthogonality of $L_i$. Because $ \norm{L_i}^2 > 0 $, we have $ \norm{f} \ge \norm{g} $, and the equality is reached iff $ c_i = 0 \crbr{ 0 \le i \le n - 1 } $, which is equivalent to $ f = g $. Consequently,
\begin{gather}
g = \argmin_{ f \mtx{monic}, \deg f = n } \norm{f}, \\
\norm{g} = \min_{ f \mtx{monic}, \deg f = n } \norm{f}.
\end{gather}

\sqed
\end{thmproof}
\end{thmquestion}

\begin{thmanswer}
\ 
\begin{thmquestion}
Suppose the best uniform approximation is $ g \rbr{x} = a x + b $, and therefore there exists a Chebyshev alternance of $3$ points for the $ R \rbr{x} = \sin \frac{ \spi x }{2} - a x - b $. Becuase $R$ is concave over $ \rbr{ 0, 1 } $, therefore two of the points are $ 0, 1 $ respectively. Because $ R \rbr{0} = R \rbr{1} $, it follows that $ a = 1 $ and consequently the third point is
\begin{equation}
\xi = \frac{2}{\spi} \arccos \frac{2}{\spi}.
\end{equation}
From $ R \rbr{0} = -R \rbr{\xi} $, we derive that
\begin{equation}
b = \frac{1}{2} \sin \arccos \frac{2}{\spi} - \frac{1}{\spi} \arccos \frac{2}{\spi} =  \frac{\sqrt{ \spi^2 - 4 }}{ 2 \spi } - \frac{1}{\spi} \arccos \frac{2}{\spi}
\end{equation}
and
\begin{equation}
g \rbr{x} = x + \frac{\sqrt{ \spi^2 - 4 }}{ 2 \spi } - \frac{1}{\spi} \arccos \frac{2}{\spi}.
\end{equation}
\end{thmquestion}
\end{thmanswer}

\begin{thmanswer}
\ 
\begin{thmquestion}
Let $T_n$ be the Chebyshev polynomial of degree $n$, $ r_n = \frac{1}{2^{ n - 1 }} T_n $ and $ f_n = x^n - r_n $. Because $r_n$ is monic, therefore $ f_n \in \mathcal{P}_{ n - 1 } $.

Note that $ r_n = x^n - f_n $ have a Chebyshev alternance of $ n + 1 $ points: let
\begin{equation}
x_k = \cos \frac{ k \spi }{ 2 n }, \crbr{ 0 \le k \le n },
\end{equation}
and then
\begin{partlist}
\item $x_k$ are $ n + 1 $ distinct points arranged from right to left on the axis;
\item
\begin{equation}
r_n \rbr{x_k} = \frac{1}{2^{ n - 1 }} T_n \rbr{x_k} = \frac{1}{2^{ n - 1 }} \rbr{-1}^k;
\end{equation}
\item $ \abs{ r_n \rbr{x} } \le \frac{1}{2^{ n - 1 }} $; \label{Item:NormBound}
\item the equality in \ref{Item:NormBound} is only reached at $x_k$.
\end{partlist}
Because $ f_n \in \mathcal{P}_{ n - 1 } $, therefore $f_n$ is the best uniform approximation of $x^n$.

In conclusion, $ f_n = x^n - \frac{1}{2^{ n - 1 }} T_n $ is the best uniform approximation of $x^n$, and the Chebyshev alternance consists of $ x_k = \cos \frac{ k \spi }{ 2 n } \crbr{ 0 \le k \le n } $.
\end{thmquestion}
\end{thmanswer}

\begin{thmquestion}
\
\begin{thmproof}
Consider
\begin{equation}
u_i = \msbr{ \phi_i \rbr{x_1} & \phi_i \rbr{x_2} & \phi_i \rbr{x_3} & \cdots & \phi_i \rbr{x_{ i + 1 }} }^{\rmut}
\end{equation}
and the matrix
\begin{equation}
M = \msbr{ u_i & u_1 & u_2 & u_3 & \cdots & u_i }.
\end{equation}
Because $M$ is linear dependent in terms of columns, we have
\begin{align}
0 &= \det M \\
&= \sume{j}{1}{n}{ \rbr{-1}^{ j + 1 } \phi_i \rbr{x_j} \rbr{x_j} D_j } \\
&= -\sume{j}{1}{n}{ \phi_i \rbr{x_j} \sigma_j }.
\end{align}
Therefore,
\begin{equation}
\sume{j}{1}{n}{ \phi_i \rbr{x_j} \sigma_j } = 0
\end{equation}
follows as desired.

\sqed
\end{thmproof}
\end{thmquestion}

\begin{thmquestion}
\ 
\begin{thmanswer}
The Python code is placed in the file \verb"Problem8.ipynb". The algorithm succeeded in converging in the $ k = 2 $nd iteration, such that
\begin{equation}
\maxe{j}{1}{ n + 1 }{\abs{\epsilon^{\rbr{k}}_j}} - \mine{j}{1}{ n + 1 }{\abs{\epsilon^{\rbr{k}}_j}} = \text{8.720947e-06} < \text{1e-4}.
\end{equation}
\end{thmanswer}
Plot of $f$ and $p_{\cdot}$ is shown in Figure \ref{Fig:Origin}, and plot of residue is shown in Figure \ref{Fig:Residue}.
\begin{figure}[htbp]
\centering \scalebox{0.8}{%% Creator: Matplotlib, PGF backend
%%
%% To include the figure in your LaTeX document, write
%%   \input{<filename>.pgf}
%%
%% Make sure the required packages are loaded in your preamble
%%   \usepackage{pgf}
%%
%% Figures using additional raster images can only be included by \input if
%% they are in the same directory as the main LaTeX file. For loading figures
%% from other directories you can use the `import` package
%%   \usepackage{import}
%% and then include the figures with
%%   \import{<path to file>}{<filename>.pgf}
%%
%% Matplotlib used the following preamble
%%   \usepackage{fontspec}
%%
\begingroup%
\makeatletter%
\begin{pgfpicture}%
\pgfpathrectangle{\pgfpointorigin}{\pgfqpoint{6.000000in}{4.000000in}}%
\pgfusepath{use as bounding box, clip}%
\begin{pgfscope}%
\pgfsetbuttcap%
\pgfsetmiterjoin%
\definecolor{currentfill}{rgb}{1.000000,1.000000,1.000000}%
\pgfsetfillcolor{currentfill}%
\pgfsetlinewidth{0.000000pt}%
\definecolor{currentstroke}{rgb}{1.000000,1.000000,1.000000}%
\pgfsetstrokecolor{currentstroke}%
\pgfsetdash{}{0pt}%
\pgfpathmoveto{\pgfqpoint{0.000000in}{0.000000in}}%
\pgfpathlineto{\pgfqpoint{6.000000in}{0.000000in}}%
\pgfpathlineto{\pgfqpoint{6.000000in}{4.000000in}}%
\pgfpathlineto{\pgfqpoint{0.000000in}{4.000000in}}%
\pgfpathclose%
\pgfusepath{fill}%
\end{pgfscope}%
\begin{pgfscope}%
\pgfsetbuttcap%
\pgfsetmiterjoin%
\definecolor{currentfill}{rgb}{1.000000,1.000000,1.000000}%
\pgfsetfillcolor{currentfill}%
\pgfsetlinewidth{0.000000pt}%
\definecolor{currentstroke}{rgb}{0.000000,0.000000,0.000000}%
\pgfsetstrokecolor{currentstroke}%
\pgfsetstrokeopacity{0.000000}%
\pgfsetdash{}{0pt}%
\pgfpathmoveto{\pgfqpoint{0.750000in}{0.500000in}}%
\pgfpathlineto{\pgfqpoint{5.400000in}{0.500000in}}%
\pgfpathlineto{\pgfqpoint{5.400000in}{3.520000in}}%
\pgfpathlineto{\pgfqpoint{0.750000in}{3.520000in}}%
\pgfpathclose%
\pgfusepath{fill}%
\end{pgfscope}%
\begin{pgfscope}%
\pgfpathrectangle{\pgfqpoint{0.750000in}{0.500000in}}{\pgfqpoint{4.650000in}{3.020000in}}%
\pgfusepath{clip}%
\pgfsetrectcap%
\pgfsetroundjoin%
\pgfsetlinewidth{0.803000pt}%
\definecolor{currentstroke}{rgb}{0.690196,0.690196,0.690196}%
\pgfsetstrokecolor{currentstroke}%
\pgfsetdash{}{0pt}%
\pgfpathmoveto{\pgfqpoint{0.961364in}{0.500000in}}%
\pgfpathlineto{\pgfqpoint{0.961364in}{3.520000in}}%
\pgfusepath{stroke}%
\end{pgfscope}%
\begin{pgfscope}%
\pgfsetbuttcap%
\pgfsetroundjoin%
\definecolor{currentfill}{rgb}{0.000000,0.000000,0.000000}%
\pgfsetfillcolor{currentfill}%
\pgfsetlinewidth{0.803000pt}%
\definecolor{currentstroke}{rgb}{0.000000,0.000000,0.000000}%
\pgfsetstrokecolor{currentstroke}%
\pgfsetdash{}{0pt}%
\pgfsys@defobject{currentmarker}{\pgfqpoint{0.000000in}{-0.048611in}}{\pgfqpoint{0.000000in}{0.000000in}}{%
\pgfpathmoveto{\pgfqpoint{0.000000in}{0.000000in}}%
\pgfpathlineto{\pgfqpoint{0.000000in}{-0.048611in}}%
\pgfusepath{stroke,fill}%
}%
\begin{pgfscope}%
\pgfsys@transformshift{0.961364in}{0.500000in}%
\pgfsys@useobject{currentmarker}{}%
\end{pgfscope}%
\end{pgfscope}%
\begin{pgfscope}%
\pgftext[x=0.961364in,y=0.402778in,,top]{\rmfamily\fontsize{10.000000}{12.000000}\selectfont \(\displaystyle -1.00\)}%
\end{pgfscope}%
\begin{pgfscope}%
\pgfpathrectangle{\pgfqpoint{0.750000in}{0.500000in}}{\pgfqpoint{4.650000in}{3.020000in}}%
\pgfusepath{clip}%
\pgfsetrectcap%
\pgfsetroundjoin%
\pgfsetlinewidth{0.803000pt}%
\definecolor{currentstroke}{rgb}{0.690196,0.690196,0.690196}%
\pgfsetstrokecolor{currentstroke}%
\pgfsetdash{}{0pt}%
\pgfpathmoveto{\pgfqpoint{1.489773in}{0.500000in}}%
\pgfpathlineto{\pgfqpoint{1.489773in}{3.520000in}}%
\pgfusepath{stroke}%
\end{pgfscope}%
\begin{pgfscope}%
\pgfsetbuttcap%
\pgfsetroundjoin%
\definecolor{currentfill}{rgb}{0.000000,0.000000,0.000000}%
\pgfsetfillcolor{currentfill}%
\pgfsetlinewidth{0.803000pt}%
\definecolor{currentstroke}{rgb}{0.000000,0.000000,0.000000}%
\pgfsetstrokecolor{currentstroke}%
\pgfsetdash{}{0pt}%
\pgfsys@defobject{currentmarker}{\pgfqpoint{0.000000in}{-0.048611in}}{\pgfqpoint{0.000000in}{0.000000in}}{%
\pgfpathmoveto{\pgfqpoint{0.000000in}{0.000000in}}%
\pgfpathlineto{\pgfqpoint{0.000000in}{-0.048611in}}%
\pgfusepath{stroke,fill}%
}%
\begin{pgfscope}%
\pgfsys@transformshift{1.489773in}{0.500000in}%
\pgfsys@useobject{currentmarker}{}%
\end{pgfscope}%
\end{pgfscope}%
\begin{pgfscope}%
\pgftext[x=1.489773in,y=0.402778in,,top]{\rmfamily\fontsize{10.000000}{12.000000}\selectfont \(\displaystyle -0.75\)}%
\end{pgfscope}%
\begin{pgfscope}%
\pgfpathrectangle{\pgfqpoint{0.750000in}{0.500000in}}{\pgfqpoint{4.650000in}{3.020000in}}%
\pgfusepath{clip}%
\pgfsetrectcap%
\pgfsetroundjoin%
\pgfsetlinewidth{0.803000pt}%
\definecolor{currentstroke}{rgb}{0.690196,0.690196,0.690196}%
\pgfsetstrokecolor{currentstroke}%
\pgfsetdash{}{0pt}%
\pgfpathmoveto{\pgfqpoint{2.018182in}{0.500000in}}%
\pgfpathlineto{\pgfqpoint{2.018182in}{3.520000in}}%
\pgfusepath{stroke}%
\end{pgfscope}%
\begin{pgfscope}%
\pgfsetbuttcap%
\pgfsetroundjoin%
\definecolor{currentfill}{rgb}{0.000000,0.000000,0.000000}%
\pgfsetfillcolor{currentfill}%
\pgfsetlinewidth{0.803000pt}%
\definecolor{currentstroke}{rgb}{0.000000,0.000000,0.000000}%
\pgfsetstrokecolor{currentstroke}%
\pgfsetdash{}{0pt}%
\pgfsys@defobject{currentmarker}{\pgfqpoint{0.000000in}{-0.048611in}}{\pgfqpoint{0.000000in}{0.000000in}}{%
\pgfpathmoveto{\pgfqpoint{0.000000in}{0.000000in}}%
\pgfpathlineto{\pgfqpoint{0.000000in}{-0.048611in}}%
\pgfusepath{stroke,fill}%
}%
\begin{pgfscope}%
\pgfsys@transformshift{2.018182in}{0.500000in}%
\pgfsys@useobject{currentmarker}{}%
\end{pgfscope}%
\end{pgfscope}%
\begin{pgfscope}%
\pgftext[x=2.018182in,y=0.402778in,,top]{\rmfamily\fontsize{10.000000}{12.000000}\selectfont \(\displaystyle -0.50\)}%
\end{pgfscope}%
\begin{pgfscope}%
\pgfpathrectangle{\pgfqpoint{0.750000in}{0.500000in}}{\pgfqpoint{4.650000in}{3.020000in}}%
\pgfusepath{clip}%
\pgfsetrectcap%
\pgfsetroundjoin%
\pgfsetlinewidth{0.803000pt}%
\definecolor{currentstroke}{rgb}{0.690196,0.690196,0.690196}%
\pgfsetstrokecolor{currentstroke}%
\pgfsetdash{}{0pt}%
\pgfpathmoveto{\pgfqpoint{2.546591in}{0.500000in}}%
\pgfpathlineto{\pgfqpoint{2.546591in}{3.520000in}}%
\pgfusepath{stroke}%
\end{pgfscope}%
\begin{pgfscope}%
\pgfsetbuttcap%
\pgfsetroundjoin%
\definecolor{currentfill}{rgb}{0.000000,0.000000,0.000000}%
\pgfsetfillcolor{currentfill}%
\pgfsetlinewidth{0.803000pt}%
\definecolor{currentstroke}{rgb}{0.000000,0.000000,0.000000}%
\pgfsetstrokecolor{currentstroke}%
\pgfsetdash{}{0pt}%
\pgfsys@defobject{currentmarker}{\pgfqpoint{0.000000in}{-0.048611in}}{\pgfqpoint{0.000000in}{0.000000in}}{%
\pgfpathmoveto{\pgfqpoint{0.000000in}{0.000000in}}%
\pgfpathlineto{\pgfqpoint{0.000000in}{-0.048611in}}%
\pgfusepath{stroke,fill}%
}%
\begin{pgfscope}%
\pgfsys@transformshift{2.546591in}{0.500000in}%
\pgfsys@useobject{currentmarker}{}%
\end{pgfscope}%
\end{pgfscope}%
\begin{pgfscope}%
\pgftext[x=2.546591in,y=0.402778in,,top]{\rmfamily\fontsize{10.000000}{12.000000}\selectfont \(\displaystyle -0.25\)}%
\end{pgfscope}%
\begin{pgfscope}%
\pgfpathrectangle{\pgfqpoint{0.750000in}{0.500000in}}{\pgfqpoint{4.650000in}{3.020000in}}%
\pgfusepath{clip}%
\pgfsetrectcap%
\pgfsetroundjoin%
\pgfsetlinewidth{0.803000pt}%
\definecolor{currentstroke}{rgb}{0.690196,0.690196,0.690196}%
\pgfsetstrokecolor{currentstroke}%
\pgfsetdash{}{0pt}%
\pgfpathmoveto{\pgfqpoint{3.075000in}{0.500000in}}%
\pgfpathlineto{\pgfqpoint{3.075000in}{3.520000in}}%
\pgfusepath{stroke}%
\end{pgfscope}%
\begin{pgfscope}%
\pgfsetbuttcap%
\pgfsetroundjoin%
\definecolor{currentfill}{rgb}{0.000000,0.000000,0.000000}%
\pgfsetfillcolor{currentfill}%
\pgfsetlinewidth{0.803000pt}%
\definecolor{currentstroke}{rgb}{0.000000,0.000000,0.000000}%
\pgfsetstrokecolor{currentstroke}%
\pgfsetdash{}{0pt}%
\pgfsys@defobject{currentmarker}{\pgfqpoint{0.000000in}{-0.048611in}}{\pgfqpoint{0.000000in}{0.000000in}}{%
\pgfpathmoveto{\pgfqpoint{0.000000in}{0.000000in}}%
\pgfpathlineto{\pgfqpoint{0.000000in}{-0.048611in}}%
\pgfusepath{stroke,fill}%
}%
\begin{pgfscope}%
\pgfsys@transformshift{3.075000in}{0.500000in}%
\pgfsys@useobject{currentmarker}{}%
\end{pgfscope}%
\end{pgfscope}%
\begin{pgfscope}%
\pgftext[x=3.075000in,y=0.402778in,,top]{\rmfamily\fontsize{10.000000}{12.000000}\selectfont \(\displaystyle 0.00\)}%
\end{pgfscope}%
\begin{pgfscope}%
\pgfpathrectangle{\pgfqpoint{0.750000in}{0.500000in}}{\pgfqpoint{4.650000in}{3.020000in}}%
\pgfusepath{clip}%
\pgfsetrectcap%
\pgfsetroundjoin%
\pgfsetlinewidth{0.803000pt}%
\definecolor{currentstroke}{rgb}{0.690196,0.690196,0.690196}%
\pgfsetstrokecolor{currentstroke}%
\pgfsetdash{}{0pt}%
\pgfpathmoveto{\pgfqpoint{3.603409in}{0.500000in}}%
\pgfpathlineto{\pgfqpoint{3.603409in}{3.520000in}}%
\pgfusepath{stroke}%
\end{pgfscope}%
\begin{pgfscope}%
\pgfsetbuttcap%
\pgfsetroundjoin%
\definecolor{currentfill}{rgb}{0.000000,0.000000,0.000000}%
\pgfsetfillcolor{currentfill}%
\pgfsetlinewidth{0.803000pt}%
\definecolor{currentstroke}{rgb}{0.000000,0.000000,0.000000}%
\pgfsetstrokecolor{currentstroke}%
\pgfsetdash{}{0pt}%
\pgfsys@defobject{currentmarker}{\pgfqpoint{0.000000in}{-0.048611in}}{\pgfqpoint{0.000000in}{0.000000in}}{%
\pgfpathmoveto{\pgfqpoint{0.000000in}{0.000000in}}%
\pgfpathlineto{\pgfqpoint{0.000000in}{-0.048611in}}%
\pgfusepath{stroke,fill}%
}%
\begin{pgfscope}%
\pgfsys@transformshift{3.603409in}{0.500000in}%
\pgfsys@useobject{currentmarker}{}%
\end{pgfscope}%
\end{pgfscope}%
\begin{pgfscope}%
\pgftext[x=3.603409in,y=0.402778in,,top]{\rmfamily\fontsize{10.000000}{12.000000}\selectfont \(\displaystyle 0.25\)}%
\end{pgfscope}%
\begin{pgfscope}%
\pgfpathrectangle{\pgfqpoint{0.750000in}{0.500000in}}{\pgfqpoint{4.650000in}{3.020000in}}%
\pgfusepath{clip}%
\pgfsetrectcap%
\pgfsetroundjoin%
\pgfsetlinewidth{0.803000pt}%
\definecolor{currentstroke}{rgb}{0.690196,0.690196,0.690196}%
\pgfsetstrokecolor{currentstroke}%
\pgfsetdash{}{0pt}%
\pgfpathmoveto{\pgfqpoint{4.131818in}{0.500000in}}%
\pgfpathlineto{\pgfqpoint{4.131818in}{3.520000in}}%
\pgfusepath{stroke}%
\end{pgfscope}%
\begin{pgfscope}%
\pgfsetbuttcap%
\pgfsetroundjoin%
\definecolor{currentfill}{rgb}{0.000000,0.000000,0.000000}%
\pgfsetfillcolor{currentfill}%
\pgfsetlinewidth{0.803000pt}%
\definecolor{currentstroke}{rgb}{0.000000,0.000000,0.000000}%
\pgfsetstrokecolor{currentstroke}%
\pgfsetdash{}{0pt}%
\pgfsys@defobject{currentmarker}{\pgfqpoint{0.000000in}{-0.048611in}}{\pgfqpoint{0.000000in}{0.000000in}}{%
\pgfpathmoveto{\pgfqpoint{0.000000in}{0.000000in}}%
\pgfpathlineto{\pgfqpoint{0.000000in}{-0.048611in}}%
\pgfusepath{stroke,fill}%
}%
\begin{pgfscope}%
\pgfsys@transformshift{4.131818in}{0.500000in}%
\pgfsys@useobject{currentmarker}{}%
\end{pgfscope}%
\end{pgfscope}%
\begin{pgfscope}%
\pgftext[x=4.131818in,y=0.402778in,,top]{\rmfamily\fontsize{10.000000}{12.000000}\selectfont \(\displaystyle 0.50\)}%
\end{pgfscope}%
\begin{pgfscope}%
\pgfpathrectangle{\pgfqpoint{0.750000in}{0.500000in}}{\pgfqpoint{4.650000in}{3.020000in}}%
\pgfusepath{clip}%
\pgfsetrectcap%
\pgfsetroundjoin%
\pgfsetlinewidth{0.803000pt}%
\definecolor{currentstroke}{rgb}{0.690196,0.690196,0.690196}%
\pgfsetstrokecolor{currentstroke}%
\pgfsetdash{}{0pt}%
\pgfpathmoveto{\pgfqpoint{4.660227in}{0.500000in}}%
\pgfpathlineto{\pgfqpoint{4.660227in}{3.520000in}}%
\pgfusepath{stroke}%
\end{pgfscope}%
\begin{pgfscope}%
\pgfsetbuttcap%
\pgfsetroundjoin%
\definecolor{currentfill}{rgb}{0.000000,0.000000,0.000000}%
\pgfsetfillcolor{currentfill}%
\pgfsetlinewidth{0.803000pt}%
\definecolor{currentstroke}{rgb}{0.000000,0.000000,0.000000}%
\pgfsetstrokecolor{currentstroke}%
\pgfsetdash{}{0pt}%
\pgfsys@defobject{currentmarker}{\pgfqpoint{0.000000in}{-0.048611in}}{\pgfqpoint{0.000000in}{0.000000in}}{%
\pgfpathmoveto{\pgfqpoint{0.000000in}{0.000000in}}%
\pgfpathlineto{\pgfqpoint{0.000000in}{-0.048611in}}%
\pgfusepath{stroke,fill}%
}%
\begin{pgfscope}%
\pgfsys@transformshift{4.660227in}{0.500000in}%
\pgfsys@useobject{currentmarker}{}%
\end{pgfscope}%
\end{pgfscope}%
\begin{pgfscope}%
\pgftext[x=4.660227in,y=0.402778in,,top]{\rmfamily\fontsize{10.000000}{12.000000}\selectfont \(\displaystyle 0.75\)}%
\end{pgfscope}%
\begin{pgfscope}%
\pgfpathrectangle{\pgfqpoint{0.750000in}{0.500000in}}{\pgfqpoint{4.650000in}{3.020000in}}%
\pgfusepath{clip}%
\pgfsetrectcap%
\pgfsetroundjoin%
\pgfsetlinewidth{0.803000pt}%
\definecolor{currentstroke}{rgb}{0.690196,0.690196,0.690196}%
\pgfsetstrokecolor{currentstroke}%
\pgfsetdash{}{0pt}%
\pgfpathmoveto{\pgfqpoint{5.188636in}{0.500000in}}%
\pgfpathlineto{\pgfqpoint{5.188636in}{3.520000in}}%
\pgfusepath{stroke}%
\end{pgfscope}%
\begin{pgfscope}%
\pgfsetbuttcap%
\pgfsetroundjoin%
\definecolor{currentfill}{rgb}{0.000000,0.000000,0.000000}%
\pgfsetfillcolor{currentfill}%
\pgfsetlinewidth{0.803000pt}%
\definecolor{currentstroke}{rgb}{0.000000,0.000000,0.000000}%
\pgfsetstrokecolor{currentstroke}%
\pgfsetdash{}{0pt}%
\pgfsys@defobject{currentmarker}{\pgfqpoint{0.000000in}{-0.048611in}}{\pgfqpoint{0.000000in}{0.000000in}}{%
\pgfpathmoveto{\pgfqpoint{0.000000in}{0.000000in}}%
\pgfpathlineto{\pgfqpoint{0.000000in}{-0.048611in}}%
\pgfusepath{stroke,fill}%
}%
\begin{pgfscope}%
\pgfsys@transformshift{5.188636in}{0.500000in}%
\pgfsys@useobject{currentmarker}{}%
\end{pgfscope}%
\end{pgfscope}%
\begin{pgfscope}%
\pgftext[x=5.188636in,y=0.402778in,,top]{\rmfamily\fontsize{10.000000}{12.000000}\selectfont \(\displaystyle 1.00\)}%
\end{pgfscope}%
\begin{pgfscope}%
\pgfpathrectangle{\pgfqpoint{0.750000in}{0.500000in}}{\pgfqpoint{4.650000in}{3.020000in}}%
\pgfusepath{clip}%
\pgfsetrectcap%
\pgfsetroundjoin%
\pgfsetlinewidth{0.803000pt}%
\definecolor{currentstroke}{rgb}{0.690196,0.690196,0.690196}%
\pgfsetstrokecolor{currentstroke}%
\pgfsetdash{}{0pt}%
\pgfpathmoveto{\pgfqpoint{0.750000in}{0.751667in}}%
\pgfpathlineto{\pgfqpoint{5.400000in}{0.751667in}}%
\pgfusepath{stroke}%
\end{pgfscope}%
\begin{pgfscope}%
\pgfsetbuttcap%
\pgfsetroundjoin%
\definecolor{currentfill}{rgb}{0.000000,0.000000,0.000000}%
\pgfsetfillcolor{currentfill}%
\pgfsetlinewidth{0.803000pt}%
\definecolor{currentstroke}{rgb}{0.000000,0.000000,0.000000}%
\pgfsetstrokecolor{currentstroke}%
\pgfsetdash{}{0pt}%
\pgfsys@defobject{currentmarker}{\pgfqpoint{-0.048611in}{0.000000in}}{\pgfqpoint{0.000000in}{0.000000in}}{%
\pgfpathmoveto{\pgfqpoint{0.000000in}{0.000000in}}%
\pgfpathlineto{\pgfqpoint{-0.048611in}{0.000000in}}%
\pgfusepath{stroke,fill}%
}%
\begin{pgfscope}%
\pgfsys@transformshift{0.750000in}{0.751667in}%
\pgfsys@useobject{currentmarker}{}%
\end{pgfscope}%
\end{pgfscope}%
\begin{pgfscope}%
\pgftext[x=0.475308in,y=0.703472in,left,base]{\rmfamily\fontsize{10.000000}{12.000000}\selectfont \(\displaystyle 0.0\)}%
\end{pgfscope}%
\begin{pgfscope}%
\pgfpathrectangle{\pgfqpoint{0.750000in}{0.500000in}}{\pgfqpoint{4.650000in}{3.020000in}}%
\pgfusepath{clip}%
\pgfsetrectcap%
\pgfsetroundjoin%
\pgfsetlinewidth{0.803000pt}%
\definecolor{currentstroke}{rgb}{0.690196,0.690196,0.690196}%
\pgfsetstrokecolor{currentstroke}%
\pgfsetdash{}{0pt}%
\pgfpathmoveto{\pgfqpoint{0.750000in}{1.255000in}}%
\pgfpathlineto{\pgfqpoint{5.400000in}{1.255000in}}%
\pgfusepath{stroke}%
\end{pgfscope}%
\begin{pgfscope}%
\pgfsetbuttcap%
\pgfsetroundjoin%
\definecolor{currentfill}{rgb}{0.000000,0.000000,0.000000}%
\pgfsetfillcolor{currentfill}%
\pgfsetlinewidth{0.803000pt}%
\definecolor{currentstroke}{rgb}{0.000000,0.000000,0.000000}%
\pgfsetstrokecolor{currentstroke}%
\pgfsetdash{}{0pt}%
\pgfsys@defobject{currentmarker}{\pgfqpoint{-0.048611in}{0.000000in}}{\pgfqpoint{0.000000in}{0.000000in}}{%
\pgfpathmoveto{\pgfqpoint{0.000000in}{0.000000in}}%
\pgfpathlineto{\pgfqpoint{-0.048611in}{0.000000in}}%
\pgfusepath{stroke,fill}%
}%
\begin{pgfscope}%
\pgfsys@transformshift{0.750000in}{1.255000in}%
\pgfsys@useobject{currentmarker}{}%
\end{pgfscope}%
\end{pgfscope}%
\begin{pgfscope}%
\pgftext[x=0.475308in,y=1.206806in,left,base]{\rmfamily\fontsize{10.000000}{12.000000}\selectfont \(\displaystyle 0.2\)}%
\end{pgfscope}%
\begin{pgfscope}%
\pgfpathrectangle{\pgfqpoint{0.750000in}{0.500000in}}{\pgfqpoint{4.650000in}{3.020000in}}%
\pgfusepath{clip}%
\pgfsetrectcap%
\pgfsetroundjoin%
\pgfsetlinewidth{0.803000pt}%
\definecolor{currentstroke}{rgb}{0.690196,0.690196,0.690196}%
\pgfsetstrokecolor{currentstroke}%
\pgfsetdash{}{0pt}%
\pgfpathmoveto{\pgfqpoint{0.750000in}{1.758333in}}%
\pgfpathlineto{\pgfqpoint{5.400000in}{1.758333in}}%
\pgfusepath{stroke}%
\end{pgfscope}%
\begin{pgfscope}%
\pgfsetbuttcap%
\pgfsetroundjoin%
\definecolor{currentfill}{rgb}{0.000000,0.000000,0.000000}%
\pgfsetfillcolor{currentfill}%
\pgfsetlinewidth{0.803000pt}%
\definecolor{currentstroke}{rgb}{0.000000,0.000000,0.000000}%
\pgfsetstrokecolor{currentstroke}%
\pgfsetdash{}{0pt}%
\pgfsys@defobject{currentmarker}{\pgfqpoint{-0.048611in}{0.000000in}}{\pgfqpoint{0.000000in}{0.000000in}}{%
\pgfpathmoveto{\pgfqpoint{0.000000in}{0.000000in}}%
\pgfpathlineto{\pgfqpoint{-0.048611in}{0.000000in}}%
\pgfusepath{stroke,fill}%
}%
\begin{pgfscope}%
\pgfsys@transformshift{0.750000in}{1.758333in}%
\pgfsys@useobject{currentmarker}{}%
\end{pgfscope}%
\end{pgfscope}%
\begin{pgfscope}%
\pgftext[x=0.475308in,y=1.710139in,left,base]{\rmfamily\fontsize{10.000000}{12.000000}\selectfont \(\displaystyle 0.4\)}%
\end{pgfscope}%
\begin{pgfscope}%
\pgfpathrectangle{\pgfqpoint{0.750000in}{0.500000in}}{\pgfqpoint{4.650000in}{3.020000in}}%
\pgfusepath{clip}%
\pgfsetrectcap%
\pgfsetroundjoin%
\pgfsetlinewidth{0.803000pt}%
\definecolor{currentstroke}{rgb}{0.690196,0.690196,0.690196}%
\pgfsetstrokecolor{currentstroke}%
\pgfsetdash{}{0pt}%
\pgfpathmoveto{\pgfqpoint{0.750000in}{2.261667in}}%
\pgfpathlineto{\pgfqpoint{5.400000in}{2.261667in}}%
\pgfusepath{stroke}%
\end{pgfscope}%
\begin{pgfscope}%
\pgfsetbuttcap%
\pgfsetroundjoin%
\definecolor{currentfill}{rgb}{0.000000,0.000000,0.000000}%
\pgfsetfillcolor{currentfill}%
\pgfsetlinewidth{0.803000pt}%
\definecolor{currentstroke}{rgb}{0.000000,0.000000,0.000000}%
\pgfsetstrokecolor{currentstroke}%
\pgfsetdash{}{0pt}%
\pgfsys@defobject{currentmarker}{\pgfqpoint{-0.048611in}{0.000000in}}{\pgfqpoint{0.000000in}{0.000000in}}{%
\pgfpathmoveto{\pgfqpoint{0.000000in}{0.000000in}}%
\pgfpathlineto{\pgfqpoint{-0.048611in}{0.000000in}}%
\pgfusepath{stroke,fill}%
}%
\begin{pgfscope}%
\pgfsys@transformshift{0.750000in}{2.261667in}%
\pgfsys@useobject{currentmarker}{}%
\end{pgfscope}%
\end{pgfscope}%
\begin{pgfscope}%
\pgftext[x=0.475308in,y=2.213472in,left,base]{\rmfamily\fontsize{10.000000}{12.000000}\selectfont \(\displaystyle 0.6\)}%
\end{pgfscope}%
\begin{pgfscope}%
\pgfpathrectangle{\pgfqpoint{0.750000in}{0.500000in}}{\pgfqpoint{4.650000in}{3.020000in}}%
\pgfusepath{clip}%
\pgfsetrectcap%
\pgfsetroundjoin%
\pgfsetlinewidth{0.803000pt}%
\definecolor{currentstroke}{rgb}{0.690196,0.690196,0.690196}%
\pgfsetstrokecolor{currentstroke}%
\pgfsetdash{}{0pt}%
\pgfpathmoveto{\pgfqpoint{0.750000in}{2.765000in}}%
\pgfpathlineto{\pgfqpoint{5.400000in}{2.765000in}}%
\pgfusepath{stroke}%
\end{pgfscope}%
\begin{pgfscope}%
\pgfsetbuttcap%
\pgfsetroundjoin%
\definecolor{currentfill}{rgb}{0.000000,0.000000,0.000000}%
\pgfsetfillcolor{currentfill}%
\pgfsetlinewidth{0.803000pt}%
\definecolor{currentstroke}{rgb}{0.000000,0.000000,0.000000}%
\pgfsetstrokecolor{currentstroke}%
\pgfsetdash{}{0pt}%
\pgfsys@defobject{currentmarker}{\pgfqpoint{-0.048611in}{0.000000in}}{\pgfqpoint{0.000000in}{0.000000in}}{%
\pgfpathmoveto{\pgfqpoint{0.000000in}{0.000000in}}%
\pgfpathlineto{\pgfqpoint{-0.048611in}{0.000000in}}%
\pgfusepath{stroke,fill}%
}%
\begin{pgfscope}%
\pgfsys@transformshift{0.750000in}{2.765000in}%
\pgfsys@useobject{currentmarker}{}%
\end{pgfscope}%
\end{pgfscope}%
\begin{pgfscope}%
\pgftext[x=0.475308in,y=2.716806in,left,base]{\rmfamily\fontsize{10.000000}{12.000000}\selectfont \(\displaystyle 0.8\)}%
\end{pgfscope}%
\begin{pgfscope}%
\pgfpathrectangle{\pgfqpoint{0.750000in}{0.500000in}}{\pgfqpoint{4.650000in}{3.020000in}}%
\pgfusepath{clip}%
\pgfsetrectcap%
\pgfsetroundjoin%
\pgfsetlinewidth{0.803000pt}%
\definecolor{currentstroke}{rgb}{0.690196,0.690196,0.690196}%
\pgfsetstrokecolor{currentstroke}%
\pgfsetdash{}{0pt}%
\pgfpathmoveto{\pgfqpoint{0.750000in}{3.268333in}}%
\pgfpathlineto{\pgfqpoint{5.400000in}{3.268333in}}%
\pgfusepath{stroke}%
\end{pgfscope}%
\begin{pgfscope}%
\pgfsetbuttcap%
\pgfsetroundjoin%
\definecolor{currentfill}{rgb}{0.000000,0.000000,0.000000}%
\pgfsetfillcolor{currentfill}%
\pgfsetlinewidth{0.803000pt}%
\definecolor{currentstroke}{rgb}{0.000000,0.000000,0.000000}%
\pgfsetstrokecolor{currentstroke}%
\pgfsetdash{}{0pt}%
\pgfsys@defobject{currentmarker}{\pgfqpoint{-0.048611in}{0.000000in}}{\pgfqpoint{0.000000in}{0.000000in}}{%
\pgfpathmoveto{\pgfqpoint{0.000000in}{0.000000in}}%
\pgfpathlineto{\pgfqpoint{-0.048611in}{0.000000in}}%
\pgfusepath{stroke,fill}%
}%
\begin{pgfscope}%
\pgfsys@transformshift{0.750000in}{3.268333in}%
\pgfsys@useobject{currentmarker}{}%
\end{pgfscope}%
\end{pgfscope}%
\begin{pgfscope}%
\pgftext[x=0.475308in,y=3.220139in,left,base]{\rmfamily\fontsize{10.000000}{12.000000}\selectfont \(\displaystyle 1.0\)}%
\end{pgfscope}%
\begin{pgfscope}%
\pgfpathrectangle{\pgfqpoint{0.750000in}{0.500000in}}{\pgfqpoint{4.650000in}{3.020000in}}%
\pgfusepath{clip}%
\pgfsetrectcap%
\pgfsetroundjoin%
\pgfsetlinewidth{1.505625pt}%
\definecolor{currentstroke}{rgb}{0.121569,0.466667,0.705882}%
\pgfsetstrokecolor{currentstroke}%
\pgfsetdash{}{0pt}%
\pgfpathmoveto{\pgfqpoint{0.961364in}{3.268333in}}%
\pgfpathlineto{\pgfqpoint{1.020605in}{3.062597in}}%
\pgfpathlineto{\pgfqpoint{1.079846in}{2.868391in}}%
\pgfpathlineto{\pgfqpoint{1.134855in}{2.698090in}}%
\pgfpathlineto{\pgfqpoint{1.189865in}{2.537177in}}%
\pgfpathlineto{\pgfqpoint{1.244874in}{2.385386in}}%
\pgfpathlineto{\pgfqpoint{1.299884in}{2.242452in}}%
\pgfpathlineto{\pgfqpoint{1.354894in}{2.108108in}}%
\pgfpathlineto{\pgfqpoint{1.409903in}{1.982087in}}%
\pgfpathlineto{\pgfqpoint{1.464913in}{1.864124in}}%
\pgfpathlineto{\pgfqpoint{1.515691in}{1.762156in}}%
\pgfpathlineto{\pgfqpoint{1.566469in}{1.666618in}}%
\pgfpathlineto{\pgfqpoint{1.617247in}{1.577299in}}%
\pgfpathlineto{\pgfqpoint{1.668025in}{1.493992in}}%
\pgfpathlineto{\pgfqpoint{1.718803in}{1.416486in}}%
\pgfpathlineto{\pgfqpoint{1.769581in}{1.344571in}}%
\pgfpathlineto{\pgfqpoint{1.820359in}{1.278039in}}%
\pgfpathlineto{\pgfqpoint{1.871137in}{1.216681in}}%
\pgfpathlineto{\pgfqpoint{1.921915in}{1.160286in}}%
\pgfpathlineto{\pgfqpoint{1.972693in}{1.108645in}}%
\pgfpathlineto{\pgfqpoint{2.023471in}{1.061550in}}%
\pgfpathlineto{\pgfqpoint{2.074249in}{1.018791in}}%
\pgfpathlineto{\pgfqpoint{2.125027in}{0.980157in}}%
\pgfpathlineto{\pgfqpoint{2.175805in}{0.945441in}}%
\pgfpathlineto{\pgfqpoint{2.226583in}{0.914432in}}%
\pgfpathlineto{\pgfqpoint{2.277361in}{0.886922in}}%
\pgfpathlineto{\pgfqpoint{2.328140in}{0.862700in}}%
\pgfpathlineto{\pgfqpoint{2.383149in}{0.839928in}}%
\pgfpathlineto{\pgfqpoint{2.438159in}{0.820505in}}%
\pgfpathlineto{\pgfqpoint{2.497400in}{0.803026in}}%
\pgfpathlineto{\pgfqpoint{2.556641in}{0.788788in}}%
\pgfpathlineto{\pgfqpoint{2.620113in}{0.776753in}}%
\pgfpathlineto{\pgfqpoint{2.687817in}{0.767136in}}%
\pgfpathlineto{\pgfqpoint{2.759753in}{0.760017in}}%
\pgfpathlineto{\pgfqpoint{2.840152in}{0.755119in}}%
\pgfpathlineto{\pgfqpoint{2.937476in}{0.752360in}}%
\pgfpathlineto{\pgfqpoint{3.089810in}{0.751668in}}%
\pgfpathlineto{\pgfqpoint{3.254839in}{0.753217in}}%
\pgfpathlineto{\pgfqpoint{3.347932in}{0.757085in}}%
\pgfpathlineto{\pgfqpoint{3.428331in}{0.763423in}}%
\pgfpathlineto{\pgfqpoint{3.500266in}{0.772165in}}%
\pgfpathlineto{\pgfqpoint{3.567970in}{0.783597in}}%
\pgfpathlineto{\pgfqpoint{3.631443in}{0.797586in}}%
\pgfpathlineto{\pgfqpoint{3.690684in}{0.813869in}}%
\pgfpathlineto{\pgfqpoint{3.745693in}{0.832076in}}%
\pgfpathlineto{\pgfqpoint{3.800703in}{0.853529in}}%
\pgfpathlineto{\pgfqpoint{3.855713in}{0.878493in}}%
\pgfpathlineto{\pgfqpoint{3.906491in}{0.904884in}}%
\pgfpathlineto{\pgfqpoint{3.957269in}{0.934703in}}%
\pgfpathlineto{\pgfqpoint{4.008047in}{0.968160in}}%
\pgfpathlineto{\pgfqpoint{4.058825in}{1.005465in}}%
\pgfpathlineto{\pgfqpoint{4.109603in}{1.046826in}}%
\pgfpathlineto{\pgfqpoint{4.160381in}{1.092452in}}%
\pgfpathlineto{\pgfqpoint{4.211159in}{1.142554in}}%
\pgfpathlineto{\pgfqpoint{4.261937in}{1.197341in}}%
\pgfpathlineto{\pgfqpoint{4.312715in}{1.257022in}}%
\pgfpathlineto{\pgfqpoint{4.363493in}{1.321806in}}%
\pgfpathlineto{\pgfqpoint{4.414271in}{1.391903in}}%
\pgfpathlineto{\pgfqpoint{4.465049in}{1.467522in}}%
\pgfpathlineto{\pgfqpoint{4.515827in}{1.548873in}}%
\pgfpathlineto{\pgfqpoint{4.566605in}{1.636164in}}%
\pgfpathlineto{\pgfqpoint{4.617383in}{1.729606in}}%
\pgfpathlineto{\pgfqpoint{4.668161in}{1.829407in}}%
\pgfpathlineto{\pgfqpoint{4.718939in}{1.935777in}}%
\pgfpathlineto{\pgfqpoint{4.769717in}{2.048926in}}%
\pgfpathlineto{\pgfqpoint{4.824727in}{2.179396in}}%
\pgfpathlineto{\pgfqpoint{4.879737in}{2.318332in}}%
\pgfpathlineto{\pgfqpoint{4.934746in}{2.466002in}}%
\pgfpathlineto{\pgfqpoint{4.989756in}{2.622672in}}%
\pgfpathlineto{\pgfqpoint{5.044765in}{2.788607in}}%
\pgfpathlineto{\pgfqpoint{5.099775in}{2.964074in}}%
\pgfpathlineto{\pgfqpoint{5.154784in}{3.149339in}}%
\pgfpathlineto{\pgfqpoint{5.188636in}{3.268333in}}%
\pgfpathlineto{\pgfqpoint{5.188636in}{3.268333in}}%
\pgfusepath{stroke}%
\end{pgfscope}%
\begin{pgfscope}%
\pgfpathrectangle{\pgfqpoint{0.750000in}{0.500000in}}{\pgfqpoint{4.650000in}{3.020000in}}%
\pgfusepath{clip}%
\pgfsetrectcap%
\pgfsetroundjoin%
\pgfsetlinewidth{1.505625pt}%
\definecolor{currentstroke}{rgb}{1.000000,0.498039,0.054902}%
\pgfsetstrokecolor{currentstroke}%
\pgfsetdash{}{0pt}%
\pgfpathmoveto{\pgfqpoint{0.961364in}{3.288902in}}%
\pgfpathlineto{\pgfqpoint{1.012142in}{3.099057in}}%
\pgfpathlineto{\pgfqpoint{1.062920in}{2.920401in}}%
\pgfpathlineto{\pgfqpoint{1.113698in}{2.752449in}}%
\pgfpathlineto{\pgfqpoint{1.164476in}{2.594728in}}%
\pgfpathlineto{\pgfqpoint{1.215254in}{2.446777in}}%
\pgfpathlineto{\pgfqpoint{1.266032in}{2.308147in}}%
\pgfpathlineto{\pgfqpoint{1.316810in}{2.178400in}}%
\pgfpathlineto{\pgfqpoint{1.367588in}{2.057112in}}%
\pgfpathlineto{\pgfqpoint{1.418366in}{1.943869in}}%
\pgfpathlineto{\pgfqpoint{1.469144in}{1.838272in}}%
\pgfpathlineto{\pgfqpoint{1.519922in}{1.739930in}}%
\pgfpathlineto{\pgfqpoint{1.570700in}{1.648469in}}%
\pgfpathlineto{\pgfqpoint{1.621478in}{1.563523in}}%
\pgfpathlineto{\pgfqpoint{1.672256in}{1.484741in}}%
\pgfpathlineto{\pgfqpoint{1.723034in}{1.411781in}}%
\pgfpathlineto{\pgfqpoint{1.773812in}{1.344316in}}%
\pgfpathlineto{\pgfqpoint{1.824590in}{1.282029in}}%
\pgfpathlineto{\pgfqpoint{1.875369in}{1.224617in}}%
\pgfpathlineto{\pgfqpoint{1.926147in}{1.171789in}}%
\pgfpathlineto{\pgfqpoint{1.976925in}{1.123263in}}%
\pgfpathlineto{\pgfqpoint{2.027703in}{1.078772in}}%
\pgfpathlineto{\pgfqpoint{2.082712in}{1.034832in}}%
\pgfpathlineto{\pgfqpoint{2.137722in}{0.995019in}}%
\pgfpathlineto{\pgfqpoint{2.192731in}{0.959042in}}%
\pgfpathlineto{\pgfqpoint{2.247741in}{0.926624in}}%
\pgfpathlineto{\pgfqpoint{2.306982in}{0.895396in}}%
\pgfpathlineto{\pgfqpoint{2.366223in}{0.867695in}}%
\pgfpathlineto{\pgfqpoint{2.425464in}{0.843241in}}%
\pgfpathlineto{\pgfqpoint{2.488937in}{0.820352in}}%
\pgfpathlineto{\pgfqpoint{2.556641in}{0.799398in}}%
\pgfpathlineto{\pgfqpoint{2.624345in}{0.781711in}}%
\pgfpathlineto{\pgfqpoint{2.696280in}{0.766194in}}%
\pgfpathlineto{\pgfqpoint{2.768216in}{0.753776in}}%
\pgfpathlineto{\pgfqpoint{2.844383in}{0.743748in}}%
\pgfpathlineto{\pgfqpoint{2.924782in}{0.736413in}}%
\pgfpathlineto{\pgfqpoint{3.005180in}{0.732240in}}%
\pgfpathlineto{\pgfqpoint{3.085579in}{0.731124in}}%
\pgfpathlineto{\pgfqpoint{3.165977in}{0.733038in}}%
\pgfpathlineto{\pgfqpoint{3.246376in}{0.738031in}}%
\pgfpathlineto{\pgfqpoint{3.326775in}{0.746224in}}%
\pgfpathlineto{\pgfqpoint{3.402942in}{0.757121in}}%
\pgfpathlineto{\pgfqpoint{3.474877in}{0.770425in}}%
\pgfpathlineto{\pgfqpoint{3.546813in}{0.786905in}}%
\pgfpathlineto{\pgfqpoint{3.614517in}{0.805582in}}%
\pgfpathlineto{\pgfqpoint{3.677989in}{0.826137in}}%
\pgfpathlineto{\pgfqpoint{3.741462in}{0.849912in}}%
\pgfpathlineto{\pgfqpoint{3.800703in}{0.875266in}}%
\pgfpathlineto{\pgfqpoint{3.859944in}{0.903946in}}%
\pgfpathlineto{\pgfqpoint{3.919185in}{0.936235in}}%
\pgfpathlineto{\pgfqpoint{3.974195in}{0.969720in}}%
\pgfpathlineto{\pgfqpoint{4.029204in}{1.006848in}}%
\pgfpathlineto{\pgfqpoint{4.084214in}{1.047899in}}%
\pgfpathlineto{\pgfqpoint{4.134992in}{1.089531in}}%
\pgfpathlineto{\pgfqpoint{4.185770in}{1.135006in}}%
\pgfpathlineto{\pgfqpoint{4.236548in}{1.184582in}}%
\pgfpathlineto{\pgfqpoint{4.287326in}{1.238530in}}%
\pgfpathlineto{\pgfqpoint{4.338104in}{1.297132in}}%
\pgfpathlineto{\pgfqpoint{4.388882in}{1.360684in}}%
\pgfpathlineto{\pgfqpoint{4.439660in}{1.429493in}}%
\pgfpathlineto{\pgfqpoint{4.490438in}{1.503877in}}%
\pgfpathlineto{\pgfqpoint{4.541216in}{1.584169in}}%
\pgfpathlineto{\pgfqpoint{4.591994in}{1.670710in}}%
\pgfpathlineto{\pgfqpoint{4.642772in}{1.763856in}}%
\pgfpathlineto{\pgfqpoint{4.693550in}{1.863976in}}%
\pgfpathlineto{\pgfqpoint{4.744328in}{1.971448in}}%
\pgfpathlineto{\pgfqpoint{4.795106in}{2.086663in}}%
\pgfpathlineto{\pgfqpoint{4.845885in}{2.210027in}}%
\pgfpathlineto{\pgfqpoint{4.896663in}{2.341955in}}%
\pgfpathlineto{\pgfqpoint{4.947441in}{2.482874in}}%
\pgfpathlineto{\pgfqpoint{4.998219in}{2.633225in}}%
\pgfpathlineto{\pgfqpoint{5.048997in}{2.793460in}}%
\pgfpathlineto{\pgfqpoint{5.099775in}{2.964042in}}%
\pgfpathlineto{\pgfqpoint{5.150553in}{3.145450in}}%
\pgfpathlineto{\pgfqpoint{5.188636in}{3.288902in}}%
\pgfpathlineto{\pgfqpoint{5.188636in}{3.288902in}}%
\pgfusepath{stroke}%
\end{pgfscope}%
\begin{pgfscope}%
\pgfpathrectangle{\pgfqpoint{0.750000in}{0.500000in}}{\pgfqpoint{4.650000in}{3.020000in}}%
\pgfusepath{clip}%
\pgfsetrectcap%
\pgfsetroundjoin%
\pgfsetlinewidth{1.505625pt}%
\definecolor{currentstroke}{rgb}{0.172549,0.627451,0.172549}%
\pgfsetstrokecolor{currentstroke}%
\pgfsetdash{}{0pt}%
\pgfpathmoveto{\pgfqpoint{0.961364in}{3.290684in}}%
\pgfpathlineto{\pgfqpoint{1.012142in}{3.100408in}}%
\pgfpathlineto{\pgfqpoint{1.062920in}{2.921357in}}%
\pgfpathlineto{\pgfqpoint{1.113698in}{2.753045in}}%
\pgfpathlineto{\pgfqpoint{1.164476in}{2.594994in}}%
\pgfpathlineto{\pgfqpoint{1.215254in}{2.446744in}}%
\pgfpathlineto{\pgfqpoint{1.266032in}{2.307843in}}%
\pgfpathlineto{\pgfqpoint{1.316810in}{2.177852in}}%
\pgfpathlineto{\pgfqpoint{1.367588in}{2.056345in}}%
\pgfpathlineto{\pgfqpoint{1.418366in}{1.942907in}}%
\pgfpathlineto{\pgfqpoint{1.469144in}{1.837136in}}%
\pgfpathlineto{\pgfqpoint{1.519922in}{1.738642in}}%
\pgfpathlineto{\pgfqpoint{1.570700in}{1.647048in}}%
\pgfpathlineto{\pgfqpoint{1.621478in}{1.561986in}}%
\pgfpathlineto{\pgfqpoint{1.672256in}{1.483104in}}%
\pgfpathlineto{\pgfqpoint{1.723034in}{1.410060in}}%
\pgfpathlineto{\pgfqpoint{1.773812in}{1.342524in}}%
\pgfpathlineto{\pgfqpoint{1.824590in}{1.280179in}}%
\pgfpathlineto{\pgfqpoint{1.875369in}{1.222721in}}%
\pgfpathlineto{\pgfqpoint{1.926147in}{1.169856in}}%
\pgfpathlineto{\pgfqpoint{1.976925in}{1.121303in}}%
\pgfpathlineto{\pgfqpoint{2.027703in}{1.076794in}}%
\pgfpathlineto{\pgfqpoint{2.082712in}{1.032842in}}%
\pgfpathlineto{\pgfqpoint{2.137722in}{0.993025in}}%
\pgfpathlineto{\pgfqpoint{2.192731in}{0.957049in}}%
\pgfpathlineto{\pgfqpoint{2.247741in}{0.924637in}}%
\pgfpathlineto{\pgfqpoint{2.306982in}{0.893422in}}%
\pgfpathlineto{\pgfqpoint{2.366223in}{0.865737in}}%
\pgfpathlineto{\pgfqpoint{2.425464in}{0.841301in}}%
\pgfpathlineto{\pgfqpoint{2.488937in}{0.818434in}}%
\pgfpathlineto{\pgfqpoint{2.556641in}{0.797504in}}%
\pgfpathlineto{\pgfqpoint{2.624345in}{0.779840in}}%
\pgfpathlineto{\pgfqpoint{2.696280in}{0.764346in}}%
\pgfpathlineto{\pgfqpoint{2.768216in}{0.751949in}}%
\pgfpathlineto{\pgfqpoint{2.844383in}{0.741940in}}%
\pgfpathlineto{\pgfqpoint{2.924782in}{0.734619in}}%
\pgfpathlineto{\pgfqpoint{3.005180in}{0.730455in}}%
\pgfpathlineto{\pgfqpoint{3.085579in}{0.729342in}}%
\pgfpathlineto{\pgfqpoint{3.165977in}{0.731252in}}%
\pgfpathlineto{\pgfqpoint{3.246376in}{0.736234in}}%
\pgfpathlineto{\pgfqpoint{3.326775in}{0.744412in}}%
\pgfpathlineto{\pgfqpoint{3.402942in}{0.755289in}}%
\pgfpathlineto{\pgfqpoint{3.474877in}{0.768570in}}%
\pgfpathlineto{\pgfqpoint{3.546813in}{0.785026in}}%
\pgfpathlineto{\pgfqpoint{3.614517in}{0.803679in}}%
\pgfpathlineto{\pgfqpoint{3.677989in}{0.824213in}}%
\pgfpathlineto{\pgfqpoint{3.741462in}{0.847966in}}%
\pgfpathlineto{\pgfqpoint{3.800703in}{0.873303in}}%
\pgfpathlineto{\pgfqpoint{3.859944in}{0.901968in}}%
\pgfpathlineto{\pgfqpoint{3.919185in}{0.934246in}}%
\pgfpathlineto{\pgfqpoint{3.974195in}{0.967726in}}%
\pgfpathlineto{\pgfqpoint{4.029204in}{1.004854in}}%
\pgfpathlineto{\pgfqpoint{4.084214in}{1.045911in}}%
\pgfpathlineto{\pgfqpoint{4.134992in}{1.087557in}}%
\pgfpathlineto{\pgfqpoint{4.185770in}{1.133052in}}%
\pgfpathlineto{\pgfqpoint{4.236548in}{1.182657in}}%
\pgfpathlineto{\pgfqpoint{4.287326in}{1.236644in}}%
\pgfpathlineto{\pgfqpoint{4.338104in}{1.295296in}}%
\pgfpathlineto{\pgfqpoint{4.388882in}{1.358909in}}%
\pgfpathlineto{\pgfqpoint{4.439660in}{1.427792in}}%
\pgfpathlineto{\pgfqpoint{4.490438in}{1.502264in}}%
\pgfpathlineto{\pgfqpoint{4.541216in}{1.582658in}}%
\pgfpathlineto{\pgfqpoint{4.591994in}{1.669320in}}%
\pgfpathlineto{\pgfqpoint{4.642772in}{1.762604in}}%
\pgfpathlineto{\pgfqpoint{4.693550in}{1.862882in}}%
\pgfpathlineto{\pgfqpoint{4.744328in}{1.970532in}}%
\pgfpathlineto{\pgfqpoint{4.795106in}{2.085949in}}%
\pgfpathlineto{\pgfqpoint{4.845885in}{2.209538in}}%
\pgfpathlineto{\pgfqpoint{4.896663in}{2.341716in}}%
\pgfpathlineto{\pgfqpoint{4.947441in}{2.482913in}}%
\pgfpathlineto{\pgfqpoint{4.998219in}{2.633570in}}%
\pgfpathlineto{\pgfqpoint{5.048997in}{2.794142in}}%
\pgfpathlineto{\pgfqpoint{5.099775in}{2.965095in}}%
\pgfpathlineto{\pgfqpoint{5.150553in}{3.146906in}}%
\pgfpathlineto{\pgfqpoint{5.188636in}{3.290684in}}%
\pgfpathlineto{\pgfqpoint{5.188636in}{3.290684in}}%
\pgfusepath{stroke}%
\end{pgfscope}%
\begin{pgfscope}%
\pgfsetrectcap%
\pgfsetmiterjoin%
\pgfsetlinewidth{0.803000pt}%
\definecolor{currentstroke}{rgb}{0.000000,0.000000,0.000000}%
\pgfsetstrokecolor{currentstroke}%
\pgfsetdash{}{0pt}%
\pgfpathmoveto{\pgfqpoint{0.750000in}{0.500000in}}%
\pgfpathlineto{\pgfqpoint{0.750000in}{3.520000in}}%
\pgfusepath{stroke}%
\end{pgfscope}%
\begin{pgfscope}%
\pgfsetrectcap%
\pgfsetmiterjoin%
\pgfsetlinewidth{0.803000pt}%
\definecolor{currentstroke}{rgb}{0.000000,0.000000,0.000000}%
\pgfsetstrokecolor{currentstroke}%
\pgfsetdash{}{0pt}%
\pgfpathmoveto{\pgfqpoint{5.400000in}{0.500000in}}%
\pgfpathlineto{\pgfqpoint{5.400000in}{3.520000in}}%
\pgfusepath{stroke}%
\end{pgfscope}%
\begin{pgfscope}%
\pgfsetrectcap%
\pgfsetmiterjoin%
\pgfsetlinewidth{0.803000pt}%
\definecolor{currentstroke}{rgb}{0.000000,0.000000,0.000000}%
\pgfsetstrokecolor{currentstroke}%
\pgfsetdash{}{0pt}%
\pgfpathmoveto{\pgfqpoint{0.750000in}{0.500000in}}%
\pgfpathlineto{\pgfqpoint{5.400000in}{0.500000in}}%
\pgfusepath{stroke}%
\end{pgfscope}%
\begin{pgfscope}%
\pgfsetrectcap%
\pgfsetmiterjoin%
\pgfsetlinewidth{0.803000pt}%
\definecolor{currentstroke}{rgb}{0.000000,0.000000,0.000000}%
\pgfsetstrokecolor{currentstroke}%
\pgfsetdash{}{0pt}%
\pgfpathmoveto{\pgfqpoint{0.750000in}{3.520000in}}%
\pgfpathlineto{\pgfqpoint{5.400000in}{3.520000in}}%
\pgfusepath{stroke}%
\end{pgfscope}%
\begin{pgfscope}%
\pgfsetbuttcap%
\pgfsetmiterjoin%
\definecolor{currentfill}{rgb}{1.000000,1.000000,1.000000}%
\pgfsetfillcolor{currentfill}%
\pgfsetfillopacity{0.800000}%
\pgfsetlinewidth{1.003750pt}%
\definecolor{currentstroke}{rgb}{0.800000,0.800000,0.800000}%
\pgfsetstrokecolor{currentstroke}%
\pgfsetstrokeopacity{0.800000}%
\pgfsetdash{}{0pt}%
\pgfpathmoveto{\pgfqpoint{0.847222in}{0.569444in}}%
\pgfpathlineto{\pgfqpoint{1.423852in}{0.569444in}}%
\pgfpathquadraticcurveto{\pgfqpoint{1.451630in}{0.569444in}}{\pgfqpoint{1.451630in}{0.597222in}}%
\pgfpathlineto{\pgfqpoint{1.451630in}{1.164352in}}%
\pgfpathquadraticcurveto{\pgfqpoint{1.451630in}{1.192129in}}{\pgfqpoint{1.423852in}{1.192129in}}%
\pgfpathlineto{\pgfqpoint{0.847222in}{1.192129in}}%
\pgfpathquadraticcurveto{\pgfqpoint{0.819444in}{1.192129in}}{\pgfqpoint{0.819444in}{1.164352in}}%
\pgfpathlineto{\pgfqpoint{0.819444in}{0.597222in}}%
\pgfpathquadraticcurveto{\pgfqpoint{0.819444in}{0.569444in}}{\pgfqpoint{0.847222in}{0.569444in}}%
\pgfpathclose%
\pgfusepath{stroke,fill}%
\end{pgfscope}%
\begin{pgfscope}%
\pgfsetrectcap%
\pgfsetroundjoin%
\pgfsetlinewidth{1.505625pt}%
\definecolor{currentstroke}{rgb}{0.121569,0.466667,0.705882}%
\pgfsetstrokecolor{currentstroke}%
\pgfsetdash{}{0pt}%
\pgfpathmoveto{\pgfqpoint{0.875000in}{1.087963in}}%
\pgfpathlineto{\pgfqpoint{1.152778in}{1.087963in}}%
\pgfusepath{stroke}%
\end{pgfscope}%
\begin{pgfscope}%
\pgftext[x=1.263889in,y=1.039352in,left,base]{\rmfamily\fontsize{10.000000}{12.000000}\selectfont \(\displaystyle f\)}%
\end{pgfscope}%
\begin{pgfscope}%
\pgfsetrectcap%
\pgfsetroundjoin%
\pgfsetlinewidth{1.505625pt}%
\definecolor{currentstroke}{rgb}{1.000000,0.498039,0.054902}%
\pgfsetstrokecolor{currentstroke}%
\pgfsetdash{}{0pt}%
\pgfpathmoveto{\pgfqpoint{0.875000in}{0.894290in}}%
\pgfpathlineto{\pgfqpoint{1.152778in}{0.894290in}}%
\pgfusepath{stroke}%
\end{pgfscope}%
\begin{pgfscope}%
\pgftext[x=1.263889in,y=0.845679in,left,base]{\rmfamily\fontsize{10.000000}{12.000000}\selectfont \(\displaystyle p_1\)}%
\end{pgfscope}%
\begin{pgfscope}%
\pgfsetrectcap%
\pgfsetroundjoin%
\pgfsetlinewidth{1.505625pt}%
\definecolor{currentstroke}{rgb}{0.172549,0.627451,0.172549}%
\pgfsetstrokecolor{currentstroke}%
\pgfsetdash{}{0pt}%
\pgfpathmoveto{\pgfqpoint{0.875000in}{0.700617in}}%
\pgfpathlineto{\pgfqpoint{1.152778in}{0.700617in}}%
\pgfusepath{stroke}%
\end{pgfscope}%
\begin{pgfscope}%
\pgftext[x=1.263889in,y=0.652006in,left,base]{\rmfamily\fontsize{10.000000}{12.000000}\selectfont \(\displaystyle p_2\)}%
\end{pgfscope}%
\end{pgfpicture}%
\makeatother%
\endgroup%
}
\caption{Graph of $f$ and $p_{\cdot}$} \label{Fig:Origin}
\end{figure}
\begin{figure}[htbp]
\centering \scalebox{0.8}{%% Creator: Matplotlib, PGF backend
%%
%% To include the figure in your LaTeX document, write
%%   \input{<filename>.pgf}
%%
%% Make sure the required packages are loaded in your preamble
%%   \usepackage{pgf}
%%
%% Figures using additional raster images can only be included by \input if
%% they are in the same directory as the main LaTeX file. For loading figures
%% from other directories you can use the `import` package
%%   \usepackage{import}
%% and then include the figures with
%%   \import{<path to file>}{<filename>.pgf}
%%
%% Matplotlib used the following preamble
%%   \usepackage{fontspec}
%%
\begingroup%
\makeatletter%
\begin{pgfpicture}%
\pgfpathrectangle{\pgfpointorigin}{\pgfqpoint{6.000000in}{4.000000in}}%
\pgfusepath{use as bounding box, clip}%
\begin{pgfscope}%
\pgfsetbuttcap%
\pgfsetmiterjoin%
\definecolor{currentfill}{rgb}{1.000000,1.000000,1.000000}%
\pgfsetfillcolor{currentfill}%
\pgfsetlinewidth{0.000000pt}%
\definecolor{currentstroke}{rgb}{1.000000,1.000000,1.000000}%
\pgfsetstrokecolor{currentstroke}%
\pgfsetdash{}{0pt}%
\pgfpathmoveto{\pgfqpoint{0.000000in}{0.000000in}}%
\pgfpathlineto{\pgfqpoint{6.000000in}{0.000000in}}%
\pgfpathlineto{\pgfqpoint{6.000000in}{4.000000in}}%
\pgfpathlineto{\pgfqpoint{0.000000in}{4.000000in}}%
\pgfpathclose%
\pgfusepath{fill}%
\end{pgfscope}%
\begin{pgfscope}%
\pgfsetbuttcap%
\pgfsetmiterjoin%
\definecolor{currentfill}{rgb}{1.000000,1.000000,1.000000}%
\pgfsetfillcolor{currentfill}%
\pgfsetlinewidth{0.000000pt}%
\definecolor{currentstroke}{rgb}{0.000000,0.000000,0.000000}%
\pgfsetstrokecolor{currentstroke}%
\pgfsetstrokeopacity{0.000000}%
\pgfsetdash{}{0pt}%
\pgfpathmoveto{\pgfqpoint{0.750000in}{0.500000in}}%
\pgfpathlineto{\pgfqpoint{5.400000in}{0.500000in}}%
\pgfpathlineto{\pgfqpoint{5.400000in}{3.520000in}}%
\pgfpathlineto{\pgfqpoint{0.750000in}{3.520000in}}%
\pgfpathclose%
\pgfusepath{fill}%
\end{pgfscope}%
\begin{pgfscope}%
\pgfpathrectangle{\pgfqpoint{0.750000in}{0.500000in}}{\pgfqpoint{4.650000in}{3.020000in}}%
\pgfusepath{clip}%
\pgfsetrectcap%
\pgfsetroundjoin%
\pgfsetlinewidth{0.803000pt}%
\definecolor{currentstroke}{rgb}{0.690196,0.690196,0.690196}%
\pgfsetstrokecolor{currentstroke}%
\pgfsetdash{}{0pt}%
\pgfpathmoveto{\pgfqpoint{0.961364in}{0.500000in}}%
\pgfpathlineto{\pgfqpoint{0.961364in}{3.520000in}}%
\pgfusepath{stroke}%
\end{pgfscope}%
\begin{pgfscope}%
\pgfsetbuttcap%
\pgfsetroundjoin%
\definecolor{currentfill}{rgb}{0.000000,0.000000,0.000000}%
\pgfsetfillcolor{currentfill}%
\pgfsetlinewidth{0.803000pt}%
\definecolor{currentstroke}{rgb}{0.000000,0.000000,0.000000}%
\pgfsetstrokecolor{currentstroke}%
\pgfsetdash{}{0pt}%
\pgfsys@defobject{currentmarker}{\pgfqpoint{0.000000in}{-0.048611in}}{\pgfqpoint{0.000000in}{0.000000in}}{%
\pgfpathmoveto{\pgfqpoint{0.000000in}{0.000000in}}%
\pgfpathlineto{\pgfqpoint{0.000000in}{-0.048611in}}%
\pgfusepath{stroke,fill}%
}%
\begin{pgfscope}%
\pgfsys@transformshift{0.961364in}{0.500000in}%
\pgfsys@useobject{currentmarker}{}%
\end{pgfscope}%
\end{pgfscope}%
\begin{pgfscope}%
\pgftext[x=0.961364in,y=0.402778in,,top]{\rmfamily\fontsize{10.000000}{12.000000}\selectfont \(\displaystyle -1.00\)}%
\end{pgfscope}%
\begin{pgfscope}%
\pgfpathrectangle{\pgfqpoint{0.750000in}{0.500000in}}{\pgfqpoint{4.650000in}{3.020000in}}%
\pgfusepath{clip}%
\pgfsetrectcap%
\pgfsetroundjoin%
\pgfsetlinewidth{0.803000pt}%
\definecolor{currentstroke}{rgb}{0.690196,0.690196,0.690196}%
\pgfsetstrokecolor{currentstroke}%
\pgfsetdash{}{0pt}%
\pgfpathmoveto{\pgfqpoint{1.489773in}{0.500000in}}%
\pgfpathlineto{\pgfqpoint{1.489773in}{3.520000in}}%
\pgfusepath{stroke}%
\end{pgfscope}%
\begin{pgfscope}%
\pgfsetbuttcap%
\pgfsetroundjoin%
\definecolor{currentfill}{rgb}{0.000000,0.000000,0.000000}%
\pgfsetfillcolor{currentfill}%
\pgfsetlinewidth{0.803000pt}%
\definecolor{currentstroke}{rgb}{0.000000,0.000000,0.000000}%
\pgfsetstrokecolor{currentstroke}%
\pgfsetdash{}{0pt}%
\pgfsys@defobject{currentmarker}{\pgfqpoint{0.000000in}{-0.048611in}}{\pgfqpoint{0.000000in}{0.000000in}}{%
\pgfpathmoveto{\pgfqpoint{0.000000in}{0.000000in}}%
\pgfpathlineto{\pgfqpoint{0.000000in}{-0.048611in}}%
\pgfusepath{stroke,fill}%
}%
\begin{pgfscope}%
\pgfsys@transformshift{1.489773in}{0.500000in}%
\pgfsys@useobject{currentmarker}{}%
\end{pgfscope}%
\end{pgfscope}%
\begin{pgfscope}%
\pgftext[x=1.489773in,y=0.402778in,,top]{\rmfamily\fontsize{10.000000}{12.000000}\selectfont \(\displaystyle -0.75\)}%
\end{pgfscope}%
\begin{pgfscope}%
\pgfpathrectangle{\pgfqpoint{0.750000in}{0.500000in}}{\pgfqpoint{4.650000in}{3.020000in}}%
\pgfusepath{clip}%
\pgfsetrectcap%
\pgfsetroundjoin%
\pgfsetlinewidth{0.803000pt}%
\definecolor{currentstroke}{rgb}{0.690196,0.690196,0.690196}%
\pgfsetstrokecolor{currentstroke}%
\pgfsetdash{}{0pt}%
\pgfpathmoveto{\pgfqpoint{2.018182in}{0.500000in}}%
\pgfpathlineto{\pgfqpoint{2.018182in}{3.520000in}}%
\pgfusepath{stroke}%
\end{pgfscope}%
\begin{pgfscope}%
\pgfsetbuttcap%
\pgfsetroundjoin%
\definecolor{currentfill}{rgb}{0.000000,0.000000,0.000000}%
\pgfsetfillcolor{currentfill}%
\pgfsetlinewidth{0.803000pt}%
\definecolor{currentstroke}{rgb}{0.000000,0.000000,0.000000}%
\pgfsetstrokecolor{currentstroke}%
\pgfsetdash{}{0pt}%
\pgfsys@defobject{currentmarker}{\pgfqpoint{0.000000in}{-0.048611in}}{\pgfqpoint{0.000000in}{0.000000in}}{%
\pgfpathmoveto{\pgfqpoint{0.000000in}{0.000000in}}%
\pgfpathlineto{\pgfqpoint{0.000000in}{-0.048611in}}%
\pgfusepath{stroke,fill}%
}%
\begin{pgfscope}%
\pgfsys@transformshift{2.018182in}{0.500000in}%
\pgfsys@useobject{currentmarker}{}%
\end{pgfscope}%
\end{pgfscope}%
\begin{pgfscope}%
\pgftext[x=2.018182in,y=0.402778in,,top]{\rmfamily\fontsize{10.000000}{12.000000}\selectfont \(\displaystyle -0.50\)}%
\end{pgfscope}%
\begin{pgfscope}%
\pgfpathrectangle{\pgfqpoint{0.750000in}{0.500000in}}{\pgfqpoint{4.650000in}{3.020000in}}%
\pgfusepath{clip}%
\pgfsetrectcap%
\pgfsetroundjoin%
\pgfsetlinewidth{0.803000pt}%
\definecolor{currentstroke}{rgb}{0.690196,0.690196,0.690196}%
\pgfsetstrokecolor{currentstroke}%
\pgfsetdash{}{0pt}%
\pgfpathmoveto{\pgfqpoint{2.546591in}{0.500000in}}%
\pgfpathlineto{\pgfqpoint{2.546591in}{3.520000in}}%
\pgfusepath{stroke}%
\end{pgfscope}%
\begin{pgfscope}%
\pgfsetbuttcap%
\pgfsetroundjoin%
\definecolor{currentfill}{rgb}{0.000000,0.000000,0.000000}%
\pgfsetfillcolor{currentfill}%
\pgfsetlinewidth{0.803000pt}%
\definecolor{currentstroke}{rgb}{0.000000,0.000000,0.000000}%
\pgfsetstrokecolor{currentstroke}%
\pgfsetdash{}{0pt}%
\pgfsys@defobject{currentmarker}{\pgfqpoint{0.000000in}{-0.048611in}}{\pgfqpoint{0.000000in}{0.000000in}}{%
\pgfpathmoveto{\pgfqpoint{0.000000in}{0.000000in}}%
\pgfpathlineto{\pgfqpoint{0.000000in}{-0.048611in}}%
\pgfusepath{stroke,fill}%
}%
\begin{pgfscope}%
\pgfsys@transformshift{2.546591in}{0.500000in}%
\pgfsys@useobject{currentmarker}{}%
\end{pgfscope}%
\end{pgfscope}%
\begin{pgfscope}%
\pgftext[x=2.546591in,y=0.402778in,,top]{\rmfamily\fontsize{10.000000}{12.000000}\selectfont \(\displaystyle -0.25\)}%
\end{pgfscope}%
\begin{pgfscope}%
\pgfpathrectangle{\pgfqpoint{0.750000in}{0.500000in}}{\pgfqpoint{4.650000in}{3.020000in}}%
\pgfusepath{clip}%
\pgfsetrectcap%
\pgfsetroundjoin%
\pgfsetlinewidth{0.803000pt}%
\definecolor{currentstroke}{rgb}{0.690196,0.690196,0.690196}%
\pgfsetstrokecolor{currentstroke}%
\pgfsetdash{}{0pt}%
\pgfpathmoveto{\pgfqpoint{3.075000in}{0.500000in}}%
\pgfpathlineto{\pgfqpoint{3.075000in}{3.520000in}}%
\pgfusepath{stroke}%
\end{pgfscope}%
\begin{pgfscope}%
\pgfsetbuttcap%
\pgfsetroundjoin%
\definecolor{currentfill}{rgb}{0.000000,0.000000,0.000000}%
\pgfsetfillcolor{currentfill}%
\pgfsetlinewidth{0.803000pt}%
\definecolor{currentstroke}{rgb}{0.000000,0.000000,0.000000}%
\pgfsetstrokecolor{currentstroke}%
\pgfsetdash{}{0pt}%
\pgfsys@defobject{currentmarker}{\pgfqpoint{0.000000in}{-0.048611in}}{\pgfqpoint{0.000000in}{0.000000in}}{%
\pgfpathmoveto{\pgfqpoint{0.000000in}{0.000000in}}%
\pgfpathlineto{\pgfqpoint{0.000000in}{-0.048611in}}%
\pgfusepath{stroke,fill}%
}%
\begin{pgfscope}%
\pgfsys@transformshift{3.075000in}{0.500000in}%
\pgfsys@useobject{currentmarker}{}%
\end{pgfscope}%
\end{pgfscope}%
\begin{pgfscope}%
\pgftext[x=3.075000in,y=0.402778in,,top]{\rmfamily\fontsize{10.000000}{12.000000}\selectfont \(\displaystyle 0.00\)}%
\end{pgfscope}%
\begin{pgfscope}%
\pgfpathrectangle{\pgfqpoint{0.750000in}{0.500000in}}{\pgfqpoint{4.650000in}{3.020000in}}%
\pgfusepath{clip}%
\pgfsetrectcap%
\pgfsetroundjoin%
\pgfsetlinewidth{0.803000pt}%
\definecolor{currentstroke}{rgb}{0.690196,0.690196,0.690196}%
\pgfsetstrokecolor{currentstroke}%
\pgfsetdash{}{0pt}%
\pgfpathmoveto{\pgfqpoint{3.603409in}{0.500000in}}%
\pgfpathlineto{\pgfqpoint{3.603409in}{3.520000in}}%
\pgfusepath{stroke}%
\end{pgfscope}%
\begin{pgfscope}%
\pgfsetbuttcap%
\pgfsetroundjoin%
\definecolor{currentfill}{rgb}{0.000000,0.000000,0.000000}%
\pgfsetfillcolor{currentfill}%
\pgfsetlinewidth{0.803000pt}%
\definecolor{currentstroke}{rgb}{0.000000,0.000000,0.000000}%
\pgfsetstrokecolor{currentstroke}%
\pgfsetdash{}{0pt}%
\pgfsys@defobject{currentmarker}{\pgfqpoint{0.000000in}{-0.048611in}}{\pgfqpoint{0.000000in}{0.000000in}}{%
\pgfpathmoveto{\pgfqpoint{0.000000in}{0.000000in}}%
\pgfpathlineto{\pgfqpoint{0.000000in}{-0.048611in}}%
\pgfusepath{stroke,fill}%
}%
\begin{pgfscope}%
\pgfsys@transformshift{3.603409in}{0.500000in}%
\pgfsys@useobject{currentmarker}{}%
\end{pgfscope}%
\end{pgfscope}%
\begin{pgfscope}%
\pgftext[x=3.603409in,y=0.402778in,,top]{\rmfamily\fontsize{10.000000}{12.000000}\selectfont \(\displaystyle 0.25\)}%
\end{pgfscope}%
\begin{pgfscope}%
\pgfpathrectangle{\pgfqpoint{0.750000in}{0.500000in}}{\pgfqpoint{4.650000in}{3.020000in}}%
\pgfusepath{clip}%
\pgfsetrectcap%
\pgfsetroundjoin%
\pgfsetlinewidth{0.803000pt}%
\definecolor{currentstroke}{rgb}{0.690196,0.690196,0.690196}%
\pgfsetstrokecolor{currentstroke}%
\pgfsetdash{}{0pt}%
\pgfpathmoveto{\pgfqpoint{4.131818in}{0.500000in}}%
\pgfpathlineto{\pgfqpoint{4.131818in}{3.520000in}}%
\pgfusepath{stroke}%
\end{pgfscope}%
\begin{pgfscope}%
\pgfsetbuttcap%
\pgfsetroundjoin%
\definecolor{currentfill}{rgb}{0.000000,0.000000,0.000000}%
\pgfsetfillcolor{currentfill}%
\pgfsetlinewidth{0.803000pt}%
\definecolor{currentstroke}{rgb}{0.000000,0.000000,0.000000}%
\pgfsetstrokecolor{currentstroke}%
\pgfsetdash{}{0pt}%
\pgfsys@defobject{currentmarker}{\pgfqpoint{0.000000in}{-0.048611in}}{\pgfqpoint{0.000000in}{0.000000in}}{%
\pgfpathmoveto{\pgfqpoint{0.000000in}{0.000000in}}%
\pgfpathlineto{\pgfqpoint{0.000000in}{-0.048611in}}%
\pgfusepath{stroke,fill}%
}%
\begin{pgfscope}%
\pgfsys@transformshift{4.131818in}{0.500000in}%
\pgfsys@useobject{currentmarker}{}%
\end{pgfscope}%
\end{pgfscope}%
\begin{pgfscope}%
\pgftext[x=4.131818in,y=0.402778in,,top]{\rmfamily\fontsize{10.000000}{12.000000}\selectfont \(\displaystyle 0.50\)}%
\end{pgfscope}%
\begin{pgfscope}%
\pgfpathrectangle{\pgfqpoint{0.750000in}{0.500000in}}{\pgfqpoint{4.650000in}{3.020000in}}%
\pgfusepath{clip}%
\pgfsetrectcap%
\pgfsetroundjoin%
\pgfsetlinewidth{0.803000pt}%
\definecolor{currentstroke}{rgb}{0.690196,0.690196,0.690196}%
\pgfsetstrokecolor{currentstroke}%
\pgfsetdash{}{0pt}%
\pgfpathmoveto{\pgfqpoint{4.660227in}{0.500000in}}%
\pgfpathlineto{\pgfqpoint{4.660227in}{3.520000in}}%
\pgfusepath{stroke}%
\end{pgfscope}%
\begin{pgfscope}%
\pgfsetbuttcap%
\pgfsetroundjoin%
\definecolor{currentfill}{rgb}{0.000000,0.000000,0.000000}%
\pgfsetfillcolor{currentfill}%
\pgfsetlinewidth{0.803000pt}%
\definecolor{currentstroke}{rgb}{0.000000,0.000000,0.000000}%
\pgfsetstrokecolor{currentstroke}%
\pgfsetdash{}{0pt}%
\pgfsys@defobject{currentmarker}{\pgfqpoint{0.000000in}{-0.048611in}}{\pgfqpoint{0.000000in}{0.000000in}}{%
\pgfpathmoveto{\pgfqpoint{0.000000in}{0.000000in}}%
\pgfpathlineto{\pgfqpoint{0.000000in}{-0.048611in}}%
\pgfusepath{stroke,fill}%
}%
\begin{pgfscope}%
\pgfsys@transformshift{4.660227in}{0.500000in}%
\pgfsys@useobject{currentmarker}{}%
\end{pgfscope}%
\end{pgfscope}%
\begin{pgfscope}%
\pgftext[x=4.660227in,y=0.402778in,,top]{\rmfamily\fontsize{10.000000}{12.000000}\selectfont \(\displaystyle 0.75\)}%
\end{pgfscope}%
\begin{pgfscope}%
\pgfpathrectangle{\pgfqpoint{0.750000in}{0.500000in}}{\pgfqpoint{4.650000in}{3.020000in}}%
\pgfusepath{clip}%
\pgfsetrectcap%
\pgfsetroundjoin%
\pgfsetlinewidth{0.803000pt}%
\definecolor{currentstroke}{rgb}{0.690196,0.690196,0.690196}%
\pgfsetstrokecolor{currentstroke}%
\pgfsetdash{}{0pt}%
\pgfpathmoveto{\pgfqpoint{5.188636in}{0.500000in}}%
\pgfpathlineto{\pgfqpoint{5.188636in}{3.520000in}}%
\pgfusepath{stroke}%
\end{pgfscope}%
\begin{pgfscope}%
\pgfsetbuttcap%
\pgfsetroundjoin%
\definecolor{currentfill}{rgb}{0.000000,0.000000,0.000000}%
\pgfsetfillcolor{currentfill}%
\pgfsetlinewidth{0.803000pt}%
\definecolor{currentstroke}{rgb}{0.000000,0.000000,0.000000}%
\pgfsetstrokecolor{currentstroke}%
\pgfsetdash{}{0pt}%
\pgfsys@defobject{currentmarker}{\pgfqpoint{0.000000in}{-0.048611in}}{\pgfqpoint{0.000000in}{0.000000in}}{%
\pgfpathmoveto{\pgfqpoint{0.000000in}{0.000000in}}%
\pgfpathlineto{\pgfqpoint{0.000000in}{-0.048611in}}%
\pgfusepath{stroke,fill}%
}%
\begin{pgfscope}%
\pgfsys@transformshift{5.188636in}{0.500000in}%
\pgfsys@useobject{currentmarker}{}%
\end{pgfscope}%
\end{pgfscope}%
\begin{pgfscope}%
\pgftext[x=5.188636in,y=0.402778in,,top]{\rmfamily\fontsize{10.000000}{12.000000}\selectfont \(\displaystyle 1.00\)}%
\end{pgfscope}%
\begin{pgfscope}%
\pgfpathrectangle{\pgfqpoint{0.750000in}{0.500000in}}{\pgfqpoint{4.650000in}{3.020000in}}%
\pgfusepath{clip}%
\pgfsetrectcap%
\pgfsetroundjoin%
\pgfsetlinewidth{0.803000pt}%
\definecolor{currentstroke}{rgb}{0.690196,0.690196,0.690196}%
\pgfsetstrokecolor{currentstroke}%
\pgfsetdash{}{0pt}%
\pgfpathmoveto{\pgfqpoint{0.750000in}{0.500000in}}%
\pgfpathlineto{\pgfqpoint{5.400000in}{0.500000in}}%
\pgfusepath{stroke}%
\end{pgfscope}%
\begin{pgfscope}%
\pgfsetbuttcap%
\pgfsetroundjoin%
\definecolor{currentfill}{rgb}{0.000000,0.000000,0.000000}%
\pgfsetfillcolor{currentfill}%
\pgfsetlinewidth{0.803000pt}%
\definecolor{currentstroke}{rgb}{0.000000,0.000000,0.000000}%
\pgfsetstrokecolor{currentstroke}%
\pgfsetdash{}{0pt}%
\pgfsys@defobject{currentmarker}{\pgfqpoint{-0.048611in}{0.000000in}}{\pgfqpoint{0.000000in}{0.000000in}}{%
\pgfpathmoveto{\pgfqpoint{0.000000in}{0.000000in}}%
\pgfpathlineto{\pgfqpoint{-0.048611in}{0.000000in}}%
\pgfusepath{stroke,fill}%
}%
\begin{pgfscope}%
\pgfsys@transformshift{0.750000in}{0.500000in}%
\pgfsys@useobject{currentmarker}{}%
\end{pgfscope}%
\end{pgfscope}%
\begin{pgfscope}%
\pgftext[x=0.228394in,y=0.451806in,left,base]{\rmfamily\fontsize{10.000000}{12.000000}\selectfont \(\displaystyle -0.015\)}%
\end{pgfscope}%
\begin{pgfscope}%
\pgfpathrectangle{\pgfqpoint{0.750000in}{0.500000in}}{\pgfqpoint{4.650000in}{3.020000in}}%
\pgfusepath{clip}%
\pgfsetrectcap%
\pgfsetroundjoin%
\pgfsetlinewidth{0.803000pt}%
\definecolor{currentstroke}{rgb}{0.690196,0.690196,0.690196}%
\pgfsetstrokecolor{currentstroke}%
\pgfsetdash{}{0pt}%
\pgfpathmoveto{\pgfqpoint{0.750000in}{1.003333in}}%
\pgfpathlineto{\pgfqpoint{5.400000in}{1.003333in}}%
\pgfusepath{stroke}%
\end{pgfscope}%
\begin{pgfscope}%
\pgfsetbuttcap%
\pgfsetroundjoin%
\definecolor{currentfill}{rgb}{0.000000,0.000000,0.000000}%
\pgfsetfillcolor{currentfill}%
\pgfsetlinewidth{0.803000pt}%
\definecolor{currentstroke}{rgb}{0.000000,0.000000,0.000000}%
\pgfsetstrokecolor{currentstroke}%
\pgfsetdash{}{0pt}%
\pgfsys@defobject{currentmarker}{\pgfqpoint{-0.048611in}{0.000000in}}{\pgfqpoint{0.000000in}{0.000000in}}{%
\pgfpathmoveto{\pgfqpoint{0.000000in}{0.000000in}}%
\pgfpathlineto{\pgfqpoint{-0.048611in}{0.000000in}}%
\pgfusepath{stroke,fill}%
}%
\begin{pgfscope}%
\pgfsys@transformshift{0.750000in}{1.003333in}%
\pgfsys@useobject{currentmarker}{}%
\end{pgfscope}%
\end{pgfscope}%
\begin{pgfscope}%
\pgftext[x=0.228394in,y=0.955139in,left,base]{\rmfamily\fontsize{10.000000}{12.000000}\selectfont \(\displaystyle -0.010\)}%
\end{pgfscope}%
\begin{pgfscope}%
\pgfpathrectangle{\pgfqpoint{0.750000in}{0.500000in}}{\pgfqpoint{4.650000in}{3.020000in}}%
\pgfusepath{clip}%
\pgfsetrectcap%
\pgfsetroundjoin%
\pgfsetlinewidth{0.803000pt}%
\definecolor{currentstroke}{rgb}{0.690196,0.690196,0.690196}%
\pgfsetstrokecolor{currentstroke}%
\pgfsetdash{}{0pt}%
\pgfpathmoveto{\pgfqpoint{0.750000in}{1.506667in}}%
\pgfpathlineto{\pgfqpoint{5.400000in}{1.506667in}}%
\pgfusepath{stroke}%
\end{pgfscope}%
\begin{pgfscope}%
\pgfsetbuttcap%
\pgfsetroundjoin%
\definecolor{currentfill}{rgb}{0.000000,0.000000,0.000000}%
\pgfsetfillcolor{currentfill}%
\pgfsetlinewidth{0.803000pt}%
\definecolor{currentstroke}{rgb}{0.000000,0.000000,0.000000}%
\pgfsetstrokecolor{currentstroke}%
\pgfsetdash{}{0pt}%
\pgfsys@defobject{currentmarker}{\pgfqpoint{-0.048611in}{0.000000in}}{\pgfqpoint{0.000000in}{0.000000in}}{%
\pgfpathmoveto{\pgfqpoint{0.000000in}{0.000000in}}%
\pgfpathlineto{\pgfqpoint{-0.048611in}{0.000000in}}%
\pgfusepath{stroke,fill}%
}%
\begin{pgfscope}%
\pgfsys@transformshift{0.750000in}{1.506667in}%
\pgfsys@useobject{currentmarker}{}%
\end{pgfscope}%
\end{pgfscope}%
\begin{pgfscope}%
\pgftext[x=0.228394in,y=1.458472in,left,base]{\rmfamily\fontsize{10.000000}{12.000000}\selectfont \(\displaystyle -0.005\)}%
\end{pgfscope}%
\begin{pgfscope}%
\pgfpathrectangle{\pgfqpoint{0.750000in}{0.500000in}}{\pgfqpoint{4.650000in}{3.020000in}}%
\pgfusepath{clip}%
\pgfsetrectcap%
\pgfsetroundjoin%
\pgfsetlinewidth{0.803000pt}%
\definecolor{currentstroke}{rgb}{0.690196,0.690196,0.690196}%
\pgfsetstrokecolor{currentstroke}%
\pgfsetdash{}{0pt}%
\pgfpathmoveto{\pgfqpoint{0.750000in}{2.010000in}}%
\pgfpathlineto{\pgfqpoint{5.400000in}{2.010000in}}%
\pgfusepath{stroke}%
\end{pgfscope}%
\begin{pgfscope}%
\pgfsetbuttcap%
\pgfsetroundjoin%
\definecolor{currentfill}{rgb}{0.000000,0.000000,0.000000}%
\pgfsetfillcolor{currentfill}%
\pgfsetlinewidth{0.803000pt}%
\definecolor{currentstroke}{rgb}{0.000000,0.000000,0.000000}%
\pgfsetstrokecolor{currentstroke}%
\pgfsetdash{}{0pt}%
\pgfsys@defobject{currentmarker}{\pgfqpoint{-0.048611in}{0.000000in}}{\pgfqpoint{0.000000in}{0.000000in}}{%
\pgfpathmoveto{\pgfqpoint{0.000000in}{0.000000in}}%
\pgfpathlineto{\pgfqpoint{-0.048611in}{0.000000in}}%
\pgfusepath{stroke,fill}%
}%
\begin{pgfscope}%
\pgfsys@transformshift{0.750000in}{2.010000in}%
\pgfsys@useobject{currentmarker}{}%
\end{pgfscope}%
\end{pgfscope}%
\begin{pgfscope}%
\pgftext[x=0.336419in,y=1.961806in,left,base]{\rmfamily\fontsize{10.000000}{12.000000}\selectfont \(\displaystyle 0.000\)}%
\end{pgfscope}%
\begin{pgfscope}%
\pgfpathrectangle{\pgfqpoint{0.750000in}{0.500000in}}{\pgfqpoint{4.650000in}{3.020000in}}%
\pgfusepath{clip}%
\pgfsetrectcap%
\pgfsetroundjoin%
\pgfsetlinewidth{0.803000pt}%
\definecolor{currentstroke}{rgb}{0.690196,0.690196,0.690196}%
\pgfsetstrokecolor{currentstroke}%
\pgfsetdash{}{0pt}%
\pgfpathmoveto{\pgfqpoint{0.750000in}{2.513333in}}%
\pgfpathlineto{\pgfqpoint{5.400000in}{2.513333in}}%
\pgfusepath{stroke}%
\end{pgfscope}%
\begin{pgfscope}%
\pgfsetbuttcap%
\pgfsetroundjoin%
\definecolor{currentfill}{rgb}{0.000000,0.000000,0.000000}%
\pgfsetfillcolor{currentfill}%
\pgfsetlinewidth{0.803000pt}%
\definecolor{currentstroke}{rgb}{0.000000,0.000000,0.000000}%
\pgfsetstrokecolor{currentstroke}%
\pgfsetdash{}{0pt}%
\pgfsys@defobject{currentmarker}{\pgfqpoint{-0.048611in}{0.000000in}}{\pgfqpoint{0.000000in}{0.000000in}}{%
\pgfpathmoveto{\pgfqpoint{0.000000in}{0.000000in}}%
\pgfpathlineto{\pgfqpoint{-0.048611in}{0.000000in}}%
\pgfusepath{stroke,fill}%
}%
\begin{pgfscope}%
\pgfsys@transformshift{0.750000in}{2.513333in}%
\pgfsys@useobject{currentmarker}{}%
\end{pgfscope}%
\end{pgfscope}%
\begin{pgfscope}%
\pgftext[x=0.336419in,y=2.465139in,left,base]{\rmfamily\fontsize{10.000000}{12.000000}\selectfont \(\displaystyle 0.005\)}%
\end{pgfscope}%
\begin{pgfscope}%
\pgfpathrectangle{\pgfqpoint{0.750000in}{0.500000in}}{\pgfqpoint{4.650000in}{3.020000in}}%
\pgfusepath{clip}%
\pgfsetrectcap%
\pgfsetroundjoin%
\pgfsetlinewidth{0.803000pt}%
\definecolor{currentstroke}{rgb}{0.690196,0.690196,0.690196}%
\pgfsetstrokecolor{currentstroke}%
\pgfsetdash{}{0pt}%
\pgfpathmoveto{\pgfqpoint{0.750000in}{3.016667in}}%
\pgfpathlineto{\pgfqpoint{5.400000in}{3.016667in}}%
\pgfusepath{stroke}%
\end{pgfscope}%
\begin{pgfscope}%
\pgfsetbuttcap%
\pgfsetroundjoin%
\definecolor{currentfill}{rgb}{0.000000,0.000000,0.000000}%
\pgfsetfillcolor{currentfill}%
\pgfsetlinewidth{0.803000pt}%
\definecolor{currentstroke}{rgb}{0.000000,0.000000,0.000000}%
\pgfsetstrokecolor{currentstroke}%
\pgfsetdash{}{0pt}%
\pgfsys@defobject{currentmarker}{\pgfqpoint{-0.048611in}{0.000000in}}{\pgfqpoint{0.000000in}{0.000000in}}{%
\pgfpathmoveto{\pgfqpoint{0.000000in}{0.000000in}}%
\pgfpathlineto{\pgfqpoint{-0.048611in}{0.000000in}}%
\pgfusepath{stroke,fill}%
}%
\begin{pgfscope}%
\pgfsys@transformshift{0.750000in}{3.016667in}%
\pgfsys@useobject{currentmarker}{}%
\end{pgfscope}%
\end{pgfscope}%
\begin{pgfscope}%
\pgftext[x=0.336419in,y=2.968472in,left,base]{\rmfamily\fontsize{10.000000}{12.000000}\selectfont \(\displaystyle 0.010\)}%
\end{pgfscope}%
\begin{pgfscope}%
\pgfpathrectangle{\pgfqpoint{0.750000in}{0.500000in}}{\pgfqpoint{4.650000in}{3.020000in}}%
\pgfusepath{clip}%
\pgfsetrectcap%
\pgfsetroundjoin%
\pgfsetlinewidth{0.803000pt}%
\definecolor{currentstroke}{rgb}{0.690196,0.690196,0.690196}%
\pgfsetstrokecolor{currentstroke}%
\pgfsetdash{}{0pt}%
\pgfpathmoveto{\pgfqpoint{0.750000in}{3.520000in}}%
\pgfpathlineto{\pgfqpoint{5.400000in}{3.520000in}}%
\pgfusepath{stroke}%
\end{pgfscope}%
\begin{pgfscope}%
\pgfsetbuttcap%
\pgfsetroundjoin%
\definecolor{currentfill}{rgb}{0.000000,0.000000,0.000000}%
\pgfsetfillcolor{currentfill}%
\pgfsetlinewidth{0.803000pt}%
\definecolor{currentstroke}{rgb}{0.000000,0.000000,0.000000}%
\pgfsetstrokecolor{currentstroke}%
\pgfsetdash{}{0pt}%
\pgfsys@defobject{currentmarker}{\pgfqpoint{-0.048611in}{0.000000in}}{\pgfqpoint{0.000000in}{0.000000in}}{%
\pgfpathmoveto{\pgfqpoint{0.000000in}{0.000000in}}%
\pgfpathlineto{\pgfqpoint{-0.048611in}{0.000000in}}%
\pgfusepath{stroke,fill}%
}%
\begin{pgfscope}%
\pgfsys@transformshift{0.750000in}{3.520000in}%
\pgfsys@useobject{currentmarker}{}%
\end{pgfscope}%
\end{pgfscope}%
\begin{pgfscope}%
\pgftext[x=0.336419in,y=3.471806in,left,base]{\rmfamily\fontsize{10.000000}{12.000000}\selectfont \(\displaystyle 0.015\)}%
\end{pgfscope}%
\begin{pgfscope}%
\pgfpathrectangle{\pgfqpoint{0.750000in}{0.500000in}}{\pgfqpoint{4.650000in}{3.020000in}}%
\pgfusepath{clip}%
\pgfsetrectcap%
\pgfsetroundjoin%
\pgfsetlinewidth{1.505625pt}%
\definecolor{currentstroke}{rgb}{0.121569,0.466667,0.705882}%
\pgfsetstrokecolor{currentstroke}%
\pgfsetdash{}{0pt}%
\pgfpathmoveto{\pgfqpoint{0.961364in}{2.904044in}}%
\pgfpathlineto{\pgfqpoint{0.990984in}{2.581764in}}%
\pgfpathlineto{\pgfqpoint{1.016373in}{2.336359in}}%
\pgfpathlineto{\pgfqpoint{1.041762in}{2.117906in}}%
\pgfpathlineto{\pgfqpoint{1.067151in}{1.925005in}}%
\pgfpathlineto{\pgfqpoint{1.092540in}{1.756288in}}%
\pgfpathlineto{\pgfqpoint{1.117929in}{1.610414in}}%
\pgfpathlineto{\pgfqpoint{1.139087in}{1.505358in}}%
\pgfpathlineto{\pgfqpoint{1.160244in}{1.414515in}}%
\pgfpathlineto{\pgfqpoint{1.181402in}{1.337160in}}%
\pgfpathlineto{\pgfqpoint{1.202559in}{1.272580in}}%
\pgfpathlineto{\pgfqpoint{1.219485in}{1.229645in}}%
\pgfpathlineto{\pgfqpoint{1.236411in}{1.194089in}}%
\pgfpathlineto{\pgfqpoint{1.253337in}{1.165568in}}%
\pgfpathlineto{\pgfqpoint{1.270263in}{1.143744in}}%
\pgfpathlineto{\pgfqpoint{1.282958in}{1.131571in}}%
\pgfpathlineto{\pgfqpoint{1.295652in}{1.122840in}}%
\pgfpathlineto{\pgfqpoint{1.308347in}{1.117415in}}%
\pgfpathlineto{\pgfqpoint{1.321041in}{1.115162in}}%
\pgfpathlineto{\pgfqpoint{1.333736in}{1.115948in}}%
\pgfpathlineto{\pgfqpoint{1.346431in}{1.119644in}}%
\pgfpathlineto{\pgfqpoint{1.359125in}{1.126120in}}%
\pgfpathlineto{\pgfqpoint{1.376051in}{1.138863in}}%
\pgfpathlineto{\pgfqpoint{1.392977in}{1.156029in}}%
\pgfpathlineto{\pgfqpoint{1.409903in}{1.177330in}}%
\pgfpathlineto{\pgfqpoint{1.431061in}{1.209342in}}%
\pgfpathlineto{\pgfqpoint{1.452218in}{1.246834in}}%
\pgfpathlineto{\pgfqpoint{1.477607in}{1.298322in}}%
\pgfpathlineto{\pgfqpoint{1.507228in}{1.366243in}}%
\pgfpathlineto{\pgfqpoint{1.541080in}{1.452581in}}%
\pgfpathlineto{\pgfqpoint{1.579163in}{1.558540in}}%
\pgfpathlineto{\pgfqpoint{1.629941in}{1.710079in}}%
\pgfpathlineto{\pgfqpoint{1.718803in}{1.987847in}}%
\pgfpathlineto{\pgfqpoint{1.794970in}{2.221536in}}%
\pgfpathlineto{\pgfqpoint{1.845748in}{2.367412in}}%
\pgfpathlineto{\pgfqpoint{1.888063in}{2.479769in}}%
\pgfpathlineto{\pgfqpoint{1.926147in}{2.572091in}}%
\pgfpathlineto{\pgfqpoint{1.959999in}{2.646159in}}%
\pgfpathlineto{\pgfqpoint{1.989619in}{2.704194in}}%
\pgfpathlineto{\pgfqpoint{2.019240in}{2.755457in}}%
\pgfpathlineto{\pgfqpoint{2.044629in}{2.793725in}}%
\pgfpathlineto{\pgfqpoint{2.070018in}{2.826543in}}%
\pgfpathlineto{\pgfqpoint{2.095407in}{2.853743in}}%
\pgfpathlineto{\pgfqpoint{2.116564in}{2.872015in}}%
\pgfpathlineto{\pgfqpoint{2.137722in}{2.886223in}}%
\pgfpathlineto{\pgfqpoint{2.158879in}{2.896316in}}%
\pgfpathlineto{\pgfqpoint{2.175805in}{2.901404in}}%
\pgfpathlineto{\pgfqpoint{2.192731in}{2.903824in}}%
\pgfpathlineto{\pgfqpoint{2.209657in}{2.903572in}}%
\pgfpathlineto{\pgfqpoint{2.226583in}{2.900647in}}%
\pgfpathlineto{\pgfqpoint{2.243509in}{2.895056in}}%
\pgfpathlineto{\pgfqpoint{2.264667in}{2.884338in}}%
\pgfpathlineto{\pgfqpoint{2.285824in}{2.869512in}}%
\pgfpathlineto{\pgfqpoint{2.306982in}{2.850631in}}%
\pgfpathlineto{\pgfqpoint{2.328140in}{2.827762in}}%
\pgfpathlineto{\pgfqpoint{2.353529in}{2.795171in}}%
\pgfpathlineto{\pgfqpoint{2.378918in}{2.757125in}}%
\pgfpathlineto{\pgfqpoint{2.404307in}{2.713822in}}%
\pgfpathlineto{\pgfqpoint{2.433927in}{2.656966in}}%
\pgfpathlineto{\pgfqpoint{2.463548in}{2.593687in}}%
\pgfpathlineto{\pgfqpoint{2.497400in}{2.514112in}}%
\pgfpathlineto{\pgfqpoint{2.535483in}{2.416297in}}%
\pgfpathlineto{\pgfqpoint{2.577798in}{2.298826in}}%
\pgfpathlineto{\pgfqpoint{2.628576in}{2.148370in}}%
\pgfpathlineto{\pgfqpoint{2.704743in}{1.911488in}}%
\pgfpathlineto{\pgfqpoint{2.793605in}{1.636510in}}%
\pgfpathlineto{\pgfqpoint{2.840152in}{1.501874in}}%
\pgfpathlineto{\pgfqpoint{2.878235in}{1.400642in}}%
\pgfpathlineto{\pgfqpoint{2.912087in}{1.319706in}}%
\pgfpathlineto{\pgfqpoint{2.941708in}{1.257487in}}%
\pgfpathlineto{\pgfqpoint{2.967097in}{1.211640in}}%
\pgfpathlineto{\pgfqpoint{2.988254in}{1.179399in}}%
\pgfpathlineto{\pgfqpoint{3.009412in}{1.153153in}}%
\pgfpathlineto{\pgfqpoint{3.026338in}{1.136846in}}%
\pgfpathlineto{\pgfqpoint{3.043264in}{1.125019in}}%
\pgfpathlineto{\pgfqpoint{3.055958in}{1.119267in}}%
\pgfpathlineto{\pgfqpoint{3.068653in}{1.116330in}}%
\pgfpathlineto{\pgfqpoint{3.081347in}{1.116330in}}%
\pgfpathlineto{\pgfqpoint{3.094042in}{1.119267in}}%
\pgfpathlineto{\pgfqpoint{3.106736in}{1.125019in}}%
\pgfpathlineto{\pgfqpoint{3.123662in}{1.136846in}}%
\pgfpathlineto{\pgfqpoint{3.140588in}{1.153153in}}%
\pgfpathlineto{\pgfqpoint{3.157514in}{1.173652in}}%
\pgfpathlineto{\pgfqpoint{3.178672in}{1.204738in}}%
\pgfpathlineto{\pgfqpoint{3.199829in}{1.241388in}}%
\pgfpathlineto{\pgfqpoint{3.225218in}{1.291971in}}%
\pgfpathlineto{\pgfqpoint{3.254839in}{1.358979in}}%
\pgfpathlineto{\pgfqpoint{3.288691in}{1.444452in}}%
\pgfpathlineto{\pgfqpoint{3.326775in}{1.549658in}}%
\pgfpathlineto{\pgfqpoint{3.373321in}{1.687616in}}%
\pgfpathlineto{\pgfqpoint{3.449488in}{1.924795in}}%
\pgfpathlineto{\pgfqpoint{3.538350in}{2.199466in}}%
\pgfpathlineto{\pgfqpoint{3.589128in}{2.346810in}}%
\pgfpathlineto{\pgfqpoint{3.631443in}{2.460781in}}%
\pgfpathlineto{\pgfqpoint{3.669526in}{2.554823in}}%
\pgfpathlineto{\pgfqpoint{3.703378in}{2.630607in}}%
\pgfpathlineto{\pgfqpoint{3.732999in}{2.690270in}}%
\pgfpathlineto{\pgfqpoint{3.762619in}{2.743264in}}%
\pgfpathlineto{\pgfqpoint{3.788008in}{2.783086in}}%
\pgfpathlineto{\pgfqpoint{3.813397in}{2.817516in}}%
\pgfpathlineto{\pgfqpoint{3.838787in}{2.846374in}}%
\pgfpathlineto{\pgfqpoint{3.859944in}{2.866058in}}%
\pgfpathlineto{\pgfqpoint{3.881102in}{2.881700in}}%
\pgfpathlineto{\pgfqpoint{3.902259in}{2.893243in}}%
\pgfpathlineto{\pgfqpoint{3.923417in}{2.900647in}}%
\pgfpathlineto{\pgfqpoint{3.940343in}{2.903572in}}%
\pgfpathlineto{\pgfqpoint{3.957269in}{2.903824in}}%
\pgfpathlineto{\pgfqpoint{3.974195in}{2.901404in}}%
\pgfpathlineto{\pgfqpoint{3.991121in}{2.896316in}}%
\pgfpathlineto{\pgfqpoint{4.012278in}{2.886223in}}%
\pgfpathlineto{\pgfqpoint{4.033436in}{2.872015in}}%
\pgfpathlineto{\pgfqpoint{4.054593in}{2.853743in}}%
\pgfpathlineto{\pgfqpoint{4.075751in}{2.831471in}}%
\pgfpathlineto{\pgfqpoint{4.101140in}{2.799578in}}%
\pgfpathlineto{\pgfqpoint{4.126529in}{2.762205in}}%
\pgfpathlineto{\pgfqpoint{4.151918in}{2.719549in}}%
\pgfpathlineto{\pgfqpoint{4.181538in}{2.663408in}}%
\pgfpathlineto{\pgfqpoint{4.211159in}{2.600793in}}%
\pgfpathlineto{\pgfqpoint{4.245011in}{2.521910in}}%
\pgfpathlineto{\pgfqpoint{4.283094in}{2.424774in}}%
\pgfpathlineto{\pgfqpoint{4.325410in}{2.307920in}}%
\pgfpathlineto{\pgfqpoint{4.376188in}{2.157988in}}%
\pgfpathlineto{\pgfqpoint{4.448123in}{1.934692in}}%
\pgfpathlineto{\pgfqpoint{4.545448in}{1.633148in}}%
\pgfpathlineto{\pgfqpoint{4.591994in}{1.498664in}}%
\pgfpathlineto{\pgfqpoint{4.630078in}{1.397626in}}%
\pgfpathlineto{\pgfqpoint{4.663930in}{1.316919in}}%
\pgfpathlineto{\pgfqpoint{4.693550in}{1.254944in}}%
\pgfpathlineto{\pgfqpoint{4.718939in}{1.209342in}}%
\pgfpathlineto{\pgfqpoint{4.740097in}{1.177330in}}%
\pgfpathlineto{\pgfqpoint{4.761254in}{1.151338in}}%
\pgfpathlineto{\pgfqpoint{4.778180in}{1.135251in}}%
\pgfpathlineto{\pgfqpoint{4.795106in}{1.123660in}}%
\pgfpathlineto{\pgfqpoint{4.807801in}{1.118097in}}%
\pgfpathlineto{\pgfqpoint{4.820495in}{1.115357in}}%
\pgfpathlineto{\pgfqpoint{4.833190in}{1.115569in}}%
\pgfpathlineto{\pgfqpoint{4.845885in}{1.118865in}}%
\pgfpathlineto{\pgfqpoint{4.858579in}{1.125377in}}%
\pgfpathlineto{\pgfqpoint{4.871274in}{1.135240in}}%
\pgfpathlineto{\pgfqpoint{4.883968in}{1.148591in}}%
\pgfpathlineto{\pgfqpoint{4.896663in}{1.165568in}}%
\pgfpathlineto{\pgfqpoint{4.913589in}{1.194089in}}%
\pgfpathlineto{\pgfqpoint{4.930515in}{1.229645in}}%
\pgfpathlineto{\pgfqpoint{4.947441in}{1.272580in}}%
\pgfpathlineto{\pgfqpoint{4.964367in}{1.323244in}}%
\pgfpathlineto{\pgfqpoint{4.985524in}{1.397988in}}%
\pgfpathlineto{\pgfqpoint{5.006682in}{1.486076in}}%
\pgfpathlineto{\pgfqpoint{5.027839in}{1.588230in}}%
\pgfpathlineto{\pgfqpoint{5.048997in}{1.705190in}}%
\pgfpathlineto{\pgfqpoint{5.070154in}{1.837708in}}%
\pgfpathlineto{\pgfqpoint{5.095543in}{2.018348in}}%
\pgfpathlineto{\pgfqpoint{5.120932in}{2.223852in}}%
\pgfpathlineto{\pgfqpoint{5.146321in}{2.455604in}}%
\pgfpathlineto{\pgfqpoint{5.171710in}{2.715019in}}%
\pgfpathlineto{\pgfqpoint{5.188636in}{2.904044in}}%
\pgfpathlineto{\pgfqpoint{5.188636in}{2.904044in}}%
\pgfusepath{stroke}%
\end{pgfscope}%
\begin{pgfscope}%
\pgfsetrectcap%
\pgfsetmiterjoin%
\pgfsetlinewidth{0.803000pt}%
\definecolor{currentstroke}{rgb}{0.000000,0.000000,0.000000}%
\pgfsetstrokecolor{currentstroke}%
\pgfsetdash{}{0pt}%
\pgfpathmoveto{\pgfqpoint{0.750000in}{0.500000in}}%
\pgfpathlineto{\pgfqpoint{0.750000in}{3.520000in}}%
\pgfusepath{stroke}%
\end{pgfscope}%
\begin{pgfscope}%
\pgfsetrectcap%
\pgfsetmiterjoin%
\pgfsetlinewidth{0.803000pt}%
\definecolor{currentstroke}{rgb}{0.000000,0.000000,0.000000}%
\pgfsetstrokecolor{currentstroke}%
\pgfsetdash{}{0pt}%
\pgfpathmoveto{\pgfqpoint{5.400000in}{0.500000in}}%
\pgfpathlineto{\pgfqpoint{5.400000in}{3.520000in}}%
\pgfusepath{stroke}%
\end{pgfscope}%
\begin{pgfscope}%
\pgfsetrectcap%
\pgfsetmiterjoin%
\pgfsetlinewidth{0.803000pt}%
\definecolor{currentstroke}{rgb}{0.000000,0.000000,0.000000}%
\pgfsetstrokecolor{currentstroke}%
\pgfsetdash{}{0pt}%
\pgfpathmoveto{\pgfqpoint{0.750000in}{0.500000in}}%
\pgfpathlineto{\pgfqpoint{5.400000in}{0.500000in}}%
\pgfusepath{stroke}%
\end{pgfscope}%
\begin{pgfscope}%
\pgfsetrectcap%
\pgfsetmiterjoin%
\pgfsetlinewidth{0.803000pt}%
\definecolor{currentstroke}{rgb}{0.000000,0.000000,0.000000}%
\pgfsetstrokecolor{currentstroke}%
\pgfsetdash{}{0pt}%
\pgfpathmoveto{\pgfqpoint{0.750000in}{3.520000in}}%
\pgfpathlineto{\pgfqpoint{5.400000in}{3.520000in}}%
\pgfusepath{stroke}%
\end{pgfscope}%
\begin{pgfscope}%
\pgfsetbuttcap%
\pgfsetmiterjoin%
\definecolor{currentfill}{rgb}{1.000000,1.000000,1.000000}%
\pgfsetfillcolor{currentfill}%
\pgfsetfillopacity{0.800000}%
\pgfsetlinewidth{1.003750pt}%
\definecolor{currentstroke}{rgb}{0.800000,0.800000,0.800000}%
\pgfsetstrokecolor{currentstroke}%
\pgfsetstrokeopacity{0.800000}%
\pgfsetdash{}{0pt}%
\pgfpathmoveto{\pgfqpoint{4.473447in}{3.215216in}}%
\pgfpathlineto{\pgfqpoint{5.302778in}{3.215216in}}%
\pgfpathquadraticcurveto{\pgfqpoint{5.330556in}{3.215216in}}{\pgfqpoint{5.330556in}{3.242994in}}%
\pgfpathlineto{\pgfqpoint{5.330556in}{3.422778in}}%
\pgfpathquadraticcurveto{\pgfqpoint{5.330556in}{3.450556in}}{\pgfqpoint{5.302778in}{3.450556in}}%
\pgfpathlineto{\pgfqpoint{4.473447in}{3.450556in}}%
\pgfpathquadraticcurveto{\pgfqpoint{4.445670in}{3.450556in}}{\pgfqpoint{4.445670in}{3.422778in}}%
\pgfpathlineto{\pgfqpoint{4.445670in}{3.242994in}}%
\pgfpathquadraticcurveto{\pgfqpoint{4.445670in}{3.215216in}}{\pgfqpoint{4.473447in}{3.215216in}}%
\pgfpathclose%
\pgfusepath{stroke,fill}%
\end{pgfscope}%
\begin{pgfscope}%
\pgfsetrectcap%
\pgfsetroundjoin%
\pgfsetlinewidth{1.505625pt}%
\definecolor{currentstroke}{rgb}{0.121569,0.466667,0.705882}%
\pgfsetstrokecolor{currentstroke}%
\pgfsetdash{}{0pt}%
\pgfpathmoveto{\pgfqpoint{4.501225in}{3.346389in}}%
\pgfpathlineto{\pgfqpoint{4.779003in}{3.346389in}}%
\pgfusepath{stroke}%
\end{pgfscope}%
\begin{pgfscope}%
\pgftext[x=4.890114in,y=3.297778in,left,base]{\rmfamily\fontsize{10.000000}{12.000000}\selectfont \(\displaystyle  p_2 - f \)}%
\end{pgfscope}%
\end{pgfpicture}%
\makeatother%
\endgroup%
}
\caption{Graph of $ p_2 - f $} \label{Fig:Residue}
\end{figure}
The final polynomial is
\begin{equation}
\begin{split}
p_2 \rbr{x} &= - 9.600345 \cdot 10^{-11} x^{5} + 0.603579 x^{4} \\
&+ 1.136689 \cdot 10^{-10} x^{3} + 0.414182 x^{2} - 1.766548 \cdot 10^{-11} x - 0.008881,
\end{split}
\end{equation}
while the final control points are
\begin{equation}
-1.000000, -0.832979, -0.414353, -0.000000, 0.414353, 0.832979, 1.000000.
\end{equation}
\end{thmquestion}

\begin{thmquestion}
\
\begin{thmproof}
Consider the space to be continuos function on an interval $ \sbr{ a, b } $ and the inner product to be
\begin{equation}
\pbr{ f, g } = \intb{a}{b}{ f \rbr{x} g \rbr{x} \sd x }.
\end{equation}

Suppose zeros of $p_i$ are $ \xi_1 < \xi_2 < \cdots < \xi_k $ with $ k < i $, and without loss of generality assume $p_i$ are positive, negative, positive, \ldots on $ \rsbr{ \xi_k, b }, \rbr{ \xi_{ k - 1 }, \xi_k }, \cdots, \rbr{ \xi_1, \xi_2 }, \srbr{ a , \xi_1 } $ respectively. Consider
\begin{equation}
g = \rbr{ x - \xi_1 } \rbr{ x - \xi_2 } \cdots \rbr{ x - \xi_k }.
\end{equation}
Because $ p_i g \ge 0 $, and $ p_i g $ have some strictly positive points, therefore
\begin{equation}
\pbr{ p_i, g } = \intb{a}{b}{ p_i \rbr{x} g \rbr{x} \sd x } > 0.
\end{equation}
However, because $ \deg g = k < i $, therefore $ g \in \opspan \cbr{ p_0, p_1, \cdots, p_{ i - 1 } } $ and $ \pbr{ p_i, g } = 0 $, which leads to contradiction. Consequently, $p_i$ has at least $i$ zeros. Because $p_i$ is a polynomial of degree $i$, $p_i$ has exactly $i$ zeros in $ \sbr{ a, b } $.
\end{thmproof}
\end{thmquestion}

\end{document}
