% !TeX encoding = UTF-8
% !TeX program = lualatex
% !TeX spellcheck = en_US

% Author : Zhihan Li
% Description : Report for Lecture 4 --- Best Approximation

\documentclass[english, nochinese]{../TeXTemplate/pkupaper}

\usepackage[paper, cmrgreekup]{../TeXTemplate/def}

\newcommand{\cuniversity}{Peking University}
\newcommand{\cthesisname}{Introduction to Applied Mathematics}
\newcommand{\titlemark}{Assignment for Lecture 4}

\DeclareRobustCommand{\authoring}%
{%
\begin{tabular}{c}%
Zhihan Li \\%
1600010653%
\end{tabular}%
}

\title{\titlemark}
\author{\authoring}
\begin{document}

\maketitle

\begin{thmquestion}
\
\begin{thmproof}
Prove by contradiction. Otherwise, there exists $ c = \msbr{ c_1 & c_2 & \cdots & c_m }^{\rmut} $, such that $ A c = 0 $, and therefore $ c^{\dag} A c = 0 $.

Note that
\begin{equation}
\begin{split}
c^{\dag} A c &= \sume{i}{1}{m}{\sume{j}{1}{m}{ \overline{c_j} \pbr{ g_j, g_i } c_i }} \\
&= \pbr{ \sume{j}{1}{m}{ \overline{c_j} g_j }, \sume{i}{1}{m}{ \overline{c_i} g_i } },
\end{split}
\end{equation}
and consequently from the positively definiteness
\begin{equation}
\sume{i}{1}{m}{ \overline{c_i} g_i } = 0.
\end{equation}
However this contradicts the given linear independence, which shows $A$ is non-singular.

\sqed
\end{thmproof}
\end{thmquestion}

\begin{thmquestion}
\ 
\begin{thmanswer}
Let the best approximation be $ g = \alpha x + \beta x^3 + \gamma x^5 $. From
\begin{equation}
f - g \perp \opspan \cbr{ x, x^3, x^5 },
\end{equation}
it can be derived that
\begin{gather}
\pbr{ f - g, x } = \pbr{ f - g, x^3 } = \pbr{ f - g, x^5 } = 0.
\end{gather}
Direct computation leads to the system
\begin{equation}
\left[\begin{matrix}\frac{2}{3} & \frac{2}{5} & \frac{2}{7}\\\frac{2}{5} & \frac{2}{7} & \frac{2}{9}\\\frac{2}{7} & \frac{2}{9} & \frac{2}{11}\end{matrix}\right] \msbr{ \alpha \\ \beta \\ \gamma } = \left[\begin{matrix}- 2 \cos{\left (1 \right )} + 2 \sin{\left (1 \right )}\\- 6 \sin{\left (1 \right )} + 10 \cos{\left (1 \right )}\\- 202 \cos{\left (1 \right )} + 130 \sin{\left (1 \right )}\end{matrix}\right],
\end{equation}
whose solution is
\begin{equation}
\msbr{ \alpha \\ \beta \\ \gamma } = \left[\begin{matrix}- \frac{139965}{8} \cos{\left (1 \right )} + 11235 \sin{\left (1 \right )}\\- \frac{104265}{2} \sin{\left (1 \right )} + \frac{324765}{4} \cos{\left (1 \right )}\\- \frac{582813}{8} \cos{\left (1 \right )} + \frac{93555}{2} \sin{\left (1 \right )}\end{matrix}\right].
\end{equation}
Therefore, the final solution is
\begin{equation}
\begin{split}
g &= x^{5} \left(- \frac{582813}{8} \cos{\left (1 \right )} + \frac{93555}{2} \sin{\left (1 \right )}\right) \\
&+ x^{3} \left(- \frac{104265}{2} \sin{\left (1 \right )} + \frac{324765}{4} \cos{\left (1 \right )}\right) \\
&+ x \left(- \frac{139965}{8} \cos{\left (1 \right )} + 11235 \sin{\left (1 \right )}\right).
\end{split}
\end{equation}
\end{thmanswer}
\end{thmquestion}

\begin{thmquestion}
\
\begin{thmproof}
Because
\begin{equation}
\norm{u_i} = \frac{\norm{ v_i - \sume{j}{1}{ i - 1 }{ \pbr{ v_i, u_j } u_j } }}{\norm{ v_i - \sume{j}{1}{ i - 1 }{ \pbr{ v_i, u_j } u_j } }} = 1
\end{equation}
for $ 1 \le i \le n $, it remains to prove $u_i$ are pairwise perpendicular.

Perform mathematical induction to prove that for a given $i$ with $ 1 \le i \le n $, $ u_j \perp u_k $ is always true for $ 1 \le j < k \le i $. Suppose the case $ i = m - 1  $ is done and then consider the case $ i = m $, where $ 1 < m \le n $. It is sufficient to prove $ u_m \perp u_k $ for $ 1 \le k < m $, which directly follows from
\begin{equation}
\pbr{ u_m, u_k } = \pbr{ v_m, u_k  } - \sume{j}{1}{ m - 1 }{ \pbr{ v_m, u_j } \pbr{ u_j, u_k } } = \pbr{ v_m, u_k } - \pbr{ v_m, u_k } \norm{u_k} = 0.
\end{equation}
Therefore, the induction is finished and we obtain $ u_j \perp u_k $ for $ 1 \le j < k \le n $.

As shown above, $u_i$ are pairwise orthogonal and normalized, which means they form a orthonormal basis.

\sqed
\end{thmproof}
\end{thmquestion}

\begin{thmquestion}
\
\begin{thmproof}
The leading coefficient of $L_n$ is
\begin{equation}
\frac{ \rbr{ 2 n } ! }{ 2^n n ! n ! } = \frac{ \rbr{ 2 n - 1 } !! }{ n ! },
\end{equation}
and therefore
\begin{equation}
g = \frac{ n ! }{ \rbr{ 2 n - 1 } !! } L_n
\end{equation}
is monic. Note that $ L_0, L_1, \cdots, L_{ n - 1 } $ form a basis of $ \opspan \cbr{ 1, x, \cdots, x^{ n - 1 } } $. Therefore, for any monic polynomial $f$ of degree $n$, there exists $ c_i \crbr{ 0 \le i \le n - 1 } $ such that
\begin{equation}
f - g = \sume{i}{0}{ n - 1 }{ c_i L_i }.
\end{equation}
Therefore,
\begin{equation}
\begin{split}
\norm{f}^2 &= \norm{ g + \sume{i}{0}{ n - 1 }{ c_i L_i }}^2 \\
&= \norm{g}^2 + \sume{i}{0}{ n - 1 }{ c_i^2 \norm{L_i}^2 }
\end{split}
\end{equation}
where the last equality follows from the orthogonality of $L_i$. Because $ \norm{L_i}^2 > 0 $, we have $ \norm{f} \ge \norm{g} $, and the equality is reached iff $ c_i = 0 \crbr{ 0 \le i \le n - 1 } $, which is equivalent to $ f = g $. Consequently,
\begin{gather}
g = \argmin_{ f \mtx{monic}, \deg f = n } \norm{f}, \\
\norm{g} = \min_{ f \mtx{monic}, \deg f = n } \norm{f}.
\end{gather}

\sqed
\end{thmproof}
\end{thmquestion}

\begin{thmanswer}
\ 
\begin{thmquestion}
Suppose the best uniform approximation is $ g \rbr{x} = a x + b $, and therefore there exists a Chebyshev alternance of $3$ points for the $ R \rbr{x} = \sin \frac{ \spi x }{2} - a x - b $. Becuase $R$ is concave over $ \rbr{ 0, 1 } $, therefore two of the points are $ 0, 1 $ respectively. Because $ R \rbr{0} = R \rbr{1} $, it follows that $ a = 1 $ and consequently the third point is
\begin{equation}
\xi = \frac{2}{\spi} \arccos \frac{2}{\spi}.
\end{equation}
From $ R \rbr{0} = -R \rbr{\xi} $, we derive that
\begin{equation}
b = \frac{1}{2} \sin \arccos \frac{2}{\spi} - \frac{1}{\spi} \arccos \frac{2}{\spi} =  \frac{\sqrt{ \spi^2 - 4 }}{ 2 \spi } - \frac{1}{\spi} \arccos \frac{2}{\spi}
\end{equation}
and
\begin{equation}
g \rbr{x} = x + \frac{\sqrt{ \spi^2 - 4 }}{ 2 \spi } - \frac{1}{\spi} \arccos \frac{2}{\spi}.
\end{equation}
\end{thmquestion}
\end{thmanswer}

\begin{thmanswer}
\ 
\begin{thmquestion}
Let $T_n$ be the Chebyshev polynomial of degree $n$, $ r_n = \frac{1}{2^{ n - 1 }} T_n $ and $ f_n = x^n - r_n $. Because $r_n$ is monic, therefore $ f_n \in \mathcal{P}_{ n - 1 } $.

Note that $ r_n = x^n - f_n $ have a Chebyshev alternance of $ n + 1 $ points: let
\begin{equation}
x_k = \cos \frac{ k \spi }{ 2 n }, \crbr{ 0 \le k \le n },
\end{equation}
and then
\begin{partlist}
\item $x_k$ are $ n + 1 $ distinct points arranged from right to left on the axis;
\item
\begin{equation}
r_n \rbr{x_k} = \frac{1}{2^{ n - 1 }} T_n \rbr{x_k} = \frac{1}{2^{ n - 1 }} \rbr{-1}^k;
\end{equation}
\item $ \abs{ r_n \rbr{x} } \le \frac{1}{2^{ n - 1 }} $; \label{Item:NormBound}
\item the equality in \ref{Item:NormBound} is only reached at $x_k$.
\end{partlist}
Because $ f_n \in \mathcal{P}_{ n - 1 } $, therefore $f_n$ is the best uniform approximation of $x^n$.

In conclusion, $ f_n = x^n - \frac{1}{2^{ n - 1 }} T_n $ is the best uniform approximation of $x^n$, and the Chebyshev alternance consists of $ x_k = \cos \frac{ k \spi }{ 2 n } \crbr{ 0 \le k \le n } $.
\end{thmquestion}
\end{thmanswer}

\begin{thmquestion}
\
\begin{thmproof}
Consider
\begin{equation}
u_i = \msbr{ \phi_i \rbr{x_1} & \phi_i \rbr{x_2} & \phi_i \rbr{x_3} & \cdots & \phi_i \rbr{x_{ i + 1 }} }^{\rmut}
\end{equation}
and the matrix
\begin{equation}
M = \msbr{ u_i & u_1 & u_2 & u_3 & \cdots & u_i }.
\end{equation}
Because $M$ is linear dependent in terms of columns, we have
\begin{align}
0 &= \det M \\
&= \sume{j}{1}{n}{ \rbr{-1}^{ j + 1 } \phi_i \rbr{x_j} \rbr{x_j} D_j } \\
&= -\sume{j}{1}{n}{ \phi_i \rbr{x_j} \sigma_j }.
\end{align}
Therefore,
\begin{equation}
\sume{j}{1}{n}{ \phi_i \rbr{x_j} \sigma_j } = 0
\end{equation}
follows as desired.

\sqed
\end{thmproof}
\end{thmquestion}

\begin{thmquestion}
\ 
\begin{thmanswer}
The Python code is placed in the file \verb"Problem8.ipynb". The algorithm succeeded in converging in the $ k = 2 $nd iteration, such that
\begin{equation}
\maxe{j}{1}{ n + 1 }{\abs{\epsilon^{\rbr{k}}_j}} - \mine{j}{1}{ n + 1 }{\abs{\epsilon^{\rbr{k}}_j}} = \text{8.720947e-06} < \text{1e-4}.
\end{equation}
\end{thmanswer}
Plot of $f$ and $p_{\cdot}$ is shown in Figure \ref{Fig:Origin}, and plot of residue is shown in Figure \ref{Fig:Residue}.
\begin{figure}[htbp]
\centering \scalebox{0.8}{%% Creator: Matplotlib, PGF backend
%%
%% To include the figure in your LaTeX document, write
%%   \input{<filename>.pgf}
%%
%% Make sure the required packages are loaded in your preamble
%%   \usepackage{pgf}
%%
%% Figures using additional raster images can only be included by \input if
%% they are in the same directory as the main LaTeX file. For loading figures
%% from other directories you can use the `import` package
%%   \usepackage{import}
%% and then include the figures with
%%   \import{<path to file>}{<filename>.pgf}
%%
%% Matplotlib used the following preamble
%%   \usepackage{fontspec}
%%
\begingroup%
\makeatletter%
\begin{pgfpicture}%
\pgfpathrectangle{\pgfpointorigin}{\pgfqpoint{6.000000in}{4.000000in}}%
\pgfusepath{use as bounding box, clip}%
\begin{pgfscope}%
\pgfsetbuttcap%
\pgfsetmiterjoin%
\definecolor{currentfill}{rgb}{1.000000,1.000000,1.000000}%
\pgfsetfillcolor{currentfill}%
\pgfsetlinewidth{0.000000pt}%
\definecolor{currentstroke}{rgb}{1.000000,1.000000,1.000000}%
\pgfsetstrokecolor{currentstroke}%
\pgfsetdash{}{0pt}%
\pgfpathmoveto{\pgfqpoint{0.000000in}{0.000000in}}%
\pgfpathlineto{\pgfqpoint{6.000000in}{0.000000in}}%
\pgfpathlineto{\pgfqpoint{6.000000in}{4.000000in}}%
\pgfpathlineto{\pgfqpoint{0.000000in}{4.000000in}}%
\pgfpathclose%
\pgfusepath{fill}%
\end{pgfscope}%
\begin{pgfscope}%
\pgfsetbuttcap%
\pgfsetmiterjoin%
\definecolor{currentfill}{rgb}{1.000000,1.000000,1.000000}%
\pgfsetfillcolor{currentfill}%
\pgfsetlinewidth{0.000000pt}%
\definecolor{currentstroke}{rgb}{0.000000,0.000000,0.000000}%
\pgfsetstrokecolor{currentstroke}%
\pgfsetstrokeopacity{0.000000}%
\pgfsetdash{}{0pt}%
\pgfpathmoveto{\pgfqpoint{0.750000in}{0.500000in}}%
\pgfpathlineto{\pgfqpoint{5.400000in}{0.500000in}}%
\pgfpathlineto{\pgfqpoint{5.400000in}{3.520000in}}%
\pgfpathlineto{\pgfqpoint{0.750000in}{3.520000in}}%
\pgfpathclose%
\pgfusepath{fill}%
\end{pgfscope}%
\begin{pgfscope}%
\pgfpathrectangle{\pgfqpoint{0.750000in}{0.500000in}}{\pgfqpoint{4.650000in}{3.020000in}}%
\pgfusepath{clip}%
\pgfsetrectcap%
\pgfsetroundjoin%
\pgfsetlinewidth{0.803000pt}%
\definecolor{currentstroke}{rgb}{0.690196,0.690196,0.690196}%
\pgfsetstrokecolor{currentstroke}%
\pgfsetdash{}{0pt}%
\pgfpathmoveto{\pgfqpoint{0.750000in}{0.500000in}}%
\pgfpathlineto{\pgfqpoint{0.750000in}{3.520000in}}%
\pgfusepath{stroke}%
\end{pgfscope}%
\begin{pgfscope}%
\pgfsetbuttcap%
\pgfsetroundjoin%
\definecolor{currentfill}{rgb}{0.000000,0.000000,0.000000}%
\pgfsetfillcolor{currentfill}%
\pgfsetlinewidth{0.803000pt}%
\definecolor{currentstroke}{rgb}{0.000000,0.000000,0.000000}%
\pgfsetstrokecolor{currentstroke}%
\pgfsetdash{}{0pt}%
\pgfsys@defobject{currentmarker}{\pgfqpoint{0.000000in}{-0.048611in}}{\pgfqpoint{0.000000in}{0.000000in}}{%
\pgfpathmoveto{\pgfqpoint{0.000000in}{0.000000in}}%
\pgfpathlineto{\pgfqpoint{0.000000in}{-0.048611in}}%
\pgfusepath{stroke,fill}%
}%
\begin{pgfscope}%
\pgfsys@transformshift{0.750000in}{0.500000in}%
\pgfsys@useobject{currentmarker}{}%
\end{pgfscope}%
\end{pgfscope}%
\begin{pgfscope}%
\pgftext[x=0.750000in,y=0.402778in,,top]{\rmfamily\fontsize{10.000000}{12.000000}\selectfont \(\displaystyle -6\)}%
\end{pgfscope}%
\begin{pgfscope}%
\pgfpathrectangle{\pgfqpoint{0.750000in}{0.500000in}}{\pgfqpoint{4.650000in}{3.020000in}}%
\pgfusepath{clip}%
\pgfsetrectcap%
\pgfsetroundjoin%
\pgfsetlinewidth{0.803000pt}%
\definecolor{currentstroke}{rgb}{0.690196,0.690196,0.690196}%
\pgfsetstrokecolor{currentstroke}%
\pgfsetdash{}{0pt}%
\pgfpathmoveto{\pgfqpoint{1.525000in}{0.500000in}}%
\pgfpathlineto{\pgfqpoint{1.525000in}{3.520000in}}%
\pgfusepath{stroke}%
\end{pgfscope}%
\begin{pgfscope}%
\pgfsetbuttcap%
\pgfsetroundjoin%
\definecolor{currentfill}{rgb}{0.000000,0.000000,0.000000}%
\pgfsetfillcolor{currentfill}%
\pgfsetlinewidth{0.803000pt}%
\definecolor{currentstroke}{rgb}{0.000000,0.000000,0.000000}%
\pgfsetstrokecolor{currentstroke}%
\pgfsetdash{}{0pt}%
\pgfsys@defobject{currentmarker}{\pgfqpoint{0.000000in}{-0.048611in}}{\pgfqpoint{0.000000in}{0.000000in}}{%
\pgfpathmoveto{\pgfqpoint{0.000000in}{0.000000in}}%
\pgfpathlineto{\pgfqpoint{0.000000in}{-0.048611in}}%
\pgfusepath{stroke,fill}%
}%
\begin{pgfscope}%
\pgfsys@transformshift{1.525000in}{0.500000in}%
\pgfsys@useobject{currentmarker}{}%
\end{pgfscope}%
\end{pgfscope}%
\begin{pgfscope}%
\pgftext[x=1.525000in,y=0.402778in,,top]{\rmfamily\fontsize{10.000000}{12.000000}\selectfont \(\displaystyle -4\)}%
\end{pgfscope}%
\begin{pgfscope}%
\pgfpathrectangle{\pgfqpoint{0.750000in}{0.500000in}}{\pgfqpoint{4.650000in}{3.020000in}}%
\pgfusepath{clip}%
\pgfsetrectcap%
\pgfsetroundjoin%
\pgfsetlinewidth{0.803000pt}%
\definecolor{currentstroke}{rgb}{0.690196,0.690196,0.690196}%
\pgfsetstrokecolor{currentstroke}%
\pgfsetdash{}{0pt}%
\pgfpathmoveto{\pgfqpoint{2.300000in}{0.500000in}}%
\pgfpathlineto{\pgfqpoint{2.300000in}{3.520000in}}%
\pgfusepath{stroke}%
\end{pgfscope}%
\begin{pgfscope}%
\pgfsetbuttcap%
\pgfsetroundjoin%
\definecolor{currentfill}{rgb}{0.000000,0.000000,0.000000}%
\pgfsetfillcolor{currentfill}%
\pgfsetlinewidth{0.803000pt}%
\definecolor{currentstroke}{rgb}{0.000000,0.000000,0.000000}%
\pgfsetstrokecolor{currentstroke}%
\pgfsetdash{}{0pt}%
\pgfsys@defobject{currentmarker}{\pgfqpoint{0.000000in}{-0.048611in}}{\pgfqpoint{0.000000in}{0.000000in}}{%
\pgfpathmoveto{\pgfqpoint{0.000000in}{0.000000in}}%
\pgfpathlineto{\pgfqpoint{0.000000in}{-0.048611in}}%
\pgfusepath{stroke,fill}%
}%
\begin{pgfscope}%
\pgfsys@transformshift{2.300000in}{0.500000in}%
\pgfsys@useobject{currentmarker}{}%
\end{pgfscope}%
\end{pgfscope}%
\begin{pgfscope}%
\pgftext[x=2.300000in,y=0.402778in,,top]{\rmfamily\fontsize{10.000000}{12.000000}\selectfont \(\displaystyle -2\)}%
\end{pgfscope}%
\begin{pgfscope}%
\pgfpathrectangle{\pgfqpoint{0.750000in}{0.500000in}}{\pgfqpoint{4.650000in}{3.020000in}}%
\pgfusepath{clip}%
\pgfsetrectcap%
\pgfsetroundjoin%
\pgfsetlinewidth{0.803000pt}%
\definecolor{currentstroke}{rgb}{0.690196,0.690196,0.690196}%
\pgfsetstrokecolor{currentstroke}%
\pgfsetdash{}{0pt}%
\pgfpathmoveto{\pgfqpoint{3.075000in}{0.500000in}}%
\pgfpathlineto{\pgfqpoint{3.075000in}{3.520000in}}%
\pgfusepath{stroke}%
\end{pgfscope}%
\begin{pgfscope}%
\pgfsetbuttcap%
\pgfsetroundjoin%
\definecolor{currentfill}{rgb}{0.000000,0.000000,0.000000}%
\pgfsetfillcolor{currentfill}%
\pgfsetlinewidth{0.803000pt}%
\definecolor{currentstroke}{rgb}{0.000000,0.000000,0.000000}%
\pgfsetstrokecolor{currentstroke}%
\pgfsetdash{}{0pt}%
\pgfsys@defobject{currentmarker}{\pgfqpoint{0.000000in}{-0.048611in}}{\pgfqpoint{0.000000in}{0.000000in}}{%
\pgfpathmoveto{\pgfqpoint{0.000000in}{0.000000in}}%
\pgfpathlineto{\pgfqpoint{0.000000in}{-0.048611in}}%
\pgfusepath{stroke,fill}%
}%
\begin{pgfscope}%
\pgfsys@transformshift{3.075000in}{0.500000in}%
\pgfsys@useobject{currentmarker}{}%
\end{pgfscope}%
\end{pgfscope}%
\begin{pgfscope}%
\pgftext[x=3.075000in,y=0.402778in,,top]{\rmfamily\fontsize{10.000000}{12.000000}\selectfont \(\displaystyle 0\)}%
\end{pgfscope}%
\begin{pgfscope}%
\pgfpathrectangle{\pgfqpoint{0.750000in}{0.500000in}}{\pgfqpoint{4.650000in}{3.020000in}}%
\pgfusepath{clip}%
\pgfsetrectcap%
\pgfsetroundjoin%
\pgfsetlinewidth{0.803000pt}%
\definecolor{currentstroke}{rgb}{0.690196,0.690196,0.690196}%
\pgfsetstrokecolor{currentstroke}%
\pgfsetdash{}{0pt}%
\pgfpathmoveto{\pgfqpoint{3.850000in}{0.500000in}}%
\pgfpathlineto{\pgfqpoint{3.850000in}{3.520000in}}%
\pgfusepath{stroke}%
\end{pgfscope}%
\begin{pgfscope}%
\pgfsetbuttcap%
\pgfsetroundjoin%
\definecolor{currentfill}{rgb}{0.000000,0.000000,0.000000}%
\pgfsetfillcolor{currentfill}%
\pgfsetlinewidth{0.803000pt}%
\definecolor{currentstroke}{rgb}{0.000000,0.000000,0.000000}%
\pgfsetstrokecolor{currentstroke}%
\pgfsetdash{}{0pt}%
\pgfsys@defobject{currentmarker}{\pgfqpoint{0.000000in}{-0.048611in}}{\pgfqpoint{0.000000in}{0.000000in}}{%
\pgfpathmoveto{\pgfqpoint{0.000000in}{0.000000in}}%
\pgfpathlineto{\pgfqpoint{0.000000in}{-0.048611in}}%
\pgfusepath{stroke,fill}%
}%
\begin{pgfscope}%
\pgfsys@transformshift{3.850000in}{0.500000in}%
\pgfsys@useobject{currentmarker}{}%
\end{pgfscope}%
\end{pgfscope}%
\begin{pgfscope}%
\pgftext[x=3.850000in,y=0.402778in,,top]{\rmfamily\fontsize{10.000000}{12.000000}\selectfont \(\displaystyle 2\)}%
\end{pgfscope}%
\begin{pgfscope}%
\pgfpathrectangle{\pgfqpoint{0.750000in}{0.500000in}}{\pgfqpoint{4.650000in}{3.020000in}}%
\pgfusepath{clip}%
\pgfsetrectcap%
\pgfsetroundjoin%
\pgfsetlinewidth{0.803000pt}%
\definecolor{currentstroke}{rgb}{0.690196,0.690196,0.690196}%
\pgfsetstrokecolor{currentstroke}%
\pgfsetdash{}{0pt}%
\pgfpathmoveto{\pgfqpoint{4.625000in}{0.500000in}}%
\pgfpathlineto{\pgfqpoint{4.625000in}{3.520000in}}%
\pgfusepath{stroke}%
\end{pgfscope}%
\begin{pgfscope}%
\pgfsetbuttcap%
\pgfsetroundjoin%
\definecolor{currentfill}{rgb}{0.000000,0.000000,0.000000}%
\pgfsetfillcolor{currentfill}%
\pgfsetlinewidth{0.803000pt}%
\definecolor{currentstroke}{rgb}{0.000000,0.000000,0.000000}%
\pgfsetstrokecolor{currentstroke}%
\pgfsetdash{}{0pt}%
\pgfsys@defobject{currentmarker}{\pgfqpoint{0.000000in}{-0.048611in}}{\pgfqpoint{0.000000in}{0.000000in}}{%
\pgfpathmoveto{\pgfqpoint{0.000000in}{0.000000in}}%
\pgfpathlineto{\pgfqpoint{0.000000in}{-0.048611in}}%
\pgfusepath{stroke,fill}%
}%
\begin{pgfscope}%
\pgfsys@transformshift{4.625000in}{0.500000in}%
\pgfsys@useobject{currentmarker}{}%
\end{pgfscope}%
\end{pgfscope}%
\begin{pgfscope}%
\pgftext[x=4.625000in,y=0.402778in,,top]{\rmfamily\fontsize{10.000000}{12.000000}\selectfont \(\displaystyle 4\)}%
\end{pgfscope}%
\begin{pgfscope}%
\pgfpathrectangle{\pgfqpoint{0.750000in}{0.500000in}}{\pgfqpoint{4.650000in}{3.020000in}}%
\pgfusepath{clip}%
\pgfsetrectcap%
\pgfsetroundjoin%
\pgfsetlinewidth{0.803000pt}%
\definecolor{currentstroke}{rgb}{0.690196,0.690196,0.690196}%
\pgfsetstrokecolor{currentstroke}%
\pgfsetdash{}{0pt}%
\pgfpathmoveto{\pgfqpoint{5.400000in}{0.500000in}}%
\pgfpathlineto{\pgfqpoint{5.400000in}{3.520000in}}%
\pgfusepath{stroke}%
\end{pgfscope}%
\begin{pgfscope}%
\pgfsetbuttcap%
\pgfsetroundjoin%
\definecolor{currentfill}{rgb}{0.000000,0.000000,0.000000}%
\pgfsetfillcolor{currentfill}%
\pgfsetlinewidth{0.803000pt}%
\definecolor{currentstroke}{rgb}{0.000000,0.000000,0.000000}%
\pgfsetstrokecolor{currentstroke}%
\pgfsetdash{}{0pt}%
\pgfsys@defobject{currentmarker}{\pgfqpoint{0.000000in}{-0.048611in}}{\pgfqpoint{0.000000in}{0.000000in}}{%
\pgfpathmoveto{\pgfqpoint{0.000000in}{0.000000in}}%
\pgfpathlineto{\pgfqpoint{0.000000in}{-0.048611in}}%
\pgfusepath{stroke,fill}%
}%
\begin{pgfscope}%
\pgfsys@transformshift{5.400000in}{0.500000in}%
\pgfsys@useobject{currentmarker}{}%
\end{pgfscope}%
\end{pgfscope}%
\begin{pgfscope}%
\pgftext[x=5.400000in,y=0.402778in,,top]{\rmfamily\fontsize{10.000000}{12.000000}\selectfont \(\displaystyle 6\)}%
\end{pgfscope}%
\begin{pgfscope}%
\pgfpathrectangle{\pgfqpoint{0.750000in}{0.500000in}}{\pgfqpoint{4.650000in}{3.020000in}}%
\pgfusepath{clip}%
\pgfsetrectcap%
\pgfsetroundjoin%
\pgfsetlinewidth{0.803000pt}%
\definecolor{currentstroke}{rgb}{0.690196,0.690196,0.690196}%
\pgfsetstrokecolor{currentstroke}%
\pgfsetdash{}{0pt}%
\pgfpathmoveto{\pgfqpoint{0.750000in}{0.500000in}}%
\pgfpathlineto{\pgfqpoint{5.400000in}{0.500000in}}%
\pgfusepath{stroke}%
\end{pgfscope}%
\begin{pgfscope}%
\pgfsetbuttcap%
\pgfsetroundjoin%
\definecolor{currentfill}{rgb}{0.000000,0.000000,0.000000}%
\pgfsetfillcolor{currentfill}%
\pgfsetlinewidth{0.803000pt}%
\definecolor{currentstroke}{rgb}{0.000000,0.000000,0.000000}%
\pgfsetstrokecolor{currentstroke}%
\pgfsetdash{}{0pt}%
\pgfsys@defobject{currentmarker}{\pgfqpoint{-0.048611in}{0.000000in}}{\pgfqpoint{0.000000in}{0.000000in}}{%
\pgfpathmoveto{\pgfqpoint{0.000000in}{0.000000in}}%
\pgfpathlineto{\pgfqpoint{-0.048611in}{0.000000in}}%
\pgfusepath{stroke,fill}%
}%
\begin{pgfscope}%
\pgfsys@transformshift{0.750000in}{0.500000in}%
\pgfsys@useobject{currentmarker}{}%
\end{pgfscope}%
\end{pgfscope}%
\begin{pgfscope}%
\pgftext[x=0.297838in,y=0.451806in,left,base]{\rmfamily\fontsize{10.000000}{12.000000}\selectfont \(\displaystyle -0.50\)}%
\end{pgfscope}%
\begin{pgfscope}%
\pgfpathrectangle{\pgfqpoint{0.750000in}{0.500000in}}{\pgfqpoint{4.650000in}{3.020000in}}%
\pgfusepath{clip}%
\pgfsetrectcap%
\pgfsetroundjoin%
\pgfsetlinewidth{0.803000pt}%
\definecolor{currentstroke}{rgb}{0.690196,0.690196,0.690196}%
\pgfsetstrokecolor{currentstroke}%
\pgfsetdash{}{0pt}%
\pgfpathmoveto{\pgfqpoint{0.750000in}{0.877500in}}%
\pgfpathlineto{\pgfqpoint{5.400000in}{0.877500in}}%
\pgfusepath{stroke}%
\end{pgfscope}%
\begin{pgfscope}%
\pgfsetbuttcap%
\pgfsetroundjoin%
\definecolor{currentfill}{rgb}{0.000000,0.000000,0.000000}%
\pgfsetfillcolor{currentfill}%
\pgfsetlinewidth{0.803000pt}%
\definecolor{currentstroke}{rgb}{0.000000,0.000000,0.000000}%
\pgfsetstrokecolor{currentstroke}%
\pgfsetdash{}{0pt}%
\pgfsys@defobject{currentmarker}{\pgfqpoint{-0.048611in}{0.000000in}}{\pgfqpoint{0.000000in}{0.000000in}}{%
\pgfpathmoveto{\pgfqpoint{0.000000in}{0.000000in}}%
\pgfpathlineto{\pgfqpoint{-0.048611in}{0.000000in}}%
\pgfusepath{stroke,fill}%
}%
\begin{pgfscope}%
\pgfsys@transformshift{0.750000in}{0.877500in}%
\pgfsys@useobject{currentmarker}{}%
\end{pgfscope}%
\end{pgfscope}%
\begin{pgfscope}%
\pgftext[x=0.297838in,y=0.829306in,left,base]{\rmfamily\fontsize{10.000000}{12.000000}\selectfont \(\displaystyle -0.25\)}%
\end{pgfscope}%
\begin{pgfscope}%
\pgfpathrectangle{\pgfqpoint{0.750000in}{0.500000in}}{\pgfqpoint{4.650000in}{3.020000in}}%
\pgfusepath{clip}%
\pgfsetrectcap%
\pgfsetroundjoin%
\pgfsetlinewidth{0.803000pt}%
\definecolor{currentstroke}{rgb}{0.690196,0.690196,0.690196}%
\pgfsetstrokecolor{currentstroke}%
\pgfsetdash{}{0pt}%
\pgfpathmoveto{\pgfqpoint{0.750000in}{1.255000in}}%
\pgfpathlineto{\pgfqpoint{5.400000in}{1.255000in}}%
\pgfusepath{stroke}%
\end{pgfscope}%
\begin{pgfscope}%
\pgfsetbuttcap%
\pgfsetroundjoin%
\definecolor{currentfill}{rgb}{0.000000,0.000000,0.000000}%
\pgfsetfillcolor{currentfill}%
\pgfsetlinewidth{0.803000pt}%
\definecolor{currentstroke}{rgb}{0.000000,0.000000,0.000000}%
\pgfsetstrokecolor{currentstroke}%
\pgfsetdash{}{0pt}%
\pgfsys@defobject{currentmarker}{\pgfqpoint{-0.048611in}{0.000000in}}{\pgfqpoint{0.000000in}{0.000000in}}{%
\pgfpathmoveto{\pgfqpoint{0.000000in}{0.000000in}}%
\pgfpathlineto{\pgfqpoint{-0.048611in}{0.000000in}}%
\pgfusepath{stroke,fill}%
}%
\begin{pgfscope}%
\pgfsys@transformshift{0.750000in}{1.255000in}%
\pgfsys@useobject{currentmarker}{}%
\end{pgfscope}%
\end{pgfscope}%
\begin{pgfscope}%
\pgftext[x=0.405863in,y=1.206806in,left,base]{\rmfamily\fontsize{10.000000}{12.000000}\selectfont \(\displaystyle 0.00\)}%
\end{pgfscope}%
\begin{pgfscope}%
\pgfpathrectangle{\pgfqpoint{0.750000in}{0.500000in}}{\pgfqpoint{4.650000in}{3.020000in}}%
\pgfusepath{clip}%
\pgfsetrectcap%
\pgfsetroundjoin%
\pgfsetlinewidth{0.803000pt}%
\definecolor{currentstroke}{rgb}{0.690196,0.690196,0.690196}%
\pgfsetstrokecolor{currentstroke}%
\pgfsetdash{}{0pt}%
\pgfpathmoveto{\pgfqpoint{0.750000in}{1.632500in}}%
\pgfpathlineto{\pgfqpoint{5.400000in}{1.632500in}}%
\pgfusepath{stroke}%
\end{pgfscope}%
\begin{pgfscope}%
\pgfsetbuttcap%
\pgfsetroundjoin%
\definecolor{currentfill}{rgb}{0.000000,0.000000,0.000000}%
\pgfsetfillcolor{currentfill}%
\pgfsetlinewidth{0.803000pt}%
\definecolor{currentstroke}{rgb}{0.000000,0.000000,0.000000}%
\pgfsetstrokecolor{currentstroke}%
\pgfsetdash{}{0pt}%
\pgfsys@defobject{currentmarker}{\pgfqpoint{-0.048611in}{0.000000in}}{\pgfqpoint{0.000000in}{0.000000in}}{%
\pgfpathmoveto{\pgfqpoint{0.000000in}{0.000000in}}%
\pgfpathlineto{\pgfqpoint{-0.048611in}{0.000000in}}%
\pgfusepath{stroke,fill}%
}%
\begin{pgfscope}%
\pgfsys@transformshift{0.750000in}{1.632500in}%
\pgfsys@useobject{currentmarker}{}%
\end{pgfscope}%
\end{pgfscope}%
\begin{pgfscope}%
\pgftext[x=0.405863in,y=1.584306in,left,base]{\rmfamily\fontsize{10.000000}{12.000000}\selectfont \(\displaystyle 0.25\)}%
\end{pgfscope}%
\begin{pgfscope}%
\pgfpathrectangle{\pgfqpoint{0.750000in}{0.500000in}}{\pgfqpoint{4.650000in}{3.020000in}}%
\pgfusepath{clip}%
\pgfsetrectcap%
\pgfsetroundjoin%
\pgfsetlinewidth{0.803000pt}%
\definecolor{currentstroke}{rgb}{0.690196,0.690196,0.690196}%
\pgfsetstrokecolor{currentstroke}%
\pgfsetdash{}{0pt}%
\pgfpathmoveto{\pgfqpoint{0.750000in}{2.010000in}}%
\pgfpathlineto{\pgfqpoint{5.400000in}{2.010000in}}%
\pgfusepath{stroke}%
\end{pgfscope}%
\begin{pgfscope}%
\pgfsetbuttcap%
\pgfsetroundjoin%
\definecolor{currentfill}{rgb}{0.000000,0.000000,0.000000}%
\pgfsetfillcolor{currentfill}%
\pgfsetlinewidth{0.803000pt}%
\definecolor{currentstroke}{rgb}{0.000000,0.000000,0.000000}%
\pgfsetstrokecolor{currentstroke}%
\pgfsetdash{}{0pt}%
\pgfsys@defobject{currentmarker}{\pgfqpoint{-0.048611in}{0.000000in}}{\pgfqpoint{0.000000in}{0.000000in}}{%
\pgfpathmoveto{\pgfqpoint{0.000000in}{0.000000in}}%
\pgfpathlineto{\pgfqpoint{-0.048611in}{0.000000in}}%
\pgfusepath{stroke,fill}%
}%
\begin{pgfscope}%
\pgfsys@transformshift{0.750000in}{2.010000in}%
\pgfsys@useobject{currentmarker}{}%
\end{pgfscope}%
\end{pgfscope}%
\begin{pgfscope}%
\pgftext[x=0.405863in,y=1.961806in,left,base]{\rmfamily\fontsize{10.000000}{12.000000}\selectfont \(\displaystyle 0.50\)}%
\end{pgfscope}%
\begin{pgfscope}%
\pgfpathrectangle{\pgfqpoint{0.750000in}{0.500000in}}{\pgfqpoint{4.650000in}{3.020000in}}%
\pgfusepath{clip}%
\pgfsetrectcap%
\pgfsetroundjoin%
\pgfsetlinewidth{0.803000pt}%
\definecolor{currentstroke}{rgb}{0.690196,0.690196,0.690196}%
\pgfsetstrokecolor{currentstroke}%
\pgfsetdash{}{0pt}%
\pgfpathmoveto{\pgfqpoint{0.750000in}{2.387500in}}%
\pgfpathlineto{\pgfqpoint{5.400000in}{2.387500in}}%
\pgfusepath{stroke}%
\end{pgfscope}%
\begin{pgfscope}%
\pgfsetbuttcap%
\pgfsetroundjoin%
\definecolor{currentfill}{rgb}{0.000000,0.000000,0.000000}%
\pgfsetfillcolor{currentfill}%
\pgfsetlinewidth{0.803000pt}%
\definecolor{currentstroke}{rgb}{0.000000,0.000000,0.000000}%
\pgfsetstrokecolor{currentstroke}%
\pgfsetdash{}{0pt}%
\pgfsys@defobject{currentmarker}{\pgfqpoint{-0.048611in}{0.000000in}}{\pgfqpoint{0.000000in}{0.000000in}}{%
\pgfpathmoveto{\pgfqpoint{0.000000in}{0.000000in}}%
\pgfpathlineto{\pgfqpoint{-0.048611in}{0.000000in}}%
\pgfusepath{stroke,fill}%
}%
\begin{pgfscope}%
\pgfsys@transformshift{0.750000in}{2.387500in}%
\pgfsys@useobject{currentmarker}{}%
\end{pgfscope}%
\end{pgfscope}%
\begin{pgfscope}%
\pgftext[x=0.405863in,y=2.339306in,left,base]{\rmfamily\fontsize{10.000000}{12.000000}\selectfont \(\displaystyle 0.75\)}%
\end{pgfscope}%
\begin{pgfscope}%
\pgfpathrectangle{\pgfqpoint{0.750000in}{0.500000in}}{\pgfqpoint{4.650000in}{3.020000in}}%
\pgfusepath{clip}%
\pgfsetrectcap%
\pgfsetroundjoin%
\pgfsetlinewidth{0.803000pt}%
\definecolor{currentstroke}{rgb}{0.690196,0.690196,0.690196}%
\pgfsetstrokecolor{currentstroke}%
\pgfsetdash{}{0pt}%
\pgfpathmoveto{\pgfqpoint{0.750000in}{2.765000in}}%
\pgfpathlineto{\pgfqpoint{5.400000in}{2.765000in}}%
\pgfusepath{stroke}%
\end{pgfscope}%
\begin{pgfscope}%
\pgfsetbuttcap%
\pgfsetroundjoin%
\definecolor{currentfill}{rgb}{0.000000,0.000000,0.000000}%
\pgfsetfillcolor{currentfill}%
\pgfsetlinewidth{0.803000pt}%
\definecolor{currentstroke}{rgb}{0.000000,0.000000,0.000000}%
\pgfsetstrokecolor{currentstroke}%
\pgfsetdash{}{0pt}%
\pgfsys@defobject{currentmarker}{\pgfqpoint{-0.048611in}{0.000000in}}{\pgfqpoint{0.000000in}{0.000000in}}{%
\pgfpathmoveto{\pgfqpoint{0.000000in}{0.000000in}}%
\pgfpathlineto{\pgfqpoint{-0.048611in}{0.000000in}}%
\pgfusepath{stroke,fill}%
}%
\begin{pgfscope}%
\pgfsys@transformshift{0.750000in}{2.765000in}%
\pgfsys@useobject{currentmarker}{}%
\end{pgfscope}%
\end{pgfscope}%
\begin{pgfscope}%
\pgftext[x=0.405863in,y=2.716806in,left,base]{\rmfamily\fontsize{10.000000}{12.000000}\selectfont \(\displaystyle 1.00\)}%
\end{pgfscope}%
\begin{pgfscope}%
\pgfpathrectangle{\pgfqpoint{0.750000in}{0.500000in}}{\pgfqpoint{4.650000in}{3.020000in}}%
\pgfusepath{clip}%
\pgfsetrectcap%
\pgfsetroundjoin%
\pgfsetlinewidth{0.803000pt}%
\definecolor{currentstroke}{rgb}{0.690196,0.690196,0.690196}%
\pgfsetstrokecolor{currentstroke}%
\pgfsetdash{}{0pt}%
\pgfpathmoveto{\pgfqpoint{0.750000in}{3.142500in}}%
\pgfpathlineto{\pgfqpoint{5.400000in}{3.142500in}}%
\pgfusepath{stroke}%
\end{pgfscope}%
\begin{pgfscope}%
\pgfsetbuttcap%
\pgfsetroundjoin%
\definecolor{currentfill}{rgb}{0.000000,0.000000,0.000000}%
\pgfsetfillcolor{currentfill}%
\pgfsetlinewidth{0.803000pt}%
\definecolor{currentstroke}{rgb}{0.000000,0.000000,0.000000}%
\pgfsetstrokecolor{currentstroke}%
\pgfsetdash{}{0pt}%
\pgfsys@defobject{currentmarker}{\pgfqpoint{-0.048611in}{0.000000in}}{\pgfqpoint{0.000000in}{0.000000in}}{%
\pgfpathmoveto{\pgfqpoint{0.000000in}{0.000000in}}%
\pgfpathlineto{\pgfqpoint{-0.048611in}{0.000000in}}%
\pgfusepath{stroke,fill}%
}%
\begin{pgfscope}%
\pgfsys@transformshift{0.750000in}{3.142500in}%
\pgfsys@useobject{currentmarker}{}%
\end{pgfscope}%
\end{pgfscope}%
\begin{pgfscope}%
\pgftext[x=0.405863in,y=3.094306in,left,base]{\rmfamily\fontsize{10.000000}{12.000000}\selectfont \(\displaystyle 1.25\)}%
\end{pgfscope}%
\begin{pgfscope}%
\pgfpathrectangle{\pgfqpoint{0.750000in}{0.500000in}}{\pgfqpoint{4.650000in}{3.020000in}}%
\pgfusepath{clip}%
\pgfsetrectcap%
\pgfsetroundjoin%
\pgfsetlinewidth{0.803000pt}%
\definecolor{currentstroke}{rgb}{0.690196,0.690196,0.690196}%
\pgfsetstrokecolor{currentstroke}%
\pgfsetdash{}{0pt}%
\pgfpathmoveto{\pgfqpoint{0.750000in}{3.520000in}}%
\pgfpathlineto{\pgfqpoint{5.400000in}{3.520000in}}%
\pgfusepath{stroke}%
\end{pgfscope}%
\begin{pgfscope}%
\pgfsetbuttcap%
\pgfsetroundjoin%
\definecolor{currentfill}{rgb}{0.000000,0.000000,0.000000}%
\pgfsetfillcolor{currentfill}%
\pgfsetlinewidth{0.803000pt}%
\definecolor{currentstroke}{rgb}{0.000000,0.000000,0.000000}%
\pgfsetstrokecolor{currentstroke}%
\pgfsetdash{}{0pt}%
\pgfsys@defobject{currentmarker}{\pgfqpoint{-0.048611in}{0.000000in}}{\pgfqpoint{0.000000in}{0.000000in}}{%
\pgfpathmoveto{\pgfqpoint{0.000000in}{0.000000in}}%
\pgfpathlineto{\pgfqpoint{-0.048611in}{0.000000in}}%
\pgfusepath{stroke,fill}%
}%
\begin{pgfscope}%
\pgfsys@transformshift{0.750000in}{3.520000in}%
\pgfsys@useobject{currentmarker}{}%
\end{pgfscope}%
\end{pgfscope}%
\begin{pgfscope}%
\pgftext[x=0.405863in,y=3.471806in,left,base]{\rmfamily\fontsize{10.000000}{12.000000}\selectfont \(\displaystyle 1.50\)}%
\end{pgfscope}%
\begin{pgfscope}%
\pgfpathrectangle{\pgfqpoint{0.750000in}{0.500000in}}{\pgfqpoint{4.650000in}{3.020000in}}%
\pgfusepath{clip}%
\pgfsetrectcap%
\pgfsetroundjoin%
\pgfsetlinewidth{1.505625pt}%
\definecolor{currentstroke}{rgb}{0.121569,0.466667,0.705882}%
\pgfsetstrokecolor{currentstroke}%
\pgfsetdash{}{0pt}%
\pgfpathmoveto{\pgfqpoint{0.750000in}{1.313077in}}%
\pgfpathlineto{\pgfqpoint{5.400000in}{1.313077in}}%
\pgfpathlineto{\pgfqpoint{5.400000in}{1.313077in}}%
\pgfusepath{stroke}%
\end{pgfscope}%
\begin{pgfscope}%
\pgfpathrectangle{\pgfqpoint{0.750000in}{0.500000in}}{\pgfqpoint{4.650000in}{3.020000in}}%
\pgfusepath{clip}%
\pgfsetrectcap%
\pgfsetroundjoin%
\pgfsetlinewidth{1.505625pt}%
\definecolor{currentstroke}{rgb}{1.000000,0.498039,0.054902}%
\pgfsetstrokecolor{currentstroke}%
\pgfsetdash{}{0pt}%
\pgfpathmoveto{\pgfqpoint{0.750000in}{1.143971in}}%
\pgfpathlineto{\pgfqpoint{0.880330in}{1.204279in}}%
\pgfpathlineto{\pgfqpoint{1.010661in}{1.261109in}}%
\pgfpathlineto{\pgfqpoint{1.136336in}{1.312615in}}%
\pgfpathlineto{\pgfqpoint{1.262012in}{1.360887in}}%
\pgfpathlineto{\pgfqpoint{1.387688in}{1.405925in}}%
\pgfpathlineto{\pgfqpoint{1.513363in}{1.447730in}}%
\pgfpathlineto{\pgfqpoint{1.639039in}{1.486299in}}%
\pgfpathlineto{\pgfqpoint{1.764715in}{1.521635in}}%
\pgfpathlineto{\pgfqpoint{1.890390in}{1.553737in}}%
\pgfpathlineto{\pgfqpoint{2.016066in}{1.582604in}}%
\pgfpathlineto{\pgfqpoint{2.141742in}{1.608238in}}%
\pgfpathlineto{\pgfqpoint{2.267417in}{1.630637in}}%
\pgfpathlineto{\pgfqpoint{2.393093in}{1.649802in}}%
\pgfpathlineto{\pgfqpoint{2.518769in}{1.665733in}}%
\pgfpathlineto{\pgfqpoint{2.639790in}{1.678018in}}%
\pgfpathlineto{\pgfqpoint{2.760811in}{1.687303in}}%
\pgfpathlineto{\pgfqpoint{2.881832in}{1.693589in}}%
\pgfpathlineto{\pgfqpoint{3.002853in}{1.696877in}}%
\pgfpathlineto{\pgfqpoint{3.123874in}{1.697165in}}%
\pgfpathlineto{\pgfqpoint{3.244895in}{1.694454in}}%
\pgfpathlineto{\pgfqpoint{3.365916in}{1.688745in}}%
\pgfpathlineto{\pgfqpoint{3.486937in}{1.680036in}}%
\pgfpathlineto{\pgfqpoint{3.607958in}{1.668328in}}%
\pgfpathlineto{\pgfqpoint{3.728979in}{1.653622in}}%
\pgfpathlineto{\pgfqpoint{3.850000in}{1.635916in}}%
\pgfpathlineto{\pgfqpoint{3.975676in}{1.614355in}}%
\pgfpathlineto{\pgfqpoint{4.101351in}{1.589561in}}%
\pgfpathlineto{\pgfqpoint{4.227027in}{1.561532in}}%
\pgfpathlineto{\pgfqpoint{4.352703in}{1.530268in}}%
\pgfpathlineto{\pgfqpoint{4.478378in}{1.495771in}}%
\pgfpathlineto{\pgfqpoint{4.604054in}{1.458040in}}%
\pgfpathlineto{\pgfqpoint{4.729730in}{1.417074in}}%
\pgfpathlineto{\pgfqpoint{4.855405in}{1.372874in}}%
\pgfpathlineto{\pgfqpoint{4.981081in}{1.325441in}}%
\pgfpathlineto{\pgfqpoint{5.106757in}{1.274773in}}%
\pgfpathlineto{\pgfqpoint{5.232432in}{1.220871in}}%
\pgfpathlineto{\pgfqpoint{5.362763in}{1.161556in}}%
\pgfpathlineto{\pgfqpoint{5.400000in}{1.143971in}}%
\pgfpathlineto{\pgfqpoint{5.400000in}{1.143971in}}%
\pgfusepath{stroke}%
\end{pgfscope}%
\begin{pgfscope}%
\pgfpathrectangle{\pgfqpoint{0.750000in}{0.500000in}}{\pgfqpoint{4.650000in}{3.020000in}}%
\pgfusepath{clip}%
\pgfsetrectcap%
\pgfsetroundjoin%
\pgfsetlinewidth{1.505625pt}%
\definecolor{currentstroke}{rgb}{0.172549,0.627451,0.172549}%
\pgfsetstrokecolor{currentstroke}%
\pgfsetdash{}{0pt}%
\pgfpathmoveto{\pgfqpoint{0.750000in}{2.111635in}}%
\pgfpathlineto{\pgfqpoint{0.787237in}{1.995851in}}%
\pgfpathlineto{\pgfqpoint{0.824474in}{1.889353in}}%
\pgfpathlineto{\pgfqpoint{0.861712in}{1.791779in}}%
\pgfpathlineto{\pgfqpoint{0.898949in}{1.702773in}}%
\pgfpathlineto{\pgfqpoint{0.936186in}{1.621984in}}%
\pgfpathlineto{\pgfqpoint{0.973423in}{1.549068in}}%
\pgfpathlineto{\pgfqpoint{1.006006in}{1.491459in}}%
\pgfpathlineto{\pgfqpoint{1.038589in}{1.439395in}}%
\pgfpathlineto{\pgfqpoint{1.071171in}{1.392658in}}%
\pgfpathlineto{\pgfqpoint{1.103754in}{1.351030in}}%
\pgfpathlineto{\pgfqpoint{1.136336in}{1.314301in}}%
\pgfpathlineto{\pgfqpoint{1.168919in}{1.282260in}}%
\pgfpathlineto{\pgfqpoint{1.201502in}{1.254702in}}%
\pgfpathlineto{\pgfqpoint{1.234084in}{1.231425in}}%
\pgfpathlineto{\pgfqpoint{1.266667in}{1.212231in}}%
\pgfpathlineto{\pgfqpoint{1.299249in}{1.196924in}}%
\pgfpathlineto{\pgfqpoint{1.331832in}{1.185313in}}%
\pgfpathlineto{\pgfqpoint{1.364414in}{1.177210in}}%
\pgfpathlineto{\pgfqpoint{1.396997in}{1.172430in}}%
\pgfpathlineto{\pgfqpoint{1.429580in}{1.170791in}}%
\pgfpathlineto{\pgfqpoint{1.462162in}{1.172117in}}%
\pgfpathlineto{\pgfqpoint{1.494745in}{1.176233in}}%
\pgfpathlineto{\pgfqpoint{1.531982in}{1.184134in}}%
\pgfpathlineto{\pgfqpoint{1.569219in}{1.195208in}}%
\pgfpathlineto{\pgfqpoint{1.611111in}{1.211155in}}%
\pgfpathlineto{\pgfqpoint{1.653003in}{1.230471in}}%
\pgfpathlineto{\pgfqpoint{1.699550in}{1.255487in}}%
\pgfpathlineto{\pgfqpoint{1.750751in}{1.286821in}}%
\pgfpathlineto{\pgfqpoint{1.806607in}{1.324928in}}%
\pgfpathlineto{\pgfqpoint{1.867117in}{1.370044in}}%
\pgfpathlineto{\pgfqpoint{1.936937in}{1.425977in}}%
\pgfpathlineto{\pgfqpoint{2.025375in}{1.500948in}}%
\pgfpathlineto{\pgfqpoint{2.365165in}{1.793542in}}%
\pgfpathlineto{\pgfqpoint{2.439640in}{1.851579in}}%
\pgfpathlineto{\pgfqpoint{2.509459in}{1.902140in}}%
\pgfpathlineto{\pgfqpoint{2.574625in}{1.945397in}}%
\pgfpathlineto{\pgfqpoint{2.635135in}{1.981755in}}%
\pgfpathlineto{\pgfqpoint{2.690991in}{2.011772in}}%
\pgfpathlineto{\pgfqpoint{2.746847in}{2.038158in}}%
\pgfpathlineto{\pgfqpoint{2.798048in}{2.058992in}}%
\pgfpathlineto{\pgfqpoint{2.849249in}{2.076489in}}%
\pgfpathlineto{\pgfqpoint{2.900450in}{2.090543in}}%
\pgfpathlineto{\pgfqpoint{2.946997in}{2.100262in}}%
\pgfpathlineto{\pgfqpoint{2.993544in}{2.107021in}}%
\pgfpathlineto{\pgfqpoint{3.040090in}{2.110786in}}%
\pgfpathlineto{\pgfqpoint{3.086637in}{2.111540in}}%
\pgfpathlineto{\pgfqpoint{3.133183in}{2.109279in}}%
\pgfpathlineto{\pgfqpoint{3.179730in}{2.104014in}}%
\pgfpathlineto{\pgfqpoint{3.226276in}{2.095770in}}%
\pgfpathlineto{\pgfqpoint{3.272823in}{2.084587in}}%
\pgfpathlineto{\pgfqpoint{3.324024in}{2.068957in}}%
\pgfpathlineto{\pgfqpoint{3.375225in}{2.049929in}}%
\pgfpathlineto{\pgfqpoint{3.426426in}{2.027618in}}%
\pgfpathlineto{\pgfqpoint{3.482282in}{1.999694in}}%
\pgfpathlineto{\pgfqpoint{3.538138in}{1.968228in}}%
\pgfpathlineto{\pgfqpoint{3.598649in}{1.930415in}}%
\pgfpathlineto{\pgfqpoint{3.663814in}{1.885743in}}%
\pgfpathlineto{\pgfqpoint{3.733634in}{1.833861in}}%
\pgfpathlineto{\pgfqpoint{3.808108in}{1.774666in}}%
\pgfpathlineto{\pgfqpoint{3.896547in}{1.700414in}}%
\pgfpathlineto{\pgfqpoint{4.031532in}{1.582463in}}%
\pgfpathlineto{\pgfqpoint{4.180480in}{1.453170in}}%
\pgfpathlineto{\pgfqpoint{4.264264in}{1.384600in}}%
\pgfpathlineto{\pgfqpoint{4.329429in}{1.335015in}}%
\pgfpathlineto{\pgfqpoint{4.389940in}{1.292906in}}%
\pgfpathlineto{\pgfqpoint{4.441141in}{1.260902in}}%
\pgfpathlineto{\pgfqpoint{4.487688in}{1.235188in}}%
\pgfpathlineto{\pgfqpoint{4.534234in}{1.213141in}}%
\pgfpathlineto{\pgfqpoint{4.576126in}{1.196803in}}%
\pgfpathlineto{\pgfqpoint{4.613363in}{1.185350in}}%
\pgfpathlineto{\pgfqpoint{4.650601in}{1.177039in}}%
\pgfpathlineto{\pgfqpoint{4.687838in}{1.172117in}}%
\pgfpathlineto{\pgfqpoint{4.720420in}{1.170791in}}%
\pgfpathlineto{\pgfqpoint{4.753003in}{1.172430in}}%
\pgfpathlineto{\pgfqpoint{4.785586in}{1.177210in}}%
\pgfpathlineto{\pgfqpoint{4.818168in}{1.185313in}}%
\pgfpathlineto{\pgfqpoint{4.850751in}{1.196924in}}%
\pgfpathlineto{\pgfqpoint{4.883333in}{1.212231in}}%
\pgfpathlineto{\pgfqpoint{4.915916in}{1.231425in}}%
\pgfpathlineto{\pgfqpoint{4.948498in}{1.254702in}}%
\pgfpathlineto{\pgfqpoint{4.981081in}{1.282260in}}%
\pgfpathlineto{\pgfqpoint{5.013664in}{1.314301in}}%
\pgfpathlineto{\pgfqpoint{5.046246in}{1.351030in}}%
\pgfpathlineto{\pgfqpoint{5.078829in}{1.392658in}}%
\pgfpathlineto{\pgfqpoint{5.111411in}{1.439395in}}%
\pgfpathlineto{\pgfqpoint{5.143994in}{1.491459in}}%
\pgfpathlineto{\pgfqpoint{5.176577in}{1.549068in}}%
\pgfpathlineto{\pgfqpoint{5.209159in}{1.612446in}}%
\pgfpathlineto{\pgfqpoint{5.246396in}{1.692232in}}%
\pgfpathlineto{\pgfqpoint{5.283634in}{1.780192in}}%
\pgfpathlineto{\pgfqpoint{5.320871in}{1.876676in}}%
\pgfpathlineto{\pgfqpoint{5.358108in}{1.982038in}}%
\pgfpathlineto{\pgfqpoint{5.395345in}{2.096641in}}%
\pgfpathlineto{\pgfqpoint{5.400000in}{2.111635in}}%
\pgfpathlineto{\pgfqpoint{5.400000in}{2.111635in}}%
\pgfusepath{stroke}%
\end{pgfscope}%
\begin{pgfscope}%
\pgfpathrectangle{\pgfqpoint{0.750000in}{0.500000in}}{\pgfqpoint{4.650000in}{3.020000in}}%
\pgfusepath{clip}%
\pgfsetrectcap%
\pgfsetroundjoin%
\pgfsetlinewidth{1.505625pt}%
\definecolor{currentstroke}{rgb}{0.839216,0.152941,0.156863}%
\pgfsetstrokecolor{currentstroke}%
\pgfsetdash{}{0pt}%
\pgfpathmoveto{\pgfqpoint{0.979582in}{0.486111in}}%
\pgfpathlineto{\pgfqpoint{1.006006in}{0.679070in}}%
\pgfpathlineto{\pgfqpoint{1.029279in}{0.828662in}}%
\pgfpathlineto{\pgfqpoint{1.052553in}{0.960663in}}%
\pgfpathlineto{\pgfqpoint{1.075826in}{1.076356in}}%
\pgfpathlineto{\pgfqpoint{1.099099in}{1.176964in}}%
\pgfpathlineto{\pgfqpoint{1.122372in}{1.263657in}}%
\pgfpathlineto{\pgfqpoint{1.145646in}{1.337552in}}%
\pgfpathlineto{\pgfqpoint{1.168919in}{1.399709in}}%
\pgfpathlineto{\pgfqpoint{1.187538in}{1.441666in}}%
\pgfpathlineto{\pgfqpoint{1.206156in}{1.477254in}}%
\pgfpathlineto{\pgfqpoint{1.224775in}{1.506949in}}%
\pgfpathlineto{\pgfqpoint{1.243393in}{1.531206in}}%
\pgfpathlineto{\pgfqpoint{1.262012in}{1.550460in}}%
\pgfpathlineto{\pgfqpoint{1.280631in}{1.565131in}}%
\pgfpathlineto{\pgfqpoint{1.299249in}{1.575617in}}%
\pgfpathlineto{\pgfqpoint{1.317868in}{1.582301in}}%
\pgfpathlineto{\pgfqpoint{1.336486in}{1.585545in}}%
\pgfpathlineto{\pgfqpoint{1.355105in}{1.585699in}}%
\pgfpathlineto{\pgfqpoint{1.373724in}{1.583092in}}%
\pgfpathlineto{\pgfqpoint{1.396997in}{1.576431in}}%
\pgfpathlineto{\pgfqpoint{1.420270in}{1.566528in}}%
\pgfpathlineto{\pgfqpoint{1.448198in}{1.551129in}}%
\pgfpathlineto{\pgfqpoint{1.480781in}{1.529409in}}%
\pgfpathlineto{\pgfqpoint{1.522673in}{1.497464in}}%
\pgfpathlineto{\pgfqpoint{1.685586in}{1.368505in}}%
\pgfpathlineto{\pgfqpoint{1.722823in}{1.344687in}}%
\pgfpathlineto{\pgfqpoint{1.755405in}{1.327017in}}%
\pgfpathlineto{\pgfqpoint{1.787988in}{1.312709in}}%
\pgfpathlineto{\pgfqpoint{1.820571in}{1.302070in}}%
\pgfpathlineto{\pgfqpoint{1.848498in}{1.296049in}}%
\pgfpathlineto{\pgfqpoint{1.876426in}{1.293002in}}%
\pgfpathlineto{\pgfqpoint{1.904354in}{1.293000in}}%
\pgfpathlineto{\pgfqpoint{1.932282in}{1.296083in}}%
\pgfpathlineto{\pgfqpoint{1.960210in}{1.302261in}}%
\pgfpathlineto{\pgfqpoint{1.988138in}{1.311516in}}%
\pgfpathlineto{\pgfqpoint{2.016066in}{1.323802in}}%
\pgfpathlineto{\pgfqpoint{2.048649in}{1.341875in}}%
\pgfpathlineto{\pgfqpoint{2.081231in}{1.363835in}}%
\pgfpathlineto{\pgfqpoint{2.113814in}{1.389497in}}%
\pgfpathlineto{\pgfqpoint{2.151051in}{1.423080in}}%
\pgfpathlineto{\pgfqpoint{2.188288in}{1.460848in}}%
\pgfpathlineto{\pgfqpoint{2.230180in}{1.507824in}}%
\pgfpathlineto{\pgfqpoint{2.276727in}{1.564841in}}%
\pgfpathlineto{\pgfqpoint{2.332583in}{1.638671in}}%
\pgfpathlineto{\pgfqpoint{2.402402in}{1.736766in}}%
\pgfpathlineto{\pgfqpoint{2.621171in}{2.048985in}}%
\pgfpathlineto{\pgfqpoint{2.677027in}{2.121182in}}%
\pgfpathlineto{\pgfqpoint{2.723574in}{2.176704in}}%
\pgfpathlineto{\pgfqpoint{2.765465in}{2.222364in}}%
\pgfpathlineto{\pgfqpoint{2.807357in}{2.263403in}}%
\pgfpathlineto{\pgfqpoint{2.844595in}{2.295625in}}%
\pgfpathlineto{\pgfqpoint{2.881832in}{2.323543in}}%
\pgfpathlineto{\pgfqpoint{2.914414in}{2.344245in}}%
\pgfpathlineto{\pgfqpoint{2.946997in}{2.361322in}}%
\pgfpathlineto{\pgfqpoint{2.979580in}{2.374661in}}%
\pgfpathlineto{\pgfqpoint{3.007508in}{2.383050in}}%
\pgfpathlineto{\pgfqpoint{3.035435in}{2.388584in}}%
\pgfpathlineto{\pgfqpoint{3.063363in}{2.391237in}}%
\pgfpathlineto{\pgfqpoint{3.091291in}{2.390996in}}%
\pgfpathlineto{\pgfqpoint{3.119219in}{2.387861in}}%
\pgfpathlineto{\pgfqpoint{3.147147in}{2.381849in}}%
\pgfpathlineto{\pgfqpoint{3.175075in}{2.372988in}}%
\pgfpathlineto{\pgfqpoint{3.207658in}{2.359109in}}%
\pgfpathlineto{\pgfqpoint{3.240240in}{2.341506in}}%
\pgfpathlineto{\pgfqpoint{3.272823in}{2.320298in}}%
\pgfpathlineto{\pgfqpoint{3.305405in}{2.295625in}}%
\pgfpathlineto{\pgfqpoint{3.342643in}{2.263403in}}%
\pgfpathlineto{\pgfqpoint{3.379880in}{2.227161in}}%
\pgfpathlineto{\pgfqpoint{3.421772in}{2.181990in}}%
\pgfpathlineto{\pgfqpoint{3.468318in}{2.126940in}}%
\pgfpathlineto{\pgfqpoint{3.519520in}{2.061383in}}%
\pgfpathlineto{\pgfqpoint{3.580030in}{1.978667in}}%
\pgfpathlineto{\pgfqpoint{3.663814in}{1.858410in}}%
\pgfpathlineto{\pgfqpoint{3.798799in}{1.664318in}}%
\pgfpathlineto{\pgfqpoint{3.859309in}{1.582802in}}%
\pgfpathlineto{\pgfqpoint{3.910511in}{1.518849in}}%
\pgfpathlineto{\pgfqpoint{3.957057in}{1.465843in}}%
\pgfpathlineto{\pgfqpoint{3.998949in}{1.423080in}}%
\pgfpathlineto{\pgfqpoint{4.036186in}{1.389497in}}%
\pgfpathlineto{\pgfqpoint{4.068769in}{1.363835in}}%
\pgfpathlineto{\pgfqpoint{4.101351in}{1.341875in}}%
\pgfpathlineto{\pgfqpoint{4.133934in}{1.323802in}}%
\pgfpathlineto{\pgfqpoint{4.161862in}{1.311516in}}%
\pgfpathlineto{\pgfqpoint{4.189790in}{1.302261in}}%
\pgfpathlineto{\pgfqpoint{4.217718in}{1.296083in}}%
\pgfpathlineto{\pgfqpoint{4.245646in}{1.293000in}}%
\pgfpathlineto{\pgfqpoint{4.273574in}{1.293002in}}%
\pgfpathlineto{\pgfqpoint{4.301502in}{1.296049in}}%
\pgfpathlineto{\pgfqpoint{4.329429in}{1.302070in}}%
\pgfpathlineto{\pgfqpoint{4.357357in}{1.310959in}}%
\pgfpathlineto{\pgfqpoint{4.389940in}{1.324759in}}%
\pgfpathlineto{\pgfqpoint{4.422523in}{1.341971in}}%
\pgfpathlineto{\pgfqpoint{4.459760in}{1.365338in}}%
\pgfpathlineto{\pgfqpoint{4.501652in}{1.395505in}}%
\pgfpathlineto{\pgfqpoint{4.557508in}{1.440050in}}%
\pgfpathlineto{\pgfqpoint{4.659910in}{1.522629in}}%
\pgfpathlineto{\pgfqpoint{4.697147in}{1.548247in}}%
\pgfpathlineto{\pgfqpoint{4.725075in}{1.564207in}}%
\pgfpathlineto{\pgfqpoint{4.753003in}{1.576431in}}%
\pgfpathlineto{\pgfqpoint{4.776276in}{1.583092in}}%
\pgfpathlineto{\pgfqpoint{4.794895in}{1.585699in}}%
\pgfpathlineto{\pgfqpoint{4.813514in}{1.585545in}}%
\pgfpathlineto{\pgfqpoint{4.832132in}{1.582301in}}%
\pgfpathlineto{\pgfqpoint{4.850751in}{1.575617in}}%
\pgfpathlineto{\pgfqpoint{4.869369in}{1.565131in}}%
\pgfpathlineto{\pgfqpoint{4.887988in}{1.550460in}}%
\pgfpathlineto{\pgfqpoint{4.906607in}{1.531206in}}%
\pgfpathlineto{\pgfqpoint{4.925225in}{1.506949in}}%
\pgfpathlineto{\pgfqpoint{4.943844in}{1.477254in}}%
\pgfpathlineto{\pgfqpoint{4.962462in}{1.441666in}}%
\pgfpathlineto{\pgfqpoint{4.981081in}{1.399709in}}%
\pgfpathlineto{\pgfqpoint{4.999700in}{1.350889in}}%
\pgfpathlineto{\pgfqpoint{5.018318in}{1.294690in}}%
\pgfpathlineto{\pgfqpoint{5.041592in}{1.213247in}}%
\pgfpathlineto{\pgfqpoint{5.064865in}{1.118342in}}%
\pgfpathlineto{\pgfqpoint{5.088138in}{1.008827in}}%
\pgfpathlineto{\pgfqpoint{5.111411in}{0.883500in}}%
\pgfpathlineto{\pgfqpoint{5.134685in}{0.741102in}}%
\pgfpathlineto{\pgfqpoint{5.157958in}{0.580316in}}%
\pgfpathlineto{\pgfqpoint{5.170418in}{0.486111in}}%
\pgfpathlineto{\pgfqpoint{5.170418in}{0.486111in}}%
\pgfusepath{stroke}%
\end{pgfscope}%
\begin{pgfscope}%
\pgfpathrectangle{\pgfqpoint{0.750000in}{0.500000in}}{\pgfqpoint{4.650000in}{3.020000in}}%
\pgfusepath{clip}%
\pgfsetrectcap%
\pgfsetroundjoin%
\pgfsetlinewidth{1.505625pt}%
\definecolor{currentstroke}{rgb}{0.580392,0.403922,0.741176}%
\pgfsetstrokecolor{currentstroke}%
\pgfsetdash{}{0pt}%
\pgfpathmoveto{\pgfqpoint{0.997377in}{3.533889in}}%
\pgfpathlineto{\pgfqpoint{1.019970in}{2.984893in}}%
\pgfpathlineto{\pgfqpoint{1.038589in}{2.597532in}}%
\pgfpathlineto{\pgfqpoint{1.057207in}{2.263052in}}%
\pgfpathlineto{\pgfqpoint{1.075826in}{1.976481in}}%
\pgfpathlineto{\pgfqpoint{1.094444in}{1.733185in}}%
\pgfpathlineto{\pgfqpoint{1.113063in}{1.528845in}}%
\pgfpathlineto{\pgfqpoint{1.131682in}{1.359445in}}%
\pgfpathlineto{\pgfqpoint{1.150300in}{1.221260in}}%
\pgfpathlineto{\pgfqpoint{1.168919in}{1.110835in}}%
\pgfpathlineto{\pgfqpoint{1.182883in}{1.044303in}}%
\pgfpathlineto{\pgfqpoint{1.196847in}{0.990328in}}%
\pgfpathlineto{\pgfqpoint{1.210811in}{0.947730in}}%
\pgfpathlineto{\pgfqpoint{1.224775in}{0.915399in}}%
\pgfpathlineto{\pgfqpoint{1.238739in}{0.892295in}}%
\pgfpathlineto{\pgfqpoint{1.248048in}{0.881534in}}%
\pgfpathlineto{\pgfqpoint{1.257357in}{0.874167in}}%
\pgfpathlineto{\pgfqpoint{1.266667in}{0.869932in}}%
\pgfpathlineto{\pgfqpoint{1.275976in}{0.868581in}}%
\pgfpathlineto{\pgfqpoint{1.285285in}{0.869876in}}%
\pgfpathlineto{\pgfqpoint{1.294595in}{0.873590in}}%
\pgfpathlineto{\pgfqpoint{1.308559in}{0.883229in}}%
\pgfpathlineto{\pgfqpoint{1.322523in}{0.897144in}}%
\pgfpathlineto{\pgfqpoint{1.341141in}{0.921262in}}%
\pgfpathlineto{\pgfqpoint{1.359760in}{0.950491in}}%
\pgfpathlineto{\pgfqpoint{1.387688in}{1.001271in}}%
\pgfpathlineto{\pgfqpoint{1.424925in}{1.076511in}}%
\pgfpathlineto{\pgfqpoint{1.504054in}{1.238151in}}%
\pgfpathlineto{\pgfqpoint{1.536637in}{1.297705in}}%
\pgfpathlineto{\pgfqpoint{1.564565in}{1.343339in}}%
\pgfpathlineto{\pgfqpoint{1.592492in}{1.383263in}}%
\pgfpathlineto{\pgfqpoint{1.615766in}{1.411886in}}%
\pgfpathlineto{\pgfqpoint{1.639039in}{1.436185in}}%
\pgfpathlineto{\pgfqpoint{1.662312in}{1.456169in}}%
\pgfpathlineto{\pgfqpoint{1.685586in}{1.471940in}}%
\pgfpathlineto{\pgfqpoint{1.708859in}{1.483676in}}%
\pgfpathlineto{\pgfqpoint{1.732132in}{1.491616in}}%
\pgfpathlineto{\pgfqpoint{1.755405in}{1.496054in}}%
\pgfpathlineto{\pgfqpoint{1.778679in}{1.497327in}}%
\pgfpathlineto{\pgfqpoint{1.801952in}{1.495804in}}%
\pgfpathlineto{\pgfqpoint{1.829880in}{1.490841in}}%
\pgfpathlineto{\pgfqpoint{1.862462in}{1.481632in}}%
\pgfpathlineto{\pgfqpoint{1.904354in}{1.466162in}}%
\pgfpathlineto{\pgfqpoint{2.020721in}{1.420800in}}%
\pgfpathlineto{\pgfqpoint{2.053303in}{1.412039in}}%
\pgfpathlineto{\pgfqpoint{2.085886in}{1.406626in}}%
\pgfpathlineto{\pgfqpoint{2.113814in}{1.405182in}}%
\pgfpathlineto{\pgfqpoint{2.141742in}{1.407078in}}%
\pgfpathlineto{\pgfqpoint{2.165015in}{1.411428in}}%
\pgfpathlineto{\pgfqpoint{2.188288in}{1.418445in}}%
\pgfpathlineto{\pgfqpoint{2.211562in}{1.428231in}}%
\pgfpathlineto{\pgfqpoint{2.239489in}{1.443728in}}%
\pgfpathlineto{\pgfqpoint{2.267417in}{1.463374in}}%
\pgfpathlineto{\pgfqpoint{2.295345in}{1.487162in}}%
\pgfpathlineto{\pgfqpoint{2.323273in}{1.515023in}}%
\pgfpathlineto{\pgfqpoint{2.355856in}{1.552498in}}%
\pgfpathlineto{\pgfqpoint{2.388438in}{1.595035in}}%
\pgfpathlineto{\pgfqpoint{2.425676in}{1.649329in}}%
\pgfpathlineto{\pgfqpoint{2.467568in}{1.716763in}}%
\pgfpathlineto{\pgfqpoint{2.514114in}{1.798117in}}%
\pgfpathlineto{\pgfqpoint{2.574625in}{1.910876in}}%
\pgfpathlineto{\pgfqpoint{2.732883in}{2.210075in}}%
\pgfpathlineto{\pgfqpoint{2.779429in}{2.289909in}}%
\pgfpathlineto{\pgfqpoint{2.821321in}{2.355379in}}%
\pgfpathlineto{\pgfqpoint{2.858559in}{2.407349in}}%
\pgfpathlineto{\pgfqpoint{2.891141in}{2.447307in}}%
\pgfpathlineto{\pgfqpoint{2.919069in}{2.477055in}}%
\pgfpathlineto{\pgfqpoint{2.946997in}{2.502355in}}%
\pgfpathlineto{\pgfqpoint{2.970270in}{2.519870in}}%
\pgfpathlineto{\pgfqpoint{2.993544in}{2.534019in}}%
\pgfpathlineto{\pgfqpoint{3.016817in}{2.544711in}}%
\pgfpathlineto{\pgfqpoint{3.040090in}{2.551878in}}%
\pgfpathlineto{\pgfqpoint{3.063363in}{2.555473in}}%
\pgfpathlineto{\pgfqpoint{3.086637in}{2.555473in}}%
\pgfpathlineto{\pgfqpoint{3.109910in}{2.551878in}}%
\pgfpathlineto{\pgfqpoint{3.133183in}{2.544711in}}%
\pgfpathlineto{\pgfqpoint{3.156456in}{2.534019in}}%
\pgfpathlineto{\pgfqpoint{3.179730in}{2.519870in}}%
\pgfpathlineto{\pgfqpoint{3.203003in}{2.502355in}}%
\pgfpathlineto{\pgfqpoint{3.230931in}{2.477055in}}%
\pgfpathlineto{\pgfqpoint{3.258859in}{2.447307in}}%
\pgfpathlineto{\pgfqpoint{3.291441in}{2.407349in}}%
\pgfpathlineto{\pgfqpoint{3.324024in}{2.362217in}}%
\pgfpathlineto{\pgfqpoint{3.361261in}{2.305035in}}%
\pgfpathlineto{\pgfqpoint{3.403153in}{2.234697in}}%
\pgfpathlineto{\pgfqpoint{3.454354in}{2.142161in}}%
\pgfpathlineto{\pgfqpoint{3.533483in}{1.991325in}}%
\pgfpathlineto{\pgfqpoint{3.626577in}{1.815040in}}%
\pgfpathlineto{\pgfqpoint{3.677778in}{1.724621in}}%
\pgfpathlineto{\pgfqpoint{3.719670in}{1.656509in}}%
\pgfpathlineto{\pgfqpoint{3.756907in}{1.601503in}}%
\pgfpathlineto{\pgfqpoint{3.789489in}{1.558274in}}%
\pgfpathlineto{\pgfqpoint{3.822072in}{1.520054in}}%
\pgfpathlineto{\pgfqpoint{3.850000in}{1.491525in}}%
\pgfpathlineto{\pgfqpoint{3.877928in}{1.467052in}}%
\pgfpathlineto{\pgfqpoint{3.905856in}{1.446714in}}%
\pgfpathlineto{\pgfqpoint{3.933784in}{1.430528in}}%
\pgfpathlineto{\pgfqpoint{3.961712in}{1.418445in}}%
\pgfpathlineto{\pgfqpoint{3.984985in}{1.411428in}}%
\pgfpathlineto{\pgfqpoint{4.012913in}{1.406516in}}%
\pgfpathlineto{\pgfqpoint{4.040841in}{1.405200in}}%
\pgfpathlineto{\pgfqpoint{4.068769in}{1.407167in}}%
\pgfpathlineto{\pgfqpoint{4.096697in}{1.412039in}}%
\pgfpathlineto{\pgfqpoint{4.129279in}{1.420800in}}%
\pgfpathlineto{\pgfqpoint{4.171171in}{1.435667in}}%
\pgfpathlineto{\pgfqpoint{4.306156in}{1.487285in}}%
\pgfpathlineto{\pgfqpoint{4.338739in}{1.494499in}}%
\pgfpathlineto{\pgfqpoint{4.366667in}{1.497234in}}%
\pgfpathlineto{\pgfqpoint{4.389940in}{1.496550in}}%
\pgfpathlineto{\pgfqpoint{4.413213in}{1.492773in}}%
\pgfpathlineto{\pgfqpoint{4.436486in}{1.485559in}}%
\pgfpathlineto{\pgfqpoint{4.459760in}{1.474604in}}%
\pgfpathlineto{\pgfqpoint{4.483033in}{1.459656in}}%
\pgfpathlineto{\pgfqpoint{4.506306in}{1.440524in}}%
\pgfpathlineto{\pgfqpoint{4.529580in}{1.417093in}}%
\pgfpathlineto{\pgfqpoint{4.552853in}{1.389331in}}%
\pgfpathlineto{\pgfqpoint{4.580781in}{1.350403in}}%
\pgfpathlineto{\pgfqpoint{4.608709in}{1.305683in}}%
\pgfpathlineto{\pgfqpoint{4.641291in}{1.247024in}}%
\pgfpathlineto{\pgfqpoint{4.678529in}{1.173364in}}%
\pgfpathlineto{\pgfqpoint{4.780931in}{0.966633in}}%
\pgfpathlineto{\pgfqpoint{4.804204in}{0.928140in}}%
\pgfpathlineto{\pgfqpoint{4.822823in}{0.902622in}}%
\pgfpathlineto{\pgfqpoint{4.836787in}{0.887425in}}%
\pgfpathlineto{\pgfqpoint{4.850751in}{0.876287in}}%
\pgfpathlineto{\pgfqpoint{4.864715in}{0.869876in}}%
\pgfpathlineto{\pgfqpoint{4.874024in}{0.868581in}}%
\pgfpathlineto{\pgfqpoint{4.883333in}{0.869932in}}%
\pgfpathlineto{\pgfqpoint{4.892643in}{0.874167in}}%
\pgfpathlineto{\pgfqpoint{4.901952in}{0.881534in}}%
\pgfpathlineto{\pgfqpoint{4.911261in}{0.892295in}}%
\pgfpathlineto{\pgfqpoint{4.920571in}{0.906722in}}%
\pgfpathlineto{\pgfqpoint{4.934535in}{0.935865in}}%
\pgfpathlineto{\pgfqpoint{4.948498in}{0.974921in}}%
\pgfpathlineto{\pgfqpoint{4.962462in}{1.024976in}}%
\pgfpathlineto{\pgfqpoint{4.976426in}{1.087186in}}%
\pgfpathlineto{\pgfqpoint{4.990390in}{1.162783in}}%
\pgfpathlineto{\pgfqpoint{5.004354in}{1.253073in}}%
\pgfpathlineto{\pgfqpoint{5.018318in}{1.359445in}}%
\pgfpathlineto{\pgfqpoint{5.036937in}{1.528845in}}%
\pgfpathlineto{\pgfqpoint{5.055556in}{1.733185in}}%
\pgfpathlineto{\pgfqpoint{5.074174in}{1.976481in}}%
\pgfpathlineto{\pgfqpoint{5.092793in}{2.263052in}}%
\pgfpathlineto{\pgfqpoint{5.111411in}{2.597532in}}%
\pgfpathlineto{\pgfqpoint{5.130030in}{2.984893in}}%
\pgfpathlineto{\pgfqpoint{5.148649in}{3.430455in}}%
\pgfpathlineto{\pgfqpoint{5.152623in}{3.533889in}}%
\pgfpathlineto{\pgfqpoint{5.152623in}{3.533889in}}%
\pgfusepath{stroke}%
\end{pgfscope}%
\begin{pgfscope}%
\pgfpathrectangle{\pgfqpoint{0.750000in}{0.500000in}}{\pgfqpoint{4.650000in}{3.020000in}}%
\pgfusepath{clip}%
\pgfsetrectcap%
\pgfsetroundjoin%
\pgfsetlinewidth{1.505625pt}%
\definecolor{currentstroke}{rgb}{0.549020,0.337255,0.294118}%
\pgfsetstrokecolor{currentstroke}%
\pgfsetdash{}{0pt}%
\pgfpathmoveto{\pgfqpoint{0.750000in}{1.295811in}}%
\pgfpathlineto{\pgfqpoint{1.001351in}{1.305950in}}%
\pgfpathlineto{\pgfqpoint{1.206156in}{1.317243in}}%
\pgfpathlineto{\pgfqpoint{1.373724in}{1.329474in}}%
\pgfpathlineto{\pgfqpoint{1.518018in}{1.343075in}}%
\pgfpathlineto{\pgfqpoint{1.639039in}{1.357496in}}%
\pgfpathlineto{\pgfqpoint{1.746096in}{1.373329in}}%
\pgfpathlineto{\pgfqpoint{1.839189in}{1.390172in}}%
\pgfpathlineto{\pgfqpoint{1.922973in}{1.408478in}}%
\pgfpathlineto{\pgfqpoint{1.997447in}{1.427912in}}%
\pgfpathlineto{\pgfqpoint{2.062613in}{1.447953in}}%
\pgfpathlineto{\pgfqpoint{2.123123in}{1.469666in}}%
\pgfpathlineto{\pgfqpoint{2.178979in}{1.492916in}}%
\pgfpathlineto{\pgfqpoint{2.230180in}{1.517463in}}%
\pgfpathlineto{\pgfqpoint{2.276727in}{1.542957in}}%
\pgfpathlineto{\pgfqpoint{2.323273in}{1.572003in}}%
\pgfpathlineto{\pgfqpoint{2.365165in}{1.601680in}}%
\pgfpathlineto{\pgfqpoint{2.407057in}{1.635235in}}%
\pgfpathlineto{\pgfqpoint{2.444294in}{1.668793in}}%
\pgfpathlineto{\pgfqpoint{2.481532in}{1.706341in}}%
\pgfpathlineto{\pgfqpoint{2.518769in}{1.748387in}}%
\pgfpathlineto{\pgfqpoint{2.556006in}{1.795477in}}%
\pgfpathlineto{\pgfqpoint{2.593243in}{1.848168in}}%
\pgfpathlineto{\pgfqpoint{2.630480in}{1.907001in}}%
\pgfpathlineto{\pgfqpoint{2.667718in}{1.972439in}}%
\pgfpathlineto{\pgfqpoint{2.704955in}{2.044774in}}%
\pgfpathlineto{\pgfqpoint{2.742192in}{2.123995in}}%
\pgfpathlineto{\pgfqpoint{2.784084in}{2.220704in}}%
\pgfpathlineto{\pgfqpoint{2.839940in}{2.358824in}}%
\pgfpathlineto{\pgfqpoint{2.914414in}{2.543683in}}%
\pgfpathlineto{\pgfqpoint{2.946997in}{2.616442in}}%
\pgfpathlineto{\pgfqpoint{2.970270in}{2.662209in}}%
\pgfpathlineto{\pgfqpoint{2.993544in}{2.701099in}}%
\pgfpathlineto{\pgfqpoint{3.012162in}{2.726310in}}%
\pgfpathlineto{\pgfqpoint{3.030781in}{2.745589in}}%
\pgfpathlineto{\pgfqpoint{3.044745in}{2.755851in}}%
\pgfpathlineto{\pgfqpoint{3.058709in}{2.762336in}}%
\pgfpathlineto{\pgfqpoint{3.072673in}{2.764946in}}%
\pgfpathlineto{\pgfqpoint{3.086637in}{2.763640in}}%
\pgfpathlineto{\pgfqpoint{3.100601in}{2.758438in}}%
\pgfpathlineto{\pgfqpoint{3.114565in}{2.749421in}}%
\pgfpathlineto{\pgfqpoint{3.128529in}{2.736725in}}%
\pgfpathlineto{\pgfqpoint{3.147147in}{2.714409in}}%
\pgfpathlineto{\pgfqpoint{3.165766in}{2.686462in}}%
\pgfpathlineto{\pgfqpoint{3.189039in}{2.644644in}}%
\pgfpathlineto{\pgfqpoint{3.216967in}{2.586306in}}%
\pgfpathlineto{\pgfqpoint{3.249550in}{2.510294in}}%
\pgfpathlineto{\pgfqpoint{3.310060in}{2.358824in}}%
\pgfpathlineto{\pgfqpoint{3.370571in}{2.209604in}}%
\pgfpathlineto{\pgfqpoint{3.412462in}{2.113727in}}%
\pgfpathlineto{\pgfqpoint{3.449700in}{2.035352in}}%
\pgfpathlineto{\pgfqpoint{3.486937in}{1.963886in}}%
\pgfpathlineto{\pgfqpoint{3.524174in}{1.899294in}}%
\pgfpathlineto{\pgfqpoint{3.561411in}{1.841256in}}%
\pgfpathlineto{\pgfqpoint{3.598649in}{1.789296in}}%
\pgfpathlineto{\pgfqpoint{3.635886in}{1.742867in}}%
\pgfpathlineto{\pgfqpoint{3.673123in}{1.701412in}}%
\pgfpathlineto{\pgfqpoint{3.710360in}{1.664390in}}%
\pgfpathlineto{\pgfqpoint{3.752252in}{1.627414in}}%
\pgfpathlineto{\pgfqpoint{3.794144in}{1.594769in}}%
\pgfpathlineto{\pgfqpoint{3.836036in}{1.565882in}}%
\pgfpathlineto{\pgfqpoint{3.882583in}{1.537591in}}%
\pgfpathlineto{\pgfqpoint{3.929129in}{1.512744in}}%
\pgfpathlineto{\pgfqpoint{3.980330in}{1.488802in}}%
\pgfpathlineto{\pgfqpoint{4.036186in}{1.466107in}}%
\pgfpathlineto{\pgfqpoint{4.096697in}{1.444893in}}%
\pgfpathlineto{\pgfqpoint{4.161862in}{1.425296in}}%
\pgfpathlineto{\pgfqpoint{4.236336in}{1.406272in}}%
\pgfpathlineto{\pgfqpoint{4.315465in}{1.389250in}}%
\pgfpathlineto{\pgfqpoint{4.403904in}{1.373329in}}%
\pgfpathlineto{\pgfqpoint{4.506306in}{1.358118in}}%
\pgfpathlineto{\pgfqpoint{4.622673in}{1.344075in}}%
\pgfpathlineto{\pgfqpoint{4.757658in}{1.331048in}}%
\pgfpathlineto{\pgfqpoint{4.911261in}{1.319377in}}%
\pgfpathlineto{\pgfqpoint{5.097447in}{1.308470in}}%
\pgfpathlineto{\pgfqpoint{5.320871in}{1.298653in}}%
\pgfpathlineto{\pgfqpoint{5.400000in}{1.295811in}}%
\pgfpathlineto{\pgfqpoint{5.400000in}{1.295811in}}%
\pgfusepath{stroke}%
\end{pgfscope}%
\begin{pgfscope}%
\pgfsetrectcap%
\pgfsetmiterjoin%
\pgfsetlinewidth{0.803000pt}%
\definecolor{currentstroke}{rgb}{0.000000,0.000000,0.000000}%
\pgfsetstrokecolor{currentstroke}%
\pgfsetdash{}{0pt}%
\pgfpathmoveto{\pgfqpoint{0.750000in}{0.500000in}}%
\pgfpathlineto{\pgfqpoint{0.750000in}{3.520000in}}%
\pgfusepath{stroke}%
\end{pgfscope}%
\begin{pgfscope}%
\pgfsetrectcap%
\pgfsetmiterjoin%
\pgfsetlinewidth{0.803000pt}%
\definecolor{currentstroke}{rgb}{0.000000,0.000000,0.000000}%
\pgfsetstrokecolor{currentstroke}%
\pgfsetdash{}{0pt}%
\pgfpathmoveto{\pgfqpoint{5.400000in}{0.500000in}}%
\pgfpathlineto{\pgfqpoint{5.400000in}{3.520000in}}%
\pgfusepath{stroke}%
\end{pgfscope}%
\begin{pgfscope}%
\pgfsetrectcap%
\pgfsetmiterjoin%
\pgfsetlinewidth{0.803000pt}%
\definecolor{currentstroke}{rgb}{0.000000,0.000000,0.000000}%
\pgfsetstrokecolor{currentstroke}%
\pgfsetdash{}{0pt}%
\pgfpathmoveto{\pgfqpoint{0.750000in}{0.500000in}}%
\pgfpathlineto{\pgfqpoint{5.400000in}{0.500000in}}%
\pgfusepath{stroke}%
\end{pgfscope}%
\begin{pgfscope}%
\pgfsetrectcap%
\pgfsetmiterjoin%
\pgfsetlinewidth{0.803000pt}%
\definecolor{currentstroke}{rgb}{0.000000,0.000000,0.000000}%
\pgfsetstrokecolor{currentstroke}%
\pgfsetdash{}{0pt}%
\pgfpathmoveto{\pgfqpoint{0.750000in}{3.520000in}}%
\pgfpathlineto{\pgfqpoint{5.400000in}{3.520000in}}%
\pgfusepath{stroke}%
\end{pgfscope}%
\begin{pgfscope}%
\pgfsetbuttcap%
\pgfsetmiterjoin%
\definecolor{currentfill}{rgb}{1.000000,1.000000,1.000000}%
\pgfsetfillcolor{currentfill}%
\pgfsetfillopacity{0.800000}%
\pgfsetlinewidth{1.003750pt}%
\definecolor{currentstroke}{rgb}{0.800000,0.800000,0.800000}%
\pgfsetstrokecolor{currentstroke}%
\pgfsetstrokeopacity{0.800000}%
\pgfsetdash{}{0pt}%
\pgfpathmoveto{\pgfqpoint{4.520339in}{2.247161in}}%
\pgfpathlineto{\pgfqpoint{5.302778in}{2.247161in}}%
\pgfpathquadraticcurveto{\pgfqpoint{5.330556in}{2.247161in}}{\pgfqpoint{5.330556in}{2.274939in}}%
\pgfpathlineto{\pgfqpoint{5.330556in}{3.422778in}}%
\pgfpathquadraticcurveto{\pgfqpoint{5.330556in}{3.450556in}}{\pgfqpoint{5.302778in}{3.450556in}}%
\pgfpathlineto{\pgfqpoint{4.520339in}{3.450556in}}%
\pgfpathquadraticcurveto{\pgfqpoint{4.492561in}{3.450556in}}{\pgfqpoint{4.492561in}{3.422778in}}%
\pgfpathlineto{\pgfqpoint{4.492561in}{2.274939in}}%
\pgfpathquadraticcurveto{\pgfqpoint{4.492561in}{2.247161in}}{\pgfqpoint{4.520339in}{2.247161in}}%
\pgfpathclose%
\pgfusepath{stroke,fill}%
\end{pgfscope}%
\begin{pgfscope}%
\pgfsetrectcap%
\pgfsetroundjoin%
\pgfsetlinewidth{1.505625pt}%
\definecolor{currentstroke}{rgb}{0.121569,0.466667,0.705882}%
\pgfsetstrokecolor{currentstroke}%
\pgfsetdash{}{0pt}%
\pgfpathmoveto{\pgfqpoint{4.548117in}{3.346389in}}%
\pgfpathlineto{\pgfqpoint{4.825895in}{3.346389in}}%
\pgfusepath{stroke}%
\end{pgfscope}%
\begin{pgfscope}%
\pgftext[x=4.937006in,y=3.297778in,left,base]{\rmfamily\fontsize{10.000000}{12.000000}\selectfont \(\displaystyle  n = 1 \)}%
\end{pgfscope}%
\begin{pgfscope}%
\pgfsetrectcap%
\pgfsetroundjoin%
\pgfsetlinewidth{1.505625pt}%
\definecolor{currentstroke}{rgb}{1.000000,0.498039,0.054902}%
\pgfsetstrokecolor{currentstroke}%
\pgfsetdash{}{0pt}%
\pgfpathmoveto{\pgfqpoint{4.548117in}{3.152778in}}%
\pgfpathlineto{\pgfqpoint{4.825895in}{3.152778in}}%
\pgfusepath{stroke}%
\end{pgfscope}%
\begin{pgfscope}%
\pgftext[x=4.937006in,y=3.104167in,left,base]{\rmfamily\fontsize{10.000000}{12.000000}\selectfont \(\displaystyle  n = 3 \)}%
\end{pgfscope}%
\begin{pgfscope}%
\pgfsetrectcap%
\pgfsetroundjoin%
\pgfsetlinewidth{1.505625pt}%
\definecolor{currentstroke}{rgb}{0.172549,0.627451,0.172549}%
\pgfsetstrokecolor{currentstroke}%
\pgfsetdash{}{0pt}%
\pgfpathmoveto{\pgfqpoint{4.548117in}{2.959167in}}%
\pgfpathlineto{\pgfqpoint{4.825895in}{2.959167in}}%
\pgfusepath{stroke}%
\end{pgfscope}%
\begin{pgfscope}%
\pgftext[x=4.937006in,y=2.910556in,left,base]{\rmfamily\fontsize{10.000000}{12.000000}\selectfont \(\displaystyle  n = 5 \)}%
\end{pgfscope}%
\begin{pgfscope}%
\pgfsetrectcap%
\pgfsetroundjoin%
\pgfsetlinewidth{1.505625pt}%
\definecolor{currentstroke}{rgb}{0.839216,0.152941,0.156863}%
\pgfsetstrokecolor{currentstroke}%
\pgfsetdash{}{0pt}%
\pgfpathmoveto{\pgfqpoint{4.548117in}{2.765556in}}%
\pgfpathlineto{\pgfqpoint{4.825895in}{2.765556in}}%
\pgfusepath{stroke}%
\end{pgfscope}%
\begin{pgfscope}%
\pgftext[x=4.937006in,y=2.716945in,left,base]{\rmfamily\fontsize{10.000000}{12.000000}\selectfont \(\displaystyle  n = 7 \)}%
\end{pgfscope}%
\begin{pgfscope}%
\pgfsetrectcap%
\pgfsetroundjoin%
\pgfsetlinewidth{1.505625pt}%
\definecolor{currentstroke}{rgb}{0.580392,0.403922,0.741176}%
\pgfsetstrokecolor{currentstroke}%
\pgfsetdash{}{0pt}%
\pgfpathmoveto{\pgfqpoint{4.548117in}{2.571945in}}%
\pgfpathlineto{\pgfqpoint{4.825895in}{2.571945in}}%
\pgfusepath{stroke}%
\end{pgfscope}%
\begin{pgfscope}%
\pgftext[x=4.937006in,y=2.523334in,left,base]{\rmfamily\fontsize{10.000000}{12.000000}\selectfont \(\displaystyle  n = 9 \)}%
\end{pgfscope}%
\begin{pgfscope}%
\pgfsetrectcap%
\pgfsetroundjoin%
\pgfsetlinewidth{1.505625pt}%
\definecolor{currentstroke}{rgb}{0.549020,0.337255,0.294118}%
\pgfsetstrokecolor{currentstroke}%
\pgfsetdash{}{0pt}%
\pgfpathmoveto{\pgfqpoint{4.548117in}{2.378334in}}%
\pgfpathlineto{\pgfqpoint{4.825895in}{2.378334in}}%
\pgfusepath{stroke}%
\end{pgfscope}%
\begin{pgfscope}%
\pgftext[x=4.937006in,y=2.329723in,left,base]{\rmfamily\fontsize{10.000000}{12.000000}\selectfont \(\displaystyle f_1\)}%
\end{pgfscope}%
\end{pgfpicture}%
\makeatother%
\endgroup%
}
\caption{Graph of $f$ and $p_{\cdot}$} \label{Fig:Origin}
\end{figure}
\begin{figure}[htbp]
\centering \scalebox{0.8}{%% Creator: Matplotlib, PGF backend
%%
%% To include the figure in your LaTeX document, write
%%   \input{<filename>.pgf}
%%
%% Make sure the required packages are loaded in your preamble
%%   \usepackage{pgf}
%%
%% Figures using additional raster images can only be included by \input if
%% they are in the same directory as the main LaTeX file. For loading figures
%% from other directories you can use the `import` package
%%   \usepackage{import}
%% and then include the figures with
%%   \import{<path to file>}{<filename>.pgf}
%%
%% Matplotlib used the following preamble
%%   \usepackage{fontspec}
%%
\begingroup%
\makeatletter%
\begin{pgfpicture}%
\pgfpathrectangle{\pgfpointorigin}{\pgfqpoint{6.000000in}{4.000000in}}%
\pgfusepath{use as bounding box, clip}%
\begin{pgfscope}%
\pgfsetbuttcap%
\pgfsetmiterjoin%
\definecolor{currentfill}{rgb}{1.000000,1.000000,1.000000}%
\pgfsetfillcolor{currentfill}%
\pgfsetlinewidth{0.000000pt}%
\definecolor{currentstroke}{rgb}{1.000000,1.000000,1.000000}%
\pgfsetstrokecolor{currentstroke}%
\pgfsetdash{}{0pt}%
\pgfpathmoveto{\pgfqpoint{0.000000in}{0.000000in}}%
\pgfpathlineto{\pgfqpoint{6.000000in}{0.000000in}}%
\pgfpathlineto{\pgfqpoint{6.000000in}{4.000000in}}%
\pgfpathlineto{\pgfqpoint{0.000000in}{4.000000in}}%
\pgfpathclose%
\pgfusepath{fill}%
\end{pgfscope}%
\begin{pgfscope}%
\pgfsetbuttcap%
\pgfsetmiterjoin%
\definecolor{currentfill}{rgb}{1.000000,1.000000,1.000000}%
\pgfsetfillcolor{currentfill}%
\pgfsetlinewidth{0.000000pt}%
\definecolor{currentstroke}{rgb}{0.000000,0.000000,0.000000}%
\pgfsetstrokecolor{currentstroke}%
\pgfsetstrokeopacity{0.000000}%
\pgfsetdash{}{0pt}%
\pgfpathmoveto{\pgfqpoint{0.750000in}{0.500000in}}%
\pgfpathlineto{\pgfqpoint{5.400000in}{0.500000in}}%
\pgfpathlineto{\pgfqpoint{5.400000in}{3.520000in}}%
\pgfpathlineto{\pgfqpoint{0.750000in}{3.520000in}}%
\pgfpathclose%
\pgfusepath{fill}%
\end{pgfscope}%
\begin{pgfscope}%
\pgfpathrectangle{\pgfqpoint{0.750000in}{0.500000in}}{\pgfqpoint{4.650000in}{3.020000in}}%
\pgfusepath{clip}%
\pgfsetrectcap%
\pgfsetroundjoin%
\pgfsetlinewidth{0.803000pt}%
\definecolor{currentstroke}{rgb}{0.690196,0.690196,0.690196}%
\pgfsetstrokecolor{currentstroke}%
\pgfsetdash{}{0pt}%
\pgfpathmoveto{\pgfqpoint{0.750000in}{0.500000in}}%
\pgfpathlineto{\pgfqpoint{0.750000in}{3.520000in}}%
\pgfusepath{stroke}%
\end{pgfscope}%
\begin{pgfscope}%
\pgfsetbuttcap%
\pgfsetroundjoin%
\definecolor{currentfill}{rgb}{0.000000,0.000000,0.000000}%
\pgfsetfillcolor{currentfill}%
\pgfsetlinewidth{0.803000pt}%
\definecolor{currentstroke}{rgb}{0.000000,0.000000,0.000000}%
\pgfsetstrokecolor{currentstroke}%
\pgfsetdash{}{0pt}%
\pgfsys@defobject{currentmarker}{\pgfqpoint{0.000000in}{-0.048611in}}{\pgfqpoint{0.000000in}{0.000000in}}{%
\pgfpathmoveto{\pgfqpoint{0.000000in}{0.000000in}}%
\pgfpathlineto{\pgfqpoint{0.000000in}{-0.048611in}}%
\pgfusepath{stroke,fill}%
}%
\begin{pgfscope}%
\pgfsys@transformshift{0.750000in}{0.500000in}%
\pgfsys@useobject{currentmarker}{}%
\end{pgfscope}%
\end{pgfscope}%
\begin{pgfscope}%
\pgftext[x=0.750000in,y=0.402778in,,top]{\rmfamily\fontsize{10.000000}{12.000000}\selectfont \(\displaystyle -6\)}%
\end{pgfscope}%
\begin{pgfscope}%
\pgfpathrectangle{\pgfqpoint{0.750000in}{0.500000in}}{\pgfqpoint{4.650000in}{3.020000in}}%
\pgfusepath{clip}%
\pgfsetrectcap%
\pgfsetroundjoin%
\pgfsetlinewidth{0.803000pt}%
\definecolor{currentstroke}{rgb}{0.690196,0.690196,0.690196}%
\pgfsetstrokecolor{currentstroke}%
\pgfsetdash{}{0pt}%
\pgfpathmoveto{\pgfqpoint{1.525000in}{0.500000in}}%
\pgfpathlineto{\pgfqpoint{1.525000in}{3.520000in}}%
\pgfusepath{stroke}%
\end{pgfscope}%
\begin{pgfscope}%
\pgfsetbuttcap%
\pgfsetroundjoin%
\definecolor{currentfill}{rgb}{0.000000,0.000000,0.000000}%
\pgfsetfillcolor{currentfill}%
\pgfsetlinewidth{0.803000pt}%
\definecolor{currentstroke}{rgb}{0.000000,0.000000,0.000000}%
\pgfsetstrokecolor{currentstroke}%
\pgfsetdash{}{0pt}%
\pgfsys@defobject{currentmarker}{\pgfqpoint{0.000000in}{-0.048611in}}{\pgfqpoint{0.000000in}{0.000000in}}{%
\pgfpathmoveto{\pgfqpoint{0.000000in}{0.000000in}}%
\pgfpathlineto{\pgfqpoint{0.000000in}{-0.048611in}}%
\pgfusepath{stroke,fill}%
}%
\begin{pgfscope}%
\pgfsys@transformshift{1.525000in}{0.500000in}%
\pgfsys@useobject{currentmarker}{}%
\end{pgfscope}%
\end{pgfscope}%
\begin{pgfscope}%
\pgftext[x=1.525000in,y=0.402778in,,top]{\rmfamily\fontsize{10.000000}{12.000000}\selectfont \(\displaystyle -4\)}%
\end{pgfscope}%
\begin{pgfscope}%
\pgfpathrectangle{\pgfqpoint{0.750000in}{0.500000in}}{\pgfqpoint{4.650000in}{3.020000in}}%
\pgfusepath{clip}%
\pgfsetrectcap%
\pgfsetroundjoin%
\pgfsetlinewidth{0.803000pt}%
\definecolor{currentstroke}{rgb}{0.690196,0.690196,0.690196}%
\pgfsetstrokecolor{currentstroke}%
\pgfsetdash{}{0pt}%
\pgfpathmoveto{\pgfqpoint{2.300000in}{0.500000in}}%
\pgfpathlineto{\pgfqpoint{2.300000in}{3.520000in}}%
\pgfusepath{stroke}%
\end{pgfscope}%
\begin{pgfscope}%
\pgfsetbuttcap%
\pgfsetroundjoin%
\definecolor{currentfill}{rgb}{0.000000,0.000000,0.000000}%
\pgfsetfillcolor{currentfill}%
\pgfsetlinewidth{0.803000pt}%
\definecolor{currentstroke}{rgb}{0.000000,0.000000,0.000000}%
\pgfsetstrokecolor{currentstroke}%
\pgfsetdash{}{0pt}%
\pgfsys@defobject{currentmarker}{\pgfqpoint{0.000000in}{-0.048611in}}{\pgfqpoint{0.000000in}{0.000000in}}{%
\pgfpathmoveto{\pgfqpoint{0.000000in}{0.000000in}}%
\pgfpathlineto{\pgfqpoint{0.000000in}{-0.048611in}}%
\pgfusepath{stroke,fill}%
}%
\begin{pgfscope}%
\pgfsys@transformshift{2.300000in}{0.500000in}%
\pgfsys@useobject{currentmarker}{}%
\end{pgfscope}%
\end{pgfscope}%
\begin{pgfscope}%
\pgftext[x=2.300000in,y=0.402778in,,top]{\rmfamily\fontsize{10.000000}{12.000000}\selectfont \(\displaystyle -2\)}%
\end{pgfscope}%
\begin{pgfscope}%
\pgfpathrectangle{\pgfqpoint{0.750000in}{0.500000in}}{\pgfqpoint{4.650000in}{3.020000in}}%
\pgfusepath{clip}%
\pgfsetrectcap%
\pgfsetroundjoin%
\pgfsetlinewidth{0.803000pt}%
\definecolor{currentstroke}{rgb}{0.690196,0.690196,0.690196}%
\pgfsetstrokecolor{currentstroke}%
\pgfsetdash{}{0pt}%
\pgfpathmoveto{\pgfqpoint{3.075000in}{0.500000in}}%
\pgfpathlineto{\pgfqpoint{3.075000in}{3.520000in}}%
\pgfusepath{stroke}%
\end{pgfscope}%
\begin{pgfscope}%
\pgfsetbuttcap%
\pgfsetroundjoin%
\definecolor{currentfill}{rgb}{0.000000,0.000000,0.000000}%
\pgfsetfillcolor{currentfill}%
\pgfsetlinewidth{0.803000pt}%
\definecolor{currentstroke}{rgb}{0.000000,0.000000,0.000000}%
\pgfsetstrokecolor{currentstroke}%
\pgfsetdash{}{0pt}%
\pgfsys@defobject{currentmarker}{\pgfqpoint{0.000000in}{-0.048611in}}{\pgfqpoint{0.000000in}{0.000000in}}{%
\pgfpathmoveto{\pgfqpoint{0.000000in}{0.000000in}}%
\pgfpathlineto{\pgfqpoint{0.000000in}{-0.048611in}}%
\pgfusepath{stroke,fill}%
}%
\begin{pgfscope}%
\pgfsys@transformshift{3.075000in}{0.500000in}%
\pgfsys@useobject{currentmarker}{}%
\end{pgfscope}%
\end{pgfscope}%
\begin{pgfscope}%
\pgftext[x=3.075000in,y=0.402778in,,top]{\rmfamily\fontsize{10.000000}{12.000000}\selectfont \(\displaystyle 0\)}%
\end{pgfscope}%
\begin{pgfscope}%
\pgfpathrectangle{\pgfqpoint{0.750000in}{0.500000in}}{\pgfqpoint{4.650000in}{3.020000in}}%
\pgfusepath{clip}%
\pgfsetrectcap%
\pgfsetroundjoin%
\pgfsetlinewidth{0.803000pt}%
\definecolor{currentstroke}{rgb}{0.690196,0.690196,0.690196}%
\pgfsetstrokecolor{currentstroke}%
\pgfsetdash{}{0pt}%
\pgfpathmoveto{\pgfqpoint{3.850000in}{0.500000in}}%
\pgfpathlineto{\pgfqpoint{3.850000in}{3.520000in}}%
\pgfusepath{stroke}%
\end{pgfscope}%
\begin{pgfscope}%
\pgfsetbuttcap%
\pgfsetroundjoin%
\definecolor{currentfill}{rgb}{0.000000,0.000000,0.000000}%
\pgfsetfillcolor{currentfill}%
\pgfsetlinewidth{0.803000pt}%
\definecolor{currentstroke}{rgb}{0.000000,0.000000,0.000000}%
\pgfsetstrokecolor{currentstroke}%
\pgfsetdash{}{0pt}%
\pgfsys@defobject{currentmarker}{\pgfqpoint{0.000000in}{-0.048611in}}{\pgfqpoint{0.000000in}{0.000000in}}{%
\pgfpathmoveto{\pgfqpoint{0.000000in}{0.000000in}}%
\pgfpathlineto{\pgfqpoint{0.000000in}{-0.048611in}}%
\pgfusepath{stroke,fill}%
}%
\begin{pgfscope}%
\pgfsys@transformshift{3.850000in}{0.500000in}%
\pgfsys@useobject{currentmarker}{}%
\end{pgfscope}%
\end{pgfscope}%
\begin{pgfscope}%
\pgftext[x=3.850000in,y=0.402778in,,top]{\rmfamily\fontsize{10.000000}{12.000000}\selectfont \(\displaystyle 2\)}%
\end{pgfscope}%
\begin{pgfscope}%
\pgfpathrectangle{\pgfqpoint{0.750000in}{0.500000in}}{\pgfqpoint{4.650000in}{3.020000in}}%
\pgfusepath{clip}%
\pgfsetrectcap%
\pgfsetroundjoin%
\pgfsetlinewidth{0.803000pt}%
\definecolor{currentstroke}{rgb}{0.690196,0.690196,0.690196}%
\pgfsetstrokecolor{currentstroke}%
\pgfsetdash{}{0pt}%
\pgfpathmoveto{\pgfqpoint{4.625000in}{0.500000in}}%
\pgfpathlineto{\pgfqpoint{4.625000in}{3.520000in}}%
\pgfusepath{stroke}%
\end{pgfscope}%
\begin{pgfscope}%
\pgfsetbuttcap%
\pgfsetroundjoin%
\definecolor{currentfill}{rgb}{0.000000,0.000000,0.000000}%
\pgfsetfillcolor{currentfill}%
\pgfsetlinewidth{0.803000pt}%
\definecolor{currentstroke}{rgb}{0.000000,0.000000,0.000000}%
\pgfsetstrokecolor{currentstroke}%
\pgfsetdash{}{0pt}%
\pgfsys@defobject{currentmarker}{\pgfqpoint{0.000000in}{-0.048611in}}{\pgfqpoint{0.000000in}{0.000000in}}{%
\pgfpathmoveto{\pgfqpoint{0.000000in}{0.000000in}}%
\pgfpathlineto{\pgfqpoint{0.000000in}{-0.048611in}}%
\pgfusepath{stroke,fill}%
}%
\begin{pgfscope}%
\pgfsys@transformshift{4.625000in}{0.500000in}%
\pgfsys@useobject{currentmarker}{}%
\end{pgfscope}%
\end{pgfscope}%
\begin{pgfscope}%
\pgftext[x=4.625000in,y=0.402778in,,top]{\rmfamily\fontsize{10.000000}{12.000000}\selectfont \(\displaystyle 4\)}%
\end{pgfscope}%
\begin{pgfscope}%
\pgfpathrectangle{\pgfqpoint{0.750000in}{0.500000in}}{\pgfqpoint{4.650000in}{3.020000in}}%
\pgfusepath{clip}%
\pgfsetrectcap%
\pgfsetroundjoin%
\pgfsetlinewidth{0.803000pt}%
\definecolor{currentstroke}{rgb}{0.690196,0.690196,0.690196}%
\pgfsetstrokecolor{currentstroke}%
\pgfsetdash{}{0pt}%
\pgfpathmoveto{\pgfqpoint{5.400000in}{0.500000in}}%
\pgfpathlineto{\pgfqpoint{5.400000in}{3.520000in}}%
\pgfusepath{stroke}%
\end{pgfscope}%
\begin{pgfscope}%
\pgfsetbuttcap%
\pgfsetroundjoin%
\definecolor{currentfill}{rgb}{0.000000,0.000000,0.000000}%
\pgfsetfillcolor{currentfill}%
\pgfsetlinewidth{0.803000pt}%
\definecolor{currentstroke}{rgb}{0.000000,0.000000,0.000000}%
\pgfsetstrokecolor{currentstroke}%
\pgfsetdash{}{0pt}%
\pgfsys@defobject{currentmarker}{\pgfqpoint{0.000000in}{-0.048611in}}{\pgfqpoint{0.000000in}{0.000000in}}{%
\pgfpathmoveto{\pgfqpoint{0.000000in}{0.000000in}}%
\pgfpathlineto{\pgfqpoint{0.000000in}{-0.048611in}}%
\pgfusepath{stroke,fill}%
}%
\begin{pgfscope}%
\pgfsys@transformshift{5.400000in}{0.500000in}%
\pgfsys@useobject{currentmarker}{}%
\end{pgfscope}%
\end{pgfscope}%
\begin{pgfscope}%
\pgftext[x=5.400000in,y=0.402778in,,top]{\rmfamily\fontsize{10.000000}{12.000000}\selectfont \(\displaystyle 6\)}%
\end{pgfscope}%
\begin{pgfscope}%
\pgfpathrectangle{\pgfqpoint{0.750000in}{0.500000in}}{\pgfqpoint{4.650000in}{3.020000in}}%
\pgfusepath{clip}%
\pgfsetrectcap%
\pgfsetroundjoin%
\pgfsetlinewidth{0.803000pt}%
\definecolor{currentstroke}{rgb}{0.690196,0.690196,0.690196}%
\pgfsetstrokecolor{currentstroke}%
\pgfsetdash{}{0pt}%
\pgfpathmoveto{\pgfqpoint{0.750000in}{0.500000in}}%
\pgfpathlineto{\pgfqpoint{5.400000in}{0.500000in}}%
\pgfusepath{stroke}%
\end{pgfscope}%
\begin{pgfscope}%
\pgfsetbuttcap%
\pgfsetroundjoin%
\definecolor{currentfill}{rgb}{0.000000,0.000000,0.000000}%
\pgfsetfillcolor{currentfill}%
\pgfsetlinewidth{0.803000pt}%
\definecolor{currentstroke}{rgb}{0.000000,0.000000,0.000000}%
\pgfsetstrokecolor{currentstroke}%
\pgfsetdash{}{0pt}%
\pgfsys@defobject{currentmarker}{\pgfqpoint{-0.048611in}{0.000000in}}{\pgfqpoint{0.000000in}{0.000000in}}{%
\pgfpathmoveto{\pgfqpoint{0.000000in}{0.000000in}}%
\pgfpathlineto{\pgfqpoint{-0.048611in}{0.000000in}}%
\pgfusepath{stroke,fill}%
}%
\begin{pgfscope}%
\pgfsys@transformshift{0.750000in}{0.500000in}%
\pgfsys@useobject{currentmarker}{}%
\end{pgfscope}%
\end{pgfscope}%
\begin{pgfscope}%
\pgftext[x=0.297838in,y=0.451806in,left,base]{\rmfamily\fontsize{10.000000}{12.000000}\selectfont \(\displaystyle -0.50\)}%
\end{pgfscope}%
\begin{pgfscope}%
\pgfpathrectangle{\pgfqpoint{0.750000in}{0.500000in}}{\pgfqpoint{4.650000in}{3.020000in}}%
\pgfusepath{clip}%
\pgfsetrectcap%
\pgfsetroundjoin%
\pgfsetlinewidth{0.803000pt}%
\definecolor{currentstroke}{rgb}{0.690196,0.690196,0.690196}%
\pgfsetstrokecolor{currentstroke}%
\pgfsetdash{}{0pt}%
\pgfpathmoveto{\pgfqpoint{0.750000in}{0.877500in}}%
\pgfpathlineto{\pgfqpoint{5.400000in}{0.877500in}}%
\pgfusepath{stroke}%
\end{pgfscope}%
\begin{pgfscope}%
\pgfsetbuttcap%
\pgfsetroundjoin%
\definecolor{currentfill}{rgb}{0.000000,0.000000,0.000000}%
\pgfsetfillcolor{currentfill}%
\pgfsetlinewidth{0.803000pt}%
\definecolor{currentstroke}{rgb}{0.000000,0.000000,0.000000}%
\pgfsetstrokecolor{currentstroke}%
\pgfsetdash{}{0pt}%
\pgfsys@defobject{currentmarker}{\pgfqpoint{-0.048611in}{0.000000in}}{\pgfqpoint{0.000000in}{0.000000in}}{%
\pgfpathmoveto{\pgfqpoint{0.000000in}{0.000000in}}%
\pgfpathlineto{\pgfqpoint{-0.048611in}{0.000000in}}%
\pgfusepath{stroke,fill}%
}%
\begin{pgfscope}%
\pgfsys@transformshift{0.750000in}{0.877500in}%
\pgfsys@useobject{currentmarker}{}%
\end{pgfscope}%
\end{pgfscope}%
\begin{pgfscope}%
\pgftext[x=0.297838in,y=0.829306in,left,base]{\rmfamily\fontsize{10.000000}{12.000000}\selectfont \(\displaystyle -0.25\)}%
\end{pgfscope}%
\begin{pgfscope}%
\pgfpathrectangle{\pgfqpoint{0.750000in}{0.500000in}}{\pgfqpoint{4.650000in}{3.020000in}}%
\pgfusepath{clip}%
\pgfsetrectcap%
\pgfsetroundjoin%
\pgfsetlinewidth{0.803000pt}%
\definecolor{currentstroke}{rgb}{0.690196,0.690196,0.690196}%
\pgfsetstrokecolor{currentstroke}%
\pgfsetdash{}{0pt}%
\pgfpathmoveto{\pgfqpoint{0.750000in}{1.255000in}}%
\pgfpathlineto{\pgfqpoint{5.400000in}{1.255000in}}%
\pgfusepath{stroke}%
\end{pgfscope}%
\begin{pgfscope}%
\pgfsetbuttcap%
\pgfsetroundjoin%
\definecolor{currentfill}{rgb}{0.000000,0.000000,0.000000}%
\pgfsetfillcolor{currentfill}%
\pgfsetlinewidth{0.803000pt}%
\definecolor{currentstroke}{rgb}{0.000000,0.000000,0.000000}%
\pgfsetstrokecolor{currentstroke}%
\pgfsetdash{}{0pt}%
\pgfsys@defobject{currentmarker}{\pgfqpoint{-0.048611in}{0.000000in}}{\pgfqpoint{0.000000in}{0.000000in}}{%
\pgfpathmoveto{\pgfqpoint{0.000000in}{0.000000in}}%
\pgfpathlineto{\pgfqpoint{-0.048611in}{0.000000in}}%
\pgfusepath{stroke,fill}%
}%
\begin{pgfscope}%
\pgfsys@transformshift{0.750000in}{1.255000in}%
\pgfsys@useobject{currentmarker}{}%
\end{pgfscope}%
\end{pgfscope}%
\begin{pgfscope}%
\pgftext[x=0.405863in,y=1.206806in,left,base]{\rmfamily\fontsize{10.000000}{12.000000}\selectfont \(\displaystyle 0.00\)}%
\end{pgfscope}%
\begin{pgfscope}%
\pgfpathrectangle{\pgfqpoint{0.750000in}{0.500000in}}{\pgfqpoint{4.650000in}{3.020000in}}%
\pgfusepath{clip}%
\pgfsetrectcap%
\pgfsetroundjoin%
\pgfsetlinewidth{0.803000pt}%
\definecolor{currentstroke}{rgb}{0.690196,0.690196,0.690196}%
\pgfsetstrokecolor{currentstroke}%
\pgfsetdash{}{0pt}%
\pgfpathmoveto{\pgfqpoint{0.750000in}{1.632500in}}%
\pgfpathlineto{\pgfqpoint{5.400000in}{1.632500in}}%
\pgfusepath{stroke}%
\end{pgfscope}%
\begin{pgfscope}%
\pgfsetbuttcap%
\pgfsetroundjoin%
\definecolor{currentfill}{rgb}{0.000000,0.000000,0.000000}%
\pgfsetfillcolor{currentfill}%
\pgfsetlinewidth{0.803000pt}%
\definecolor{currentstroke}{rgb}{0.000000,0.000000,0.000000}%
\pgfsetstrokecolor{currentstroke}%
\pgfsetdash{}{0pt}%
\pgfsys@defobject{currentmarker}{\pgfqpoint{-0.048611in}{0.000000in}}{\pgfqpoint{0.000000in}{0.000000in}}{%
\pgfpathmoveto{\pgfqpoint{0.000000in}{0.000000in}}%
\pgfpathlineto{\pgfqpoint{-0.048611in}{0.000000in}}%
\pgfusepath{stroke,fill}%
}%
\begin{pgfscope}%
\pgfsys@transformshift{0.750000in}{1.632500in}%
\pgfsys@useobject{currentmarker}{}%
\end{pgfscope}%
\end{pgfscope}%
\begin{pgfscope}%
\pgftext[x=0.405863in,y=1.584306in,left,base]{\rmfamily\fontsize{10.000000}{12.000000}\selectfont \(\displaystyle 0.25\)}%
\end{pgfscope}%
\begin{pgfscope}%
\pgfpathrectangle{\pgfqpoint{0.750000in}{0.500000in}}{\pgfqpoint{4.650000in}{3.020000in}}%
\pgfusepath{clip}%
\pgfsetrectcap%
\pgfsetroundjoin%
\pgfsetlinewidth{0.803000pt}%
\definecolor{currentstroke}{rgb}{0.690196,0.690196,0.690196}%
\pgfsetstrokecolor{currentstroke}%
\pgfsetdash{}{0pt}%
\pgfpathmoveto{\pgfqpoint{0.750000in}{2.010000in}}%
\pgfpathlineto{\pgfqpoint{5.400000in}{2.010000in}}%
\pgfusepath{stroke}%
\end{pgfscope}%
\begin{pgfscope}%
\pgfsetbuttcap%
\pgfsetroundjoin%
\definecolor{currentfill}{rgb}{0.000000,0.000000,0.000000}%
\pgfsetfillcolor{currentfill}%
\pgfsetlinewidth{0.803000pt}%
\definecolor{currentstroke}{rgb}{0.000000,0.000000,0.000000}%
\pgfsetstrokecolor{currentstroke}%
\pgfsetdash{}{0pt}%
\pgfsys@defobject{currentmarker}{\pgfqpoint{-0.048611in}{0.000000in}}{\pgfqpoint{0.000000in}{0.000000in}}{%
\pgfpathmoveto{\pgfqpoint{0.000000in}{0.000000in}}%
\pgfpathlineto{\pgfqpoint{-0.048611in}{0.000000in}}%
\pgfusepath{stroke,fill}%
}%
\begin{pgfscope}%
\pgfsys@transformshift{0.750000in}{2.010000in}%
\pgfsys@useobject{currentmarker}{}%
\end{pgfscope}%
\end{pgfscope}%
\begin{pgfscope}%
\pgftext[x=0.405863in,y=1.961806in,left,base]{\rmfamily\fontsize{10.000000}{12.000000}\selectfont \(\displaystyle 0.50\)}%
\end{pgfscope}%
\begin{pgfscope}%
\pgfpathrectangle{\pgfqpoint{0.750000in}{0.500000in}}{\pgfqpoint{4.650000in}{3.020000in}}%
\pgfusepath{clip}%
\pgfsetrectcap%
\pgfsetroundjoin%
\pgfsetlinewidth{0.803000pt}%
\definecolor{currentstroke}{rgb}{0.690196,0.690196,0.690196}%
\pgfsetstrokecolor{currentstroke}%
\pgfsetdash{}{0pt}%
\pgfpathmoveto{\pgfqpoint{0.750000in}{2.387500in}}%
\pgfpathlineto{\pgfqpoint{5.400000in}{2.387500in}}%
\pgfusepath{stroke}%
\end{pgfscope}%
\begin{pgfscope}%
\pgfsetbuttcap%
\pgfsetroundjoin%
\definecolor{currentfill}{rgb}{0.000000,0.000000,0.000000}%
\pgfsetfillcolor{currentfill}%
\pgfsetlinewidth{0.803000pt}%
\definecolor{currentstroke}{rgb}{0.000000,0.000000,0.000000}%
\pgfsetstrokecolor{currentstroke}%
\pgfsetdash{}{0pt}%
\pgfsys@defobject{currentmarker}{\pgfqpoint{-0.048611in}{0.000000in}}{\pgfqpoint{0.000000in}{0.000000in}}{%
\pgfpathmoveto{\pgfqpoint{0.000000in}{0.000000in}}%
\pgfpathlineto{\pgfqpoint{-0.048611in}{0.000000in}}%
\pgfusepath{stroke,fill}%
}%
\begin{pgfscope}%
\pgfsys@transformshift{0.750000in}{2.387500in}%
\pgfsys@useobject{currentmarker}{}%
\end{pgfscope}%
\end{pgfscope}%
\begin{pgfscope}%
\pgftext[x=0.405863in,y=2.339306in,left,base]{\rmfamily\fontsize{10.000000}{12.000000}\selectfont \(\displaystyle 0.75\)}%
\end{pgfscope}%
\begin{pgfscope}%
\pgfpathrectangle{\pgfqpoint{0.750000in}{0.500000in}}{\pgfqpoint{4.650000in}{3.020000in}}%
\pgfusepath{clip}%
\pgfsetrectcap%
\pgfsetroundjoin%
\pgfsetlinewidth{0.803000pt}%
\definecolor{currentstroke}{rgb}{0.690196,0.690196,0.690196}%
\pgfsetstrokecolor{currentstroke}%
\pgfsetdash{}{0pt}%
\pgfpathmoveto{\pgfqpoint{0.750000in}{2.765000in}}%
\pgfpathlineto{\pgfqpoint{5.400000in}{2.765000in}}%
\pgfusepath{stroke}%
\end{pgfscope}%
\begin{pgfscope}%
\pgfsetbuttcap%
\pgfsetroundjoin%
\definecolor{currentfill}{rgb}{0.000000,0.000000,0.000000}%
\pgfsetfillcolor{currentfill}%
\pgfsetlinewidth{0.803000pt}%
\definecolor{currentstroke}{rgb}{0.000000,0.000000,0.000000}%
\pgfsetstrokecolor{currentstroke}%
\pgfsetdash{}{0pt}%
\pgfsys@defobject{currentmarker}{\pgfqpoint{-0.048611in}{0.000000in}}{\pgfqpoint{0.000000in}{0.000000in}}{%
\pgfpathmoveto{\pgfqpoint{0.000000in}{0.000000in}}%
\pgfpathlineto{\pgfqpoint{-0.048611in}{0.000000in}}%
\pgfusepath{stroke,fill}%
}%
\begin{pgfscope}%
\pgfsys@transformshift{0.750000in}{2.765000in}%
\pgfsys@useobject{currentmarker}{}%
\end{pgfscope}%
\end{pgfscope}%
\begin{pgfscope}%
\pgftext[x=0.405863in,y=2.716806in,left,base]{\rmfamily\fontsize{10.000000}{12.000000}\selectfont \(\displaystyle 1.00\)}%
\end{pgfscope}%
\begin{pgfscope}%
\pgfpathrectangle{\pgfqpoint{0.750000in}{0.500000in}}{\pgfqpoint{4.650000in}{3.020000in}}%
\pgfusepath{clip}%
\pgfsetrectcap%
\pgfsetroundjoin%
\pgfsetlinewidth{0.803000pt}%
\definecolor{currentstroke}{rgb}{0.690196,0.690196,0.690196}%
\pgfsetstrokecolor{currentstroke}%
\pgfsetdash{}{0pt}%
\pgfpathmoveto{\pgfqpoint{0.750000in}{3.142500in}}%
\pgfpathlineto{\pgfqpoint{5.400000in}{3.142500in}}%
\pgfusepath{stroke}%
\end{pgfscope}%
\begin{pgfscope}%
\pgfsetbuttcap%
\pgfsetroundjoin%
\definecolor{currentfill}{rgb}{0.000000,0.000000,0.000000}%
\pgfsetfillcolor{currentfill}%
\pgfsetlinewidth{0.803000pt}%
\definecolor{currentstroke}{rgb}{0.000000,0.000000,0.000000}%
\pgfsetstrokecolor{currentstroke}%
\pgfsetdash{}{0pt}%
\pgfsys@defobject{currentmarker}{\pgfqpoint{-0.048611in}{0.000000in}}{\pgfqpoint{0.000000in}{0.000000in}}{%
\pgfpathmoveto{\pgfqpoint{0.000000in}{0.000000in}}%
\pgfpathlineto{\pgfqpoint{-0.048611in}{0.000000in}}%
\pgfusepath{stroke,fill}%
}%
\begin{pgfscope}%
\pgfsys@transformshift{0.750000in}{3.142500in}%
\pgfsys@useobject{currentmarker}{}%
\end{pgfscope}%
\end{pgfscope}%
\begin{pgfscope}%
\pgftext[x=0.405863in,y=3.094306in,left,base]{\rmfamily\fontsize{10.000000}{12.000000}\selectfont \(\displaystyle 1.25\)}%
\end{pgfscope}%
\begin{pgfscope}%
\pgfpathrectangle{\pgfqpoint{0.750000in}{0.500000in}}{\pgfqpoint{4.650000in}{3.020000in}}%
\pgfusepath{clip}%
\pgfsetrectcap%
\pgfsetroundjoin%
\pgfsetlinewidth{0.803000pt}%
\definecolor{currentstroke}{rgb}{0.690196,0.690196,0.690196}%
\pgfsetstrokecolor{currentstroke}%
\pgfsetdash{}{0pt}%
\pgfpathmoveto{\pgfqpoint{0.750000in}{3.520000in}}%
\pgfpathlineto{\pgfqpoint{5.400000in}{3.520000in}}%
\pgfusepath{stroke}%
\end{pgfscope}%
\begin{pgfscope}%
\pgfsetbuttcap%
\pgfsetroundjoin%
\definecolor{currentfill}{rgb}{0.000000,0.000000,0.000000}%
\pgfsetfillcolor{currentfill}%
\pgfsetlinewidth{0.803000pt}%
\definecolor{currentstroke}{rgb}{0.000000,0.000000,0.000000}%
\pgfsetstrokecolor{currentstroke}%
\pgfsetdash{}{0pt}%
\pgfsys@defobject{currentmarker}{\pgfqpoint{-0.048611in}{0.000000in}}{\pgfqpoint{0.000000in}{0.000000in}}{%
\pgfpathmoveto{\pgfqpoint{0.000000in}{0.000000in}}%
\pgfpathlineto{\pgfqpoint{-0.048611in}{0.000000in}}%
\pgfusepath{stroke,fill}%
}%
\begin{pgfscope}%
\pgfsys@transformshift{0.750000in}{3.520000in}%
\pgfsys@useobject{currentmarker}{}%
\end{pgfscope}%
\end{pgfscope}%
\begin{pgfscope}%
\pgftext[x=0.405863in,y=3.471806in,left,base]{\rmfamily\fontsize{10.000000}{12.000000}\selectfont \(\displaystyle 1.50\)}%
\end{pgfscope}%
\begin{pgfscope}%
\pgfpathrectangle{\pgfqpoint{0.750000in}{0.500000in}}{\pgfqpoint{4.650000in}{3.020000in}}%
\pgfusepath{clip}%
\pgfsetrectcap%
\pgfsetroundjoin%
\pgfsetlinewidth{1.505625pt}%
\definecolor{currentstroke}{rgb}{0.121569,0.466667,0.705882}%
\pgfsetstrokecolor{currentstroke}%
\pgfsetdash{}{0pt}%
\pgfpathmoveto{\pgfqpoint{0.750000in}{0.674231in}}%
\pgfpathlineto{\pgfqpoint{0.838438in}{0.830263in}}%
\pgfpathlineto{\pgfqpoint{0.926877in}{0.980245in}}%
\pgfpathlineto{\pgfqpoint{1.015315in}{1.124177in}}%
\pgfpathlineto{\pgfqpoint{1.103754in}{1.262059in}}%
\pgfpathlineto{\pgfqpoint{1.187538in}{1.387103in}}%
\pgfpathlineto{\pgfqpoint{1.271321in}{1.506716in}}%
\pgfpathlineto{\pgfqpoint{1.355105in}{1.620900in}}%
\pgfpathlineto{\pgfqpoint{1.438889in}{1.729653in}}%
\pgfpathlineto{\pgfqpoint{1.518018in}{1.827379in}}%
\pgfpathlineto{\pgfqpoint{1.597147in}{1.920261in}}%
\pgfpathlineto{\pgfqpoint{1.676276in}{2.008299in}}%
\pgfpathlineto{\pgfqpoint{1.750751in}{2.086735in}}%
\pgfpathlineto{\pgfqpoint{1.825225in}{2.160879in}}%
\pgfpathlineto{\pgfqpoint{1.899700in}{2.230734in}}%
\pgfpathlineto{\pgfqpoint{1.974174in}{2.296297in}}%
\pgfpathlineto{\pgfqpoint{2.048649in}{2.357571in}}%
\pgfpathlineto{\pgfqpoint{2.118468in}{2.411118in}}%
\pgfpathlineto{\pgfqpoint{2.188288in}{2.460894in}}%
\pgfpathlineto{\pgfqpoint{2.258108in}{2.506899in}}%
\pgfpathlineto{\pgfqpoint{2.327928in}{2.549134in}}%
\pgfpathlineto{\pgfqpoint{2.397748in}{2.587597in}}%
\pgfpathlineto{\pgfqpoint{2.462913in}{2.620094in}}%
\pgfpathlineto{\pgfqpoint{2.528078in}{2.649306in}}%
\pgfpathlineto{\pgfqpoint{2.593243in}{2.675233in}}%
\pgfpathlineto{\pgfqpoint{2.658408in}{2.697875in}}%
\pgfpathlineto{\pgfqpoint{2.723574in}{2.717233in}}%
\pgfpathlineto{\pgfqpoint{2.788739in}{2.733305in}}%
\pgfpathlineto{\pgfqpoint{2.853904in}{2.746093in}}%
\pgfpathlineto{\pgfqpoint{2.919069in}{2.755596in}}%
\pgfpathlineto{\pgfqpoint{2.984234in}{2.761814in}}%
\pgfpathlineto{\pgfqpoint{3.049399in}{2.764747in}}%
\pgfpathlineto{\pgfqpoint{3.114565in}{2.764395in}}%
\pgfpathlineto{\pgfqpoint{3.179730in}{2.760758in}}%
\pgfpathlineto{\pgfqpoint{3.244895in}{2.753836in}}%
\pgfpathlineto{\pgfqpoint{3.310060in}{2.743629in}}%
\pgfpathlineto{\pgfqpoint{3.375225in}{2.730138in}}%
\pgfpathlineto{\pgfqpoint{3.440390in}{2.713361in}}%
\pgfpathlineto{\pgfqpoint{3.505556in}{2.693300in}}%
\pgfpathlineto{\pgfqpoint{3.570721in}{2.669954in}}%
\pgfpathlineto{\pgfqpoint{3.635886in}{2.643323in}}%
\pgfpathlineto{\pgfqpoint{3.701051in}{2.613407in}}%
\pgfpathlineto{\pgfqpoint{3.766216in}{2.580206in}}%
\pgfpathlineto{\pgfqpoint{3.831381in}{2.543720in}}%
\pgfpathlineto{\pgfqpoint{3.901201in}{2.500983in}}%
\pgfpathlineto{\pgfqpoint{3.971021in}{2.454475in}}%
\pgfpathlineto{\pgfqpoint{4.040841in}{2.404196in}}%
\pgfpathlineto{\pgfqpoint{4.110661in}{2.350146in}}%
\pgfpathlineto{\pgfqpoint{4.180480in}{2.292325in}}%
\pgfpathlineto{\pgfqpoint{4.254955in}{2.226493in}}%
\pgfpathlineto{\pgfqpoint{4.329429in}{2.156371in}}%
\pgfpathlineto{\pgfqpoint{4.403904in}{2.081958in}}%
\pgfpathlineto{\pgfqpoint{4.478378in}{2.003255in}}%
\pgfpathlineto{\pgfqpoint{4.557508in}{1.914931in}}%
\pgfpathlineto{\pgfqpoint{4.636637in}{1.821764in}}%
\pgfpathlineto{\pgfqpoint{4.715766in}{1.723754in}}%
\pgfpathlineto{\pgfqpoint{4.794895in}{1.620900in}}%
\pgfpathlineto{\pgfqpoint{4.878679in}{1.506716in}}%
\pgfpathlineto{\pgfqpoint{4.962462in}{1.387103in}}%
\pgfpathlineto{\pgfqpoint{5.046246in}{1.262059in}}%
\pgfpathlineto{\pgfqpoint{5.130030in}{1.131585in}}%
\pgfpathlineto{\pgfqpoint{5.218468in}{0.987972in}}%
\pgfpathlineto{\pgfqpoint{5.306907in}{0.838308in}}%
\pgfpathlineto{\pgfqpoint{5.395345in}{0.682594in}}%
\pgfpathlineto{\pgfqpoint{5.400000in}{0.674231in}}%
\pgfpathlineto{\pgfqpoint{5.400000in}{0.674231in}}%
\pgfusepath{stroke}%
\end{pgfscope}%
\begin{pgfscope}%
\pgfpathrectangle{\pgfqpoint{0.750000in}{0.500000in}}{\pgfqpoint{4.650000in}{3.020000in}}%
\pgfusepath{clip}%
\pgfsetrectcap%
\pgfsetroundjoin%
\pgfsetlinewidth{1.505625pt}%
\definecolor{currentstroke}{rgb}{1.000000,0.498039,0.054902}%
\pgfsetstrokecolor{currentstroke}%
\pgfsetdash{}{0pt}%
\pgfpathmoveto{\pgfqpoint{0.782751in}{3.533889in}}%
\pgfpathlineto{\pgfqpoint{0.819820in}{3.204623in}}%
\pgfpathlineto{\pgfqpoint{0.857057in}{2.899108in}}%
\pgfpathlineto{\pgfqpoint{0.894294in}{2.617904in}}%
\pgfpathlineto{\pgfqpoint{0.931532in}{2.360044in}}%
\pgfpathlineto{\pgfqpoint{0.968769in}{2.124574in}}%
\pgfpathlineto{\pgfqpoint{1.006006in}{1.910561in}}%
\pgfpathlineto{\pgfqpoint{1.038589in}{1.740176in}}%
\pgfpathlineto{\pgfqpoint{1.071171in}{1.584910in}}%
\pgfpathlineto{\pgfqpoint{1.103754in}{1.444165in}}%
\pgfpathlineto{\pgfqpoint{1.136336in}{1.317355in}}%
\pgfpathlineto{\pgfqpoint{1.168919in}{1.203905in}}%
\pgfpathlineto{\pgfqpoint{1.196847in}{1.116865in}}%
\pgfpathlineto{\pgfqpoint{1.224775in}{1.038871in}}%
\pgfpathlineto{\pgfqpoint{1.252703in}{0.969577in}}%
\pgfpathlineto{\pgfqpoint{1.280631in}{0.908643in}}%
\pgfpathlineto{\pgfqpoint{1.308559in}{0.855732in}}%
\pgfpathlineto{\pgfqpoint{1.331832in}{0.817530in}}%
\pgfpathlineto{\pgfqpoint{1.355105in}{0.784482in}}%
\pgfpathlineto{\pgfqpoint{1.378378in}{0.756400in}}%
\pgfpathlineto{\pgfqpoint{1.401652in}{0.733103in}}%
\pgfpathlineto{\pgfqpoint{1.424925in}{0.714408in}}%
\pgfpathlineto{\pgfqpoint{1.448198in}{0.700137in}}%
\pgfpathlineto{\pgfqpoint{1.471471in}{0.690114in}}%
\pgfpathlineto{\pgfqpoint{1.494745in}{0.684165in}}%
\pgfpathlineto{\pgfqpoint{1.518018in}{0.682119in}}%
\pgfpathlineto{\pgfqpoint{1.541291in}{0.683808in}}%
\pgfpathlineto{\pgfqpoint{1.564565in}{0.689065in}}%
\pgfpathlineto{\pgfqpoint{1.587838in}{0.697727in}}%
\pgfpathlineto{\pgfqpoint{1.611111in}{0.709633in}}%
\pgfpathlineto{\pgfqpoint{1.634384in}{0.724624in}}%
\pgfpathlineto{\pgfqpoint{1.662312in}{0.746466in}}%
\pgfpathlineto{\pgfqpoint{1.690240in}{0.772261in}}%
\pgfpathlineto{\pgfqpoint{1.722823in}{0.807005in}}%
\pgfpathlineto{\pgfqpoint{1.755405in}{0.846372in}}%
\pgfpathlineto{\pgfqpoint{1.792643in}{0.896519in}}%
\pgfpathlineto{\pgfqpoint{1.834535in}{0.958825in}}%
\pgfpathlineto{\pgfqpoint{1.876426in}{1.026608in}}%
\pgfpathlineto{\pgfqpoint{1.927628in}{1.115778in}}%
\pgfpathlineto{\pgfqpoint{1.983483in}{1.219508in}}%
\pgfpathlineto{\pgfqpoint{2.053303in}{1.356182in}}%
\pgfpathlineto{\pgfqpoint{2.151051in}{1.555173in}}%
\pgfpathlineto{\pgfqpoint{2.323273in}{1.906217in}}%
\pgfpathlineto{\pgfqpoint{2.397748in}{2.050607in}}%
\pgfpathlineto{\pgfqpoint{2.462913in}{2.170287in}}%
\pgfpathlineto{\pgfqpoint{2.518769in}{2.266702in}}%
\pgfpathlineto{\pgfqpoint{2.569970in}{2.349293in}}%
\pgfpathlineto{\pgfqpoint{2.616517in}{2.419040in}}%
\pgfpathlineto{\pgfqpoint{2.663063in}{2.483277in}}%
\pgfpathlineto{\pgfqpoint{2.704955in}{2.536070in}}%
\pgfpathlineto{\pgfqpoint{2.746847in}{2.583850in}}%
\pgfpathlineto{\pgfqpoint{2.784084in}{2.621936in}}%
\pgfpathlineto{\pgfqpoint{2.821321in}{2.655753in}}%
\pgfpathlineto{\pgfqpoint{2.858559in}{2.685180in}}%
\pgfpathlineto{\pgfqpoint{2.891141in}{2.707246in}}%
\pgfpathlineto{\pgfqpoint{2.923724in}{2.725814in}}%
\pgfpathlineto{\pgfqpoint{2.956306in}{2.740832in}}%
\pgfpathlineto{\pgfqpoint{2.988889in}{2.752262in}}%
\pgfpathlineto{\pgfqpoint{3.021471in}{2.760073in}}%
\pgfpathlineto{\pgfqpoint{3.054054in}{2.764245in}}%
\pgfpathlineto{\pgfqpoint{3.081982in}{2.764916in}}%
\pgfpathlineto{\pgfqpoint{3.109910in}{2.762904in}}%
\pgfpathlineto{\pgfqpoint{3.137838in}{2.758212in}}%
\pgfpathlineto{\pgfqpoint{3.170420in}{2.749364in}}%
\pgfpathlineto{\pgfqpoint{3.203003in}{2.736906in}}%
\pgfpathlineto{\pgfqpoint{3.235586in}{2.720869in}}%
\pgfpathlineto{\pgfqpoint{3.268168in}{2.701296in}}%
\pgfpathlineto{\pgfqpoint{3.300751in}{2.678241in}}%
\pgfpathlineto{\pgfqpoint{3.337988in}{2.647706in}}%
\pgfpathlineto{\pgfqpoint{3.375225in}{2.612810in}}%
\pgfpathlineto{\pgfqpoint{3.412462in}{2.573677in}}%
\pgfpathlineto{\pgfqpoint{3.454354in}{2.524763in}}%
\pgfpathlineto{\pgfqpoint{3.496246in}{2.470889in}}%
\pgfpathlineto{\pgfqpoint{3.542793in}{2.405519in}}%
\pgfpathlineto{\pgfqpoint{3.589339in}{2.334718in}}%
\pgfpathlineto{\pgfqpoint{3.640541in}{2.251070in}}%
\pgfpathlineto{\pgfqpoint{3.696396in}{2.153634in}}%
\pgfpathlineto{\pgfqpoint{3.756907in}{2.041799in}}%
\pgfpathlineto{\pgfqpoint{3.826727in}{1.906217in}}%
\pgfpathlineto{\pgfqpoint{3.919820in}{1.718034in}}%
\pgfpathlineto{\pgfqpoint{4.129279in}{1.291604in}}%
\pgfpathlineto{\pgfqpoint{4.194444in}{1.166901in}}%
\pgfpathlineto{\pgfqpoint{4.250300in}{1.066349in}}%
\pgfpathlineto{\pgfqpoint{4.296847in}{0.988321in}}%
\pgfpathlineto{\pgfqpoint{4.338739in}{0.923484in}}%
\pgfpathlineto{\pgfqpoint{4.375976in}{0.870791in}}%
\pgfpathlineto{\pgfqpoint{4.413213in}{0.823333in}}%
\pgfpathlineto{\pgfqpoint{4.445796in}{0.786559in}}%
\pgfpathlineto{\pgfqpoint{4.478378in}{0.754638in}}%
\pgfpathlineto{\pgfqpoint{4.506306in}{0.731449in}}%
\pgfpathlineto{\pgfqpoint{4.534234in}{0.712390in}}%
\pgfpathlineto{\pgfqpoint{4.557508in}{0.699854in}}%
\pgfpathlineto{\pgfqpoint{4.580781in}{0.690531in}}%
\pgfpathlineto{\pgfqpoint{4.604054in}{0.684579in}}%
\pgfpathlineto{\pgfqpoint{4.627327in}{0.682164in}}%
\pgfpathlineto{\pgfqpoint{4.650601in}{0.683449in}}%
\pgfpathlineto{\pgfqpoint{4.669219in}{0.687255in}}%
\pgfpathlineto{\pgfqpoint{4.687838in}{0.693624in}}%
\pgfpathlineto{\pgfqpoint{4.706456in}{0.702646in}}%
\pgfpathlineto{\pgfqpoint{4.729730in}{0.717787in}}%
\pgfpathlineto{\pgfqpoint{4.753003in}{0.737388in}}%
\pgfpathlineto{\pgfqpoint{4.776276in}{0.761628in}}%
\pgfpathlineto{\pgfqpoint{4.799550in}{0.790688in}}%
\pgfpathlineto{\pgfqpoint{4.822823in}{0.824752in}}%
\pgfpathlineto{\pgfqpoint{4.846096in}{0.864008in}}%
\pgfpathlineto{\pgfqpoint{4.869369in}{0.908643in}}%
\pgfpathlineto{\pgfqpoint{4.897297in}{0.969577in}}%
\pgfpathlineto{\pgfqpoint{4.925225in}{1.038871in}}%
\pgfpathlineto{\pgfqpoint{4.953153in}{1.116865in}}%
\pgfpathlineto{\pgfqpoint{4.981081in}{1.203905in}}%
\pgfpathlineto{\pgfqpoint{5.009009in}{1.300344in}}%
\pgfpathlineto{\pgfqpoint{5.041592in}{1.425209in}}%
\pgfpathlineto{\pgfqpoint{5.074174in}{1.563928in}}%
\pgfpathlineto{\pgfqpoint{5.106757in}{1.717084in}}%
\pgfpathlineto{\pgfqpoint{5.139339in}{1.885272in}}%
\pgfpathlineto{\pgfqpoint{5.171922in}{2.069096in}}%
\pgfpathlineto{\pgfqpoint{5.204505in}{2.269169in}}%
\pgfpathlineto{\pgfqpoint{5.241742in}{2.518523in}}%
\pgfpathlineto{\pgfqpoint{5.278979in}{2.790860in}}%
\pgfpathlineto{\pgfqpoint{5.316216in}{3.087144in}}%
\pgfpathlineto{\pgfqpoint{5.353453in}{3.408353in}}%
\pgfpathlineto{\pgfqpoint{5.367249in}{3.533889in}}%
\pgfpathlineto{\pgfqpoint{5.367249in}{3.533889in}}%
\pgfusepath{stroke}%
\end{pgfscope}%
\begin{pgfscope}%
\pgfpathrectangle{\pgfqpoint{0.750000in}{0.500000in}}{\pgfqpoint{4.650000in}{3.020000in}}%
\pgfusepath{clip}%
\pgfsetrectcap%
\pgfsetroundjoin%
\pgfsetlinewidth{1.505625pt}%
\definecolor{currentstroke}{rgb}{0.172549,0.627451,0.172549}%
\pgfsetstrokecolor{currentstroke}%
\pgfsetdash{}{0pt}%
\pgfpathmoveto{\pgfqpoint{1.069447in}{0.486111in}}%
\pgfpathlineto{\pgfqpoint{1.089790in}{0.769073in}}%
\pgfpathlineto{\pgfqpoint{1.113063in}{1.054324in}}%
\pgfpathlineto{\pgfqpoint{1.136336in}{1.301654in}}%
\pgfpathlineto{\pgfqpoint{1.159610in}{1.513944in}}%
\pgfpathlineto{\pgfqpoint{1.182883in}{1.693945in}}%
\pgfpathlineto{\pgfqpoint{1.201502in}{1.816462in}}%
\pgfpathlineto{\pgfqpoint{1.220120in}{1.921277in}}%
\pgfpathlineto{\pgfqpoint{1.238739in}{2.009623in}}%
\pgfpathlineto{\pgfqpoint{1.257357in}{2.082682in}}%
\pgfpathlineto{\pgfqpoint{1.275976in}{2.141591in}}%
\pgfpathlineto{\pgfqpoint{1.289940in}{2.177143in}}%
\pgfpathlineto{\pgfqpoint{1.303904in}{2.205790in}}%
\pgfpathlineto{\pgfqpoint{1.317868in}{2.227961in}}%
\pgfpathlineto{\pgfqpoint{1.331832in}{2.244068in}}%
\pgfpathlineto{\pgfqpoint{1.345796in}{2.254513in}}%
\pgfpathlineto{\pgfqpoint{1.359760in}{2.259681in}}%
\pgfpathlineto{\pgfqpoint{1.369069in}{2.260379in}}%
\pgfpathlineto{\pgfqpoint{1.383033in}{2.257575in}}%
\pgfpathlineto{\pgfqpoint{1.396997in}{2.250460in}}%
\pgfpathlineto{\pgfqpoint{1.410961in}{2.239373in}}%
\pgfpathlineto{\pgfqpoint{1.424925in}{2.224640in}}%
\pgfpathlineto{\pgfqpoint{1.443544in}{2.199868in}}%
\pgfpathlineto{\pgfqpoint{1.462162in}{2.169884in}}%
\pgfpathlineto{\pgfqpoint{1.485435in}{2.126100in}}%
\pgfpathlineto{\pgfqpoint{1.513363in}{2.065942in}}%
\pgfpathlineto{\pgfqpoint{1.545946in}{1.987743in}}%
\pgfpathlineto{\pgfqpoint{1.592492in}{1.866422in}}%
\pgfpathlineto{\pgfqpoint{1.722823in}{1.520225in}}%
\pgfpathlineto{\pgfqpoint{1.764715in}{1.420667in}}%
\pgfpathlineto{\pgfqpoint{1.797297in}{1.350425in}}%
\pgfpathlineto{\pgfqpoint{1.829880in}{1.287546in}}%
\pgfpathlineto{\pgfqpoint{1.857808in}{1.240100in}}%
\pgfpathlineto{\pgfqpoint{1.885736in}{1.199018in}}%
\pgfpathlineto{\pgfqpoint{1.909009in}{1.169861in}}%
\pgfpathlineto{\pgfqpoint{1.932282in}{1.145456in}}%
\pgfpathlineto{\pgfqpoint{1.955556in}{1.125892in}}%
\pgfpathlineto{\pgfqpoint{1.974174in}{1.113763in}}%
\pgfpathlineto{\pgfqpoint{1.992793in}{1.104777in}}%
\pgfpathlineto{\pgfqpoint{2.011411in}{1.098933in}}%
\pgfpathlineto{\pgfqpoint{2.030030in}{1.096217in}}%
\pgfpathlineto{\pgfqpoint{2.048649in}{1.096601in}}%
\pgfpathlineto{\pgfqpoint{2.067267in}{1.100048in}}%
\pgfpathlineto{\pgfqpoint{2.085886in}{1.106511in}}%
\pgfpathlineto{\pgfqpoint{2.104505in}{1.115930in}}%
\pgfpathlineto{\pgfqpoint{2.127778in}{1.131758in}}%
\pgfpathlineto{\pgfqpoint{2.151051in}{1.151949in}}%
\pgfpathlineto{\pgfqpoint{2.174324in}{1.176331in}}%
\pgfpathlineto{\pgfqpoint{2.202252in}{1.210855in}}%
\pgfpathlineto{\pgfqpoint{2.230180in}{1.250784in}}%
\pgfpathlineto{\pgfqpoint{2.262763in}{1.303683in}}%
\pgfpathlineto{\pgfqpoint{2.295345in}{1.362753in}}%
\pgfpathlineto{\pgfqpoint{2.332583in}{1.436921in}}%
\pgfpathlineto{\pgfqpoint{2.374474in}{1.527543in}}%
\pgfpathlineto{\pgfqpoint{2.425676in}{1.646352in}}%
\pgfpathlineto{\pgfqpoint{2.495495in}{1.817466in}}%
\pgfpathlineto{\pgfqpoint{2.639790in}{2.173767in}}%
\pgfpathlineto{\pgfqpoint{2.695646in}{2.301903in}}%
\pgfpathlineto{\pgfqpoint{2.742192in}{2.400680in}}%
\pgfpathlineto{\pgfqpoint{2.779429in}{2.473206in}}%
\pgfpathlineto{\pgfqpoint{2.816667in}{2.539086in}}%
\pgfpathlineto{\pgfqpoint{2.849249in}{2.590710in}}%
\pgfpathlineto{\pgfqpoint{2.881832in}{2.636263in}}%
\pgfpathlineto{\pgfqpoint{2.909760in}{2.670190in}}%
\pgfpathlineto{\pgfqpoint{2.937688in}{2.699175in}}%
\pgfpathlineto{\pgfqpoint{2.960961in}{2.719428in}}%
\pgfpathlineto{\pgfqpoint{2.984234in}{2.736043in}}%
\pgfpathlineto{\pgfqpoint{3.007508in}{2.748951in}}%
\pgfpathlineto{\pgfqpoint{3.030781in}{2.758100in}}%
\pgfpathlineto{\pgfqpoint{3.054054in}{2.763450in}}%
\pgfpathlineto{\pgfqpoint{3.077327in}{2.764981in}}%
\pgfpathlineto{\pgfqpoint{3.100601in}{2.762685in}}%
\pgfpathlineto{\pgfqpoint{3.123874in}{2.756573in}}%
\pgfpathlineto{\pgfqpoint{3.147147in}{2.746669in}}%
\pgfpathlineto{\pgfqpoint{3.170420in}{2.733014in}}%
\pgfpathlineto{\pgfqpoint{3.193694in}{2.715666in}}%
\pgfpathlineto{\pgfqpoint{3.216967in}{2.694695in}}%
\pgfpathlineto{\pgfqpoint{3.244895in}{2.664873in}}%
\pgfpathlineto{\pgfqpoint{3.272823in}{2.630142in}}%
\pgfpathlineto{\pgfqpoint{3.300751in}{2.590710in}}%
\pgfpathlineto{\pgfqpoint{3.333333in}{2.539086in}}%
\pgfpathlineto{\pgfqpoint{3.365916in}{2.481820in}}%
\pgfpathlineto{\pgfqpoint{3.403153in}{2.410082in}}%
\pgfpathlineto{\pgfqpoint{3.445045in}{2.322308in}}%
\pgfpathlineto{\pgfqpoint{3.491592in}{2.217465in}}%
\pgfpathlineto{\pgfqpoint{3.552102in}{2.072645in}}%
\pgfpathlineto{\pgfqpoint{3.659159in}{1.805865in}}%
\pgfpathlineto{\pgfqpoint{3.733634in}{1.624214in}}%
\pgfpathlineto{\pgfqpoint{3.784835in}{1.506824in}}%
\pgfpathlineto{\pgfqpoint{3.826727in}{1.417768in}}%
\pgfpathlineto{\pgfqpoint{3.863964in}{1.345286in}}%
\pgfpathlineto{\pgfqpoint{3.896547in}{1.287910in}}%
\pgfpathlineto{\pgfqpoint{3.929129in}{1.236897in}}%
\pgfpathlineto{\pgfqpoint{3.957057in}{1.198728in}}%
\pgfpathlineto{\pgfqpoint{3.984985in}{1.166087in}}%
\pgfpathlineto{\pgfqpoint{4.008258in}{1.143360in}}%
\pgfpathlineto{\pgfqpoint{4.031532in}{1.124894in}}%
\pgfpathlineto{\pgfqpoint{4.054805in}{1.110855in}}%
\pgfpathlineto{\pgfqpoint{4.073423in}{1.102906in}}%
\pgfpathlineto{\pgfqpoint{4.092042in}{1.097944in}}%
\pgfpathlineto{\pgfqpoint{4.110661in}{1.096023in}}%
\pgfpathlineto{\pgfqpoint{4.129279in}{1.097185in}}%
\pgfpathlineto{\pgfqpoint{4.147898in}{1.101463in}}%
\pgfpathlineto{\pgfqpoint{4.166517in}{1.108877in}}%
\pgfpathlineto{\pgfqpoint{4.185135in}{1.119435in}}%
\pgfpathlineto{\pgfqpoint{4.203754in}{1.133132in}}%
\pgfpathlineto{\pgfqpoint{4.227027in}{1.154641in}}%
\pgfpathlineto{\pgfqpoint{4.250300in}{1.180960in}}%
\pgfpathlineto{\pgfqpoint{4.273574in}{1.211985in}}%
\pgfpathlineto{\pgfqpoint{4.301502in}{1.255225in}}%
\pgfpathlineto{\pgfqpoint{4.329429in}{1.304712in}}%
\pgfpathlineto{\pgfqpoint{4.362012in}{1.369781in}}%
\pgfpathlineto{\pgfqpoint{4.394595in}{1.441958in}}%
\pgfpathlineto{\pgfqpoint{4.431832in}{1.531835in}}%
\pgfpathlineto{\pgfqpoint{4.478378in}{1.652508in}}%
\pgfpathlineto{\pgfqpoint{4.618018in}{2.022141in}}%
\pgfpathlineto{\pgfqpoint{4.650601in}{2.096946in}}%
\pgfpathlineto{\pgfqpoint{4.678529in}{2.153150in}}%
\pgfpathlineto{\pgfqpoint{4.701802in}{2.192833in}}%
\pgfpathlineto{\pgfqpoint{4.720420in}{2.218974in}}%
\pgfpathlineto{\pgfqpoint{4.739039in}{2.239373in}}%
\pgfpathlineto{\pgfqpoint{4.753003in}{2.250460in}}%
\pgfpathlineto{\pgfqpoint{4.766967in}{2.257575in}}%
\pgfpathlineto{\pgfqpoint{4.780931in}{2.260379in}}%
\pgfpathlineto{\pgfqpoint{4.790240in}{2.259681in}}%
\pgfpathlineto{\pgfqpoint{4.799550in}{2.256803in}}%
\pgfpathlineto{\pgfqpoint{4.808859in}{2.251637in}}%
\pgfpathlineto{\pgfqpoint{4.822823in}{2.239348in}}%
\pgfpathlineto{\pgfqpoint{4.836787in}{2.221265in}}%
\pgfpathlineto{\pgfqpoint{4.850751in}{2.196982in}}%
\pgfpathlineto{\pgfqpoint{4.864715in}{2.166081in}}%
\pgfpathlineto{\pgfqpoint{4.878679in}{2.128130in}}%
\pgfpathlineto{\pgfqpoint{4.892643in}{2.082682in}}%
\pgfpathlineto{\pgfqpoint{4.911261in}{2.009623in}}%
\pgfpathlineto{\pgfqpoint{4.929880in}{1.921277in}}%
\pgfpathlineto{\pgfqpoint{4.948498in}{1.816462in}}%
\pgfpathlineto{\pgfqpoint{4.967117in}{1.693945in}}%
\pgfpathlineto{\pgfqpoint{4.985736in}{1.552441in}}%
\pgfpathlineto{\pgfqpoint{5.004354in}{1.390617in}}%
\pgfpathlineto{\pgfqpoint{5.027628in}{1.157642in}}%
\pgfpathlineto{\pgfqpoint{5.050901in}{0.887914in}}%
\pgfpathlineto{\pgfqpoint{5.074174in}{0.578470in}}%
\pgfpathlineto{\pgfqpoint{5.080553in}{0.486111in}}%
\pgfpathlineto{\pgfqpoint{5.080553in}{0.486111in}}%
\pgfusepath{stroke}%
\end{pgfscope}%
\begin{pgfscope}%
\pgfpathrectangle{\pgfqpoint{0.750000in}{0.500000in}}{\pgfqpoint{4.650000in}{3.020000in}}%
\pgfusepath{clip}%
\pgfsetrectcap%
\pgfsetroundjoin%
\pgfsetlinewidth{1.505625pt}%
\definecolor{currentstroke}{rgb}{0.839216,0.152941,0.156863}%
\pgfsetstrokecolor{currentstroke}%
\pgfsetdash{}{0pt}%
\pgfpathmoveto{\pgfqpoint{1.072388in}{3.533889in}}%
\pgfpathlineto{\pgfqpoint{1.089790in}{2.808968in}}%
\pgfpathlineto{\pgfqpoint{1.108408in}{2.145400in}}%
\pgfpathlineto{\pgfqpoint{1.127027in}{1.585945in}}%
\pgfpathlineto{\pgfqpoint{1.145646in}{1.120219in}}%
\pgfpathlineto{\pgfqpoint{1.164264in}{0.738558in}}%
\pgfpathlineto{\pgfqpoint{1.179288in}{0.486111in}}%
\pgfpathmoveto{\pgfqpoint{1.463123in}{0.486111in}}%
\pgfpathlineto{\pgfqpoint{1.527327in}{0.877950in}}%
\pgfpathlineto{\pgfqpoint{1.564565in}{1.084929in}}%
\pgfpathlineto{\pgfqpoint{1.592492in}{1.224753in}}%
\pgfpathlineto{\pgfqpoint{1.620420in}{1.349212in}}%
\pgfpathlineto{\pgfqpoint{1.643694in}{1.440328in}}%
\pgfpathlineto{\pgfqpoint{1.666967in}{1.519595in}}%
\pgfpathlineto{\pgfqpoint{1.690240in}{1.586916in}}%
\pgfpathlineto{\pgfqpoint{1.708859in}{1.632249in}}%
\pgfpathlineto{\pgfqpoint{1.727477in}{1.670168in}}%
\pgfpathlineto{\pgfqpoint{1.746096in}{1.700895in}}%
\pgfpathlineto{\pgfqpoint{1.764715in}{1.724715in}}%
\pgfpathlineto{\pgfqpoint{1.778679in}{1.738246in}}%
\pgfpathlineto{\pgfqpoint{1.792643in}{1.748238in}}%
\pgfpathlineto{\pgfqpoint{1.806607in}{1.754860in}}%
\pgfpathlineto{\pgfqpoint{1.820571in}{1.758294in}}%
\pgfpathlineto{\pgfqpoint{1.834535in}{1.758728in}}%
\pgfpathlineto{\pgfqpoint{1.848498in}{1.756360in}}%
\pgfpathlineto{\pgfqpoint{1.862462in}{1.751393in}}%
\pgfpathlineto{\pgfqpoint{1.881081in}{1.741086in}}%
\pgfpathlineto{\pgfqpoint{1.899700in}{1.727029in}}%
\pgfpathlineto{\pgfqpoint{1.922973in}{1.704953in}}%
\pgfpathlineto{\pgfqpoint{1.950901in}{1.673162in}}%
\pgfpathlineto{\pgfqpoint{1.983483in}{1.630893in}}%
\pgfpathlineto{\pgfqpoint{2.048649in}{1.539467in}}%
\pgfpathlineto{\pgfqpoint{2.095195in}{1.476983in}}%
\pgfpathlineto{\pgfqpoint{2.127778in}{1.438321in}}%
\pgfpathlineto{\pgfqpoint{2.155706in}{1.410019in}}%
\pgfpathlineto{\pgfqpoint{2.178979in}{1.390550in}}%
\pgfpathlineto{\pgfqpoint{2.202252in}{1.375306in}}%
\pgfpathlineto{\pgfqpoint{2.220871in}{1.366407in}}%
\pgfpathlineto{\pgfqpoint{2.239489in}{1.360613in}}%
\pgfpathlineto{\pgfqpoint{2.258108in}{1.358052in}}%
\pgfpathlineto{\pgfqpoint{2.276727in}{1.358824in}}%
\pgfpathlineto{\pgfqpoint{2.295345in}{1.363000in}}%
\pgfpathlineto{\pgfqpoint{2.313964in}{1.370622in}}%
\pgfpathlineto{\pgfqpoint{2.332583in}{1.381704in}}%
\pgfpathlineto{\pgfqpoint{2.351201in}{1.396233in}}%
\pgfpathlineto{\pgfqpoint{2.369820in}{1.414172in}}%
\pgfpathlineto{\pgfqpoint{2.393093in}{1.441290in}}%
\pgfpathlineto{\pgfqpoint{2.416366in}{1.473459in}}%
\pgfpathlineto{\pgfqpoint{2.439640in}{1.510447in}}%
\pgfpathlineto{\pgfqpoint{2.467568in}{1.560798in}}%
\pgfpathlineto{\pgfqpoint{2.500150in}{1.627034in}}%
\pgfpathlineto{\pgfqpoint{2.532733in}{1.700335in}}%
\pgfpathlineto{\pgfqpoint{2.574625in}{1.803045in}}%
\pgfpathlineto{\pgfqpoint{2.625826in}{1.937736in}}%
\pgfpathlineto{\pgfqpoint{2.784084in}{2.360870in}}%
\pgfpathlineto{\pgfqpoint{2.825976in}{2.460250in}}%
\pgfpathlineto{\pgfqpoint{2.858559in}{2.530333in}}%
\pgfpathlineto{\pgfqpoint{2.891141in}{2.592872in}}%
\pgfpathlineto{\pgfqpoint{2.919069in}{2.639714in}}%
\pgfpathlineto{\pgfqpoint{2.942342in}{2.673559in}}%
\pgfpathlineto{\pgfqpoint{2.965616in}{2.702392in}}%
\pgfpathlineto{\pgfqpoint{2.988889in}{2.725979in}}%
\pgfpathlineto{\pgfqpoint{3.007508in}{2.740944in}}%
\pgfpathlineto{\pgfqpoint{3.026126in}{2.752352in}}%
\pgfpathlineto{\pgfqpoint{3.044745in}{2.760144in}}%
\pgfpathlineto{\pgfqpoint{3.063363in}{2.764281in}}%
\pgfpathlineto{\pgfqpoint{3.081982in}{2.764741in}}%
\pgfpathlineto{\pgfqpoint{3.100601in}{2.761522in}}%
\pgfpathlineto{\pgfqpoint{3.119219in}{2.754641in}}%
\pgfpathlineto{\pgfqpoint{3.137838in}{2.744132in}}%
\pgfpathlineto{\pgfqpoint{3.156456in}{2.730050in}}%
\pgfpathlineto{\pgfqpoint{3.175075in}{2.712468in}}%
\pgfpathlineto{\pgfqpoint{3.198348in}{2.685708in}}%
\pgfpathlineto{\pgfqpoint{3.221622in}{2.653837in}}%
\pgfpathlineto{\pgfqpoint{3.244895in}{2.617113in}}%
\pgfpathlineto{\pgfqpoint{3.272823in}{2.567060in}}%
\pgfpathlineto{\pgfqpoint{3.305405in}{2.501159in}}%
\pgfpathlineto{\pgfqpoint{3.337988in}{2.428178in}}%
\pgfpathlineto{\pgfqpoint{3.379880in}{2.325832in}}%
\pgfpathlineto{\pgfqpoint{3.431081in}{2.191452in}}%
\pgfpathlineto{\pgfqpoint{3.593994in}{1.756367in}}%
\pgfpathlineto{\pgfqpoint{3.631231in}{1.668128in}}%
\pgfpathlineto{\pgfqpoint{3.663814in}{1.597718in}}%
\pgfpathlineto{\pgfqpoint{3.696396in}{1.534840in}}%
\pgfpathlineto{\pgfqpoint{3.724324in}{1.487690in}}%
\pgfpathlineto{\pgfqpoint{3.747598in}{1.453563in}}%
\pgfpathlineto{\pgfqpoint{3.770871in}{1.424400in}}%
\pgfpathlineto{\pgfqpoint{3.794144in}{1.400400in}}%
\pgfpathlineto{\pgfqpoint{3.812763in}{1.385014in}}%
\pgfpathlineto{\pgfqpoint{3.831381in}{1.373068in}}%
\pgfpathlineto{\pgfqpoint{3.850000in}{1.364581in}}%
\pgfpathlineto{\pgfqpoint{3.868619in}{1.359547in}}%
\pgfpathlineto{\pgfqpoint{3.887237in}{1.357929in}}%
\pgfpathlineto{\pgfqpoint{3.905856in}{1.359665in}}%
\pgfpathlineto{\pgfqpoint{3.924474in}{1.364662in}}%
\pgfpathlineto{\pgfqpoint{3.943093in}{1.372798in}}%
\pgfpathlineto{\pgfqpoint{3.961712in}{1.383923in}}%
\pgfpathlineto{\pgfqpoint{3.984985in}{1.401751in}}%
\pgfpathlineto{\pgfqpoint{4.008258in}{1.423544in}}%
\pgfpathlineto{\pgfqpoint{4.036186in}{1.454230in}}%
\pgfpathlineto{\pgfqpoint{4.068769in}{1.495024in}}%
\pgfpathlineto{\pgfqpoint{4.115315in}{1.559132in}}%
\pgfpathlineto{\pgfqpoint{4.185135in}{1.655591in}}%
\pgfpathlineto{\pgfqpoint{4.217718in}{1.694925in}}%
\pgfpathlineto{\pgfqpoint{4.240991in}{1.718751in}}%
\pgfpathlineto{\pgfqpoint{4.264264in}{1.737904in}}%
\pgfpathlineto{\pgfqpoint{4.282883in}{1.749195in}}%
\pgfpathlineto{\pgfqpoint{4.301502in}{1.756360in}}%
\pgfpathlineto{\pgfqpoint{4.315465in}{1.758728in}}%
\pgfpathlineto{\pgfqpoint{4.329429in}{1.758294in}}%
\pgfpathlineto{\pgfqpoint{4.343393in}{1.754860in}}%
\pgfpathlineto{\pgfqpoint{4.357357in}{1.748238in}}%
\pgfpathlineto{\pgfqpoint{4.371321in}{1.738246in}}%
\pgfpathlineto{\pgfqpoint{4.385285in}{1.724715in}}%
\pgfpathlineto{\pgfqpoint{4.399249in}{1.707485in}}%
\pgfpathlineto{\pgfqpoint{4.417868in}{1.678514in}}%
\pgfpathlineto{\pgfqpoint{4.436486in}{1.642416in}}%
\pgfpathlineto{\pgfqpoint{4.455105in}{1.598954in}}%
\pgfpathlineto{\pgfqpoint{4.473724in}{1.547954in}}%
\pgfpathlineto{\pgfqpoint{4.496997in}{1.473467in}}%
\pgfpathlineto{\pgfqpoint{4.520270in}{1.387064in}}%
\pgfpathlineto{\pgfqpoint{4.543544in}{1.288992in}}%
\pgfpathlineto{\pgfqpoint{4.571471in}{1.156662in}}%
\pgfpathlineto{\pgfqpoint{4.599399in}{1.009812in}}%
\pgfpathlineto{\pgfqpoint{4.636637in}{0.795452in}}%
\pgfpathlineto{\pgfqpoint{4.686877in}{0.486111in}}%
\pgfpathmoveto{\pgfqpoint{4.970712in}{0.486111in}}%
\pgfpathlineto{\pgfqpoint{4.981081in}{0.655212in}}%
\pgfpathlineto{\pgfqpoint{4.995045in}{0.919457in}}%
\pgfpathlineto{\pgfqpoint{5.013664in}{1.341987in}}%
\pgfpathlineto{\pgfqpoint{5.032282in}{1.853325in}}%
\pgfpathlineto{\pgfqpoint{5.050901in}{2.463491in}}%
\pgfpathlineto{\pgfqpoint{5.069520in}{3.183248in}}%
\pgfpathlineto{\pgfqpoint{5.077612in}{3.533889in}}%
\pgfpathlineto{\pgfqpoint{5.077612in}{3.533889in}}%
\pgfusepath{stroke}%
\end{pgfscope}%
\begin{pgfscope}%
\pgfpathrectangle{\pgfqpoint{0.750000in}{0.500000in}}{\pgfqpoint{4.650000in}{3.020000in}}%
\pgfusepath{clip}%
\pgfsetrectcap%
\pgfsetroundjoin%
\pgfsetlinewidth{1.505625pt}%
\definecolor{currentstroke}{rgb}{0.580392,0.403922,0.741176}%
\pgfsetstrokecolor{currentstroke}%
\pgfsetdash{}{0pt}%
\pgfpathmoveto{\pgfqpoint{1.125217in}{0.486111in}}%
\pgfpathlineto{\pgfqpoint{1.136336in}{1.240825in}}%
\pgfpathlineto{\pgfqpoint{1.150300in}{2.032486in}}%
\pgfpathlineto{\pgfqpoint{1.164264in}{2.671853in}}%
\pgfpathlineto{\pgfqpoint{1.178228in}{3.176763in}}%
\pgfpathlineto{\pgfqpoint{1.191010in}{3.533889in}}%
\pgfpathmoveto{\pgfqpoint{1.339624in}{3.533889in}}%
\pgfpathlineto{\pgfqpoint{1.369069in}{3.140528in}}%
\pgfpathlineto{\pgfqpoint{1.434234in}{2.263910in}}%
\pgfpathlineto{\pgfqpoint{1.462162in}{1.931484in}}%
\pgfpathlineto{\pgfqpoint{1.485435in}{1.686145in}}%
\pgfpathlineto{\pgfqpoint{1.508709in}{1.473016in}}%
\pgfpathlineto{\pgfqpoint{1.527327in}{1.326725in}}%
\pgfpathlineto{\pgfqpoint{1.545946in}{1.202111in}}%
\pgfpathlineto{\pgfqpoint{1.564565in}{1.098825in}}%
\pgfpathlineto{\pgfqpoint{1.583183in}{1.016145in}}%
\pgfpathlineto{\pgfqpoint{1.597147in}{0.967049in}}%
\pgfpathlineto{\pgfqpoint{1.611111in}{0.928449in}}%
\pgfpathlineto{\pgfqpoint{1.625075in}{0.899762in}}%
\pgfpathlineto{\pgfqpoint{1.634384in}{0.885834in}}%
\pgfpathlineto{\pgfqpoint{1.643694in}{0.875836in}}%
\pgfpathlineto{\pgfqpoint{1.653003in}{0.869567in}}%
\pgfpathlineto{\pgfqpoint{1.662312in}{0.866819in}}%
\pgfpathlineto{\pgfqpoint{1.671622in}{0.867384in}}%
\pgfpathlineto{\pgfqpoint{1.680931in}{0.871050in}}%
\pgfpathlineto{\pgfqpoint{1.690240in}{0.877606in}}%
\pgfpathlineto{\pgfqpoint{1.704204in}{0.892396in}}%
\pgfpathlineto{\pgfqpoint{1.718168in}{0.912501in}}%
\pgfpathlineto{\pgfqpoint{1.736787in}{0.946374in}}%
\pgfpathlineto{\pgfqpoint{1.755405in}{0.986854in}}%
\pgfpathlineto{\pgfqpoint{1.778679in}{1.044424in}}%
\pgfpathlineto{\pgfqpoint{1.815916in}{1.146187in}}%
\pgfpathlineto{\pgfqpoint{1.885736in}{1.339361in}}%
\pgfpathlineto{\pgfqpoint{1.918318in}{1.419599in}}%
\pgfpathlineto{\pgfqpoint{1.946246in}{1.479766in}}%
\pgfpathlineto{\pgfqpoint{1.969520in}{1.522847in}}%
\pgfpathlineto{\pgfqpoint{1.992793in}{1.559050in}}%
\pgfpathlineto{\pgfqpoint{2.011411in}{1.582936in}}%
\pgfpathlineto{\pgfqpoint{2.030030in}{1.602324in}}%
\pgfpathlineto{\pgfqpoint{2.048649in}{1.617316in}}%
\pgfpathlineto{\pgfqpoint{2.067267in}{1.628093in}}%
\pgfpathlineto{\pgfqpoint{2.085886in}{1.634902in}}%
\pgfpathlineto{\pgfqpoint{2.104505in}{1.638051in}}%
\pgfpathlineto{\pgfqpoint{2.123123in}{1.637897in}}%
\pgfpathlineto{\pgfqpoint{2.141742in}{1.634842in}}%
\pgfpathlineto{\pgfqpoint{2.165015in}{1.627605in}}%
\pgfpathlineto{\pgfqpoint{2.192943in}{1.615102in}}%
\pgfpathlineto{\pgfqpoint{2.234835in}{1.591927in}}%
\pgfpathlineto{\pgfqpoint{2.290691in}{1.561301in}}%
\pgfpathlineto{\pgfqpoint{2.318619in}{1.549565in}}%
\pgfpathlineto{\pgfqpoint{2.341892in}{1.542955in}}%
\pgfpathlineto{\pgfqpoint{2.365165in}{1.539964in}}%
\pgfpathlineto{\pgfqpoint{2.383784in}{1.540583in}}%
\pgfpathlineto{\pgfqpoint{2.402402in}{1.544162in}}%
\pgfpathlineto{\pgfqpoint{2.421021in}{1.550917in}}%
\pgfpathlineto{\pgfqpoint{2.439640in}{1.561022in}}%
\pgfpathlineto{\pgfqpoint{2.458258in}{1.574603in}}%
\pgfpathlineto{\pgfqpoint{2.476877in}{1.591740in}}%
\pgfpathlineto{\pgfqpoint{2.495495in}{1.612469in}}%
\pgfpathlineto{\pgfqpoint{2.518769in}{1.643411in}}%
\pgfpathlineto{\pgfqpoint{2.542042in}{1.679827in}}%
\pgfpathlineto{\pgfqpoint{2.565315in}{1.721494in}}%
\pgfpathlineto{\pgfqpoint{2.593243in}{1.777965in}}%
\pgfpathlineto{\pgfqpoint{2.625826in}{1.851832in}}%
\pgfpathlineto{\pgfqpoint{2.663063in}{1.944982in}}%
\pgfpathlineto{\pgfqpoint{2.709610in}{2.070691in}}%
\pgfpathlineto{\pgfqpoint{2.839940in}{2.428938in}}%
\pgfpathlineto{\pgfqpoint{2.877177in}{2.519274in}}%
\pgfpathlineto{\pgfqpoint{2.905105in}{2.580055in}}%
\pgfpathlineto{\pgfqpoint{2.933033in}{2.633633in}}%
\pgfpathlineto{\pgfqpoint{2.956306in}{2.672076in}}%
\pgfpathlineto{\pgfqpoint{2.979580in}{2.704356in}}%
\pgfpathlineto{\pgfqpoint{2.998198in}{2.725466in}}%
\pgfpathlineto{\pgfqpoint{3.016817in}{2.742199in}}%
\pgfpathlineto{\pgfqpoint{3.035435in}{2.754419in}}%
\pgfpathlineto{\pgfqpoint{3.049399in}{2.760562in}}%
\pgfpathlineto{\pgfqpoint{3.063363in}{2.764082in}}%
\pgfpathlineto{\pgfqpoint{3.077327in}{2.764963in}}%
\pgfpathlineto{\pgfqpoint{3.091291in}{2.763201in}}%
\pgfpathlineto{\pgfqpoint{3.105255in}{2.758805in}}%
\pgfpathlineto{\pgfqpoint{3.119219in}{2.751793in}}%
\pgfpathlineto{\pgfqpoint{3.137838in}{2.738434in}}%
\pgfpathlineto{\pgfqpoint{3.156456in}{2.720593in}}%
\pgfpathlineto{\pgfqpoint{3.175075in}{2.698413in}}%
\pgfpathlineto{\pgfqpoint{3.198348in}{2.664866in}}%
\pgfpathlineto{\pgfqpoint{3.221622in}{2.625247in}}%
\pgfpathlineto{\pgfqpoint{3.249550in}{2.570397in}}%
\pgfpathlineto{\pgfqpoint{3.277477in}{2.508518in}}%
\pgfpathlineto{\pgfqpoint{3.310060in}{2.428938in}}%
\pgfpathlineto{\pgfqpoint{3.351952in}{2.317944in}}%
\pgfpathlineto{\pgfqpoint{3.431081in}{2.096616in}}%
\pgfpathlineto{\pgfqpoint{3.486937in}{1.944982in}}%
\pgfpathlineto{\pgfqpoint{3.524174in}{1.851832in}}%
\pgfpathlineto{\pgfqpoint{3.556757in}{1.777965in}}%
\pgfpathlineto{\pgfqpoint{3.584685in}{1.721494in}}%
\pgfpathlineto{\pgfqpoint{3.612613in}{1.672115in}}%
\pgfpathlineto{\pgfqpoint{3.635886in}{1.636780in}}%
\pgfpathlineto{\pgfqpoint{3.659159in}{1.606950in}}%
\pgfpathlineto{\pgfqpoint{3.682432in}{1.582724in}}%
\pgfpathlineto{\pgfqpoint{3.701051in}{1.567372in}}%
\pgfpathlineto{\pgfqpoint{3.719670in}{1.555542in}}%
\pgfpathlineto{\pgfqpoint{3.738288in}{1.547130in}}%
\pgfpathlineto{\pgfqpoint{3.756907in}{1.541988in}}%
\pgfpathlineto{\pgfqpoint{3.775526in}{1.539919in}}%
\pgfpathlineto{\pgfqpoint{3.794144in}{1.540686in}}%
\pgfpathlineto{\pgfqpoint{3.817417in}{1.545201in}}%
\pgfpathlineto{\pgfqpoint{3.840691in}{1.553071in}}%
\pgfpathlineto{\pgfqpoint{3.868619in}{1.565923in}}%
\pgfpathlineto{\pgfqpoint{3.910511in}{1.589227in}}%
\pgfpathlineto{\pgfqpoint{3.961712in}{1.617413in}}%
\pgfpathlineto{\pgfqpoint{3.989640in}{1.629319in}}%
\pgfpathlineto{\pgfqpoint{4.012913in}{1.635855in}}%
\pgfpathlineto{\pgfqpoint{4.031532in}{1.638224in}}%
\pgfpathlineto{\pgfqpoint{4.050150in}{1.637588in}}%
\pgfpathlineto{\pgfqpoint{4.068769in}{1.633556in}}%
\pgfpathlineto{\pgfqpoint{4.087387in}{1.625781in}}%
\pgfpathlineto{\pgfqpoint{4.106006in}{1.613972in}}%
\pgfpathlineto{\pgfqpoint{4.124625in}{1.597894in}}%
\pgfpathlineto{\pgfqpoint{4.143243in}{1.577389in}}%
\pgfpathlineto{\pgfqpoint{4.161862in}{1.552371in}}%
\pgfpathlineto{\pgfqpoint{4.180480in}{1.522847in}}%
\pgfpathlineto{\pgfqpoint{4.203754in}{1.479766in}}%
\pgfpathlineto{\pgfqpoint{4.227027in}{1.430226in}}%
\pgfpathlineto{\pgfqpoint{4.254955in}{1.363247in}}%
\pgfpathlineto{\pgfqpoint{4.287538in}{1.276983in}}%
\pgfpathlineto{\pgfqpoint{4.389940in}{0.997826in}}%
\pgfpathlineto{\pgfqpoint{4.413213in}{0.946374in}}%
\pgfpathlineto{\pgfqpoint{4.431832in}{0.912501in}}%
\pgfpathlineto{\pgfqpoint{4.445796in}{0.892396in}}%
\pgfpathlineto{\pgfqpoint{4.459760in}{0.877606in}}%
\pgfpathlineto{\pgfqpoint{4.469069in}{0.871050in}}%
\pgfpathlineto{\pgfqpoint{4.478378in}{0.867384in}}%
\pgfpathlineto{\pgfqpoint{4.487688in}{0.866819in}}%
\pgfpathlineto{\pgfqpoint{4.496997in}{0.869567in}}%
\pgfpathlineto{\pgfqpoint{4.506306in}{0.875836in}}%
\pgfpathlineto{\pgfqpoint{4.515616in}{0.885834in}}%
\pgfpathlineto{\pgfqpoint{4.524925in}{0.899762in}}%
\pgfpathlineto{\pgfqpoint{4.538889in}{0.928449in}}%
\pgfpathlineto{\pgfqpoint{4.552853in}{0.967049in}}%
\pgfpathlineto{\pgfqpoint{4.566817in}{1.016145in}}%
\pgfpathlineto{\pgfqpoint{4.580781in}{1.076259in}}%
\pgfpathlineto{\pgfqpoint{4.594745in}{1.147839in}}%
\pgfpathlineto{\pgfqpoint{4.613363in}{1.261720in}}%
\pgfpathlineto{\pgfqpoint{4.631982in}{1.397156in}}%
\pgfpathlineto{\pgfqpoint{4.650601in}{1.554276in}}%
\pgfpathlineto{\pgfqpoint{4.673874in}{1.780548in}}%
\pgfpathlineto{\pgfqpoint{4.697147in}{2.038008in}}%
\pgfpathlineto{\pgfqpoint{4.725075in}{2.382552in}}%
\pgfpathlineto{\pgfqpoint{4.762312in}{2.883494in}}%
\pgfpathlineto{\pgfqpoint{4.810376in}{3.533889in}}%
\pgfpathmoveto{\pgfqpoint{4.958990in}{3.533889in}}%
\pgfpathlineto{\pgfqpoint{4.971772in}{3.176763in}}%
\pgfpathlineto{\pgfqpoint{4.985736in}{2.671853in}}%
\pgfpathlineto{\pgfqpoint{4.999700in}{2.032486in}}%
\pgfpathlineto{\pgfqpoint{5.013664in}{1.240825in}}%
\pgfpathlineto{\pgfqpoint{5.024783in}{0.486111in}}%
\pgfpathlineto{\pgfqpoint{5.024783in}{0.486111in}}%
\pgfusepath{stroke}%
\end{pgfscope}%
\begin{pgfscope}%
\pgfpathrectangle{\pgfqpoint{0.750000in}{0.500000in}}{\pgfqpoint{4.650000in}{3.020000in}}%
\pgfusepath{clip}%
\pgfsetrectcap%
\pgfsetroundjoin%
\pgfsetlinewidth{1.505625pt}%
\definecolor{currentstroke}{rgb}{0.549020,0.337255,0.294118}%
\pgfsetstrokecolor{currentstroke}%
\pgfsetdash{}{0pt}%
\pgfpathmoveto{\pgfqpoint{0.750000in}{1.295811in}}%
\pgfpathlineto{\pgfqpoint{1.001351in}{1.305950in}}%
\pgfpathlineto{\pgfqpoint{1.206156in}{1.317243in}}%
\pgfpathlineto{\pgfqpoint{1.373724in}{1.329474in}}%
\pgfpathlineto{\pgfqpoint{1.518018in}{1.343075in}}%
\pgfpathlineto{\pgfqpoint{1.639039in}{1.357496in}}%
\pgfpathlineto{\pgfqpoint{1.746096in}{1.373329in}}%
\pgfpathlineto{\pgfqpoint{1.839189in}{1.390172in}}%
\pgfpathlineto{\pgfqpoint{1.922973in}{1.408478in}}%
\pgfpathlineto{\pgfqpoint{1.997447in}{1.427912in}}%
\pgfpathlineto{\pgfqpoint{2.062613in}{1.447953in}}%
\pgfpathlineto{\pgfqpoint{2.123123in}{1.469666in}}%
\pgfpathlineto{\pgfqpoint{2.178979in}{1.492916in}}%
\pgfpathlineto{\pgfqpoint{2.230180in}{1.517463in}}%
\pgfpathlineto{\pgfqpoint{2.276727in}{1.542957in}}%
\pgfpathlineto{\pgfqpoint{2.323273in}{1.572003in}}%
\pgfpathlineto{\pgfqpoint{2.365165in}{1.601680in}}%
\pgfpathlineto{\pgfqpoint{2.407057in}{1.635235in}}%
\pgfpathlineto{\pgfqpoint{2.444294in}{1.668793in}}%
\pgfpathlineto{\pgfqpoint{2.481532in}{1.706341in}}%
\pgfpathlineto{\pgfqpoint{2.518769in}{1.748387in}}%
\pgfpathlineto{\pgfqpoint{2.556006in}{1.795477in}}%
\pgfpathlineto{\pgfqpoint{2.593243in}{1.848168in}}%
\pgfpathlineto{\pgfqpoint{2.630480in}{1.907001in}}%
\pgfpathlineto{\pgfqpoint{2.667718in}{1.972439in}}%
\pgfpathlineto{\pgfqpoint{2.704955in}{2.044774in}}%
\pgfpathlineto{\pgfqpoint{2.742192in}{2.123995in}}%
\pgfpathlineto{\pgfqpoint{2.784084in}{2.220704in}}%
\pgfpathlineto{\pgfqpoint{2.839940in}{2.358824in}}%
\pgfpathlineto{\pgfqpoint{2.914414in}{2.543683in}}%
\pgfpathlineto{\pgfqpoint{2.946997in}{2.616442in}}%
\pgfpathlineto{\pgfqpoint{2.970270in}{2.662209in}}%
\pgfpathlineto{\pgfqpoint{2.993544in}{2.701099in}}%
\pgfpathlineto{\pgfqpoint{3.012162in}{2.726310in}}%
\pgfpathlineto{\pgfqpoint{3.030781in}{2.745589in}}%
\pgfpathlineto{\pgfqpoint{3.044745in}{2.755851in}}%
\pgfpathlineto{\pgfqpoint{3.058709in}{2.762336in}}%
\pgfpathlineto{\pgfqpoint{3.072673in}{2.764946in}}%
\pgfpathlineto{\pgfqpoint{3.086637in}{2.763640in}}%
\pgfpathlineto{\pgfqpoint{3.100601in}{2.758438in}}%
\pgfpathlineto{\pgfqpoint{3.114565in}{2.749421in}}%
\pgfpathlineto{\pgfqpoint{3.128529in}{2.736725in}}%
\pgfpathlineto{\pgfqpoint{3.147147in}{2.714409in}}%
\pgfpathlineto{\pgfqpoint{3.165766in}{2.686462in}}%
\pgfpathlineto{\pgfqpoint{3.189039in}{2.644644in}}%
\pgfpathlineto{\pgfqpoint{3.216967in}{2.586306in}}%
\pgfpathlineto{\pgfqpoint{3.249550in}{2.510294in}}%
\pgfpathlineto{\pgfqpoint{3.310060in}{2.358824in}}%
\pgfpathlineto{\pgfqpoint{3.370571in}{2.209604in}}%
\pgfpathlineto{\pgfqpoint{3.412462in}{2.113727in}}%
\pgfpathlineto{\pgfqpoint{3.449700in}{2.035352in}}%
\pgfpathlineto{\pgfqpoint{3.486937in}{1.963886in}}%
\pgfpathlineto{\pgfqpoint{3.524174in}{1.899294in}}%
\pgfpathlineto{\pgfqpoint{3.561411in}{1.841256in}}%
\pgfpathlineto{\pgfqpoint{3.598649in}{1.789296in}}%
\pgfpathlineto{\pgfqpoint{3.635886in}{1.742867in}}%
\pgfpathlineto{\pgfqpoint{3.673123in}{1.701412in}}%
\pgfpathlineto{\pgfqpoint{3.710360in}{1.664390in}}%
\pgfpathlineto{\pgfqpoint{3.752252in}{1.627414in}}%
\pgfpathlineto{\pgfqpoint{3.794144in}{1.594769in}}%
\pgfpathlineto{\pgfqpoint{3.836036in}{1.565882in}}%
\pgfpathlineto{\pgfqpoint{3.882583in}{1.537591in}}%
\pgfpathlineto{\pgfqpoint{3.929129in}{1.512744in}}%
\pgfpathlineto{\pgfqpoint{3.980330in}{1.488802in}}%
\pgfpathlineto{\pgfqpoint{4.036186in}{1.466107in}}%
\pgfpathlineto{\pgfqpoint{4.096697in}{1.444893in}}%
\pgfpathlineto{\pgfqpoint{4.161862in}{1.425296in}}%
\pgfpathlineto{\pgfqpoint{4.236336in}{1.406272in}}%
\pgfpathlineto{\pgfqpoint{4.315465in}{1.389250in}}%
\pgfpathlineto{\pgfqpoint{4.403904in}{1.373329in}}%
\pgfpathlineto{\pgfqpoint{4.506306in}{1.358118in}}%
\pgfpathlineto{\pgfqpoint{4.622673in}{1.344075in}}%
\pgfpathlineto{\pgfqpoint{4.757658in}{1.331048in}}%
\pgfpathlineto{\pgfqpoint{4.911261in}{1.319377in}}%
\pgfpathlineto{\pgfqpoint{5.097447in}{1.308470in}}%
\pgfpathlineto{\pgfqpoint{5.320871in}{1.298653in}}%
\pgfpathlineto{\pgfqpoint{5.400000in}{1.295811in}}%
\pgfpathlineto{\pgfqpoint{5.400000in}{1.295811in}}%
\pgfusepath{stroke}%
\end{pgfscope}%
\begin{pgfscope}%
\pgfsetrectcap%
\pgfsetmiterjoin%
\pgfsetlinewidth{0.803000pt}%
\definecolor{currentstroke}{rgb}{0.000000,0.000000,0.000000}%
\pgfsetstrokecolor{currentstroke}%
\pgfsetdash{}{0pt}%
\pgfpathmoveto{\pgfqpoint{0.750000in}{0.500000in}}%
\pgfpathlineto{\pgfqpoint{0.750000in}{3.520000in}}%
\pgfusepath{stroke}%
\end{pgfscope}%
\begin{pgfscope}%
\pgfsetrectcap%
\pgfsetmiterjoin%
\pgfsetlinewidth{0.803000pt}%
\definecolor{currentstroke}{rgb}{0.000000,0.000000,0.000000}%
\pgfsetstrokecolor{currentstroke}%
\pgfsetdash{}{0pt}%
\pgfpathmoveto{\pgfqpoint{5.400000in}{0.500000in}}%
\pgfpathlineto{\pgfqpoint{5.400000in}{3.520000in}}%
\pgfusepath{stroke}%
\end{pgfscope}%
\begin{pgfscope}%
\pgfsetrectcap%
\pgfsetmiterjoin%
\pgfsetlinewidth{0.803000pt}%
\definecolor{currentstroke}{rgb}{0.000000,0.000000,0.000000}%
\pgfsetstrokecolor{currentstroke}%
\pgfsetdash{}{0pt}%
\pgfpathmoveto{\pgfqpoint{0.750000in}{0.500000in}}%
\pgfpathlineto{\pgfqpoint{5.400000in}{0.500000in}}%
\pgfusepath{stroke}%
\end{pgfscope}%
\begin{pgfscope}%
\pgfsetrectcap%
\pgfsetmiterjoin%
\pgfsetlinewidth{0.803000pt}%
\definecolor{currentstroke}{rgb}{0.000000,0.000000,0.000000}%
\pgfsetstrokecolor{currentstroke}%
\pgfsetdash{}{0pt}%
\pgfpathmoveto{\pgfqpoint{0.750000in}{3.520000in}}%
\pgfpathlineto{\pgfqpoint{5.400000in}{3.520000in}}%
\pgfusepath{stroke}%
\end{pgfscope}%
\begin{pgfscope}%
\pgfsetbuttcap%
\pgfsetmiterjoin%
\definecolor{currentfill}{rgb}{1.000000,1.000000,1.000000}%
\pgfsetfillcolor{currentfill}%
\pgfsetfillopacity{0.800000}%
\pgfsetlinewidth{1.003750pt}%
\definecolor{currentstroke}{rgb}{0.800000,0.800000,0.800000}%
\pgfsetstrokecolor{currentstroke}%
\pgfsetstrokeopacity{0.800000}%
\pgfsetdash{}{0pt}%
\pgfpathmoveto{\pgfqpoint{2.649058in}{0.569444in}}%
\pgfpathlineto{\pgfqpoint{3.500942in}{0.569444in}}%
\pgfpathquadraticcurveto{\pgfqpoint{3.528719in}{0.569444in}}{\pgfqpoint{3.528719in}{0.597222in}}%
\pgfpathlineto{\pgfqpoint{3.528719in}{1.745061in}}%
\pgfpathquadraticcurveto{\pgfqpoint{3.528719in}{1.772839in}}{\pgfqpoint{3.500942in}{1.772839in}}%
\pgfpathlineto{\pgfqpoint{2.649058in}{1.772839in}}%
\pgfpathquadraticcurveto{\pgfqpoint{2.621281in}{1.772839in}}{\pgfqpoint{2.621281in}{1.745061in}}%
\pgfpathlineto{\pgfqpoint{2.621281in}{0.597222in}}%
\pgfpathquadraticcurveto{\pgfqpoint{2.621281in}{0.569444in}}{\pgfqpoint{2.649058in}{0.569444in}}%
\pgfpathclose%
\pgfusepath{stroke,fill}%
\end{pgfscope}%
\begin{pgfscope}%
\pgfsetrectcap%
\pgfsetroundjoin%
\pgfsetlinewidth{1.505625pt}%
\definecolor{currentstroke}{rgb}{0.121569,0.466667,0.705882}%
\pgfsetstrokecolor{currentstroke}%
\pgfsetdash{}{0pt}%
\pgfpathmoveto{\pgfqpoint{2.676836in}{1.668672in}}%
\pgfpathlineto{\pgfqpoint{2.954614in}{1.668672in}}%
\pgfusepath{stroke}%
\end{pgfscope}%
\begin{pgfscope}%
\pgftext[x=3.065725in,y=1.620061in,left,base]{\rmfamily\fontsize{10.000000}{12.000000}\selectfont \(\displaystyle  n = 2 \)}%
\end{pgfscope}%
\begin{pgfscope}%
\pgfsetrectcap%
\pgfsetroundjoin%
\pgfsetlinewidth{1.505625pt}%
\definecolor{currentstroke}{rgb}{1.000000,0.498039,0.054902}%
\pgfsetstrokecolor{currentstroke}%
\pgfsetdash{}{0pt}%
\pgfpathmoveto{\pgfqpoint{2.676836in}{1.475061in}}%
\pgfpathlineto{\pgfqpoint{2.954614in}{1.475061in}}%
\pgfusepath{stroke}%
\end{pgfscope}%
\begin{pgfscope}%
\pgftext[x=3.065725in,y=1.426450in,left,base]{\rmfamily\fontsize{10.000000}{12.000000}\selectfont \(\displaystyle  n = 4 \)}%
\end{pgfscope}%
\begin{pgfscope}%
\pgfsetrectcap%
\pgfsetroundjoin%
\pgfsetlinewidth{1.505625pt}%
\definecolor{currentstroke}{rgb}{0.172549,0.627451,0.172549}%
\pgfsetstrokecolor{currentstroke}%
\pgfsetdash{}{0pt}%
\pgfpathmoveto{\pgfqpoint{2.676836in}{1.281450in}}%
\pgfpathlineto{\pgfqpoint{2.954614in}{1.281450in}}%
\pgfusepath{stroke}%
\end{pgfscope}%
\begin{pgfscope}%
\pgftext[x=3.065725in,y=1.232839in,left,base]{\rmfamily\fontsize{10.000000}{12.000000}\selectfont \(\displaystyle  n = 6 \)}%
\end{pgfscope}%
\begin{pgfscope}%
\pgfsetrectcap%
\pgfsetroundjoin%
\pgfsetlinewidth{1.505625pt}%
\definecolor{currentstroke}{rgb}{0.839216,0.152941,0.156863}%
\pgfsetstrokecolor{currentstroke}%
\pgfsetdash{}{0pt}%
\pgfpathmoveto{\pgfqpoint{2.676836in}{1.087839in}}%
\pgfpathlineto{\pgfqpoint{2.954614in}{1.087839in}}%
\pgfusepath{stroke}%
\end{pgfscope}%
\begin{pgfscope}%
\pgftext[x=3.065725in,y=1.039228in,left,base]{\rmfamily\fontsize{10.000000}{12.000000}\selectfont \(\displaystyle  n = 8 \)}%
\end{pgfscope}%
\begin{pgfscope}%
\pgfsetrectcap%
\pgfsetroundjoin%
\pgfsetlinewidth{1.505625pt}%
\definecolor{currentstroke}{rgb}{0.580392,0.403922,0.741176}%
\pgfsetstrokecolor{currentstroke}%
\pgfsetdash{}{0pt}%
\pgfpathmoveto{\pgfqpoint{2.676836in}{0.894228in}}%
\pgfpathlineto{\pgfqpoint{2.954614in}{0.894228in}}%
\pgfusepath{stroke}%
\end{pgfscope}%
\begin{pgfscope}%
\pgftext[x=3.065725in,y=0.845617in,left,base]{\rmfamily\fontsize{10.000000}{12.000000}\selectfont \(\displaystyle  n = 10 \)}%
\end{pgfscope}%
\begin{pgfscope}%
\pgfsetrectcap%
\pgfsetroundjoin%
\pgfsetlinewidth{1.505625pt}%
\definecolor{currentstroke}{rgb}{0.549020,0.337255,0.294118}%
\pgfsetstrokecolor{currentstroke}%
\pgfsetdash{}{0pt}%
\pgfpathmoveto{\pgfqpoint{2.676836in}{0.700617in}}%
\pgfpathlineto{\pgfqpoint{2.954614in}{0.700617in}}%
\pgfusepath{stroke}%
\end{pgfscope}%
\begin{pgfscope}%
\pgftext[x=3.065725in,y=0.652006in,left,base]{\rmfamily\fontsize{10.000000}{12.000000}\selectfont \(\displaystyle f_1\)}%
\end{pgfscope}%
\end{pgfpicture}%
\makeatother%
\endgroup%
}
\caption{Graph of $ p_2 - f $} \label{Fig:Residue}
\end{figure}
The final polynomial is
\begin{equation}
\begin{split}
p_2 \rbr{x} &= - 9.600345 \cdot 10^{-11} x^{5} + 0.603579 x^{4} \\
&+ 1.136689 \cdot 10^{-10} x^{3} + 0.414182 x^{2} - 1.766548 \cdot 10^{-11} x - 0.008881,
\end{split}
\end{equation}
while the final control points are
\begin{equation}
-1.000000, -0.832979, -0.414353, -0.000000, 0.414353, 0.832979, 1.000000.
\end{equation}
\end{thmquestion}

\begin{thmquestion}
\
\begin{thmproof}
Consider the space to be continuos function on an interval $ \sbr{ a, b } $ and the inner product to be
\begin{equation}
\pbr{ f, g } = \intb{a}{b}{ f \rbr{x} g \rbr{x} \sd x }.
\end{equation}

Suppose zeros of $p_i$ are $ \xi_1 < \xi_2 < \cdots < \xi_k $ with $ k < i $, and without loss of generality assume $p_i$ are positive, negative, positive, \ldots on $ \rsbr{ \xi_k, b }, \rbr{ \xi_{ k - 1 }, \xi_k }, \cdots, \rbr{ \xi_1, \xi_2 }, \srbr{ a , \xi_1 } $ respectively. Consider
\begin{equation}
g = \rbr{ x - \xi_1 } \rbr{ x - \xi_2 } \cdots \rbr{ x - \xi_k }.
\end{equation}
Because $ p_i g \ge 0 $, and $ p_i g $ have some strictly positive points, therefore
\begin{equation}
\pbr{ p_i, g } = \intb{a}{b}{ p_i \rbr{x} g \rbr{x} \sd x } > 0.
\end{equation}
However, because $ \deg g = k < i $, therefore $ g \in \opspan \cbr{ p_0, p_1, \cdots, p_{ i - 1 } } $ and $ \pbr{ p_i, g } = 0 $, which leads to contradiction. Consequently, $p_i$ has at least $i$ zeros. Because $p_i$ is a polynomial of degree $i$, $p_i$ has exactly $i$ zeros in $ \sbr{ a, b } $.
\end{thmproof}
\end{thmquestion}

\end{document}
