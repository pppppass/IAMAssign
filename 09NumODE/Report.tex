% !TeX encoding = UTF-8
% !TeX program = LuaLaTeX
% !TeX spellcheck = en_US

% Author : Zhihan Li
% Description : Report for Lecture 9

\documentclass[english, nochinese]{pkupaper}

\usepackage[paper, extables]{def}

\newcommand{\cuniversity}{Peking University}
\newcommand{\cthesisname}{Introduction to Applied Mathematics}
\newcommand{\titlemark}{Report of Assignment for Lecture 9}

\DeclareRobustCommand{\authoring}%
{%
\begin{tabular}{c}%
Zhihan Li \\%
1600010653%
\end{tabular}%
}

\title{\titlemark}
\author{\authoring}
\begin{document}

\maketitle

\begin{thmquestion}
\ 
\begin{thmproof}
We have
\begin{align}
K_1 &= f, \\
K_2 &= f + h t \rbr{ f_x + K_1 f_y } + h^2 t^2 \rbr{ \frac{1}{2} f_{ x x } + K_1 f_{ x y } + \frac{1}{2} K_1^2 f_{ y y } } + O \rbr{h^3}, \\
K_2 &= f + h \rbr{ 1 - t } \rbr{ f_x + K_1 f_y } + h^2 \rbr{ 1 - t }^2 \rbr{ \frac{1}{2} f_{ x x } + K_1 f_{ x y } + \frac{1}{2} K_1^2 f_{ y y } } + O \rbr{h^3}
\end{align}
and
\begin{align}
y' &= f, \\
y'' &= f_x + f f_y, \\
y''' &= f_{ x x } + 2 f f_{ x y } + f f f_{ y y } + f_x f_y + f_y f_y f
\end{align}
where $f$, $f_x$ denote values at $ \rbr{ x_n, y_n } $ and $y$, $y'$ denote value at $x_n$. Combining these equations, the truncated error
\begin{equation}
\begin{split}
T_n &= y \rbr{x_{ n + 1 }} - y_{ n + 1 } \\
&= y + h y' + \frac{1}{2} h^2 y'' + \frac{1}{6} h^3 y''' + O \rbr{h^4} - y - \frac{1}{2} h \rbr{ K_2 + K_3 } \\
&= h^3 \rbr{ \frac{ -1 + 6 t - 6 t^2 }{12} \rbr{ f_{ x x } + 2 f f_{ x y } + f f f_{ y y } } + \frac{1}{6} f_x f_y + \frac{1}{6} f_y f_y f } + O \rbr{h^4}
\end{split}
\end{equation}
is $ O \rbr{h^3} $, and this scheme is of second order. However, $T_n$ being $ O \rbr{h^4} $ cannot be established because terms like $ h^3 f_x f_y $ are present.

In conclusion, the given scheme is of exact second order.

\sqed
\end{thmproof}
\end{thmquestion}

\begin{thmquestion}
\ 
\begin{thmanswer}
Consider the ordinary differential equation
\begin{gather}
y' = e^{x^2}, \\
y \rbr{0} = 0,
\end{gather}
and it suffices to find values of $y$ at $ x = 0.5, 1.0, 1.5, 2.0 $.

We deploy forward Euler method, backward Euler method and trapezoid method (which coincides the refined Euler method) here. Numerical results are shown in Table \ref{Tbl:Euler}, and codes are given in Python in \verb"Problem2.ipynb". Note that $n$ is the number of division and $ h = x / n $.

\begin{table}[htbp]
\centering
\caption{Numerical results of Euler methods}
\label{Tbl:Euler}
\begin{tabular}{|c|c|c|c|c|}
\hline
$x$ & $n$ & Forward & Backward & Trapezoid \\
\hline
\multirow{4}*{0.5} & 5 & 0.53185 & 0.56026 & 0.54606 \\
\cline{2-5}
& 10 & 0.53815 & 0.55236 & 0.54525 \\
\cline{2-5}
& 20 & 0.54150 & 0.54860 & 0.54505 \\
\cline{2-5}
& 50 & 0.54358 & 0.54642 & 0.54500 \\
\hline
\multirow{4}*{1.0} & 5 & 1.30883 & 1.65248 & 1.48065 \\
\cline{2-5}
& 10 & 1.38126 & 1.55309 & 1.46717 \\
\cline{2-5}
& 20 & 1.42083 & 1.50674 & 1.46378 \\
\cline{2-5}
& 50 & 1.44565 & 1.48002 & 1.46283 \\
\hline
\multirow{4}*{1.5} & 5 & 2.99883 & 5.54515 & 4.27199 \\
\cline{2-5}
& 10 & 3.47961 & 4.75277 & 4.11619 \\
\cline{2-5}
& 20 & 3.75815 & 4.39473 & 4.07644 \\
\cline{2-5}
& 50 & 3.93793 & 4.19256 & 4.06525 \\
\hline
\multirow{4}*{2.0} & 5 & 8.49060 & 29.92986 & 19.21023 \\
\cline{2-5}
& 10 & 11.81040 & 22.53003 & 17.17021 \\
\cline{2-5}
& 20 & 13.95405 & 19.31387 & 16.63396 \\
\cline{2-5}
& 50 & 15.40977 & 17.55369 & 16.48173 \\
\hline
\end{tabular}
\end{table}
It can be clearly seen that results of forward Euler method increases, and that of backward Euler methods decreases. Trapezoid method here also decreases, but in a slower manner.
\end{thmanswer}
\end{thmquestion}

\begin{thmquestion}
\ 
\begin{thmproof}
We have
\begin{align}
K_1 &= f, \\
K_2 &= f + h \rbr{ \lambda_2 f_x + \mu_{ 2 1 } K_1 } + h^2 \rbr{ \frac{1}{2} \lambda_2^2 f_{ x x } + \lambda_2 \mu_{ 2 1 } K_1 f_{ x y } + \frac{1}{2} \mu_{ 2 1 } K_1^2 f_{ y y } } + O \rbr{h^3}
\end{align}
and
\begin{align}
y' &= f, \\
y'' &= f_x + f f_y, \\
y''' &= f_{ x x } + 2 f f_{ x y } + f f f_{ y y } + f_x f_y + f_y f_y f.
\end{align}
Combining these equations, the truncated error
\begin{equation}
\begin{split}
T_n &= y_{x_{ n + 1 }} - y_{ n + 1 } \\
&= y + h y' + \frac{1}{2} h^2 y'' + \frac{1}{6} h^3 y''' + O \rbr{h^4} - y - h \rbr{ c_1 K_1 + c_2 K_2 } \\
&= h \rbr{ 1 - c_1 - c_2 } f + h^2 \rbr{ \rbr{ \frac{1}{2} - c_2 \lambda_2 } f_x + \rbr{ \frac{1}{2} - c_2 \mu_{ 2 1 } } f f_y } \\
&= h^3 \rbr{ \rbr{ \frac{1}{6} - \frac{1}{2} c_2 \lambda_2^2 } f_{ x x } + \rbr{ \frac{1}{3} - c_2 \lambda_2 \mu_{ 2 1 } } f f_{ x y } + \rbr{ \frac{1}{6} - \frac{1}{2} c_2 \mu_{ 2 1 }^2 } f f f_{ y y } + \frac{1}{6} f_x f_y + \frac{1}{6} f f_y f_y } + O \rbr{h^4}.
\end{split}
\end{equation}
Note that terms like $ h^3 f_x f_y $ are present and the coefficient is fixed ($ 1 / 6 $), and therefore such scheme can never reach third order.

\sqed
\end{thmproof}
\end{thmquestion}

\end{document}
